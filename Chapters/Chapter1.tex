\chapter{Motivation for a New Precise Measurement of the nEDM}
\lhead{\emph{Motivation for a New Precise Measurement of the nEDM}}\label{ch:motivation} 
This chapter highlights the scientific interest in a new precise measurement of the neutron electric dipole moment (nEDM). The measurement principle of the nEDM experiment is discussed with the importance of the magnetic environment for the successfulness of the experiment. Finally, this chapter ends with describing the TRIUMF Ultra Cold Advanced Neutron (TUCAN) EDM experiment.

\section{Baryon Asymmetry and the nEDM}

\fig{Images/SM_Pic3}{width = 0.8\textwidth}{ Standard model of elementary particles~\cite{SM_Pic}. Three generations of matter aligned column-wise and classified into two groups : quarks and leptons.\label{fig:SM_Pic}}{standard model of elementary particles.}

The universe is composed of particles which are governed by the four fundamental forces (electromagnetic, weak interaction, strong interaction and gravitational force). The Standard Model (SM) is a theory to describe particle interactions excluding gravity. Quarks, leptons, gauge bosons, and the higgs bosons are the fundamental particles. They are shown in Fig.~\ref{fig:SM_Pic}. Baryons such as the neutron are formed from three quarks. In the early universe, there were equal particle numbers of matter and antimatter. But the universe today contains mostly baryonic matter. By experimental observations of the cosmic microwave background radiation (CMBR) the baryon asymmetry can be deduced to be~\cite{expBar}

% which classify the matters based on the quantum numbers namely baryon carried by quarks and lepton number carried by leptons. 

\begin{equation}\label{eq:baryons}
    \eta =\frac{n_b-\bar{n}_b}{n_\gamma}\simeq 6.0 \times 10^{-10}
\end{equation}
where $n_b$, $\bar{n}_b$ and ${n_\gamma}$ are the number of baryons, anti-baryons and photons respectively. 

% So, this $6.0 \times 10^{-10}$ excess baryons over anti baryons per photon is called the baryon asymmetry of the universe. Scientists are continuously searching for the answer of this mystery. The breakthrough is possible by finding the non zero value of nEDM. 

% The relation of the non zero nEDM with the baryon asymmetry has been discussed next.

% \section{Sakharov Criteria and nEDM}
Baryogenesis is the process of creation of a baryon asymmetry from an initially symmetric state. In 1967, A.D. Sakharov described following three conditions on particle theories aiming to explain baryogenesis~\cite{Sakharov:1967dj}:
\begin{enumerate}
    \item {\bf Baryon number (B) violation.} Any theory which creates net baryon number obviously requires B-violating processes.
    \item {\bf Charge (C) and Charge-Parity (CP) symmetry violation.} Particle and antiparticle reaction rates must also be different otherwise net baryon creation would be balanced by net anti-baryon creation, hence CP violation is required.
    \item {\bf Departure from thermal equilibrium.} In thermal equilibrium, forward reaction rates would balance reverse rates and no net baryon number could be produced, hence a departure from thermal equilibrium is required.
\end{enumerate}

The Big Bang and subsequent cooling of the universe offers a way to provide the departure from thermal equilibrium in many models of baryogenesis~\cite{sakharov_3rd_cond}. \textcolor{red}{Electroweak baryogenesis is a scenario which uses SM processes, the electroweak phase transition are the expansion and cooling of the universe to explain baryogenesis.} One drawback of the electroweak baryogenesis is that there is not enough CP violation in the SM. This motivates searches for new sources of CP violation near the weak scale.

% The first criterion can be explained by the second and third conditions. During the expansion of the universe immediately after the big bang, most of the particles were departed from the thermal equilibrium ( defined by the temperature of the universe between $10^{2}$ GeV and $10^{12}$ GeV ) at $\sim 10^{19}$ GeV temperature as suggested by Sakharov \cite{sakharov_3rd_cond} resulting in net non zero B number. The non zero B number also exists due to C and CP asymmetry. Because in the absence of those violations, excess baryon reactions can be nullified by the excess anti-baryon reactions \cite{sakharov_1st_cond}. Though the standard model of particle physics possesses CP violation, there is an insufficient amount of CP violation to explain the observed baryon asymmetry.  Theories of new physics beyond standard model, motivated in part by the baryon asymmetry, predict new sources of CP violation.

A fundamental symmetry of quantum field theories is CPT symmetry, which implies that time-reversal (T) violation is equivalent to CP violation.  An electric dipole moment (EDM) is a measure of separation of oppositely charged particles within a system. To have a nonzero EDM, the system should not violate both parity (P) and time-reversal (T) symmetry~\cite{edm_reason}. Because of the CPT theorem, a non-zero EDM represents a search for new physics that violates CP symmetry. The neutron may have an EDM with its magnitude depending on the nature and origin of the T violation~\cite{nEDM_reason}. A precise measurement of the nEDM is a very important measurement which could help solving the baryon asymmetry by a discovery of new physics.


% on particle physics beyond the known interaction in the SM. 

% \section{Experimental Efforts So Far}\label{sec:lim}
% \fig{Images/lim}{width = \textwidth}{Experimental nEDM upperlimit over the years \cite{1_lim,2_lim,3_lim,4_lim,5_lim,6_lim,7_lim,8_lim,9_lim,10_lim,11_lim,12_lim,13_lim,14_lim,15_lim,16_lim,17_lim} along with theoretical predictions \cite{theory_lim_1, theory_lim_2, theory_lim_3}.\label{fig:lim}}

% Additional sources of CP violation beyond the standard model predict the nEDM to be in the range of $10^{-27}$ -- $10^{-28}$ ~e$\cdot$cm as compared to $10^{-33}$ -- $10^{-31}$~e$\cdot$cm predicted by the standard model \cite{theory_lim_1, theory_lim_2, theory_lim_3}. There are several experiments aiming at improving the uncertainty on the nEDM. The Fig. \ref{fig:lim} summarize all the results. Since the first measurement data published in 1957 \cite{1_lim}, the upper limit set on nEDM has been reduced by eight orders of magnitude over the last six decades. At 1980, ultra cold neutron (UCN) experiment overtook over neutron beam experiments on precision. The current best upper limit set by Sussex-RAL-ILL nEDM experiment is $3.0 \times 10^{-26}$~e$\cdot$cm (90 \% C.L) or $3.6 \times 10^{-26}$~e$\cdot$cm (96 \% C.L)  \cite{bestLim_1,bestLim_2}. The experiment has been done at the Institut Laue-Langevin (ILL) which is situated at Grenoble, France. A new generation of UCN source known as superthermal UCN source which uses a new method of cooling by transferring energy to quantum excitations in a material have recently come online. To use such a UCN source, the nEDM appa

% The apparatus for nEDM experiment has been moved from ILL to a superthermal UCN source at Paul Scherrer Institut (PSI) which is situated at Villigen, Switzerland. UCN superthermal source use the cooling technique . A new UCN source generation have recently come online. They use a technique from condensed
% matter physics involving cooling by transferring energy to quantum excitations in a material.
% UCN sources that employ this method of cooling are known as superthermal sources and they are
% beginning to transform the landscape of fundamental neutron physics at various facilities in the
% world.
% The nEDM apparatus from ILL was moved to such a UCN source at Paul Scherrer Institut
% (PSI, Villigen, Switzerland). The apparatus was also improved and upgraded in several respects.
% Data-taking for this new nEDM experiment was completed recently and the analysis of the data is
% ongoing [9]. The expectation in the community is that this new result will improve the previous
% best by a factor of about three to four.
% Next generation UCN EDM experiments are now in preparation at a variety of sites aiming to
% improve the result by an order of magnitude or more. Experiments are either ongoing or planned
% at ILL [10, 11], PSI [12], the Gatchina reactor [10], the Forchungsreaktor Munchen II (FRM2)
% reactor [13], Los Alamos National Laboratory (LANL, Los Alamos, NM, USA) [14], the Spallation
% Neutron Source (SNS, Oak Ridge, TN, USA) [15], and our eort at TRIUMF [16, 17]. We discuss
% our relationship to these experiments in Section 4. Our goal of dn < 10��27 ecm within the next
% 6-7 years (running until 2024-25, as stated earlier) is competitive with these eorts. One of the key
% factors is our unique UCN source, which we are upgrading. We envision achieving UCN counting
% rates over 100 times larger than the previous best nEDM experiment and the recently completed
% experiment at PSI, and similar to or surpassing the plans of other experiments.


% The nEDM experiment at TRIUMF is aiming to constrain the uncertainty on the nEDM at the $10^{-27}$~e$\cdot$cm sensitivity level. 


\section{Ultracold Neutrons}
Ultracold neutrons (UCN) are used to measure the nEDM. They have small kinetic energies ($<$ 300 neV).  They can be confined in a material bottle because they are reflected at any angle of incidence off suitable material walls~\cite{ucn_storage}. Their interaction with the neutron optical potential of the walls through the strong force enables them to trap. This provides long time frame (the mean lifetime of neutron is $\tau_n=881.5$~s~\cite{mike}) for observation making them ideal for the nEDM experiment. First UCN were produced in 2017 at TRIUMF using superfluid-helium at 0.9 K as the UCN production medium~\cite{TRIUMF_UCN,taraneh_theis,TRIUMF_Beamline}.

\section{Measurement Principle of nEDM}\label{sec:nEDM}

For the extraction of the nEDM, a form of Ramsey's method of separated oscillatory fields~\cite{ramsey} is used. UCN are polarized by passage through a strong magnetic field and guided to the nEDM cell. 

\fig{Images/ramsey}{width =\textwidth}{Ramsey's method of separated oscillatory fields, as applied to the measurement of the nEDM. $\bm{E_0}$ field parallel to $\bm{B_0}$ field in the left where in the right they are antiparallel. \label{fig:ramsey}}{Ramsey's method of separated oscillatory fields.}

%\FloatBarrier

The method of the nEDM measurement is shown in Fig.~\ref{fig:ramsey}. At the beginning and end of free precession, short $\pi$/2 pulses are applied. Polarized neutron detection after the pulse sequence is used to measure the free spin precession frequency $v$ (the Larmor frequency). In the first instance (left one in Fig.~\ref{fig:ramsey}), an electric field ($\bm{E_0}$) parallel to the magnetic field ($\bm{B_0}$) will be applied giving a spin-precession frequency as
\begin{equation}\label{eq:up_freq}
    h v_{\Uparrow \Uparrow}=2\mu_n B_0+2 d_n E_0
\end{equation}
where, $\mu_n$ and $d_n$ are the magnetic and electric dipole moments respectively, $h$ is Planck's constant and arrows indicate parallel orientation of $\bm{E_0}$ and $\bm{B_0}$.
Now the same experiment is repeated with anti-parallel $\bm{E_0}$ (right one in Fig.~\ref{fig:ramsey}) which gives the spin-precession frequency
\begin{equation}\label{eq:down_freq}
    h v_{\Uparrow \Downarrow}=2\mu_n B_0-2 d_n E_0
\end{equation}
where, the arrows indicate anti-parallel orientation of $\bm{E_0}$ and $\bm{B_0}$.
The measured change in the precession frequency (using Eq.~(\ref{eq:up_freq}) and Eq.~(\ref{eq:down_freq})) can be used to deduce the nEDM via
\begin{equation}\label{eq:nEDM}
    d_n=\frac{h (v_{\Uparrow \Uparrow}-v_{\Uparrow \Downarrow})}{4 E_0}~\text{.}
\end{equation}

\textcolor{red}{Since the NMR frequency is proportional to the magnetic field in the nEDM cell, the requirement is to have a very stable and homogeneous $\bm{B_0}$ field within the cell.}


\section{Experimental Efforts}\label{sec:lim}

Additional sources of CP violation beyond the standard model predict the nEDM to be in the range of $10^{-27}$ -- $10^{-28}$~e$\cdot$cm as compared to $10^{-33}$ -- $10^{-31}$~e$\cdot$cm predicted by the standard model~\cite{theory_lim_1, theory_lim_2, theory_lim_3}. There are several experiments aiming at improving the uncertainty on the nEDM. 

\fig{Images/lim}{width = 0.9\textwidth}{Experimental nEDM upper limit over the years~\cite{1_lim,2_lim,3_lim,4_lim,5_lim,6_lim,7_lim,8_lim,9_lim,10_lim,11_lim,12_lim,13_lim,14_lim,15_lim,16_lim,17_lim} along with theoretical predictions~\cite{theory_lim_1, theory_lim_2, theory_lim_3}. The vertical dashed line indicates the introduction of UCN. The light green region indicates the nEDM limit in SUSY, M-theory and others while he light red region indicates for SM. \label{fig:lim}}{Experimental nEDM upperlimit over the years}

Figure~\ref{fig:lim} summarizes the previous experimental results. Since the first data published in 1957~\cite{1_lim}, the upper limit set on the nEDM has been reduced by eight orders of magnitude over the last six decades. In 1980, ultracold neutron (UCN) experiments overtook neutron beam experiments in precision. The current best upper limit set by Sussex-RAL-ILL nEDM experiment is $3.0 \times 10^{-26}$~e$\cdot$cm (90 \% C.L)~\cite{bestLim_1,bestLim_2}. The experiment was performed at Institut Laue-Langevin (ILL, Grenoble, France). A new $^{199}\mathrm{Hg}$ EDM measurement constrains the nEDM better than direct nEDM measurements, $d_n<\mathrm{1.6\times10^{-26}}$~e$\cdot$cm~\cite{schiff_screen}, although subject to uncertainty from Schiff screening. The TUCAN nEDM experiment is aiming to constrain the uncertainty on the nEDM at the $10^{-27}$~e$\cdot$cm level. 



The Paul Scherrer Institut (PSI, Villigen, Switzerland) nEDM experiment used an improved version of the former Sussex-RAL-ILL apparatus. Several innovations were made at PSI, including a new solid deuterium ($\mathrm{SD_2}$) spallation-driven UCN source. The experiment employed several Cs magnetometers outside the EDM cell, and a $^{199}\mathrm{Hg}$ comagnetometer. Active magnetic shielding and other environmental controls were improved. A new detector that can simultaneously count both spin states of UCN was also mplemented. The final sensitivity expected is $\mathrm{10^{-26}}$~e$\cdot$cm~\cite{psi} and the final data are being analyzed. Some of the chief improvements made at PSI have been in the area of nearby alkali atom (Cs) magnetometry, Hg comagnetometry, and neutron magnetometry. An achievement at PSI is the understanding of the Cs magnetometer signals in terms of magnetic field gradients internal to the magnetic shielding. This has led to a detailed understanding of the false EDM of the Hg comagnetometer~\cite{psi_falseEDM}. Another recent achievement is in using the neutrons themselves to measure gradients~\cite{psi_n_gradient}. PSI also aims to improve their magnetometry with $^3\mathrm{He}$ magnetometers inside the electrodes of the double EDM measurement cells for their future n2EDM effort. They have performed R$\&$D using Cs magnetometers to sense the free-induction decay signal from $^3\mathrm{He}$, which resulted in a new high-precision magnetometer possessing excellent long-term stability~\cite{psi_magnetometer}. The precision goal for n2EDM is $5 \times 10^{-28}$~e$\cdot$cm~\cite{psi_n2edm_nEDM-workshop,psi_n2edm_PPNS-workshop}.

The nEDM collaboration at Spallation Neutron Source (SNS, Oak Ridge, TN, USA) plans to measure $\delta d_n<$ $3\times10^{-28}$~e$\cdot$cm~\cite{sns_nEDM-workshop}, a two orders of magnitude improvement~\cite{sns_lim}. They plan to use a unique experimental technique. A cold neutron (CN) beam from the SNS will impinge upon a volume of superfluid $^4\mathrm{He}$ creating UCN. The nEDM measurement will also be conducted in the superfluid. A small amount of polarized $^3\mathrm{He}$ introduced into the superfluid $^4\mathrm{He}$ will act as both a comagnetometer and spin analyzer for the UCN. The $^3\mathrm{He}$ neutron capture rate is strongly spin dependent, and will beat at the difference of the Larmor precession frequencies of the neutrons and $^3\mathrm{He}$. A non-zero EDM would change the beat frequency with $E$-reversal. Scintillation light produced in the superfluid will be used to detect the capture products. The false EDM of the $^3\mathrm{He}$ comagnetometer may be reduced by collisions in the surrounding $^4\mathrm{He}$~\cite{sns_false_edm}. The group aims to commission the experiment at SNS by 2022~\cite{sns_nEDM-workshop}.

% A new room-temperature nEDM experiment will be conducted using an upgraded
% Los Alamos National Laboratory (LANL, Los Alamos, NM, USA) UCN source~\cite{lanl_nEDM-workshop}. The aim of the project is to increase the UCN density by a factor of five to ten, which could then be used to carry out a $\sim$ $10^{-27}$~e$\cdot$cm determination of the nEDM. The experiment aims for completion of a $10^{-27}$ level result, to be completed in the years prior to the SNS nEDM experiment, which shares a number of collaborators. 

At Los Alamos National Laboratory (LANL, Los Alamos, NM, USA), an upgraded UCN source will be used to conduct a new room-temperature nEDM experiment~\cite{lanl_nEDM-workshop}. An upgrade of the source, and an experiment storing UCN in an nEDM-like bottle has been completed recently~\cite{sns_nEDM-workshop2,sns2}. The upgrade enables an nEDM experiment with a statistical sensitivity of $2\times10^{-27}$~e$\cdot$cm. A test nEDM apparatus, similar in scope to our prototype nEDM apparatus used at RCNP Osaka is assembled by the LANL collaboration to conduct experiments.



% Two other room temperature nEDM experiments are being pursued at the Forchungsreaktor Muncheon II (FRM2) reactor in Munich~\cite{frm2} and Gatchina reactor at ILL~\cite{PNPI}. Both experiments feature double measurement cells and Cs magnetometers internal to the innermost magnetic shield. The Munich effort features an impressive new effort in active and passive magnetic shielding~\cite{msr_design,shield_pnpi,shield_pnpi2}, and uses $^{199}\mathrm{Hg}$ comagnetometer. The ILL/Gatchina experiment has produced results at ILL~\cite{PNPI}. This could be improved in further runs at ILL in the EDM position, or in runs using the superfluid He UCN source at ILL, where a statistical sensitivity of $3.5 \times 10^{-27}$~e$\cdot$cm could be obtained~\cite{pnpi_nEDM-workshop}. The group will build a UCN source at the WWR-M reactor in Gatchina in order to increase the UCN flux.


AT ILL, groups from Russia~\cite{pnpi_nEDM-workshop} and Munich~\cite{ill2_nEDM-workshop} are pursuing two other room temperature nEDM experiments. In future, the Russian experiment will be moved at the Petersburg Nuclear Physics Institute (PNPI) in Gatchina where their own UCN source are being developed. Internal to the innermost magnetic shield of both experiments have double measurement cells and Cs magnetometers. The Munich experiment $^{199}\mathrm{Hg}$ comagnetometer and their active and passive magnetic shielding system~\cite{msr_design,shield_pnpi,shield_pnpi2} are quite impressive. The groups are at the stage of constructing major equipment and upgrades.

\section{TUCAN nEDM Experiment}


\fig{Images/tucan}{width =\textwidth}{Conceptual design of the proposed TUCAN source and nEDM Experiment. The major portion of the biological shielding is not shown. Protons strike a tungsten spallation target. Neutrons are moderated in the $\mathrm{LD_2}$ cryostat and become UCN in a super fluid $^4\mathrm{He}$ bottle within, which is cooled by the superfluid $^4\mathrm{He}$ cryostat. UCN pass through guides and the superconducting magnet (SCM) to reach the nEDM experiment located within a magnetically shielded room (MSR). Simultaneous spin analyzers (SSA's) detect the UCN at the end of each nEDM experimental cycle. \label{fig:tucan}}{Conceptual design of the proposed TUCAN source and nEDM Experiment.}

The proposed TUCAN facility and nEDM experiment is shown in Fig.~\ref{fig:tucan}. A proton beam at 480 MeV and 40 $\mu$A from the cyclotron impinges upon the tungsten spallation target liberating fast neutrons. Above the target is a neutron moderator system containing liquid deuterium ($\mathrm{LD_2}$) which creates a large flux of cold neutrons (CN). The CN enter a bottle surrounded by the $\mathrm{LD_2}$ which contains superfluid $^4\mathrm{He}$ at below 1 K. In the superfluid, the CN excite phonon and roton transitions, losing virtually all their kinetic energy to become ultracold. 

Once a sufficient density of UCN has built up, a UCN valve opens. The UCN are transported out of the source by reflection on the surfaces of UCN guides. A superconducting magnet (SCM) transmits one neutron spin orientation in the magnetic field, giving near-unity UCN polarization and facilitating transmission through a vacuum-isolation foil at room temperature. The UCN are then transported to the nEDM experiment by additional guides. At the end of each nEDM experiment cycle, simultaneous spin analyzers (SSA's) detect the UCN.


% In fChapter~\ref{ch:magnetics}, the magnetic field system around the EDM measurement cell is discussed in detail.
As mentioned above, a major challenge for all nEDM experiments is the generation of a sufficiently homogeneous and stable magnetic field in which to perform the Ramsey measurement. The particular magnetic field requirements for  the TUCAN nEDM experiment are presented in Chapter~\ref{ch:magnetics}, along with a description of the principles of active magnetic field compensation (AMC), which is the subject of this work. Chapters~\ref{ch:amcP} and~\ref{ch:operation} describe the AMC prototype and control methods developed in the thesis, while Chapter~\ref{ch:quantification} provides a detailed quantification of the systems performance. In Chapter~\ref{ch:conclusion}, I summarize the four key findings of the thesis and provide a number of recommendations for a final AMC system suitable for the TUCAN nEDM experiment.



% that are slowed in in a room-temperature neutron moderator composed of lead, graphite and ${D_2}$O and converted to cold neutron (CN) by L${D_2}$ in a bottle having superfluid He-II below 1K. Finally, they are reduced to ultra cold speeds by phonon scattering in superfluid helium. After generating sufficient density of UCN, the proton beam is turned off opening a cryogenic UCN valve. The UCN then transported by reflection on the surfaces UCN guides and to accelerate polarized UCN through barrier foils to a vacuum volume at room temperature, a superconducting magnet (SCM) has been used. At last , UCN's are transported to the nEDM experiment via additional guides. At the end of each nEDM experiment cycle, simultaneous spin analyzers (SSA's) detect the UCN. 





