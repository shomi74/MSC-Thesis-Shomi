\chapter{Motivation Behind the nEDM Experiment}
\lhead{\emph{Motivation Behind the nEDM Experiment}}\label{ch:motivation} 
This chapter highlights the scientific interest in a new precise measurement of the neutron dipole moment (nEDM). Then, the measurement principle of the nEDM experiment is discussed with the importance of the magnetic environment for the successfulness of the experiment. Finally, this chapter ends with describing the TRIUMF Ultra Cold Advenced Neutron (TUCAN) nEDM experiment.

\section{Baryon Asymmetry}
The universe is created from some fundamental particles which are governed by the four fundamental forces (electromagnetic, weak interaction, strong interaction and gravitational force). The Standard Model (SM) has been created as a theory to classify the particles and relation among three forces excluding gravity. Six of each quarks and leptons are the fundamental particles which classify the matters based on the quantum numbers namely baryon carried by quarks and lepton number carried by leptons. In the early universe, there were equal particle numbers of matter and antimatter. But the universe today contains almost only matter. By experimental observations of the cosmic microwave background radiation (CMBR) the baryon asymmetry can be deduced~\cite{expBar}.

\begin{equation}\label{eq:baryons}
    \eta =\frac{n_b-\bar{n}_b}{n_\gamma}\simeq6.0 \times 10^{-10}
\end{equation}
where $n_b$, $\bar{n}_b$ and ${n_\gamma}$ are the number of baryons, anti-baryons and photons respectively. 

% So, this $6.0 \times 10^{-10}$ excess baryons over anti baryons per photon is called the baryon asymmetry of the universe. Scientists are continuously searching for the answer of this mystery. The breakthrough is possible by finding the non zero value of nEDM. 

The relation of the non zero nEDM with the baryon asymmetry has been discussed next.

\section{Sakharov Criteria and nEDM}
Baryogenesis is the process of creation of a baryon asymmetry from an initially symmetric state. In 1967, A.D. Sakharov described following three conditions on particle theories aiming to explain barayogenesis~\cite{Sakharov:1967dj}
\begin{itemize}
    \item Baryon (B) number violation.
    \item Charge (C) and Charge-Parity (CP) symmetry violation.
    \item Departure from thermal equilibrium.
\end{itemize}

Any theory which creates net baryon number obviously requires B-violating processes. Particle and antiparticle reaction must be different otherwise net baryon creation would be balanced by net anti-baryon creation. In thermal equation, forward reaction rates would balance backwards rates and no net baryon number could be produced. A departure from thermal equation is therefore required. The Big Bang and subsequent cooling of the universe offers a way to provide the departure from thermal equation~\cite{sakharov_3rd_cond}. Electroweak baryogenesis is one scenario which uses SM processes and the electroweak phase transition to explain baryogenesis. One drawback of the electroweak baryogensis is that there is no enough CP violation in the SM. This motivates new sources of CP violation near the weak scale.

% The first criterion can be explained by the second and third conditions. During the expansion of the universe immediately after the big bang, most of the particles were departed from the thermal equilibrium ( defined by the temperature of the universe between $10^{2}$ GeV and $10^{12}$ GeV ) at $\sim 10^{19}$ GeV temperature as suggested by Sakharov \cite{sakharov_3rd_cond} resulting in net non zero B number. The non zero B number also exists due to C and CP asymmetry. Because in the absence of those violations, excess baryon reactions can be nullified by the excess anti-baryon reactions \cite{sakharov_1st_cond}. Though the standard model of particle physics possesses CP violation, there is an insufficient amount of CP violation to explain the observed baryon asymmetry.  Theories of new physics beyond standard model, motivated in part by the baryon asymmetry, predict new sources of CP violation.

A fundamental symmetry of quantum field theories is CPT symmetry, which implies that time-reversal (T) violation is equivalent to CP violation.  An electric dipole moment (EDM) is a measure of separation of oppositely charged particles within a system. To have a nonzero EDM, the system should not violate both parity (P) and time-reversal (T) inversion~\cite{edm_reason}. So, a search for a non-zero EDM represents a search for new physics that violates CP symmetry. The neutron may have an EDM with its magnitude depending on the nature and origin of the T violation~\cite{nEDM_reason}. A precise measurement of the nEDM could be a very important metric for explaining the baryon asymmetry on particle physics beyond the known interaction in the SM. 

% \section{Experimental Efforts So Far}\label{sec:lim}
% \fig{Images/lim}{width = \textwidth}{Experimental nEDM upperlimit over the years \cite{1_lim,2_lim,3_lim,4_lim,5_lim,6_lim,7_lim,8_lim,9_lim,10_lim,11_lim,12_lim,13_lim,14_lim,15_lim,16_lim,17_lim} along with theoretical predictions \cite{theory_lim_1, theory_lim_2, theory_lim_3}.\label{fig:lim}}

% Additional sources of CP violation beyond the standard model predict the nEDM to be in the range of $10^{-27}$ -- $10^{-28}$  e$\cdot$cm as compared to $10^{-33}$ -- $10^{-31}$ e$\cdot$cm predicted by the standard model \cite{theory_lim_1, theory_lim_2, theory_lim_3}. There are several experiments aiming at improving the uncertainty on the nEDM. The Fig. \ref{fig:lim} summarize all the results. Since the first measurement data published in 1957 \cite{1_lim}, the upper limit set on nEDM has been reduced by eight orders of magnitude over the last six decades. At 1980, ultra cold neutron (UCN) experiment overtook over neutron beam experiments on precision. The current best upper limit set by ILL/Sussex/RAL nEDM experiment is $3.0 \times 10^{-26}$ e-cm (90 \% C.L) or $3.6 \times 10^{-26}$ e-cm (96 \% C.L)  \cite{bestLim_1,bestLim_2}. The experiment has been done at the Institut Laue-Langevin (ILL) which is situated at Grenoble, France. A new generation of UCN source known as superthermal UCN source which uses a new method of cooling by transferring energy to quantum excitations in a material have recently come online. To use such a UCN source, the nEDM appa

% The apparatus for nEDM experiment has been moved from ILL to a superthermal UCN source at Paul Scherrer Institut (PSI) which is situated at Villigen, Switzerland. UCN superthermal source use the cooling technique . A new UCN source generation have recently come online. They use a technique from condensed
% matter physics involving cooling by transferring energy to quantum excitations in a material.
% UCN sources that employ this method of cooling are known as superthermal sources and they are
% beginning to transform the landscape of fundamental neutron physics at various facilities in the
% world.
% The nEDM apparatus from ILL was moved to such a UCN source at Paul Scherrer Institut
% (PSI, Villigen, Switzerland). The apparatus was also improved and upgraded in several respects.
% Data-taking for this new nEDM experiment was completed recently and the analysis of the data is
% ongoing [9]. The expectation in the community is that this new result will improve the previous
% best by a factor of about three to four.
% Next generation UCN EDM experiments are now in preparation at a variety of sites aiming to
% improve the result by an order of magnitude or more. Experiments are either ongoing or planned
% at ILL [10, 11], PSI [12], the Gatchina reactor [10], the Forchungsreaktor Munchen II (FRM2)
% reactor [13], Los Alamos National Laboratory (LANL, Los Alamos, NM, USA) [14], the Spallation
% Neutron Source (SNS, Oak Ridge, TN, USA) [15], and our eort at TRIUMF [16, 17]. We discuss
% our relationship to these experiments in Section 4. Our goal of dn < 10��27 ecm within the next
% 6-7 years (running until 2024-25, as stated earlier) is competitive with these eorts. One of the key
% factors is our unique UCN source, which we are upgrading. We envision achieving UCN counting
% rates over 100 times larger than the previous best nEDM experiment and the recently completed
% experiment at PSI, and similar to or surpassing the plans of other experiments.


% The nEDM experiment at TRIUMF is aiming to constrain the uncertainty on the nEDM at the $10^{-27}$ e-cm sensitivity level. 
Next the ultracold neutrons are discussed.

\section{Ultracold Neutrons}
Ultracold neutrons (UCN) are used to measure the nEDM. They have small kinetic energies ($<$ 300 neV). They can be confined in a material bottle as they are reflected at any angle of incidence off suitable material walls \cite{ucn_storage}. This provides long time frame (the mean lifetime of neutron is $\tau_n=881.5$ s \cite{mike}) for observation marking them ideal media for nEDM experiment. Their interaction with the neutron optical potential of the walls through the strong force enables to trap them. First UCN were produced in 2017 for TUCAN nEDM experiment using superfluid-helium at 0.9 K as the UCN production medium~\cite{TRIUMF_UCN,taraneh_theis}.

\section{Measurement Principle of nEDM}\label{sec:nEDM}

For the extraction of nEDM, Ramsey's method of separated oscillatory fields~\cite{ramsey} will be used. UCN will be polarized by passage through a strong magnetic field and guided to the nEDM cell. The Ramsey's method of separated oscillatory fields for nEDM measurement has been shown in detail in Fig.~\ref{fig:ramsey}. 

\fig{Images/ramsey}{width = \textwidth}{Ramsey's method of separated oscillatory fields, as applied to the measurement of the nEDM. $\bm{E_o}$ field parallel to $\bm{B_o}$ field in the left where in the right they are antiparallel. \label{fig:ramsey}}

\FloatBarrier

It is seen that at the beginning and end of free precession short $\pi$/2 pulses are applied and polarized neutron detection after the pulse sequence is used to measure the free spin precession frequency $v$ (the Larmor frequency). In the first instance (left one in Fig.~\ref{fig:ramsey}), an electric field ($\bm{E_o}$) parallel to the magnetic field ($\bm{B_o}$) will be applied giving a spin-precession frequency as
\begin{equation}\label{eq:up_freq}
    h v_{\Uparrow \Uparrow}=2\mu_n\bm{B_o}+2 d_n\bm{E_o}
\end{equation}
where, $\mu_n$ and $d_n$ are the magnetic moment and electric dipole moment respectively, $h$ is the Planck's constant and arrows indicate parallel orientation of $\bm{E_o}$ and $\bm{B_o}$.
Now the same experiment is repeated with anti-parallel $\bm{E_o}$ (right one in Fig.~\ref{fig:ramsey})which gives the spin-precession frequency
\begin{equation}\label{eq:down_freq}
    h v_{\Uparrow \Downarrow}=2\mu_n\bm{B_o}-2 d_n\bm{E_o}
\end{equation}
where, the arrows indicate anti-parallel orientation of $\bm{E_o}$ and $\bm{B_o}$.
The measured change in the precession frequency (using Eq.~(\ref{eq:up_freq}) and Eq.~(\ref{eq:down_freq})) can be used to deduce the nEDM via
\begin{equation}\label{eq:nEDM}
    d_n=\frac{h (v_{\Uparrow \Uparrow}-v_{\Uparrow \Downarrow})}{4\bm{E_o}}
\end{equation}

Since the NMR frequency is proportional to the magnetic field in the nEDM cell, the requirement is to have a very stable and homogeneous $\bm{B_o}$ field within the cell.

\section{TUCAN nEDM Experiment}

\fig{Images/tucan}{width = \textwidth}{Conceptual design of the proposed TUCAN source and nEDM Experiment. The major portion of the biological shielding is not shown. Protons strike a tungsten spallation target. Neutrons are moderated in the $\mathrm{LD_2}$ cryostat and become UCN in a super fluid $^4\mathrm{He}$ bottle within, which is cooled by the superfluid $^4\mathrm{He}$ cryostat. UCN pass through guides and the superconducting magnet (SCM) to reach the nEDM experiment located within a magnetically shielded room (MSR). Simultaneous spin analyzers (SSA's) detect the UCN at the end of each nEDM experimental cycle. \label{fig:tucan}}

The Fig. \ref{fig:tucan} gives an overview of the proposed TUCAN facility with designated place for nEDM experiment. A proton beam at 480 MeV and 40 $\mu$A from the cyclotron impinges upon the tungsten spallation target liberating fast neutrons. Above the target is a neutron moderator system containing liquid deuterium ($\mathrm{LD_2}$) which creates a large flux of cold neutrons (CN). The CN enter a bottle surrounded by the $\mathrm{LD_2}$ which contains superfluid $^4\mathrm{He}$ at below 1 K. In the superfluid, the CN excite phonon and roton transitions, losing virtually all their kinetic energy to become ultracold. Once a sufficient density of UCN has built up, a UCN valve opens. The UCN are transported out of the source by reflection on the surfaces of UCN guides. A superconducting magnet (SCM) transmits one neutron spin orientation in the magnetic field, giving near-unity UCN polarization and facilitating transmission through a vacuum-isolation foil at room temperature. The UCN are then transported to the nEDM experiment by additional guides. At the end of each nEDM experiment cycle, simultaneous spin analyzers (SSA's) detect the UCN.




In the upcoming chapter, the magnetic field system around the EDM measurement cell has been discussed in detail.



% that are slowed in in a room-temperature neutron moderator composed of lead, graphite and ${D_2}$O and converted to cold neutron (CN) by L${D_2}$ in a bottle having superfluid He-II below 1K. Finally, they are reduced to ultra cold speeds by phonon scattering in superfluid helium. After generating sufficient density of UCN, the proton beam is turned off opening a cryogenic UCN valve. The UCN then transported by reflection on the surfaces UCN guides and to accelerate polarized UCN through barrier foils to a vacuum volume at room temperature, a superconducting magnet (SCM) has been used. At last , UCN's are transported to the nEDM experiment via additional guides. At the end of each nEDM experiment cycle, simultaneous spin analyzers (SSA's) detect the UCN. 





