
\lhead{\emph{Quantification of AMC Prototype}}
%\part{Privacy Preserving Approximation of Edit Distance on Genomic Data}

\chapter{Quantification of AMC Prototype}\label{ch:quantification}
% \label{chap:image}

In Chapter~\ref{ch:operation}, the magnetic control process and the simulation methods to understand the process have been described. The quantitative measures of the magnetic field compensation performance suggested by the Refs.\cite{bea,lins,rawlik} are also discussed there. While implementing the magnetic control process discussed in Chapter~\ref{ch:operation}, we have faced problems in terms of long drift in the coil currents and also slower response of the prototype system. We have been able to make the system faster by introducing filter which enables us to change the sample frequency without adding noise and will be discussed in Section~\ref{sec:freq}. In Section~\ref{sec:r_pi_revisit}, we have discussed the studies to understand regularization parameter and P and I term of PI control algorithm. But the coil current drifting problem is still there. We have then changed the fluxgate positions randomly to try to make condition number smaller and we have also removed the outermost passive shielding layer to see if there is any effect. For confirming the experimental coil current pattern, we have also made a simulation of the prototype. All the discussions about the different position of fluxgates and shield removal are in Section~\ref{sec:flux_place}. We have also implemented the new control algorithm suggested by Ref.~\cite{rawlik} and compared with our existing one suggested by Ref.~\cite{bea}. While comparing those we came up with an interesting similarity in the control algorithms which is discussed in Section~\ref{sec:style_pi}. But the new algorithm sill had no answers for the current drifting problems which encouraged us to study the regularization parameter in different way that is discussed in Section~\ref{sec:new_study_r}. The current drifting problem can be solved but that introduced additional noise. We have then realized that coil configuration has a great impact on the current modes which we have discussed in Section~\ref{sec:coil_config}. Finally, the results of the metrics that has been discussed in Section~\ref{sec:metrics} are presented in Section~\ref{sec:metrics_res}. There are some unique points discussed here which are not explicitly present on previous studies on active compensation. They are
\begin{itemize}
    \item $\mathbf{4^{th}}$ order low pass Butterworth filter
    \item PI tuning general behaviour on active compensation system 
    \item $r$ behaviour on PI tuning.
    \item Effect of $r$ on coil current response
    \item Impact of shield on active compensation
    \item Fluxgate placement suggestions.
    \item Relation of $r$ with matrix condition number
    \item Prototype PI control simulation
    \item Relation of different PI control algorithm
    \item Coil current modes based on coil configuration
\end{itemize}

The new directions for future studies set by the unique points are presented in Chapter~\ref{ch:conclusion}.

% we have the varied some parameters to quantify the prototype system. They are 
% \begin{itemize}
%     \item Sampling frequency
%     \item $r$, P and I term
%     \item 
% \end{itemize}

% There are vast number of parameters that can be varied and studied for active compensation. For quantifying the the prototype, five among them are chosen based on importance, time limitation, resources available etc. and studied via both experiment and simulation. Those will create a huge impact for the future studies on active compensation. Some unique points have been discussed which aren't explicitly present on previous studies on active compensation. Their successfulness have been discussed using some metrics (see Section~\ref{sec:metrics}). In this chapter, studies on the parameter as well as the metrics to justify them will be discussed in terms of results.

\section{Sampling Frequency and Filtering}\label{sec:freq}


\begin{itemize}
\item We built some analog filters which were discussed in Chapter 3
\item Goal of the filter was to remove high-frequency noise (low-pass Butterworth with 10~Hz corner frequency)
\item This would allow the ADC to operate with less averaging, reducing its effective sampling time (denoted by the ``resolution index'')  (refer to Ch. 3)
\item We show the effectiveness in this section.
\end{itemize}

In Chapter~\ref{ch:amcP}, some low pass analog Butterworth filters with 10~Hz corner frequency were discussed that we built. The goal of the filter was to remove high-frequency noise. This would allow the ADC to operate with less averaging, increasing its effective sampling frequency (denoted by the 'resolution index' as shown in Table~\ref{table:t7freq}). The filter gives us more degree of freedom in terms of

\begin{itemize}
    \item using different sampling frequencies of the ADC
    \item reducing magnetic field compensation response time 
    \item reducing coil current response time
    \item most importantly maintaining our design sample rate with better reduction of noise.
\end{itemize}

The effectiveness of the filters are discussed in this section. 
% The section starts with defining the sampling frequency and how we first consider building an analog filter in the first place. Then it shows the effect improving the response time on the prototype. Finally, the Section wraps up with some comparison in
% current response time for different sampling frequency.



% \subsubsection{Filtering effect to use the fastest sampling frequency}

\begin{itemize}
\item Fix Fig. 5.1 (scales) -- ideally plot all of them on the same graph.
\item Noise reduced by factor of {\bf X}
\item This is resolution index 1
\item Without filter noise is dominated by 60 Hz and higher.
\item Compensation is off
\item Also compared with SCU... it agreed well.  Our filters gave slightly better performance (lower noise) likely due to slight difference in design.
\item Loop frequency is 100~Hz.  This is larger than 25~kHz$/14\approx 2$~kHz reported in Chapter 3 (ref) because of polling time.  Typ.~1~ms/channel read.  When read in this way, can increase resolution index to 7 (and averaging time) before significant delays are noted.  (when ADC effective sample time approaches 1~ms)
\item At times we did this alternate way:  increase resolution index.
\item Another problem:  current drifting during compensation although field would not change.  This is discussed further in Sections~\ref{drifting1} and \ref{drifting2}.
\item We thought maybe this was due to sequential reads (higher resol index $\sim$ 7) so in general we aimed for resol index 1 which makes the reads distributed in time.  (Explain better... it is leading up to Fig. 5.2 and 5.3.)  In such cases when using lower resol index (1) we did software averaging until the design loop rate was met.  This will be discussed further in the next sections.
\item Another point:  in Fig. 5.1 is that there is a time lag (slewing?) for the filter... consistent with expectation.
\item Conclusion:  the filter works and allows us to go to higher effective sample rate.
\item We would be stuck at resolution index 11 or 12 otherwise.  12 has 6 Hz effective sample rate $\approx$ 10 PLC averaging.  If doing this for 12 channels it it means $<0.5$~Hz.  Impossible to do compensation at 6~Hz this way (fails to meet design goal).
\end{itemize}

Power line induced noise is of main concern while using ADC. The analog input signal while averaging over one or more power line cycles (PLCs) can filter that noise. The noise at the line frequency which is 60 Hz or 50 Hz depending on the geographical region can be removed over one cycle. So, one PLC averaging corresponds to 60 Hz. That is the signal is sampled at 60 Hz (no. of samples per second is 60). However, as the noise is not uniform, the signals should be averaged over many PLCs. Increasing the number of PLCs increases the accuracy of the signal which ensures greater noise reduction and better resolution. The effective sampling frequency as shown in Table~\ref{table:t7freq} goes as high as 3846.2 Hz to as low as 6.3 Hz for gain 1 indicated by the resolution index. In the early days of the prototype experiment, we were stucked with resolution index 11 and resolution index 12 which corresponds to $\approx$ 4 PLC averaging and $\approx$ 10 PLC averaging respectively to get rid of high frequency noise. As we have 14 fluxgate sensors including the control ones, the effective sampling frequency reduces to $\mathbf{<1.1}$~Hz or $\mathbf{<0.5}$~Hz depending on resolution index 11 or 12 respectively which fails to meet 6 Hz design goal. 


\fig{Images/filtering}{width = \textwidth}{Filtering effect on the magnetic field signals measured by 14 fluxgate sensors including the control sensors placed at 1, 3 , 6, 8 and center. For position of the sensor see Fig.~\ref{fig: coil}. For simplicity only position '1x' is shown. Vertical axis represent $\Delta$B (see Eq.~(\ref{eq:del_B})) found from the measurement of sensor '1x' (a) without filter, (b) with Bartington$'$s SCU1 (Signal Conditioning Unit) and (c) with our filter. The 'ON' and 'OFF' vertical dashed lines indicate the time of the perturbation coil being turned 'ON' and 'OFF' respectively.  The current supplied on perturbation coil was 100 mA. The resolution index 1 (see Table~\ref{table:t7freq}) was used. The loop sampling frequency  for (a), (b) and (c) are shown in Hz. \label{fig:filtering}}{Filtering effect on the magnetic field signals.}

The Fig.~\ref{fig:filtering} shows the importance of using the filter. The data was taken by measuring the drift in the signal $\Delta$B (see Eq.~(\ref{eq:del_B})) by 14 fluxgate sensors including the control sensors placed at 1, 3 , 6, 8 and center separately without filter in Fig.~\ref{fig:filtering}\textcolor{blue}{(a)}, with Bartington$'$s SCU1 10 Hz low pass filter in Fig.~\ref{fig:filtering}\textcolor{blue}{(b)} and with our filter in Fig.~\ref{fig:filtering}\textcolor{blue}{(c)}. For simplicity only position '1x' is shown. It is seen that the signal is very noisy without any filter. The noise is reduced 10 times while using Bartington$'$s SCU1 10 Hz low pass filter. Our filters gave slightly better performance (lower noise) likely due to slight difference in design. For data acquisition, resolution index 1 of ADC was used. 100 mA current was supplied in the perturbation coil  at $\sim$ 0.32 s and cut off at $\sim$0.95 s. The total loop sampling frequency is found to be $\sim$220 Hz which is larger than the loop sampling frequency 344.7 Hz reported in Table~\ref{table:t7freq} for 14 channels because of new command time added for supplying current in perturbation coil. The typical polling time as discussed in Chapter~\ref{ch:amcP} is 1~ms/channel read. It is also seen that the outputs are limited by the slew rate of the op-amps used in the SCU1 and our filter. Slew rate is the rate of change in the output due to a step change on the input. That is output is change by certain amount at a given time and the limit is called the slew rate.


% When read in this way, the resolution index can be increased to 7 before significant delays are noted which is the same time when ADC loop sample time approaches 1~ms. We thought maybe this was due to sequential reads (higher resol index $\sim$ 7) so in general we aimed for resol index 1 which makes the reads distributed in time.  (Explain better... it is leading up to Fig. 5.2 and 5.3.)  In such cases when using lower resol index (1) we did software averaging until the design loop rate was met.  This will be discussed further in the next sections.

% In the early days of the prototype experiment, we have used the slowest sampling frequency which is corresponding to resolution index 12 in Table~\ref{table:t7freq} to get rid of high frequency noise e.g. 60 Hz and higher.
% While converting anlog dc fluxgate signal When determining the value of a dc signal, there oftentimes exists some power line-induced ac noise. Integrating the dc signal over one or more power line cycles helps to reject this noise. If the noise is at the line frequency (60Hz in the U.S., 50Hz in most other countries), it can be removed over one cycle. Sometimes however, the noise is not uniform, and the signal should be integrated over multiple cycles. The greater the number of power line cycles, the more accurate the signal value will be, i.e. greater noise rejection and better resolution. If your application simply calls for a true/false or high/low voltage reading, minor noise will not be a major factor in achieving the desired value, and you can use a very small NPLC to maximize throughput.






% Sampling frequency is the number of samples per second in a signal. For the prototype, by sampling frequency we meant that the number of signals of the fluxgate sensors that are going to the feedback algorithm via analog to digital converter (ADC) per second per measurement. But we are recording the time for a complete iteration which we call it as loop sampling frequency which includes the ADC sampling frequency plus rest of the loop frequency.   The problem with that was the response time for the current was very slow and so does the compensation of the magnetic field. Then we have decided to use the fastest sampling frequency that the ADC can offer which is corresponding to resolution index 1 in Table~\ref{table:t7freq}. But new problem has arisen in terms of noise especially the 60 Hz electrical noise from the power source which has motivated us to build the 4\textsuperscript{th} order low pass Butterworth filter (see Section~\ref{sec:filter}) to get rid of unwanted high frequency noise. This Section talks about the results that we got to solve those problems by increasing the sampling frequency.



% As we have decided to use the fastest sampling frequency that our ADC
% can offer as discussed above , we have faced problems in terms of
% noise there. So, we have build a 4\textsuperscript{th} order low pass
% Butterworth filter (see Section~\ref{sec:filter}) to get rid of
% unwanted high frequency noise.



\FloatBarrier
% The Fig.~\ref{fig:filtering} shows the importance of using the filter discussed above. The data was taken by measuring the drift in the signal $\Delta$B (see Eq.~(\ref{eq:del_B})) by '1x' sensor position two times where one with filter and another one without filter. On the part of data measurement, current has been applied in the perturbation coil at $\sim$ 0.32 s which is indicated by the vertical red dashed line and is termed as ON. At $\sim$0.95 s current supply in the perturbation coil has cut off which is indicated by the green vertical dash line and termed as OFF. The total loop sampling frequency is found to be $\sim$220 Hz. It is seen that the signal is very less attenuated while using filter and very noisy without any filter.

So our filter is capable of reducing high frequency noise and allows us to go to higher effective sample rate.
% So, for faster sampling it needs to have filter to avoid high frequency noise. Next, after applying filter let's see how the sampling frequency has impact on response time of the magnetic field compensation.


% \subsubsection{Effect on Response Time of Magnetic Field Compensation}

\begin{itemize}
\item Fig. 5.2.  Left: with filter, resol index 1, 50 software averages per point, compensation ON, loop cycle rate (correction rate) 6.57 Hz indicated in the Fig.  Right:  no filter, resolution index 12, no other averaging (other than what the ADC oes), graph is on longer timescale because loop rate is slower (0.45~Hz).  Corrected/uncorrected (projected) values are shown.
\item Main point:  loop cycle rate is good!  $\sim$ 15x faster.  Even plenty of time for additional averaging to reduce noise further.  Likely the same result if somewhat higher resolution index used, possibly even with reduced software averaging.
\end{itemize}

\fig{Images/samp_freq_iteration}{width = \textwidth,height=8 cm}{Active magnetic field compensation by sensor position '1y' using resolution index 1 (left) and resolution index 12 (right). For position of the sensors please see Fig.~\ref{fig: coil}. Vertical axis represent $\Delta$B (see Eq.~(\ref{eq:del_B})) due to '1y' with red represents uncompensated $\Delta$B. They are described in the text. The 'ON' and 'OFF' vertical dashed lines indicate the time of the perturbation coil being turned 'ON' and 'OFF' respectively. \label{fig:samp_freq_iteration}}{Active magnetic field compensation by sensor position '1y'.}

The Fig.~\ref{fig:samp_freq_iteration} shows the comparison between using highest sampling frequency (left) and the slowest one (right). The results are part of the compensation by all the 12 sensors placed just like explained by the horizontal axis in Fig.~\ref{fig:m} using regularized pseudoinverse and PI tuning in the feedback algorithm explained in Chapter~\ref{ch:operation}. The figure only shows compensation for position '1y for simplicity. Note that the left and right are two different compensation measurements with highest (resolution index 1) and lowest sampling frequency (resolution index 12) of our ADC (see Section~\ref{sec:DAQ}) having same number of feedback loop in each case. We have used filter in the left figure. We have done 50 software averages per iteration to meet the design goal frequency and it is seen that the loop frequency 6.57 Hz. We have not used any filter in the right figure and also there is no software averaging. It is seen that for same number of iteration, the graph on the right is on longer timescale because of the slower loop rate which is 0.45 Hz and it is 6.57/0.45$\approx$15 times slower. The same information can be seen by the horizontal timescale in both the graphs that is 670/45$\approx$15 times slower. 



% So after we have our filter ready to use, now let's see what I meant by increasing the sampling frequency and its effect on the magnetic field  compensation response time.




\FloatBarrier
So the filter helps us to utilize $\approx$15 times faster response time while also maintaining design goal rate.


% \subsubsection{Effect on Coil Current Response Time }

\begin{itemize}
\item Fig. 5.3.  Always filtered.  Different resolution indices indicated on the figure along with their loop cycle rates.  Always 50 software averages done.
\item By resolution index 8 or 9, the loop cycle rate for the 50 software averages slows to 1.1 and 0.39 Hz respectively.  The results look similar to resolution index 12 (without filter) but much less noisy, smoother.
\item Another option is to run with fewer averages than 50 and at higher resolution index, while still achieving the 6 Hz loop rate design goal.  This would likely also work just fine although we didn't try it.  The advantage of resolution index 1 with averaging is that the channels are all sampled at closer to the same time, which we suspected was a problem, as discussed earlier.
\end{itemize}


Form above discussion it is found that the more the sampling frequency is the less it takes for the compensation to respond. The effect of the sampling frequency has been revisited here again to show its effect in coil current response time.

\FloatBarrier
\fig{Images/samp_freq_time}{width = \textwidth,height=8 cm}{$C_x^-$ coil current (left) and active magnetic field compensation (right) by sensor position '8x' for different resolution index. The resolution indices are (see Table~\ref{fig: res} and Sec.~\ref{sec:DAQ}) represented by 1, 6, 7, 8 and 9 while loop sampling frequencies are given in the parentheses respectively. They are describe more in the text. Vertical axis of the left one represents $C_x^-$ coil current graph and the right one indicates   $\Delta$B (see Eq.~(\ref{eq:del_B})) due to '8x' with red represents uncompensated $\Delta$B for different resolution index respectively. The 'ON' and 'OFF' vertical dashed lines indicate the time of the perturbation coil being turned 'ON' and 'OFF' respectively. \label{fig:samp_freq_example_time}}{Short}

The Fig.~\ref{fig:samp_freq_example_time} shows the active compensation by fluxgate position 1, 3, 6 and 8 for resolution index 1, 3, 5, 7 and 9 for 50 software averages using filter. Different resolution indices indicated on the figure along with their loop cycle rates.  The results are again part of the compensation by all the 12 sensors placed just like explained earlier. Here, different measurements was taken for a same time frame i.e. 180s compared to earlier where time was independent for different ones. For simplicity only $C_x^-$ coil current (left) and magnetic compensation (right) on fluxgate sensor position '8x' for different loop sampling frequencies have been shown instead of showing all current coils and sensor positions. For a complete list of positions see the color map of $\bm{M}$ in Fig.~\ref{fig:m}.  As seen from the $C_x^-$ coil current (left) and $\Delta$B (right) due to '8x' graph that with decrease of loop sampling frequency, the current response time increases. For example, it takes $\sim$ 7s for $C_x^-$ coil current to settle down after the perturbation coil being turned ON when the loop sampling frequency is 6.35 Hz indicated by blue color curve compared to $\sim$ 52s when the loop sampling frequency is 0.39 Hz indicated by magenta color curve. So, the current response time has been faster by 52/7$\approx$7.5 times and we are not even considering the slowest one. It is also seen that the loop cycle rate slows to 1.1 Hz and 0.39 Hz by resolution index 8 and 9 respectively.  The results look similar to resolution index 12 (without filter) but much less noisy, smoother.

\FloatBarrier

So for better response time in the coil current and eventually in magnetic field compensation system, there should an ADC with higher sampling frequency and a filtering method to get rid of higher frequencies. 

\FloatBarrier
\fig{Images/samp_freq_same}{width = \textwidth,height=8 cm}{Active magnetic field compensation fluxgate position 1, 3, 6 and 8 for resolution index 1, 3, 5 and 7 for 50, 45, 29 and 13 software averages respectively using filter. For simplicity only $C_x^-$ coil current (left) and magnetic compensation (right) on fluxgate sensor position '8x' have been shown. Vertical axis represent $\Delta$B (see Eq.~(\ref{eq:del_B})) due to '8x' with red represents uncompensated $\Delta$B. They are described in the text. The 'ON' and 'OFF' vertical dashed lines indicate the time of the perturbation coil being turned 'ON' and 'OFF' respectively. \label{fig:samp_freq_same}}{Short}

 The Fig.~\ref{fig:samp_freq_same} shows the active compensation by fluxgate position 1, 3, 6 and 8 for resolution index 1, 3, 5 and 7 for 50, 45, 29 and 13 software averages respectively using filter. For simplicity only $C_x^-$ coil current (left) and magnetic compensation (right) on fluxgate sensor position '8x' have been shown instead of showing all current coils and sensor positions. For a complete list of positions see the color map of $\bm{M}$ in Fig.~\ref{fig:m}.  It is seen that the loop sample rate in every case is almost similar and they have similar effect on magnetic compensation (right) but the magnetic compensation due to resolution index 7 looks noisy compare to others. Moreover, the coil current (left) for different resolution index looks similar.
 
\FloatBarrier
\begin{itemize}
    \item Conclusion of the figure  ?
    \item why use resol 1?
\end{itemize}

% \FloatBarrier
% In the upcoming Section PI, regularization parameter and condition number will be revisited with results.

\section{PI Tuning General Behavior}\label{sec:pi_behave}
Main points:
\begin{itemize}
\item Trying to show that P doesn't do much (5.5) but I has a big effect.  I term gives smaller Delta B in one axis (5.6).  I term can also affect response time in that axis (5.7).  After I term is set, the P term doesn't seem to do too much. (5.8).
\item Current drifting problem is starting to become apparent.  Does not appear for P only.  Appears much more pronounced for I only, and PI combined.
\item Do we need all the figures?  Can the data be summarized better (all currents and all fields on two graphs for each setting?)
\item Should we show data?  I think so.  Need to establish that the simulation agrees with the data.  That could also be a major point of this section.  It would also lead into discussion of the current-drifting problem and that since the simulation and data agree, and we checked the hardware, that it is not a hardware problem.
\end{itemize}

The tuning method of proportional gain (P) or $k_c^p$ term  and integral reset (I) or $k_c^i$ term (see Eq.~(\ref{eq:I})) have been discussed in Section~\ref{sec:tune}. The individual effect has not been shown explicitly there. This Section describes the effect of changing P and I term individually and combined. Moreover, the necessity of having both of them or just individual one will be explained. The current drifting problem will become apparent here and how the P and I term is involved will be explained. 

% \subsection{PI Tuning General Behavior}\label{sec:pi_behave}
% Here, the individual effect of P and I term will be discussed first and then concludes with the effect of using PI together.




\subsection{Effect of changing only P term}

Proportional gain term (P) is proportionally multiplying the error (the difference between setpoint and actual measurement) with a constant gain. For the prototype it is
\begin{equation}
    \mathbf{P_{PI}}=k_c^p \Delta I_c^n
\end{equation}
where, $k_c^p$ is the proportional gain and $\Delta I_c^n$ is explained in Eq.~(\ref{eq:del_I}). Depending on the value $k_c^p$, P term tries to minimize the error level between the setpoint and the actual measurement with passage of several measurements. A large value of $k_c^p$ will result large output change for a particular error and eventually it reaches a threshold point above which the system becomes unstable. 

\fig{Images/p25p125}{width = \textwidth,height=14cm}{Currents (left vertical axis) in all six coil sides ($C_x^\pm$, $C_y^\pm$ and $C_z^\pm$) with drift $\Delta$B (right vertical axis) at 12 sensor positions with fluxgate positions 1, 3, 6 and 8 for different values of $k_c^p$ with $k_c^i$ in Eq.~(\ref{eq:I}) being zero. For position of coils and fluxgates see Fig.~\ref{fig: coil}. All grey color curves in both $\Delta$B graphs denote the drift in signal that would have been without the compensation while all other color curves denote the actual drift in signal at all 12 sensor positions found by Eq.~(\ref{eq:del_B}). The 'ON' and 'OFF' vertical dashed lines indicate the time of the perturbation coil being turned 'ON' and 'OFF' respectively. The current supplied on perturbation coil was 10  mA. The resolution index 1 (see Table~\ref{table:t7freq}) with 50 no of averages per measurement was used. The loop sampling frequency was found to be $\sim$6.5 Hz.\label{fig:p_pi}}{Coil currents with drift $\Delta$B for different values of $k_c^p$ only.}

The effect of changing $k_c^p$ has been shown in Fig.~\ref{fig:p_pi} where the currents (left) are being sent to the coils ($C_x^\pm$, $C_y^\pm$ and $C_z^\pm$) for drift $\Delta$B (right) found by Eq.~(\ref{eq:del_B}) for fluxgate positions 1, 3, 6 and 8 excluding two control sensors. It is seen that coil currents (left) ranges from 104 mA to 98 mA in Fig.~\ref{fig:p_pi}\textcolor{blue}{(a)} for  $k_c^p$=0.25 compared to  111 mA nT to 92 mA in Fig.~\ref{fig:p_pi}\textcolor{blue}{(b)} for  $k_c^p$=1.25, where initial current in all six coil sides are 100 mA. It is also seen that $\Delta$B (right) ranges from 10 nT to -40 nT in Fig.~\ref{fig:p_pi}\textcolor{blue}{(b)} for  $k_c^p$=1.25 compared to  22 nT to -42 nT in Fig.~\ref{fig:p_pi}\textcolor{blue}{(a)} for  $k_c^p$=0.25 where uncompensated field ranges from 28 nT to -50 nT. That is, with the increase of $k_c^p$, $\Delta$B magnetic field decreases in the expense of more coil currents. But, there is a limit after which with the increase of $k_c^p$, the systems becomes unstable and starts oscillating which can be seen from the coil currents (left) and some fields (right) in Fig.~\ref{fig:p_pi}\textcolor{blue}{(b)}). It is also noticeable that there is no current drifting problem.
\begin{itemize}
    \item message from the Fig.~\ref{fig:p_pi} ?
\end{itemize}

\FloatBarrier
\fig{Images/mont_p}{width = \textwidth}{The effect of $k_c^p$ on (a) current error and (b) remaining field fluctuations for fluxgate positions 1, 2, and 7. Here, 100 mA current has been supplied in the perturbation coil. For position of the fluxgates  and coil see Fig.~\ref{fig: coil}.\label{fig:mont_p}}{The effect of $k_c^p$ on fluctuations.}

The effect of $k_c^p$ on the current error and remaining field fluctuations for fluxgate signals from 1, 2 and 7 are shown in Fig.~\ref{fig:mont_p}. For position of the sensors please see Fig.~\ref{fig: coil}. For data measurement, for certain loops, PI control algorithm has been implemented for $k_c^p$ ranges from 0.8 to 1.5 in an increment of 0.1. For every measurements, signals from fluxgates at positions 1,2 and 7 are stored as $B_{\text{meas}}$ and uncompensated B field has been stored as $B_{\text{uncomp}}$ which is predicted by Eq.~\ref{eq:Buncomp}. The difference between maximum value of $B_{\text{uncomp}}$ and $B_{\text{uncomp}}$ will give the field fluctuation due to 100 mA supplied in the perturbation coil. Moreover, the difference between maximum value of $B_{\text{meas}}$ and $B_{\text{meas}}$ will give the amount of compensation of the field which when equal to field fluctuation indicates no compensation. The ratio of rms value of this to the field fluctuation will generate the remaining fluctuation $F$ which is shown for different $k_c^p$ in Fig.~\ref{fig:mont_p}\textcolor{blue}{(b)}. It is seen that for a particular filed perturbation, the remaining fluctuation decreases as $k_c^p$ which was also shown earlier. Moreover, $\Delta I_c^{\text{exp}}(r)$ has been found using Eq.(\ref{eq:del_Is}) and the rms has been determined which is shown in  Fig.~\ref{fig:mont_p}\textcolor{blue}{(a)}. It is seen that rms of current error array is taken a big jump with increased $k_c^p$ which indicates that the system becomes unstable as expected.

\FloatBarrier
\fig{Images/mont_pr}{width = \textwidth}{The effect of changing $k_c^p$ and changing $r$ on current fluctuations (a) and remaining field fluctuations (b) for fluxgate positions 1, 2, and 7. Here, 100 mA current has been supplied in the perturbation coil. For position of the fluxgates  and coil see Fig.~\ref{fig: coil}.\label{fig:mont_pr}}{Short}


% \doublefig{Images/p25}{width =\textwidth,height =8cm}{with passive shielding layers. \label{fig:shield_opera2}}{Images/p125}{width = \textwidth,height =8cm}{without outermost passive shielding layer.\label{fig:noShield_opera}}{{Currents (left vertical axis) in all six coil sides ($C_x^\pm$, $C_y^\pm$ and $C_z^\pm$) with drift $\Delta$B (right vertical axis) at sensor position '1x' for different values of $k_c^p$ with $k_c^i$ in Eq.~(\ref{eq:I}) being zero. Blue color curve denotes the actual drift in signal at position '1x' found by Eq.~(\ref{eq:del_B}) while the red curve denotes the drift that would have been without the compensation. The 'ON' and 'OFF' vertical dashed lines indicate the time of the perturbation coil being turned 'ON' and 'OFF' respectively. For position of coils and sensors see Fig.~\ref{fig: coil} \label{fig:p_pi}}}{Prototype setup in OPERA simulation software.}



% \begin{figure}[!htb]
%     \begin{subfigure}{.5\linewidth}
%         \centering
%         \includegraphics[width=\linewidth, height= 6.5 cm]{Images/p25_33}
%         \caption{at $k_c^p$=0.25}
%         \label{fig:p25}
%     \end{subfigure}%
%     \begin{subfigure}{.5\linewidth}
%         \centering
%         \includegraphics[width=\linewidth, height= 6.5 cm]{Images/p50_33}
%         \caption{at $k_c^p$=0.50}
%         \label{fig:p50}
%     \end{subfigure}\\[1ex]
%     \begin{subfigure}{.5\linewidth}
%         \centering
%         \includegraphics[width=\linewidth, height= 6.5 cm]{Images/p75_33}
%         \caption{at $k_c^p$=0.75}
%         \label{fig:p75}
%     \end{subfigure}%
%         \begin{subfigure}{.5\linewidth}
%         \centering
%         \includegraphics[width=\linewidth, height= 6.5 cm]{Images/p100_33}
%         \caption{at $k_c^p$=1.0}
%         \label{fig:p100}
%     \end{subfigure}

%     \caption[short]{Currents (left vertical axis) in all six coil sides ($C_x^\pm$, $C_y^\pm$ and $C_z^\pm$) with drift $\Delta$B (right vertical axis) at sensor position '1x' for different values of $k_c^p$ with $k_c^i$ in Eq.~(\ref{eq:I}) being zero. Blue color curve denotes the actual drift in signal at position '1x' found by Eq.~(\ref{eq:del_B}) while the red curve denotes the drift that would have been without the compensation. The 'ON' and 'OFF' vertical dashed lines indicate the time of the perturbation coil being turned 'ON' and 'OFF' respectively. For position of coils and sensors see Fig.~\ref{fig: coil}. }
%     \label{fig:p_pi2}
% \end{figure}


% It is seen that $\Delta$B=17.5 nT, 15.5 nT and 13.5 nT for $k_c^p$ = 0.25, 0.5 and 0.75 respectively (see Fig.~\ref{fig:p_pi}\textcolor{blue}{(a)}, Fig.~\ref{fig:p_pi}\textcolor{blue}{(b)}, Fig.~\ref{fig:p_pi}\textcolor{blue}{(c)}). That is, with the increase of $k_c^p$, $\Delta$B magnetic field decreases. But, it has a limit after which with the increase of $k_c^p$, the systems becomes unstable and starts oscillating which can be seen from Fig.~\ref{fig:p_pi}\textcolor{blue}{(d)}) where the currents (left) are oscillating and the drift itself also at $\Delta$B=12.5 nT (right). So, the error is reduced maximum by (20.5-12.5)/20.5 * 100$\%\approx$37$\%$ from the initial drift of $\Delta$B=20.5 nT denoted by the red curve at position '1x'. 

\FloatBarrier
The above results confirm that P term reduces difference between the setpoint and the actual measurements of the magnetic field upto a certain point and there is no current drifting problem.

\subsection{Effect of changing only I term}

The error (the difference between setpoint and actual measurement) is accumulated for the length of measurements. Accumulated error keep tracks of the offsets that should be corrected previously. Integral reset (I) term  multiplies that accumulated error with a constant gain. For the prototype it is

\begin{equation}
    \mathbf{I_{PI}}=k_c^i \sum_n \Delta I_c^n
\end{equation}
where, $k_c^i$ is the integral gain and $\Delta I_c^n$ is explained in Eq.~(\ref{eq:del_I}). I term takes care of the offset which are not corrected by the P term and thus accelerates reducing the error level. Depending on the value $k_c^i$, how fast the feedback loop will response to the drift in the signal will be determined. A large value of $k_c^p$ will result faster response to reducing the error level and eventually it reaches a threshold point above which the actual measurement will overshoot i.e. exceed the setpoint. 
% The main downfall of this is that the time required for the coil current to be settle in after reducing the error level may be very slow or never ever settle in.



% As like the effect on P, the effect of changing I has been shown in Fig.~\ref{fig:i_pi} where the change in current in all six coil sides with  $\Delta$B on a particular sensor position have been observed for $k_c^i$=0.25, 0.5, 0.75 and 1.0 . It is seen that with increase of I the level of compensation of the magnetic field is almost similar but the main difference occurs on how fast the system response in an expense of increasing current in all the coil sides (see Fig.~\ref{fig:i_pi}\textcolor{blue}{(a)}, Fig.~\ref{fig:i_pi}\textcolor{blue}{(b)}, Fig.~\ref{fig:i_pi}\textcolor{blue}{(c)} and Fig.~\ref{fig:i_pi}\textcolor{blue}{(d)}). The main problem with changing only I term is that it creates a very slow current response time. But, in terms of compensation only changing I gives very good result. The slow current response can be minimized by decreasing the value of optimized $r$ (see Section~\ref{sec:r_pi} and Section~\ref{sec:r_currentResponse} ).

\fig{Images/i125}{width = \textwidth,height =7cm}{Currents (left vertical axis) in all six coil sides ($C_x^\pm$, $C_y^\pm$ and $C_z^\pm$) with drift $\Delta$B (right vertical axis) at 12 sensor positions with fluxgate positions 1, 3, 6 and 8 for $k_c^i$=1.25 with $k_c^p$ in Eq.~(\ref{eq:I}) being zero. For position of coils and fluxgates see Fig.~\ref{fig: coil}. All grey color curves in $\Delta$B graph denote the drift in signal that would have been without the compensation while all other color curves denote the actual drift in signal at all 12 sensor positions found by Eq.~(\ref{eq:del_B}). The 'ON' and 'OFF' vertical dashed lines indicate the time of the perturbation coil being turned 'ON' and 'OFF' respectively. The current supplied on perturbation coil was 10  mA. The resolution index 1 (see Table~\ref{table:t7freq}) with 50 no of averages per measurement was used. The loop sampling frequency was found to be $\sim$6.5 Hz.\label{fig:i_pi}}{Coil currents with drift $\Delta$B for $k_c^i$ only.}

% \begin{figure}[!htb]
%     \begin{subfigure}{.5\linewidth}
%         \centering
%         \includegraphics[width=\linewidth, height= 6.5 cm]{Images/i25_33}
%         \caption{at $k_c^i$=0.25}
%         \label{fig:i25}
%     \end{subfigure}%
%     \begin{subfigure}{.5\linewidth}
%         \centering
%         \includegraphics[width=\linewidth, height= 6.5 cm]{Images/i75_33}
%         \caption{at $k_c^i$=0.75}
%         \label{fig:i50}
%     \end{subfigure}\\[1ex]
%     \begin{subfigure}{.5\linewidth}
%         \centering
%         \includegraphics[width=\linewidth, height= 6.5 cm]{Images/i100_33}
%         \caption{at $k_c^i$=1.0}
%         \label{fig:i75}
%     \end{subfigure}%
%         \begin{subfigure}{.5\linewidth}
%         \centering
%         \includegraphics[width=\linewidth, height= 6.5 cm]{Images/i125_33}
%         \caption{at $k_c^i$=1.25}
%         \label{fig:i100}
%     \end{subfigure}

%     \caption[short]{Currents (left vertical axis) in all six coil sides ($C_x^\pm$, $C_y^\pm$ and $C_z^\pm$) with drift $\Delta$B (right vertical axis) at sensor position '1x' for different values of $k_c^i$ with $k_c^p$ ( see Eq.~(\ref{eq:I}) ) being zero. Blue color curve denotes the actual drift in signal at position '1x' found by Eq.~(\ref{eq:del_B}) while the red curve denotes the drift that would have been without the compensation. The 'ON' and 'OFF' vertical dashed lines indicate the time of the perturbation coil being turned 'ON' and 'OFF' respectively. For position of coils and sensors see Fig.~\ref{fig: coil}.}
%     \label{fig:i_pi}
% \end{figure}



The effect for using $k_c^i$=1.25 only has been shown in Fig.~\ref{fig:i_pi} where the currents (left) that are being sent to the coils ($C_x^\pm$, $C_y^\pm$ and $C_z^\pm$) for drift $\Delta$B found by Eq.~(\ref{eq:del_B}) in fluxgate positions 1, 3, 6 and 8 excluding two control sensors. It is seen that coil currents (left) ranges from 137 mA to 85 mA , where initial current in all six coil sides are 100 mA which is huge compare to only P term. It is also seen that $\Delta$B (right) ranges from 10 nT to -18 nT except one at -30 nT, where uncompensated field ranges from 28 nT to -50 nT which decreases largely compare to P term . It seems like magnetic field reduction by only P term is negligible compare to I term. But, the current drifting problems appear for only I term.

% \begin{itemize}
%     \item message from the Fig.~\ref{fig:i_pi} ?
% \end{itemize}

\FloatBarrier

% \doublefig{Images/i100_33_zoom}{width =\textwidth,height =8cm}{at $k_c^i$=1.0 \label{fig:shield_opera2}}{Images/i125_33_zoom}{width = \textwidth,height =8cm}{at $k_c^i$=1.25 \label{fig:noShield_opera}}{{Simulated zoomed version of the drift $\Delta$B for fluxgate position 1x for different values of $k_c^i$ with $k_c^p$ ( see Eq.~(\ref{eq:I}) ) being zero. For position of coils and sensors see Fig.~\ref{fig: coil}. For description about simulation see Section~\ref{sec:pSim}. The red vertical dashed line indicates the time of the perturbation coil being turned 'ON'. A time delay of 0.145 s was used to match the experimental loop sampling frequency of $\sim$6.5 Hz. \label{fig:i_pi_zoom}}}{Simulated zoomed version of the drift $\Delta$B.}

% For understating the system response time and the overshoot effect, two PI simulation as explained in Section~\ref{sec:pSim} with only $k_c^i$=1.0 and $k_c^i$=1.25 have been made and  each case $\Delta$B graphs have been generated and zoomed as shown in Fig.~\ref{fig:i_pi_zoom}. It is seen that the $\Delta$B is $\sim$3.23 nT  at 25-10=15 s (the perturbation is turned 'ON' at 10 s) in Fig.~\ref{fig:i_pi_zoom}\textcolor{blue}{(a)} for $k_c^i$=1.0  compare to within 10.6-10.0=0.6 s in Fig.~\ref{fig:i_pi_zoom}\textcolor{blue}{(b)} for $k_c^i$=1.25. So, the magnetic field reduction response time increases hugely with increase value of $k_c^i$. It is also seen from the Fig.~\ref{fig:i_pi_zoom}\textcolor{blue}{(b)} that there is an overshoot in the error level before it settles in.
% \begin{itemize}
%     \item message from the Fig.~\ref{fig:i_pi_zoom} ?
% \end{itemize}

% It is seen that the error level (right) seems to be $\sim$3.5 nT in every case and the coil currents(left) settle  faster for increasing value of $k_c^i$. The figures are neither helpful to understand the system response time nor the overshoot effect in the $\Delta$B graph (right). 
% for the same duration of perturbation (here 'ON' at 11s and 'OFF' at 51s and so total 40s)
% \begin{figure}[!htb]
%     \begin{subfigure}{.5\linewidth}
%         \centering
%         \includegraphics[width=\linewidth, height= 6.5 cm]{Images/i25_33_zoom.png}
%         \caption{at $k_c^i$=0.25}
%         \label{fig:i25zoom}
%     \end{subfigure}%
%     \begin{subfigure}{.5\linewidth}
%         \centering
%         \includegraphics[width=\linewidth, height= 6.5 cm]{Images/i75_33_zoom.png}
%         \caption{at $k_c^i$=0.75}
%         \label{fig:i75zoom}
%     \end{subfigure}\\[1ex]
%     \begin{subfigure}{.5\linewidth}
%         \centering
%         \includegraphics[width=\linewidth, height= 6.5 cm]{Images/i100_33_zoom.png}
%         \caption{at $k_c^i$=1.0}
%         \label{fig:i100zoom}
%     \end{subfigure}%
%         \begin{subfigure}{.5\linewidth}
%         \centering
%         \includegraphics[width=\linewidth, height= 6.5 cm]{Images/i125_33_zoom.png}
%         \caption{at $k_c^i$=1.25}
%         \label{fig:i125zoom}
%     \end{subfigure}

%     \caption[short]{Zoomed in version of the drift $\Delta$B shown in right side of Fig.~\ref{fig:i_pi}\textcolor{blue}{(a)}, Fig.~\ref{fig:i_pi}\textcolor{blue}{(b)}, Fig.~\ref{fig:i_pi}\textcolor{blue}{(c)} and Fig.~\ref{fig:i_pi}\textcolor{blue}{(d)} respectively at sensor position '1x' for different values of $k_c^i$ with $k_c^p$ ( see Eq.~(\ref{eq:I}) ) being zero. The red vertical dashed line indicates the time of the perturbation coil being turned 'ON'. For position of coils and sensors see Fig.~\ref{fig: coil}.\label{fig:i_pi_zoom}}
% \end{figure}

% \fig{Images/i_pi_zoom}{width = \textwidth,height =10cm}{Zoomed in version of the drift $\Delta$B shown in right side of Fig.~\ref{fig:i_pi}\textcolor{blue}{(a)}, Fig.~\ref{fig:i_pi}\textcolor{blue}{(b)}, Fig.~\ref{fig:i_pi}\textcolor{blue}{(c)} and Fig.~\ref{fig:i_pi}\textcolor{blue}{(d)} respectively at sensor position '1x' for different values of $k_c^i$ with $k_c^p$ ( see Eq.~(\ref{eq:I}) ) being zero. The red vertical dashed line indicates the time of the perturbation coil being turned 'ON'. For position of coils and sensors see Fig.~\ref{fig: coil}.\label{fig:i_pi_zoom}}


\FloatBarrier
The above results confirm that magnetic reduction level for only I term is higher than that of only P term in expense of larger current changes and also current drifting problem.
% \subsection{r vs. Condition No.}\label{sec:cond}
% Instead of going through all the steps that are discussed in Section~\ref{sec:inv}, the concept of condition number of a matrix can be used. The condition number of $\bm{M}$ can be determined from the diagonal matrix $\bm{\Sigma}$ as given in eq.\ref{eq:m} by -
%  \begin{equation}
%      cond(\bm{M})=\frac{max(\sigma_d)}{min(\sigma_d)}
%  \end{equation}
 

\subsection{Effect of changing PI term combined}

P and I term have been tuned following the discussion on Section~\ref{sec:tune}. The tuned values are found to be $k_c^p$=0.60 and $k_c^i$=0.37.

\fig{Images/p60i368}{width = \textwidth,height=7cm}{Currents (left vertical axis) in all six coil sides ($C_x^\pm$, $C_y^\pm$ and $C_z^\pm$) with drift $\Delta$B (right vertical axis) at 12 sensor positions with fluxgate positions 1, 3, 6 and 8 for $k_c^p$=0.6 and $k_c^i$=0.37 (see Eq.~(\ref{eq:I})). For position of coils and fluxgates see Fig.~\ref{fig: coil}. All grey color curves in $\Delta$B graph denote the drift in signal that would have been without the compensation while all other color curves denote the actual drift in signal at fluxgate positions 1, 3, 6 and 8 found by Eq.~(\ref{eq:del_B}). The 'ON' and 'OFF' vertical dashed lines indicate the time of the perturbation coil being turned 'ON' and 'OFF' respectively. The current supplied on perturbation coil was 10  mA. The resolution index 1 (see Table~\ref{table:t7freq}) with 50 no of averages per measurement was used. The loop sampling frequency was found to be $\sim$6.5 Hz.\label{fig:tuned_vs_i}}{Coil currents with drift $\Delta$B for $k_c^p$ and $k_c^i$ combined.}

The effect for using $k_c^p$=0.60 and $k_c^i$=0.37 has been shown in Fig.~\ref{fig:pi_i_mCond30}\textcolor{blue}{(a)} where the currents (left) that are being sent to the coils ($C_x^\pm$, $C_y^\pm$ and $C_z^\pm$) for drift $\Delta$B (right) found by Eq.~(\ref{eq:del_B}) in fluxgate positions 1, 3, 6 and 8 excluding two control sensors. It is seen that the coil currents (left) and $\Delta$B (right) graphs have similar pattern to only I term pattern which is shown in Fig.~\ref{fig:pi_i_mCond30}\textcolor{blue}{(b)} where only $k_c^i$=0.37 is used.


\begin{itemize}
    \item message from the Fig.~\ref{fig:tuned_vs_i} ?
    \item Fig.~\ref{fig:tuned_vs_i} or Fig.~\ref{fig:pi_i_mCond30} ??
\end{itemize}
% For simplicity instead of showing all the drift $\Delta$B for all the fluxgate sensors for the positions given in the horizonatal axis of Fig.~\ref{fig:m}, only '1x' is shown on the right of the figure. And same as earlier the currents  that are being sent to the coils ($C_x^\pm$, $C_y^\pm$ and $C_z^\pm$) shown on the left of the same figure. But, we couldn't determine the effect of having both of them at a time. So, keeping $k_c^i$ as 0.52 and excluding P term i.e. $k_c^p$=0.0 we run the same measurement again and the results are shwon in Fig.~\ref{fig:tuned_vs_i}\textcolor{blue}{(b)}. As as matter of surprise, there is hardly any difference between the results in Fig.~\ref{fig:tuned_vs_i}\textcolor{blue}{(a)} and Fig.~\ref{fig:tuned_vs_i}\textcolor{blue}{(b)}. Why is that so ? For the moment, the Fig.~\ref{fig:tuned_vs_i} suggests that may be we don't need P term at all or maybe we need different tuning methods. So, applying P and I term at a time raises question of the necessity of the P term or importance of the tuning method describe in Section~\ref{sec:tune}. Due to lack of time, we did not further go into other tuning methods. Rather we have tried to discover the differences in the work between Ref.~\cite{bea} and Ref.~\cite{rawlik}.



\FloatBarrier
\begin{itemize}
    \item message from the Fig.~\ref{fig:pi_i_mCond30} ?
\end{itemize}

\fig{Images/pi_i_mCond30}{width = \textwidth,height=14cm}{Currents (left vertical axis) in all six coil sides ($C_x^\pm$, $C_y^\pm$ and $C_z^\pm$) with drift $\Delta$B (right vertical axis) at 12 sensor positions with fluxgate positions 1, 3, 6 and 8 for for $k_c^p$=0.6 and $k_c^i$=0.37 (see Eq.~(\ref{eq:I})) in (a) and for $k_c^p$=0.0 and $k_c^i$=0.37 (see Eq.~(\ref{eq:I})). For position of coils and fluxgates see Fig.~\ref{fig: coil}. All grey color curves in both $\Delta$B graphs denote the drift in signal that would have been without the compensation while all other color curves denote the actual drift in signal at fluxgate positions 1, 3, 6 and 8 found by Eq.~(\ref{eq:del_B}). The 'ON' and 'OFF' vertical dashed lines indicate the time of the perturbation coil being turned 'ON' and 'OFF' respectively. The current supplied on perturbation coil was 10  mA. The resolution index 1 (see Table~\ref{table:t7freq}) with 50 no of averages per measurement was used. The loop sampling frequency was found to be $\sim$6.5 Hz.\label{fig:pi_i_mCond30}}{Coil currents with drift $\Delta$B for different values of $k_c^p$ only.}


% \fig{Images/mont_i}{width = \textwidth}{The effect of $k_c^i$ on current fluctuations (a) and remaining field fluctuations (b) for fluxgate positions 1, 2, and 7. Here, 100 mA current has been supplied in the perturbation coil and $k_c^p$=0.7 and $r$=3.5 have been used. For position of the fluxgates  and coil see Fig.~\ref{fig: coil}.\label{fig:mont_i}}{The effect of $k_c^i$ on fluctuations.}
% \doublefig{Images/p43i52_33}{width =\textwidth,height =8cm}{at $k_c^p$=0.43 and $k_c^i$=0.52. \label{fig:pi_tuned}}{Images/i52_33}{width = \textwidth,height =8cm}{at $k_c^p$=0.0 and $k_c^i$=0.52..\label{fig:i52}}{{Currents (left vertical axis) in all six coil sides ($C_x^\pm$, $C_y^\pm$ and $C_z^\pm$) with drift $\Delta$B (right vertical axis) at sensor position '1x' for combine different values of $k_c^i$ and $k_c^p$ ( see Eq.~(\ref{eq:I}) ). Blue color curve denotes the actual drift in signal at position '1x' found by Eq.~(\ref{eq:del_B}), while the red curve denotes the drift that would have been without the compensation. The 'ON' and 'OFF' vertical dashed lines indicate the time of the perturbation coil being turned 'ON' and 'OFF' respectively. For position of coils and fluxgate sensor see Fig.~\ref{fig: coil}.} \label{fig:tuned_vs_i}}{short}

\FloatBarrier


So the P term has negligible effect compare to I term which is seen from comparing Fig.~\ref{fig:p_pi} with Fig.~\ref{fig:i_pi} and Fig.~\ref{fig:pi_i_mCond30}. Current drifting problem does not appear for P only but appears much more pronounced for I only, and PI combined. 
% \item Do we need all the figures?  Can the data be summarized better (all currents and all fields on two graphs for each setting?)
% \item Should we show data?  I think so.  Need to establish that the simulation agrees with the data.  That could also be a major point of this section.  It would also lead into discussion of the current-drifting problem and that since the simulation and data agree, and we checked the hardware, that it is not a hardware problem..


\section{Fluxgate Placements and Impact of Passive Shield}\label{sec:flux_place}


\begin{itemize}
\item We still had no clue so we just started changing things randomly.
\item We changed fluxgate positions to try to make cond number smaller to see if it had an effect.  Could not change the cond number or the problem significantly.
\item We were confused if the slow current response might be due to magnetic responses that we slow.  So we tried removing the shields.  No change.
\item consider a graph where there is current drifting but with no shield.
\item And then we realized this had nothing to do with it.
\end{itemize}

We had no idea with coil current not being settle properly although the field seems to be compensated on time. We changed fluxgate positions to try to make condition number to see if it had an effect. We had tried different positions mainly in corners and center of each of the coil faces. We could not change the condition number or the problem significantly. We were confused if the slow current response might be due to magnetic responses that we slow.  So we tried removing the shields which did not bring any change in the problem. In this Section, the results of those studied will be shown and discussed. 


\subsection{Fluxgate Placements}

We had no clear idea about the placements of the fluxgates from the previous studies and our coil current also did not settle down the way we thought is should. We started taking data with 12 sensors at slowest sampling frequency (see Table~\ref{table:t7freq}) in the corners but the coil currents were not properly settle down. Then we thought maybe increasing sensors would eliminate the problems. So, we bought new fluxgate sensors and build another breakout box (see Section~\ref{sec:sensor}) but still there were slow responses in the coil currents. Then we thought may be if we place in the center of each of the faces of the coils, the results would be better but unlucky us. While experimenting different fluxgate placements with different number of fluxgates we realized that fluxgate placements played vital role on compensating field in a certain positions. 

\begin{table} [htb!]
    \centering
    \begin{tabular} { |c|c|c|c|c|c|} 
        \hline
        \makecell{Fluxgates \\Position} & \makecell{$\bm{M}$\\ Condition No.} &\makecell{$\bm{M^{-1}}$\\ Condition No.} & $r$ & $r'$\\
        \hline\hline
        1, 3, 6 and 8 & 33.25 & 2.63 & 2.97 & 2.95\\ 
        \hline
        2, 4, 5 and 7 & 28.55 & 1.98 & 3.04 & 3.06 \\ 
        \hline
        \makecell{Center \\($C_x^\pm$, $C_y^\pm$ and $C_z^\pm$)} & 36.24 & 1.8 & 2.47 & 2.49 \\ 
        \hline
        \makecell{Center-6cm \\($C_x^\pm$, $C_y^\pm$ and $C_z^\pm$)} & 98.61 & 3.05 & 2.36 & 2.34 \\ 
        \hline
        \makecell{Center+6cm \\($C_x^\pm$, $C_y^\pm$ and $C_z^\pm$)} & 80.74 & 1.49 & \textcolor{red}{2.26} & \textcolor{red}{2.02} \\ 
        \hline
        \makecell{1, 2, 3, 4, \\5, 6, 7 and 8} & 28.30 & 1.94 & 3.17 & 3.19 \\ 
        \hline
        \makecell{1, 2, 3, 4, \\5, 6, 7, 8 and \\Center ($C_x^\pm$, $C_y^\pm$ and $C_z^\pm$)}  & 21.84 & 1.8 & \textcolor{red}{3.23} & \textcolor{red}{3.37} \\ 
        \hline

    \end{tabular}
    % \vspace{4mm}
    \caption[Properties of different no of fluxgate sensors for different positions.]{Properties of different no of fluxgate sensors for different positions. For the positions of the fluxgates see Fig.~\ref{fig: coil}. The column $r$ represents the regularization method which is explained in Section~\ref{sec:mont} and $r'$ represents the regularization method which will be explained in Section~\ref{sec:cond}. }\label{table:flux-pos}
\end{table}

The study on different fluxgate placements are summarized in Table~\ref{table:flux-pos}. Mainly the corner positions and the center of each of the coil faces  have been tested. Also, they have been analyzed by moving slightly in different positions. Position of the fluxgates are defined by the numbers while they are in corners and when they are in the center of each of the coil faces they are termed as 'Center ($C_x^\pm$, $C_y^\pm$ and $C_z^\pm$)'. Center-6cm means all the sensors in the center of the coil faces have been brought 6cm towards the origin from the center and center+6cm means they have brught 6cm away outside the center. For, the full picture of the positions see Fig.~\ref{fig: coil}. It is seen that that matrix condition number is from 22-36 for the fluxgates being placed either in corners or in center. But if they are slightly moved within $\pm$6cm of center then the matrix condition number becomes very large. Only considering the matrix condition number, it seems that having fluxgates in all the corners and the center of each of the coil faces should be the best choice as its giving the lowest condition number which is 22. But that is not the whole story! Our study with sensors in the middle of the coil sides reveals that only P controller is suitable while a small I term makes the current response unstable. But in the case of corner positions either P only or I only or PI combined controller option is available. It is also seen that matrix condition number in the corners can be improved by placing the fluxgates slightly away from the corners inside the coil cube which also helps to compensate the fields in certain positions. It is also noticeable that regularization by two different methods are agreeing except in two positions which are marked red.

% \doublefig{Images/cB_t_center_p}{width =\textwidth, height= 6 cm}{Position=1,2,3,4,6,8\label{fig:cb_center_p}}{Images/cB_t_center_pi}{width = \textwidth, height= 6 cm}{Position=1,2,3,4,6,8\label{fig:cb_center_pi}}{{PI Active Magnetic Field Compensation Results by both Experiment and Simulation.} \label{fig:cb_center}}

% \doublefig{Images/bt6}{width =\textwidth, height= 7 cm}{Position=1,2,3,4,6,8\label{fig:bt6}}{Images/sf6}{width = \textwidth, height= 7 cm}{Position=1,2,3,4,6,8\label{fig:sf6}}{{PI Active Magnetic Field Compensation Results by both Experiment and Simulation.} \label{fig:btSF6}}
\FloatBarrier
% \doublefig{Images/bt8}{width =\textwidth, height= 7 cm}{Position=1,2,3,4,5,6,7,8\label{fig:bt8}}{Images/sf8}{width = \textwidth, height= 7 cm}{Position=1,2,3,4,5,6,7,8\label{fig:sf8}}{{PI Active Magnetic Field Compensation Results by both Experiment and Simulation.} \label{fig:btSF8}}
The above discussion of the Table~\ref{table:flux-pos} shows that current goes crazy if the fluxgates are placed in the center of each of the coil faces and crazier if they are slightly moved inside or outside of the center. The corners position fluxgates shows normal behaviour compare to the center positions. But the study does not help us solving our original problem of slow current response.
% To quantify more we have taken the help of regenerating Fig.~\ref{fig:Isim} and Fig.~\ref{fig:fluc-sim}. From Fig.~\ref{fig:Isim}, we have recorded the maximum  $\Delta I_c^{\text{simRMS}}$ (mA) for 30 different sets of $B_s^{\text{rand}}$. For maximum compensation we have generated the Fig.\ref{fig:fluc-sim} for same for 30 different sets of $B_s^{\text{rand}}$ from which we have recorded lowest the remaining fluctuation F. For example- for fluxgate positions 1, 3, 6 and 8, the lowest F goes to 0.3. That means the maximum compensation due to the field produced by $\Delta I_c^{\text{sim}}$to counteract $B_s^{\text{rand}}$ is(1-0.3$\times$100$\%$) =70$\%$. For more details see Section~\ref{sec:mont}. It seen that the current goes crazy if the fluxgates are placed in the center of the coil faces and the maximum $\Delta I_c^{\text{simRMS}}$ ranges from 98 to 192 mA. But the maximum $\Delta I_c^{\text{simRMS}}$ is 12 mA when the center fluxgates are used with the corners one. But on that time the maximum compensation due to the field produced by $\Delta I_c^{\text{sim}}$to counteract $B_s^{\text{rand}}$ is only 15$\%$. So, anything with center seems to give crazy results. Then by looking at the all the three parameters e.g. matrix condition number, maximum $\Delta I_c^{\text{simRMS}}$ and maximum compensation for the 3 different sets of corner positions only, it is seen that the results are more balanced and surprisingly the compensation with 4*3-axis sensors are better than 8*3-axis sensors in exchange of more maximum $\Delta I_c^{\text{simRMS}}$.





% The increase in matrix condition number for the center of the coil sides is noteworthy.  So, a study has been done to see the current response effect while sensors are in the middle of the coil sides as shown in Fig.~\ref{fig:cb_center}. The Fig.~\ref{fig:cb_center_p} represents the current response on all the six coil sides with their effect on the $z$-axis in the origin as shown in the right with only choosing proportional (P) term. But just adding a small fraction of integral resest (I) term makes the current response unstable as shown in Fig.~\ref{fig:cb_center_pi}. So only P controller is suitable if fluxgates are placed in the middle of the coil sides but in the case of corner positions either P only or I only or PI controller option is available. In a summary, the more the sensors the more is the matrix condition number. Corner positions are better in terms of different freedom of controlling.

% \doublefig{Images/cB_t_center_p}{width =\textwidth, height= 6 cm}{Position=1,2,3,4,6,8\label{fig:cb_center_p}}{Images/cB_t_center_pi}{width = \textwidth, height= 6 cm}{Position=1,2,3,4,6,8\label{fig:cb_center_pi}}{{PI Active Magnetic Field Compensation Results by both Experiment and Simulation.} \label{fig:cb_center}}

% \doublefig{Images/bt6}{width =\textwidth, height= 7 cm}{Position=1,2,3,4,6,8\label{fig:bt6}}{Images/sf6}{width = \textwidth, height= 7 cm}{Position=1,2,3,4,6,8\label{fig:sf6}}{{PI Active Magnetic Field Compensation Results by both Experiment and Simulation.} \label{fig:btSF6}}

% \doublefig{Images/bt8}{width =\textwidth, height= 7 cm}{Position=1,2,3,4,5,6,7,8\label{fig:bt8}}{Images/sf8}{width = \textwidth, height= 7 cm}{Position=1,2,3,4,5,6,7,8\label{fig:sf8}}{{PI Active Magnetic Field Compensation Results by both Experiment and Simulation.} \label{fig:btSF8}}
% The above discussion of the Table~\ref{table:flux-pos} shows that current goes crazy if the fluxgates are placed in the center of each of the coil faces and crazier if they are slightly moved inside or outside of the center. The corners position fluxgates shows normal behaviour compare to the center positions and surprisingly less sensors shows better compensation in the corners in exchange more more currents.


\subsection{Impact of Passive Shield}
As fluxgate placements failed to solve the slow coil current response, we thought that increasing sampling frequency will solve the problem. So, besides the advantages discussed in Section~\ref{sec:freq}, solving the slow current response time was also a motivation to build the filters as that would allow us to use the fastest sampling frequency of the ADC. Due to time limitations and cost concern we had only built 12 filters to support 12 fluxgate sensors. Using the filter, we had reduced the response time but that was not the solution of the current unsettle problem. Then we thought the passive shielding layer had some impact on the current unsettle problem and decided to remove the outermost shield (see Section~\ref{sec:shield}). 

% After that we decided to use the fastest sampling frequency of our ADC for which we have to build the filters.  Now, due to this the current response time has increased but that was not the solution of the current unsettle problem. Finally, we have realized that in addition to fastest response we have to also consider the matrix condition number and if the condition number is large then we have to lower the value of optimized $r$ (see Section~\ref{sec:new_study_r}). Here, we will talk about the the studies we have done on fluxgate placements.
\fig{Images/shield_noShield}{width = \textwidth,height=14cm}{Currents (left) in all six coil sides ($C_x^\pm$, $C_y^\pm$ and $C_z^\pm$) with drift $\Delta$B (right) for fluxgate positions 1, 3, 6 and 8 for $k_c^p$=0.0, $k_c^i$=1.0 (see Eq.~(\ref{eq:I})) and $r$=2.8 with (a) shield and (b) without shield. For position of coils and fluxgates see Fig.~\ref{fig: coil}. All grey color curves in $\Delta$B graph denote the drift in signal that would have been without the compensation while all other color curves denote the actual drift in signal at all 12 sensor positions found by Eq.~(\ref{eq:del_B}). The 'ON' and 'OFF' vertical dashed lines indicate the time of the perturbation coil being turned 'ON' and 'OFF' respectively. The current supplied on perturbation coil was 25  mA. The resolution index 1 (see Table~\ref{table:t7freq}) with 50 no of averages per measurement used. The loop sampling frequency was found to be $\sim$6.5 Hz.\label{fig:shield_noShield}}{Coil currents with drift $\Delta$B for $k_c^p$=0.0, $k_c^i$=1.0 and $r$=2.8}

There are four layers of shields have been used for passive shielding in this prototype (see Section~\ref{sec:shield}). The active compensation effect has been seen with outermost shield and no shield. The results are shown in Fig.~\ref{fig:shield_noShield} where the currents (left) that are being sent to the coils ($C_x^\pm$, $C_y^\pm$ and $C_z^\pm$) for drift $\Delta$B (right) found by Eq.~(\ref{eq:del_B}) in fluxgate positions 1, 3, 6 and 8 excluding two control sensors with outermost passive shield in Fig.~\ref{fig:shield_noShield}\textcolor{blue}{(a)} and without shield in Fig.~\ref{fig:shield_noShield}\textcolor{blue}{(b)}. It is seen that the coil currents (left) and $\Delta$B (right) graphs have similar pattern which clears our confusion about the impact of shield on current unsettle problem.

\begin{itemize}
     \item message from Fig.~\ref{fig:shield_noShield}??
 \end{itemize}

% except for the case of shield, the matrix condition number is less. The Fig.~\ref{fig:btSF8_s} shows the AMC compensation (Fig.~\ref{fig:bt8_s}) and the corresponding allan deviation and shielding factor ( Fig.~\ref{fig:sf8_s}) with outermost layer of shield. The matrix condition number is found to be 19. Similarly, the Fig.~\ref{fig:btSF8} shows the AMC compensation (Fig.~\ref{fig:bt8}) and the corresponding allan deviation and shielding factor ( Fig.~\ref{fig:sf8}) without any shield. The matrix condition number in this case is 132. The one advantage of having shielding is that , the shielding factor for all the sensors remain $geq$1 but in case of no shield, the shielding factor may be very good at certain point but in some point it can go below 1.



% \doublefig{Images/bt8_shield}{width =\textwidth, height= 7 cm}{Position=1,2,3,4,6,8\label{fig:bt8_s}}{Images/sf8_shield}{width = \textwidth, height= 7 cm}{Position=1,2,3,4,6,8\label{fig:sf8_s}}{{Shield} \label{fig:btSF8_s}}{short}

% \doublefig{Images/bt8}{width =\textwidth, height= 7 cm}{Position=1,2,3,4,5,6,7,8\label{fig:bt8}}{Images/sf8}{width = \textwidth, height= 7 cm}{Position=1,2,3,4,5,6,7,8\label{fig:sf8}}{{No SHield} \label{fig:btSF8}}{short}
\FloatBarrier
So neither fluxgate placement nor shielding effect provide us with the solution of current unsettle problem. Next we will discuss an unique work of us which turns out to be handy for the new works we have presented in the future Sections.

\section{Simulation of Prototype Integrating Feedback Algorithm}\label{sec:pSim}

We were still in dark about current unsettle problem. So we thought simulating  the prototype would confirm whether we were getting the correct current graph. We first made a simulation without PI control algorithm to compare the experimental change in coil currents due to certain perturbation with the simulation ones (Eq.~(\ref{eq:del_I})) and the results were agreed except some sign issues. But we wanted to see simulated coil currents in time due to PI control algorithm (Eq.~(\ref{eq:I})). My supervisor made a simple one dimensional PI simulation in time in Python programming language which later I used to make a fully functional multi-dimensional PI simulation. The simulation process has been divided into three groups. They are 

\begin{enumerate}
    \item Simulation in OPERA to Generate Field Maps in Coil Cube
    \item Field Maps Processing in Python
    \item PI Simulation in Python
\end{enumerate}
    

\subsection{Simulation in OPERA to Generate Field Maps in Coil Cube}
% The simulating process will be discussed here at first and then the result of the simulation will be compared with the experiment and whether there is current unsettle problem exists in the simulation will be discovered.

OPERA 3D Finite Element Analysis~(FEA)~\cite{opera} simulation software is used for simulating the prototype and generating field map within the coil cube of the prototype. OPERA is a multi-physics software package.  We have used the Opera Static Electromagnetics module which computes magnetostatic and electrostatic fields in three dimensions. The module solves Maxwell$'$s equations for the static case in a discretized model using FEA. For calculating magnetic fields from coils, the module uses Biot-Savart integral equation. Mainly OPERA helps us to 
% \renewcommand{\labelitemi}{$\square$}
\begin{itemize}
    \item make geometry of the experimental setup.
    \item extract the magnetic field values within the geometry.
    \item energize each coil.
\end{itemize}



% which allows coupled simulations including the ability to couple a Static Module simulation with a thermal simulation or a stress simulation, for example to calculate the stress due to electromagnetic forces.The module automatically switches to the part containing magnetic sources in case of magnetostatics.or direct current (DC) flow .To calculate the potential at each node of the mesh, OPERA uses an iterative solution technique which is memory-efficient and computationally fast. In this module, the properties of the magnetic materials can be specified as required i.e. linear, non-linear, isotropic, anisotropic, laminated or permanent magnet. 



% and then move forward to all the simulations made to run a fully functional PI simulation. For, data analysis, sorting  and modifying the data and also running the PI control algorithm the Python programming language has been used alongside the simulation software. The complete simulation is made of three different simulations. 

% \begin{itemize}
%     \item Making geometry of the experimental setup and extracting the magnetic field values in that geometry
%     \item Simulation of Matrix M
%     \item Simulation of Magnetic Field Distribution
%     \item Simulation of PI Control
% \end{itemize}
% In the following  all of the above simulations  will be discussed in detail starting with the simulation software itself. The section ends with comparing the simulation results with the experimental ones.

% \subsection{About Simulation Software}
% Here, the simulation software has been discussed in detail. Mainly, the focus is on the the module of the software that we have used .



% In the upcoming Sub-sections, the geometry that has been built will be discussed along with all the steps done to finally make a PI control simulation.


\subsubsection{Geometry of Experimental Setup in OPERA}

For simulating the prototype we first need its geometry. The dimensions of the prototype passive shielding layers and the stove pipe are given in Table~\ref{table:mu-metal}. We have used the dimension of the outermost passive shielding layer along with stove pipes to draw in OPERA. The dimensions of the coils are given in Table~\ref{table:coil}. We have used the dimensions of all the coils to draw a cube in OPERA. Moreover, we have also used the dimension of the perturbation coil from the same Table to draw the perturbation coil itself in OPERA. The process to build the prototype geometry in the OPERA has been discussed both with outermost layer of passive shielding and without the shield.

\doublefig{Images/shield2}{width =\textwidth,height =8cm}{with passive shielding layers. \label{fig:shield_opera}}{Images/noShield}{width = \textwidth,height =8cm}{without outermost passive shielding layer.\label{fig:noShield_opera}}{{Prototype setup in OPERA simulation software (a) with outermost passive shielding layer  and (b) without any passive shielding layer. The square in the back appearing in both drawings is the perturbation coil. The shield dimensions and properties were given in Section~\ref{sec:shield}  and those of the coils in Section~\ref{sec:cube}. The direction of each of the three axis has been shown in (b).} \label{fig:opera_setup}}{Prototype setup in OPERA simulation software.}

A cylinder has been drawn in OPERA using the dimensions of outermost passive shielding layer and air has been chosen as material there (dimensions in Table~\ref{table:mu-metal}). Then a tube has been drawn keeping the length same but adding a thickness of 0.3175 cm with the radius of the outermost shield and the material is chosen to be linear with relative permeability 20000. Thus the outermost shield with air inside has been drawn in OPERA. Similarly, the dimensions of stove pipe has been used to draw it in both sides of the outermost shield layer with air inside.  Another cylinder has been drawn to fill the space between the coils and the outermost shield with air. The coils have been drawn using the dimensions in Table~\ref{table:coil} and thickness of the coil is 0.1 cm. Similarly, the perturbation coil has been drawn which is 106 cm away from the center of origin of the drawn prototype shield.  All the information has been written in a file format known as 'comi' file which acts as command language to run OPERA. Another comi file has been made where the shield has been replaced with air for the prototype geometry with no shield. The two 'comi' files while running individually in OPERA produce prototype geometry which has been shown with outermost passive shielding layer in Fig.~\ref{fig:opera_setup}\textcolor{blue}{(a)} and that without any passive shielding layer in Fig.~\ref{fig:opera_setup}\textcolor{blue}{(b)} respectively. The dots in Fig.~\ref{fig:opera_setup}\textcolor{blue}{(b)} represents some fluxgate positions.



\FloatBarrier

% After getting the prototype geometry in OPERA, the next is to discuss the simulation made to calculate the matrix elements and thus produce the matrix $\bm{M}$.

\subsubsection{Simulation of Field Map With Current in Compensation Coils}\label{sec:mSim}

In Section~\ref{sec:m}, matrix elements $M_{sc}$ are measured  to produce $\bm{M}$ which relates coil currents with the magnetic field. Inversion of the matrix produce current error (Eq.~(\ref{eq:del_I})) that later goes to PI control algorithm (Eq.~(\ref{eq:I})) to find new current that should be fed to compensate the drift in field. For matrix simulation, zero magnetic field is applied so that six current generated field map in T/A for six coils in the cube can be generated with current being supplied to the the coils one at a time with others having zero current. The process is discussed here. 

A block shaped background region 4 times the scale in $x$, $y$ and $z$ axis without change in the symmetry has been set in the same 'comi' file where prototype geometry has been written. To generate surface mesh, the maximum mesh element size is chosen to be 10 with maximum angle between elements is 10 degree. The maximum deviation from surface is 0 and the absolute tolerance used to check point coincidence is $\mathrm{1\times10^{-8}}$. The mesh type has been preferred to be mosaic. To generate the volume mesh the absolute tolerance is chosen to be $\mathrm{1\times10^{-8}}$. After meshing the geometry we modified the coil $C_x^+$ by introducing to 1 A/ Area, where area is determined by the OPERA based on the dimensions provided for that coil earlier. For analyzing the data, we used TOSCA Magnetostatic module of OPERA and set numerical solution convergence tolerance to be $\mathrm{1\times10^{-13}}$ and magnetic field $\bm{H}$ to be zero and the model was solved by OPERA. Once the solution has completed, in the Post-Processor of OPERA we set the unit of the magnetic fields to be Tesla and used the GRID command  which evaluates the fields over a 3D grid. The grid started from -62 cm in  $x$, $y$ and $z$ co-ordinates with an increment of 1 cm and ends at 62 cm which covers the full cube dimensions. The number of points in each of the co-ordinate are 125. The output of the GRID are 6 columns where first 3 columns indicate the the points in $x$, $y$ and $z$ co-ordinates and last 3 columns indicate the corresponding magnetic fields in Bx, By and Bz respectively and are stored in a 'csv' file. The unit is T/A as there is no external field present. In this way 6 field maps are stored in 6 'csv' files for all the 6 coil sides.

% The background region defines the sa 'comi' file is written where the magnetic field contribution due to environment is set to zero as  matrix elements $M_{sc}$ relate the coils current to the field produced by the coils current (see Eq.~(\ref{eq:B_coils})). Now as matrix elements $M_{sc}$ are in the unit of nT/A, in the same 'comi' file current for the all the six coils have set to 1A which makes the Eq.~(\ref{eq:B_coils}) as

% \begin{equation}\label{eq:B_coils_1A}
%     B_s(\text{coils})=\sum_c^{c=1-6} M_{sc}
% \end{equation}
% where, $n$=1.

% Now the 'comi' file for the above setup will interact with the 'comi' file of the prototype geometry and then by using the solver of OPERA, $B_s(\text{coils})$ i.e. $M_{sc}$ elements will be calculated.

% \renewcommand{\labelitemi}{$\blacksquare$}
% \begin{itemize}
%     \item[$\blacksquare$] \textbf{Creation of Matrix M}
% \end{itemize}

% The advantage of simulation is that we have got 6 field maps for 6 coil sides which we can use to generate any dimensions of matrix $\bm{M}$ within the cube dimensions. For generating the matrix, we used the Python programming language where the fields maps are acted as input. We convert the fields maps into nT/A and depending on the co-ordinates of the points provided by us, the matrix is generated in the output of the Python code. Note that the points must be from -62 cm to 62 cm with an increment of 1 cm in each of the $x$, $y$ and $z$ co-ordinate.


% % The best thing about simulating in this setting is that the OPERA has calculated the $M_{sc}$ elements for the whole volume of the prototype geometry which can be easily exported in a 'csv' file by running a simple 'comi' file for volume extraction. Thus the $M_{sc}$ elements of the whole prototype geometry have been stored in a 'csv' file in T/A. Then, a code has been written in the Python programming language to clean, modify and sort the data of that 'csv' file and convert the data in nT/A and  provide the co-ordinates of the points from where the magnetic field values i.e.  $M_{sc}$ elements (see Eq.~(\ref{eq:B_coils_1A})) are determined.

% % And the current in the coils are related to the magnetic filed with those $M_{sc}$ elements as given in Eq.~(\ref{eq:B_coils}). It is very easy to extract magnetic field information by providing the co-ordinates vale in OPERA. In addition to the field produced by the coil currents, there is also  values of $M_{sc}$. That means, setting current, I=1 A and offset =0 in all six coil will produce 
% % \fig{Images/noShield}{width =\textwidth,height=14 cm}{Prototype setup in OPERA simulation software without passive shielding layers. \label{fig:noShield_opera}}

% For comparison with the experimental $\bm{M}$, the simulation $\bm{M}$ has been made by choosing co-ordinates of the points as same as the sensor positions in the experiment. Then the absolute differences between the experimental $M_{sc}$ elements with that of the simulation ones  are calculated as

% \begin{equation}
%     |M_{sc}^{\text{diff}}|=|(|M_{sc}^{\text{exp}}|-|M_{sc}^{\text{sim}}|)|
% \end{equation}
% where, $|M_{sc}^{\text{exp}}|$ and $|M_{sc}^{\text{sim}}|$ indicate the absolute values of $M_{sc}$ elements both in experiment and simulation respectively.
% The absolute differences $|M_{sc}^{\text{diff}}|$ have been shown in Fig. \ref{fig:Mdiff}. It is seen that most of $|M_{sc}^{\text{diff}}|$ differences are within 100 nT/A. But that in some of the positions the differences are very big. The reason is that it is very difficult to exactly pinpoint the locations both in experiment and simulation and also the simulation runs in ideal conditions where the experiment suffers from the condition of its surrounding. The differences can be made smaller by choosing different co-ordinates of the sensor positions. We are not that concerned about that because the $M_{sc}$ element due to a particular coil and sensor position has similar response pattern both in experiment and simulation. 

% \fig{Images/Mdiff_56}{width = \textwidth}{The absolute differences between the  experimental $M_{sc}$ elements with that of simulation in nT/A. Horizontal axis indicates the various sensors, which are counted using the index $s$. Vertical axis indicates the various coils, which are counted using the index $c$. The value of the elements are increasing from dark blue to red. For position of the sensors and coils please see Fig.~\ref{fig: coil}. \label{fig:Mdiff}}{Comparison of experimental $\bm{M}$ with simulation one.}

% \FloatBarrier
% So, we have made simulation which can produce $M_{sc}$ elements for any number of the sensors placed in the prototype geometry. This will be a tool for the future students who are limited by sensor position and placing them correctly in their system to study in detail about the effect of $\bm{M}$ for any set of sensors.


% \subsection{Simulation of Magnetic Field Distribution}\label{sec:pSim}

% The main aim of active compensation is to compensate the drift that generates due to the difference between the setpoint and the actual measurement. The measurement of the field by fluxgates surrounding the prototype without any compensation system running are given in Table~\ref{table:Benvironment} which we have used to generate two field maps one with no current in any coils and another with current in the perturbation coil. The concept is that without any perturbation, the field is always same. So, the field map with no current in any coil will generate the setpoint i.e. $B_s(\text{setpoint})$ and the field map with current in perturbation coilwill generate the actual measurement i.e. $B_s(\text{measure})$. The difference between $B_s(\text{measure})$ and $B_s(\text{setpoint})$ will generate drift in field (Eq.~(\ref{eq:del_B})).

% and lastly how the perturbation affects the current in all six coil sides will be discussed. 

\subsubsection{Simulation of Field Map Without Current}

The main aim of active compensation is to compensate the drift that generates due to the difference between the setpoint and the actual measurement. The measurement of the field by fluxgates surrounding the prototype without any compensation system running are given in Table~\ref{table:Benvironment} which we have used to generate two field maps one with no current in any coils and another with current in the perturbation coil. The concept is that without any perturbation, the field is always same. So, the field map with no current in any coil will generate the setpoint i.e. $B_s(\text{setpoint})$ and the field map with current in perturbation coilwill generate the actual measurement i.e. $B_s(\text{measure})$. The difference between $B_s(\text{measure})$ and $B_s(\text{setpoint})$ will generate drift in field (Eq.~(\ref{eq:del_B})).
 
% The magnetic field distribution without any current in the coils that is $B_s(\text{setpoint})$ will be determined by simulating the prototype in OPERA.
% The 770 mA is same as providing 10 mA in the experimental perturbation coil as the experimental one has 77 turns
We used the same meshing as described earlier. After meshing the geometry, the current density on the the perturbation coil $P_z^+$ was set to 0 mA/area, where area is determined by the OPERA based on the dimensions provided for $P_z^+$ coil earlier. For analyzing the data, keeping all the settings same as earlier , magnetic field $\bm{H}$ were set according to Table~\ref{table:Benvironment} and the model was solved by OPERA. Once the solution has completed, in the Post-Processor of OPERA, as earlier we set the unit of the fields to be Tesla. Using GRID command in OPERA, the the fields over the cube dimensions starting from -62 cm in  $x$, $y$ and $z$ co-ordinates with an increment of 1 cm and ends at 62 cm have been determined as earlier and stored in a 'csv' file. The 'csv' file is the field map over the cube dimensions with no current in any coil in presence of an external field.

% \begin{itemize}
%     \item[$\blacksquare$] \textbf{Creation of Setpoints}
% \end{itemize}

% For generating setpoints, we have used the Python programming language where the fields map is acted as input. We convert the fields maps into nT as to match the experimental unit  and depending on the co-ordinates of the points provided by us, the $B_s(\text{setpoint})$ are generated in the output of the Python code. Note that the points must be from -62 cm to 62 cm with an increment of 1 cm in each of the $x$, $y$ and $z$ co-ordinate.

% While comparing experimentally measured setpoints with that determined by simulation we have found that, the values of Bx, By and Bz are differ by $\sim$2000 nT but they have the same signs with same order of magnitude in $x$, $y$ and $z$ co-ordinate. The different values can be due to not match the coordinate points exactly. 


% As long as the signs of Bx, By and Bz are same both in simulation and experiment, the difference is no issues as the main aim is to find the drift in the magnetic field.

\subsubsection{Simulation of Field Map With Current in Perturbation Coil}

To field for a particular perturbation similar steps have been taken as above except in this case, the current density on the perturbation coil $P_z^+$ was set to 770 mA/area. The reason of choosing 770 mA current in the perturbation coil is that in the experiment, 10 mA current has been provided to the perturbation coil with 77 turns. For analyzing the data, keeping all the settings same as earlier , magnetic field $\bm{H}$ were set according to Table~\ref{table:Benvironment} and the model was solved by OPERA. Once the solution has completed, in the Post-Processor of OPERA, as earlier we set the unit of the fields to be Tesla. Using GRID command in OPERA, the the fields over the cube dimensions starting from -62 cm in  $x$, $y$ and $z$ co-ordinates with an increment of 1 cm and ends at 62 cm have been determined as earlier and stored in a 'csv' file. The 'csv' file is the field map over the cube dimensions with current in perturbation coil in presence of an external field.

% \begin{itemize}
%     \item[$\blacksquare$] \textbf{Creation of Measurements}
% \end{itemize}

% For generating measurements, we have used the Python programming language where the fields map is acted as input. We convert the fields maps into nT as to match the experimental unit  and depending on the co-ordinates of the points provided by us, the measurements $B_s(\text{measure})$ are generated in the output of the Python code. Note that the points must be from -62 cm to 62 cm with an increment of 1 cm in each of the $x$, $y$ and $z$ co-ordinate.

% For comparing the simulation result with experimental one, the effect of the perturbation has been found in terms of change in the coil currents in all six coils both in simulation and experiment. For the simulation, in the above, $B_s(\text{setpoint})$ and $B_s(\text{measure})$ have been determined by choosing co-ordinates of the points as same as the sensor positions in the experiment from which the change in the signal i.e. $\Delta B_s$ have been determined using Eq.~(\ref{eq:del_B}). Finally, the change in the coils current i.e. the errors in the coils have been determined by multiplying $\Delta B_s$ with regularized pseudo-inverse of $\bm{M(\text{sim})}$ obtained from Section~\ref{sec:mSim} (see Eq.~(\ref{eq:del_I}) and Section~\ref{sec:inv}). For experiment the changes in the fluxgate signals are determined by applying 10 mA in the perturbation  coil and then find $\Delta B_s$ using Eq.~(\ref{eq:del_B}) and eventually find the change in coil currents in the same way as done for the simulation. The change in current in all six coil sides found due to experiment and simulation are given in Table~\ref{table:currentChange} where the simulation results are quite similar to the experimental ones for all six coils sides.

% \begin{table} [htb!]
%     \centering
%     \begin{tabular} { |c|c|c|c|c|c|} 
%         \hline
%         Coils & \makecell{Change in Current (mA)\\Experiment} & \makecell{Change in Current (mA)\\Simulation}\\
%         \hline\hline
%         $C_x^-$ & -26 & -32 \\ 
%         \hline
%         $C_x^+$ & 22 & 20 \\
%         \hline
%         $C_y^-$ & -30 & -24 \\
%         \hline
%         $C_y^+$ & 24 & 17 \\
%         \hline
%         $C_z^-$ & -14 & -13 \\
%         \hline
%         $C_z^+$ & 54 & 46 \\
%         \hline

%     \end{tabular}
%     % \vspace{4mm}
%     \caption[Comparison of change in current in experiment and simulation.]{Comparison of change in current in different coil sides at the time of applying perturbation both in experiment and simulation.}\label{table:currentChange}
% \end{table}

% \FloatBarrier
% In the above, simulation made via OPERA to find $B_s(\text{setpoint})$ and $B_s(\text{measure})$ have been used to find the change in current in all six coil sides and matched with experimental results. 

In the upcoming Section the field maps obtained due to various simulations in OPERA will be used to generate matrix and drift in signals which will then integrate in PI control algorithm to make a fully functional PI simulation.
% The values obtained due to this have been compared with the values due to experimental setup and the results are similar as shown in Fig. \ref{fig:noShield_cSim}.

% \fig{Images/cSim}{width =\textwidth, height=14cm}{Comparison of change in current in different Coil sides at the time of applying perturbation both in experiment and simulation. \label{fig:noShield_cSim}}

\subsection{Field Maps Processing in Python}

Three field maps are generated in OPERA which have been discussed earlier. Python programming language is used to generate matrix M and drift in field from those field maps. The generated matrix and drift in field are then used in a PI control algorithm in time to make a PI simulation in time.


\subsubsection{M Matrix Generation}
There are 6 field maps generated for currents in all six coils where each of the six field maps represents the values associated with the current bearing coil.  
The advantage of simulation is that which we can use 6 field maps to generate any dimensions of matrix $\bm{M}$ within the cube dimensions. For generating the matrix, we used the Python programming language where the fields maps are acted as input. We convert the fields maps into nT/A to match the experimental unit and depending on the co-ordinates of the points provided by us, the matrix is generated in the output of the Python code. Note that the points must be from -62 cm to 62 cm with an increment of 1 cm in each of the $x$, $y$ and $z$ co-ordinate.


% The best thing about simulating in this setting is that the OPERA has calculated the $M_{sc}$ elements for the whole volume of the prototype geometry which can be easily exported in a 'csv' file by running a simple 'comi' file for volume extraction. Thus the $M_{sc}$ elements of the whole prototype geometry have been stored in a 'csv' file in T/A. Then, a code has been written in the Python programming language to clean, modify and sort the data of that 'csv' file and convert the data in nT/A and  provide the co-ordinates of the points from where the magnetic field values i.e.  $M_{sc}$ elements (see Eq.~(\ref{eq:B_coils_1A})) are determined.

% And the current in the coils are related to the magnetic filed with those $M_{sc}$ elements as given in Eq.~(\ref{eq:B_coils}). It is very easy to extract magnetic field information by providing the co-ordinates vale in OPERA. In addition to the field produced by the coil currents, there is also  values of $M_{sc}$. That means, setting current, I=1 A and offset =0 in all six coil will produce 
% \fig{Images/noShield}{width =\textwidth,height=14 cm}{Prototype setup in OPERA simulation software without passive shielding layers. \label{fig:noShield_opera}}

For comparison with the experimental $\bm{M}$, the simulation $\bm{M}$ has been made by choosing co-ordinates of the points as same as the sensor positions in the experiment. Then the absolute differences between the experimental $M_{sc}$ elements with that of the simulation ones  are calculated as

\begin{equation}
    |M_{sc}^{\text{diff}}|=|(|M_{sc}^{\text{exp}}|-|M_{sc}^{\text{sim}}|)|
\end{equation}
where, $|M_{sc}^{\text{exp}}|$ and $|M_{sc}^{\text{sim}}|$ indicate the absolute values of $M_{sc}$ elements both in experiment and simulation respectively.

\fig{Images/Mdiff_56}{width = \textwidth}{The absolute differences between the  experimental $M_{sc}$ elements with that of simulation in nT/A. Horizontal axis indicates the various sensors, which are counted using the index $s$. Vertical axis indicates the various coils, which are counted using the index $c$. The value of the elements are increasing from dark blue to red. For position of the sensors and coils please see Fig.~\ref{fig: coil}. \label{fig:Mdiff}}{Comparison of experimental $\bm{M}$ with simulation one.}



The absolute differences $|M_{sc}^{\text{diff}}|$ have been shown in Fig. \ref{fig:Mdiff}. It is seen that most of $|M_{sc}^{\text{diff}}|$ differences are within 100 nT/A. But that in some of the positions the differences are very big. The reason is that it is very difficult to exactly pinpoint the locations both in experiment and simulation and also the simulation runs in ideal conditions where the experiment suffers from the condition of its surrounding. The differences can be made smaller by choosing different co-ordinates of the sensor positions. We are not that concerned about that because the $M_{sc}$ element due to a particular coil and sensor position has similar response pattern both in experiment and simulation. 


\FloatBarrier
\subsubsection{Change in B Field Generation}

For generating change in B field i.e. $\Delta$B field, difference between the setpoint and actual measurement is required (Eq.~(\ref{eq:del_B})). 

For generating setpoint, we have used the Python programming language where the field map generated due to no current in any coils with external field applied as discussed earlier is acted as input. We convert the field map into nT to match the experimental unit and depending on the co-ordinates of the points provided by us, the $B_s(\text{setpoint})$ are generated in the output of the Python code. 

Similarly, for generating actual measurement, we have again used the Python programming language where the field map generated due to current in perturbation coil with external field applied as discussed earlier is acted as input. We convert the field map again into nT to match the experimental unit and depending on the co-ordinates of the points provided by us, the measurements $B_s(\text{measure})$ are generated in the output of the Python code. Note that the points must be from -62 cm to 62 cm with an increment of 1 cm in each of the $x$, $y$ and $z$ co-ordinate in both case.


The difference between $B_s(\text{setpoint})$ and $B_s(\text{measure})$ will then produce $\Delta$B field. While comparing experimentally measured $\Delta$B field with that determined by simulation we have found that, the values are similar except some sign issues which we fixed by comparing with experimental values. 

\subsection{PI Simulation in Python}

Matrix $\bm{M(\text{sim})}$ and $\Delta$ B have been generated in Python from the simulated field maps discussed earlier. Errors in the coil currents (Eq.~(\ref{eq:del_I})) are generated using $\Delta$ B and regularized pseudoinverse of $\bm{M(\text{sim})}$ discussed in Section~\ref{sec:inv}. The value of $k_c^p$ and $k_c^i$ has been obtained by following the concept of PI tuning as discussed in Section~\ref{sec:tune}. After getting the tuned value of $k_c^p$ and $k_c^i$, the new current that should be fed to the coils for compensating the magnetic field drift is calculated using Eq.~(\ref{eq:I}). Moreover, in Section~\ref{sec:tune}, the time difference between two consecutive feedback loop is found to be 0.146 s. To match the same time of completing a feedback loop both in simulation and experiment, we have added a time delay of 0.146 s in the PI simulation code. After adding the time delay, The same process have been repeated for a time duration. Thus, a PI simulation in time has been made. 

For comparing experimental PI result with that simulation one, Matrix $\bm{M(\text{sim})}$ and $\Delta$ B have been generated in Python at the same position of the experimental fluxgates. Then the process discussed eralier has been used to make a PI simulation. The comparison is shown in Fig.~\ref{fig:exp_sim} where the currents (left) that are being sent to the coils ($C_x^\pm$, $C_y^\pm$ and $C_z^\pm$) for drift $\Delta$B (right) found by Eq.~(\ref{eq:del_B}) in fluxgate positions 1, 3, 6 and 8 excluding two control sensors in (a) experiment and (b). It is seen that the coil currents (left) and $\Delta$B (right) graphs have similar pattern. It confirms that the current unsettle problem being seen in the experiment is real.

\fig{Images/exp_sim}{width = \textwidth,height=14cm}{Currents (left vertical axis) in all six coil sides ($C_x^\pm$, $C_y^\pm$ and $C_z^\pm$) with drift $\Delta$B (right vertical axis) at 12 sensor positions with fluxgate positions 1, 3, 6 and 8 for $k_c^p$=0.0 and $k_c^i$=1.0 (see Eq.~(\ref{eq:I})). For position of coils and fluxgates see Fig.~\ref{fig: coil}. All grey color curves in $\Delta$B graph denote the drift in signal that would have been without the compensation while all other color curves denote the actual drift in signal at all 12 sensor positions found by Eq.~(\ref{eq:del_B}). The 'ON' and 'OFF' vertical dashed lines indicate the time of the perturbation coil being turned 'ON' and 'OFF' respectively. The current supplied on perturbation coil was 10  mA. The resolution index 1 (see Table~\ref{table:t7freq}) with 50 no of averages per measurement and $r$=3.1 were used. The loop sampling frequency was found to be $\sim$6.5 Hz.\label{fig:exp_sim}}{Coil currents with drift $\Delta$B for $k_c^p$=0.0, $k_c^i$=1.0 and $r$=3.1}


\FloatBarrier
The magnetic field compensation due to PI feedback loop and the currents in all the six coil sides results from experiment and simulation are quite similar. Next we show new ways to deal with the current unsettle problem. 

% The uncompensated magnetic field is the magnetic field without the compensation effect and predicted by
% \begin{equation}\label{eq:Buncomp}
%     B_s^n(\text{uncompensated})=B_s^n(\text{measured})- B_s^n(\text{coils})
% \end{equation}
% where, $B_s^n(\text{measured})$ is the actual measurement from coming the fluxgate sensors in experiment or determined using simulation (see Section~\ref{sec:pSim}) with $n$=(1,...N) is the no of measurements taken and $B_s^n(coils)$ is determined using Eq.~(\ref{eq:B_coils}).

% \FloatBarrier
% \doublefig{Images/pi_exp_all}{width =\textwidth, height=10cm}{Experiment\label{fig:pi_exp_all}}{Images/pi_sim_all}{width = \textwidth,height=10cm}{Simulation\label{fig:pi_sim_all}}{{PI active magnetic field compensation results by both Experiment (a) and Simulation (b). Vertical axis represents the $\Delta B_s$ (see Eq~(\ref{eq:del_B})) and the horizontal axis represent the time. All the colors has been discussed in the text. The 'ON' and 'OFF' vertical dashed lines indicate the time of the perturbation coil being turned 'ON' and 'OFF' respectively. } \label{fig:pi_ExpSim_all}}{PI active magnetic field compensation results by both experiment and simulation.}






% Again, the same result has been shown by separating $x$, $y$ and $z$ respectively in Fig. \ref{fig:pi_exp} obtained by experiment and in Fig. \ref{fig:pi_sim} due to simulation.

% \fig{Images/pi_exp}{width =\textwidth,height=14cm}{PI Active Magnetic Field Compensation Results in $x$, $y$ and $z$ axis by both Experiment. \label{fig:pi_exp}}

% \fig{Images/pi_sim}{width =\textwidth,height=14cm}{PI Active Magnetic Field Compensation Results in $x$, $y$ and $z$ axis by both Simulation. \label{fig:pi_sim}}

\section{New Studies on Regularization Parameter}\label{sec:new_study_r}

In Section~\ref{sec:inv} we have introduced the regularization parameter $r$ and then discussed regularization by random fluctuations in Section~\ref{sec:mont} which is based on the previous study from Ref.~\cite{bea}. We have also talked about the PI behaviour in Section~\ref{sec:pi_behave}. The effect of $r$ on current unsettle problem will be discussed here. The current unsettle problem will be reduced by looking from the perspective of changing $r$. We will also propose a new method of regularization based on condition number of matrix. 

% So, first the importance of matrix condition number on PI will be discussed.

% \subsection{Importance of Matrix Condition Number on PI}\label{sec:pi_behave_m}

% \begin{itemize}
% \item Consider deleting this section in favor of 5.4.2 Effect of r on PI tuning.
% \end{itemize}

% Matrix Condition number ( see Eq.~(\ref{eq:cond} ) have been introduced in Section.~\ref{sec:m} while discussing the inversion of the matrix $\bm{M}$ . An ill-conditioned matrix has large condition number and a matrix with small condition number is said to be well-conditioned. Here, the effect of matrix condition number on PI will be discussed in terms of a very large condition number that is 129.
 
% \subsubsection{Condition Number Effect on changing only P term}
% Here, the effect of changing proportional gain term (P) or $k_c^p$ of Eq.~(\ref{eq:I}) will be discussed for a matrix condition number of 129 which is very ill-conditioned compared to 33 discussed in Section~\ref{sec:pi_behave}.

% P term is proportionally multiplying the error (the difference between setpoint and actual measurement) with a constant gain and discussed more in Section~\ref{sec:pi_behave}.


% \begin{figure}[!htb]
%     \begin{subfigure}{.5\linewidth}
%         \centering
%         \includegraphics[width=\linewidth, height= 6.5 cm]{Images/p25}
%         \caption{at $k_c^p$=0.25}
%         \label{fig:p25m}
%     \end{subfigure}%
%     \begin{subfigure}{.5\linewidth}
%         \centering
%         \includegraphics[width=\linewidth, height= 6.5 cm]{Images/p50}
%         \caption{at $k_c^p$=0.50}
%         \label{fig:p50m}
%     \end{subfigure}\\[1ex]
%     \begin{subfigure}{.5\linewidth}
%         \centering
%         \includegraphics[width=\linewidth, height= 6.5 cm]{Images/p75}
%         \caption{at $k_c^p$=0.75}
%         \label{fig:p75m}
%     \end{subfigure}%
%         \begin{subfigure}{.5\linewidth}
%         \centering
%         \includegraphics[width=\linewidth, height= 6.5 cm]{Images/p100}
%         \caption{at $k_c^p$=1.0}
%         \label{fig:p100m}
%     \end{subfigure}

%     \caption[short]{Currents (left vertical axis) in all six coil sides ($C_x^\pm$, $C_y^\pm$ and $C_z^\pm$) with drift $\Delta$B (right vertical axis) at sensor position '1x' for different values of $k_c^p$ with $k_c^i$ in Eq.~(\ref{eq:I}) being zero. Blue color curve denotes the actual drift in signal at position '1x' found by Eq.~(\ref{eq:del_B}) while the red curve denotes the drift that would have been without the compensation. The 'ON' and 'OFF' vertical dashed lines indicate the time of the perturbation coil being turned 'ON' and 'OFF' respectively. For position of coils and sensors see Fig.~\ref{fig: coil}. }
%     \label{fig:p_pi_m}
% \end{figure}

% The effect of changing $k_c^p$ has been shown in Fig.~\ref{fig:p_pi_m} where the currents (left) that are being sent to the coils ($C_x^\pm$, $C_y^\pm$ and $C_z^\pm$) for drift $\Delta$B found by Eq.~(\ref{eq:del_B}) in sensor position '1x'.  It is seen that $\Delta$B=23 nT, 18.5 nT and 16.5 nT for $k_c^p$ = 0.25, 0.5 and 0.75 respectively (see Fig.~\ref{fig:p_pi_m}\textcolor{blue}{(a)}, Fig.~\ref{fig:p_pi_m}\textcolor{blue}{(b)}, Fig.~\ref{fig:p_pi_m}\textcolor{blue}{(c)}). That is, with the increase of $k_c^p$, $\Delta$B magnetic field decreases. But, it has a limit after which with the increase of $k_c^p$, the systems becomes unstable and starts oscillating which can be seen from Fig.~\ref{fig:p_pi_m}\textcolor{blue}{(d)}) where the currents (left) are oscillating and the drift itself also at $\Delta$B=14.5 nT (right). So, the error is reduced maximum by (29.5-14.5)/29.5 * 100$\%\approx$51$\%$ from the initial drift of $\Delta$B=29.5 nT denoted by the red curve at position '1x'. 

% \FloatBarrier
% The above results confirm that matrix condition number has minimal effect on P term by comparing the effect shown in Section~\ref{sec:pi_behave}. Next, we will discuss about the effect on only I term.

% \subsubsection{Condition Number Effect on Changing only I term}
% Here, the effect matrix condition number effect on changing integral reset term (I) or $k_c^i$ of Eq.~(\ref{eq:I}) will be discussed.

% The error (the difference between setpoint and actual measurement) is accumulated for the length of measurements and I term is multiplying that accumulated error  with a constant gain as discussed in Section~\ref{sec:pi_behave}.




% As like the effect on P, the effect of changing $k_c^i$ has been shown in Fig.~\ref{fig:i_pi_m} where the currents (left) that are being sent to the coils ($C_x^\pm$, $C_y^\pm$ and $C_z^\pm$) for drift $\Delta$B found by Eq.~(\ref{eq:del_B}) in sensor position '1x' for $k_c^i$=0.25, 0.5, 0.75 and 1.0.  It is seen from Fig.~\ref{fig:i_pi_m}\textcolor{blue}{(a)}, Fig.~\ref{fig:i_pi_m}\textcolor{blue}{(b)}, Fig.~\ref{fig:i_pi_m}\textcolor{blue}{(c)} and Fig.~\ref{fig:i_pi_m}\textcolor{blue}{(d)}). which are correspond to $k_c^i$ = 0.25, 0.75, 1.0 and 1.25 respectively that the error level (right) is $\sim$3.5 nT in every case, the coil currents(left) never settle in any of them which is not he case for low matrix condition number as seen from Fig.~\ref{fig:i_pi}\textcolor{blue}{(a)}, Fig.~\ref{fig:i_pi}\textcolor{blue}{(b)}, Fig.~\ref{fig:i_pi}\textcolor{blue}{(c)} and Fig.~\ref{fig:i_pi}\textcolor{blue}{(d)}) where the condition number is 33 compared to 129 here.

% \begin{figure}[!htb]
%     \begin{subfigure}{.5\linewidth}
%         \centering
%         \includegraphics[width=\linewidth, height= 6.5 cm]{Images/i25}
%         \caption{at $k_c^i$=0.25}
%         \label{fig:i25m}
%     \end{subfigure}%
%     \begin{subfigure}{.5\linewidth}
%         \centering
%         \includegraphics[width=\linewidth, height= 6.5 cm]{Images/i50}
%         \caption{at $k_c^i$=0.5}
%         \label{fig:i50m}
%     \end{subfigure}\\[1ex]
%     \begin{subfigure}{.5\linewidth}
%         \centering
%         \includegraphics[width=\linewidth, height= 6.5 cm]{Images/i75}
%         \caption{at $k_c^i$=0.75}
%         \label{fig:i75m}
%     \end{subfigure}%
%         \begin{subfigure}{.5\linewidth}
%         \centering
%         \includegraphics[width=\linewidth, height= 6.5 cm]{Images/i100}
%         \caption{at $k_c^i$=1.0}
%         \label{fig:i100m}
%     \end{subfigure}

%     \caption[short]{Currents (left vertical axis) in all six coil sides ($C_x^\pm$, $C_y^\pm$ and $C_z^\pm$) with drift $\Delta$B (right vertical axis) at sensor position '1x' for different values of $k_c^i$ with $k_c^p$ ( see Eq.~(\ref{eq:I}) ) being zero. Blue color curve denotes the actual drift in signal at position '1x' found by Eq.~(\ref{eq:del_B}) while the red curve denotes the drift that would have been without the compensation. The 'ON' and 'OFF' vertical dashed lines indicate the time of the perturbation coil being turned 'ON' and 'OFF' respectively. For position of coils and sensors see Fig.~\ref{fig: coil}.}
%     \label{fig:i_pi_m}
% \end{figure}



% % \fig{Images/i_pi_zoom}{width = \textwidth,height =10cm}{Zoomed in version of the drift $\Delta$B shown in right side of Fig.~\ref{fig:i_pi}\textcolor{blue}{(a)}, Fig.~\ref{fig:i_pi}\textcolor{blue}{(b)}, Fig.~\ref{fig:i_pi}\textcolor{blue}{(c)} and Fig.~\ref{fig:i_pi}\textcolor{blue}{(d)} respectively at sensor position '1x' for different values of $k_c^i$ with $k_c^p$ ( see Eq.~(\ref{eq:I}) ) being zero. The red vertical dashed line indicates the time of the perturbation coil being turned 'ON'. For position of coils and sensors see Fig.~\ref{fig: coil}.\label{fig:i_pi_zoom}}


% \FloatBarrier
% The above results confirm the currents in the coils never settles in if there is a ill-conditioned matrix present.. Next the condition number effect on both both of them after tuning (see Section~\ref{sec:tune}) will be discussed and may be the currents settle there!!


% \subsection{r vs. Condition No.}\label{sec:cond}
% Instead of going through all the steps that are discussed in Section~\ref{sec:inv}, the concept of condition number of a matrix can be used. The condition number of $\bm{M}$ can be determined from the diagonal matrix $\bm{\Sigma}$ as given in eq.\ref{eq:m} by -
%  \begin{equation}
%      cond(\bm{M})=\frac{max(\sigma_d)}{min(\sigma_d)}
%  \end{equation}
 

% \subsubsection{Condition Number Effect of changing PI term combined}
% Finally, the Section~\ref{sec:pi_behave_m} will be ended here with the discussion of the condition number effect on changing P and I term at a time which will complete the Eq.~(\ref{eq:I}).

% Here, first the P and I term have been tuned following the discussion on Section~\ref{sec:tune} which has generated $k_c^p$=0.43 and $k_c^i$=0.52 . The results by applying $k_c^p$ and $k_c^i$ as those tuned values are shown in Fig.~\ref{fig:tuned_vs_i}\textcolor{blue}{(a)}. For simplicity instead of showing all the drift $\Delta$B for all the fluxgate sensors for the positions given in the horizonatal axis of Fig.~\ref{fig:m}, only '1x' is shown on the right of the figure. And same as earlier the currents  that are being sent to the coils ($C_x^\pm$, $C_y^\pm$ and $C_z^\pm$) shown on the left of the same figure. But, we couldn't determine the effect of having both of them at a time. So, keeping $k_c^i$ as 0.52 and excluding P term i.e. $k_c^p$=0.0 we run the same measurement again and the results are shwon in Fig.~\ref{fig:tuned_vs_i_m}\textcolor{blue}{(b)}. The results of Fig.~\ref{fig:tuned_vs_i_m}\textcolor{blue}{(a)} are similar to that of Fig.~\ref{fig:tuned_vs_i_m}\textcolor{blue}{(b)} which is also true for low condition number as discussed in Section~\ref{sec:pi_behave}. 

% \doublefig{Images/p43i52}{width =\textwidth,height =8cm}{at $k_c^p$=0.43 and $k_c^i$=0.52. \label{fig:pi_tuned_m}}{Images/i52}{width = \textwidth,height =8cm}{at $k_c^p$=0.0 and $k_c^i$=0.52..\label{fig:i52m}}{{Currents (left vertical axis) in all six coil sides ($C_x^\pm$, $C_y^\pm$ and $C_z^\pm$) with drift $\Delta$B (right vertical axis) at sensor position '1x' for combine different values of $k_c^i$ and $k_c^p$ ( see Eq.~(\ref{eq:I}) ). Blue color curve denotes the actual drift in signal at position '1x' found by Eq.~(\ref{eq:del_B}), while the red curve denotes the drift that would have been without the compensation. The 'ON' and 'OFF' vertical dashed lines indicate the time of the perturbation coil being turned 'ON' and 'OFF' respectively. For position of coils and fluxgate sensor see Fig.~\ref{fig: coil}.} \label{fig:tuned_vs_i_m}}{short}

% \FloatBarrier
% The above results rather confirm that matrix condition number does not bring any change while using P and  term at a time. But it has got huge impact on I term for which the coil current never settles in case of ill conditioned matrix. Next we will try to improve the current settling problem by changing the regularization parameter.

\subsection{Effect of $\mathbf{r}$ on PI Tuning}\label{sec:r_pi}


\begin{itemize}
\item We noticed that if we kept P \& I fixed, and changed only $r$, then we'd also get worse current drifting.  But paradoxically, the field would respond quite fast.
\item It is natural that r should affect PI, because the error function involves $\mathbf{M^{-1}}$ and therefore $r$.  See e.g.~Eq.~(4.10) for r and (4.3) and (4.4) for how Minv enters the PI algorithm.  If $r$ is big then Minv gets small.  So it would be reasonable to expect that if r increases then we need to crank up both kp and ki accordingly see eq. 4.4.
\item Consider showing a study where you change P\& I to try to compensate?
\item We tried this but it didn't work.  Data?  Or simulation?
\item Lower r seems better Fig.~5.13(a), but what I remember is that when we did that it made the B-noise worse.  Seems like an overcorrected system.  Setting P\& I lower to try to compensate the overcorrection I think made the system slower again (?)
\item Basic conclusion:  If r or cond is bad (too big) then no amount of PI tuning will help speed up the current drifting.  If r set falsely low, it also doesn't help because it induces HF noise and setting PI large again induces slow response.
\end{itemize}

The discussion in Section~\ref{sec:pi_behave} suggests that I term
is necessary for fast system response due to any drift in the magnetic
signal as it takes care of the offset problems which is unsolvable by
using only P term. But, in doing so, it also creates problems in terms
of the coil currents which never settle in for the duration of the
perturbation in case of ill-conditioned matrix. The tuning method
described in Section~\ref{sec:tune} should have take care of this but
we have realized that having tuned P and I has similar effect like
having only I term. So, tuning method doesn't give us the
solution. Rather looking for different tuning method , we have focused
on the effect of $r$ on PI tuning. This Section will discuss that
effect with possible outcome.


\fig{Images/rp_i}{width = \textwidth,height=8 cm}{Active magnetic field compensation on fluxgate position 1, 3, 6 and 8 for resolution index 1 with $k_c^p$=0.6, $r$=3.04 and different values of $k_c^i$ for 50 software average using filter. For simplicity only $C_z^+$ coil current (left) and magnetic compensation (right) on fluxgate sensor position '8x' have been shown. Vertical axis represent $\Delta$B (see Eq.~(\ref{eq:del_B})) due to '8x' with red represents uncompensated $\Delta$B. They are described in the text. The 'ON' and 'OFF' vertical dashed lines indicate the time of the perturbation coil being turned 'ON' and 'OFF' respectively. \label{fig:rp_i}}{Active magnetic field compensation for different values of $k_c^i$}


As we have realized that having tuned P and I terms have similar effect like
having only I term, we have studied the effect of different I term keeping the P term fixed to see whether the current unsettle problems goes away. The Fig.~\ref{fig:rp_i} shows the active compensation by fluxgate position 1, 3, 6 and 8 for with $k_c^p$=0.6, $r$=3.04 and different values of $k_c^i$ for 50 software averages respectively using filter in resolution index 1. For simplicity only $C_z^+$ coil current (left) and magnetic compensation (right) on fluxgate sensor position '8x' have been shown instead of showing all current coils and sensor positions. For a complete list of positions see the color map of $\bm{M}$ in Fig.~\ref{fig:m}.  The blue color denotes current (left) in coil $C_z^+$ to compensate magnetic drift on fluxgate sensor position '8x' for $k_c^p$=0.6, $r$=3.04 and $k_c^i$=0.0. It is seen that there is no current unsettle problem (left) and the magnetic drift (right) reduces from 25 nT to 15 nT which is expected from the discussion of the effect of changing only P term in Section~\ref{sec:pi_behave}. It is also seen that for all other colors the magnetic drift vanishes very fast as different values of I term are applied which is also expected from from the discussion of the effect of changing only I term in Section~\ref{sec:pi_behave} though we have used a fixed P term here which has minimal effect as discussed earlier. Our main concern is whether the current unsettle problems appear for changing I term and it is seen that as long as there is I term, the current unsettle problem appears. But the trend shows that if I term is in increasing order, then the current unsettle problem may disappear. We have also seen the effect of increasing I term in Section~\ref{sec:pi_behave} which suggests that increasing I term after a point makes the current unstable with creating overshoot problem. So, that can not be as solution to our original problem of current unsettle.

\FloatBarrier
\fig{Images/pi_r}{width = \textwidth,height=8 cm}{Active magnetic field compensation on fluxgate position 1, 3, 6 and 8 for resolution index 1 with $k_c^p$=0.6, $k_c^i$=0.37 and different values of $r$ for 50 software average using filter. For simplicity only $C_z^+$ coil current (left) and magnetic compensation (right) on fluxgate sensor position '8x' have been shown. Vertical axis represent $\Delta$B (see Eq.~(\ref{eq:del_B})) due to '8x' with red represents uncompensated $\Delta$B. They are described in the text. The 'ON' and 'OFF' vertical dashed lines indicate the time of the perturbation coil being turned 'ON' and 'OFF' respectively. \label{fig:pi_r}}{Active magnetic field compensation for different values of $r$}


 
As increasing I term is not the solution for the current unsettle problem, we have then tried changing $r$ keeping P and I terms fixed. The Fig.~\ref{fig:pi_r} shows the active compensation for same fluxgate positions and same settings as in Fig.~\ref{fig:rp_i} except in this case we have chosen $k_c^p$=0.6, $k_c^i$=0.37 and different values of $r$ . It is seen that if we kept P and I term fixed, and changed only $r$, then for decreasing $r$ , we get the current unsettle problem goes away but doing such introduces high frequency noise. It is natural that $r$ should affect PI, because the error function involves $\mathbf{M^{-1}}$ and therefore $r$.  See e.g.~Eq.~(\ref{eq:minvR}) for r and Eq.~(\ref{eq:del_I}) and Eq.~(\ref{eq:I}) for how $\mathbf{M^{-1}}$ enters the PI algorithm.  If $r$ is big then $\mathbf{M^{-1}}$ gets small.  So it would be reasonable to expect that if r increases then we need to increase both $k_c^p$ and $k_c^i$ accordingly (see Eq.~(\ref{eq:I})).



\FloatBarrier

\fig{Images/p60i36r24}{width = \textwidth,height=7cm}{Currents (left vertical axis) in all six coil sides ($C_x^\pm$, $C_y^\pm$ and $C_z^\pm$) with drift $\Delta$B (right vertical axis) at 12 sensor positions with fluxgate positions 1, 3, 6 and 8 for $k_c^p$=.60, $k_c^i$=0.37 (see Eq.~(\ref{eq:I})) and $r$=2.4. For position of coils and fluxgates see Fig.~\ref{fig: coil}. All grey color curves in $\Delta$B graph denote the drift in signal that would have been without the compensation while all other color curves denote the actual drift in signal at all 12 sensor positions found by Eq.~(\ref{eq:del_B}). The 'ON' and 'OFF' vertical dashed lines indicate the time of the perturbation coil being turned 'ON' and 'OFF' respectively. The current supplied on perturbation coil was 10  mA. The resolution index 1 (see Table~\ref{table:t7freq}) with 50 no of averages per measurement used. The loop sampling frequency was found to be $\sim$6.5 Hz.\label{fig:pi_tuned_r24}}{Coil currents with drift $\Delta$B for $k_c^p$=0.60, $k_c^i$=0.37 and $r$=2.4}

The Fig.~\ref{fig:pi_i_mCond30}\textcolor{blue}{(a)} has been regenerated using lower value of $r$ to show its effect on all the fluxgate sensors and shown in in Fig.~\ref{fig:pi_tuned_r24} where the currents (left) that are being sent to the coils ($C_x^\pm$, $C_y^\pm$ and $C_z^\pm$) for drift $\Delta$B (right) found by Eq.~(\ref{eq:del_B}). It is seen the current unsettle problems in all the coil sides are disappeared but high frequency noises are induced.


\FloatBarrier
 
 The above results show that if $r$ or condition number is bad (too big) then no amount of PI tuning will help speed up the current drifting.  If $r$ set falsely low, it also doesn't help because it induces high frequency noises and setting PI large again induces slow response. Next we will talk about a new method of regularization based on condition number.
 



% \begin{itemize}
%     \item message from Fig.~\ref{fig:exp_sim}??
% \end{itemize}

% The experimental setup is same as discussed in Section~\ref{sec:pi_behave_m}. But in this case we have chosen $k_c^p$=0 and $k_c^i$=0.52. Among those $k_c^i$=0.52 has been found due to PI tuning (see Section~\ref{sec:tune}) and instead of choosing  $k_c^p$=0.43 we have made this zero as from the earlier discussion we saw that it barely has any effect while we use the I term. So, these values of $k_c^p$ and $k_c^i$ will be applied on Eq.~\ref{eq:I} to find the currents to be sent to the coils ($C_x^\pm$, $C_y^\pm$ and $C_z^\pm$) for drift $\Delta$B found by Eq.~(\ref{eq:del_B}) in the sensor positions given in the horizontal axis of Fig.~\ref{fig:m}. So, keeping those fixed, we will try to change the value of $r$ which will modify the Eq.~(\ref{eq:minvR}) for each change of $r$ value.

% The effect of changing $r$ with $k_c^p$=0 and $k_c^i$=0.52 has been shown in Fig.~\ref{fig:r_pi} where the currents (left) that are being sent to the coils ($C_x^\pm$, $C_y^\pm$ and $C_z^\pm$) for drift $\Delta$B found by Eq.~(\ref{eq:del_B}) in sensor position '1x'.  It is seen from Fig.~\ref{fig:r_pi}\textcolor{blue}{(a)}, Fig.~\ref{fig:r_pi}\textcolor{blue}{(b)}, Fig.~\ref{fig:r_pi}\textcolor{blue}{(c)} and Fig.~\ref{fig:r_pi}\textcolor{blue}{(d)} which are correspond to $r$ = 2.0, 2.4, 2.8 and 3.2 respectively that the changing $r$ has significant effect on the coil current graph and barely any effect on the system response time for reducing the drift in the signal. That is at $r$=2.0, the coil current graph has the fastest settling time where the current settles within 3 s after the perturbation has been applied. At $r$=2.4, it takes 10s for the coil currents to settle in. But at $r$=2.8, it seems like the coil current never settles in which is again improved at $r$=3.2. Note that the here seriously ill conditioned matrix with condtion number 129 has been used and optimized $r$ found by the simulation model is $\sim$2.9 which tells us that the coil settling of the current graph seems to have issue with the that optimized 'r for ill-conditioned matrix. So, instead of taking the optimized $r$ that has been found by the simulation model we may have to choose the lower value of $r$. 

% \fig{Images/exp_sim}{width = \textwidth,height=14cm}{Currents (left vertical axis) in all six coil sides ($C_x^\pm$, $C_y^\pm$ and $C_z^\pm$) with drift $\Delta$B (right vertical axis) at 12 sensor positions with fluxgate positions 1, 3, 6 and 8 for $k_c^p$=0.0 and $k_c^i$=1.0 (see Eq.~(\ref{eq:I})). For position of coils and fluxgates see Fig.~\ref{fig: coil}. All grey color curves in $\Delta$B graph denote the drift in signal that would have been without the compensation while all other color curves denote the actual drift in signal at all 12 sensor positions found by Eq.~(\ref{eq:del_B}). The 'ON' and 'OFF' vertical dashed lines indicate the time of the perturbation coil being turned 'ON' and 'OFF' respectively. The current supplied on perturbation coil was 10  mA. The resolution index 1 (see Table~\ref{table:t7freq}) with 50 no of averages per measurement and $r$=3.1 were used. The loop sampling frequency was found to be $\sim$6.5 Hz.\label{fig:exp_sim}}{Coil currents with drift $\Delta$B for $k_c^p$=0.0, $k_c^i$=1.0 and $r$=3.1}

\FloatBarrier
\begin{itemize}
    \item message from Fig.~\ref{fig:i100_r24}??
    \item current drifting in Fig.~\ref{fig:exp_sim}\textcolor{blue}{(a)} is hugely reduced in Fig.~\ref{fig:i100_r24}. 
\end{itemize}

\FloatBarrier
\fig{Images/p0i100r24}{width = \textwidth,height=7cm}{Currents (left vertical axis) in all six coil sides ($C_x^\pm$, $C_y^\pm$ and $C_z^\pm$) with drift $\Delta$B (right vertical axis) at 12 sensor positions with fluxgate positions 1, 3, 6 and 8 for $k_c^p$=0.0, $k_c^i$=1.0 (see Eq.~(\ref{eq:I})) and $r$=2.4. For position of coils and fluxgates see Fig.~\ref{fig: coil}. All grey color curves in $\Delta$B graph denote the drift in signal that would have been without the compensation while all other color curves denote the actual drift in signal at all 12 sensor positions found by Eq.~(\ref{eq:del_B}). The 'ON' and 'OFF' vertical dashed lines indicate the time of the perturbation coil being turned 'ON' and 'OFF' respectively. The current supplied on perturbation coil was 10  mA. The resolution index 1 (see Table~\ref{table:t7freq}) with 50 no of averages per measurement and $r$=2.4 were used. The loop sampling frequency was found to be $\sim$6.5 Hz.\label{fig:i100_r24}}{Coil currents with drift $\Delta$B for $k_c^p$=0.0, $k_c^i$=1.0 and $r$=2.4}

% % \fig{Images/cp0i100r24}{width = \textwidth}{short.\label{fig:m2}}{Short}
% \begin{itemize}
%     \item message from Fig.~\ref{fig:pi_tuned_r24}??
%     \item current drifting in Fig.~\ref{fig:tuned_vs_i} is hugely reduced in Fig.~\ref{fig:pi_tuned_r24}. 
% \end{itemize}


% \begin{figure}[!htb]
%     \begin{subfigure}{.5\linewidth}
%         \centering
%         \includegraphics[width=\linewidth, height= 6.5 cm]{Images/r20}
%         \caption{at r=2.0}
%         \label{fig:r20}
%     \end{subfigure}%
%     \begin{subfigure}{.5\linewidth}
%         \centering
%         \includegraphics[width=\linewidth, height= 6.5 cm]{Images/r24}
%         \caption{at r=2.4}
%         \label{fig:r24}
%     \end{subfigure}\\[1ex]
%     \begin{subfigure}{.5\linewidth}
%         \centering
%         \includegraphics[width=\linewidth, height= 6.5 cm]{Images/r28}
%         \caption{at r=2.8}
%         \label{fig:r28}
%     \end{subfigure}%
%         \begin{subfigure}{.5\linewidth}
%         \centering
%         \includegraphics[width=\linewidth, height= 6.5 cm]{Images/r32}
%         \caption{at r=3.2}
%         \label{fig:r32}
%     \end{subfigure}


%     \caption[short]{Currents (left vertical axis) in all six coil sides ($C_x^\pm$, $C_y^\pm$ and $C_z^\pm$) with drift $\Delta$B (right vertical axis) at sensor position '1x' for combine different values of $k_c^i$ and $k_c^p$ ( see Eq.~(\ref{eq:I}) ). Blue color curve denotes the actual drift in signal at position '1x' found by Eq.~(\ref{eq:del_B}), while the red curve denotes the drift that would have been without the compensation. The 'ON' and 'OFF' vertical dashed lines indicate the time of the perturbation coil being turned 'ON' and 'OFF' respectively. For position of coils and fluxgate sensor see Fig.~\ref{fig: coil}.}
%     \label{fig:r_pi}
% \end{figure}

\FloatBarrier
% Now the question arises about what if $r$ value is chosen more than the optimized $r$. For answering that question, we have also studied the effect for more values of $r$ with same setup which are shown in Fig.~\ref{fig:r_pi_more}. It is seen from Fig.~\ref{fig:r_pi_more}\textcolor{blue}{(a)}, Fig.~\ref{fig:r_pi_more}\textcolor{blue}{(b)}, Fig.~\ref{fig:r_pi_more}\textcolor{blue}{(c)} and Fig.~\ref{fig:r_pi_more}\textcolor{blue}{(d)} which are correspond to $r$ = 3.5, 3.6, 3.7 and 3.9 respectively that the coil current graph seems to be settle in for larger value of $r$ before it starts showing less responsive for example at $r$=3.9.   

% \begin{figure}[!htb]
%     \begin{subfigure}{.5\linewidth}
%         \centering
%         \includegraphics[width=\linewidth, height= 6.5 cm]{Images/r35}
%         \caption{at r=3.5}
%         \label{fig:r35}
%     \end{subfigure}%
%     \begin{subfigure}{.5\linewidth}
%         \centering
%         \includegraphics[width=\linewidth, height= 6.5 cm]{Images/r36}
%         \caption{at r=3.6}
%         \label{fig:r36}
%     \end{subfigure}\\[1ex]
%     \begin{subfigure}{.5\linewidth}
%         \centering
%         \includegraphics[width=\linewidth, height= 6.5 cm]{Images/r37}
%         \caption{at r=3.7}
%         \label{fig:r37}
%     \end{subfigure}%
%         \begin{subfigure}{.5\linewidth}
%         \centering
%         \includegraphics[width=\linewidth, height= 6.5 cm]{Images/r39}
%         \caption{at r=3.9}
%         \label{fig:r39}
%     \end{subfigure}


%     \caption[short]{Currents (left vertical axis) in all six coil sides ($C_x^\pm$, $C_y^\pm$ and $C_z^\pm$) with drift $\Delta$B (right vertical axis) at sensor position '1x' for combine different values of $k_c^i$ and $k_c^p$ ( see Eq.~(\ref{eq:I}) ). Blue color curve denotes the actual drift in signal at position '1x' found by Eq.~(\ref{eq:del_B}), while the red curve denotes the drift that would have been without the compensation. The 'ON' and 'OFF' vertical dashed lines indicate the time of the perturbation coil being turned 'ON' and 'OFF' respectively. For position of coils and fluxgate sensor see Fig.~\ref{fig: coil}.}
%     \label{fig:r_pi_more}
% \end{figure}

\FloatBarrier

% The above results confirm that for a matrix with large condition number, the value of $r$ has to be tuned alongside the PI tuning. Next we will talk about a new method to find $r$.

\subsection{Regularization by Matrix Condition Number Method  }\label{sec:cond}

\begin{itemize}
\item This previous stuff plus Rawlik made us believe that cond number is important.  It's important to make it small.
\item We noticed another relationship between condition number and regularization parameter for this system:  the best r determined by MC method gave the smallest cond \#.  Fig. 5.15 shows that when r is varied there is a minimum cond \#.  We tested this for a broad variety of M both theoretical and experimental and found it to be always true that the MC method gave the same best r as if we had just selected the one that gives the smallest cond number.
\item We suspect this is because Tikhonov regularization minimizes both RMS current as well as field fluctuations.  cite Wikipedia (I mean some Tikhonov thing).  In Wikipedia, it minimizes Gammax2 + Ax2.  L2 reg.
\item So it's not surprising and could be more useful, not requiring an random number technique to find the best r.
\item Maybe add example graph somehow comparing to the MC technique?
\end{itemize}


Matrix Condition number ( see Eq.~(\ref{eq:cond} ) and regularization parameter $r$ ( see Eq.~(\ref{eq:minvR} ) have been introduced in Section.~\ref{sec:m} while discussing the inversion of the matrix $\bm{M}$ . Moreover, in Section~\ref{sec:mont}, a method of regularization by random fluctuation has been discussed. We will here propose another method of regularization using the concept of matrix condition number.
 
%  Recall from Section~\ref{sec:inv}, regularization is needed in the first place while inverse of the matrix $\bm{M}$ because $\bm{M}$ itself is ill-conditioned matrix. That means the $\bm{M}$ has a large condition number which while inverse would produce large currents in some ill-positioned places that will make the system unstable. So, it is required to have a well-conditioned  $\bm{M^{-1}}$ which implies that the condition number of $\bm{M^{-1}}$ should be small and that's what regularization has been doing. So, 
 
We introduce Eq.~\ref{eq:minvR} with various values of $r$ and each time the condition number of $\bm{M^{-1}}$ is stored. Then the optimized $r$  has been determined by selecting the $r$ for which the condition number of $\bm{M^{-1}}$ is the minimum.
 
 The condition number of $\bm{M^{-1}}$ for different values of $r$ has been shown in Fig.~\ref{fig:cond}. It is seen that for $r$=0, the condition number of $\bm{M^{-1}}$=$\sim$51 that is same as the condition number of $\bm{M}$ itself. So, without regularization that is the condition number of pseudo-inverse of $\bm{M}$ would also give =$\sim$51. In regularization method, several $r$ is tried ( see Eq.~(\ref{eq:minvR}) ) and each time the condition number has been stored which are shown in the vertical axis. The red diamond symbol indicates that for $r$=2.90, the condition number of $\bm{M^{-1}}$ is minimum and that is 3.2. That by using $r$=2.99 in Eq.~(\ref{eq:minvR}), the condition number decrease from 51 to 3.1 which is 51/3.2$\approx$17 times of decrements. The Fig.~\ref{fig:cond}\textcolor{blue}{(b)} shows that the $r$=2.88 compared to $r$=2.90 that we found here. So, both method shows comparable result. This method will always produce fixed optimized $r$ for a particular  $\bm{M}$. We suspect this is because Tikhonov regularization minimizes both RMS current as well as field fluctuations. But the method by random fluctuation (see Section~\ref{sec:mont}) will produce different optimize $r$ for different run as it because depends on the random field.
 
 \begin{itemize}
     \item message from Fig.~\ref{fig:cond}??
 \end{itemize}

 
\fig{Images/mc_Mcond2}{width = \textwidth,height =7cm}{Condition Number of $\bm{M}$ (vertical axis) in (a) for different values of 'r.  For description of the figure in (b) see Fig.~\ref{fig:I-fluc}. The matrix is similar to the matrix as described by the Fig.~\ref{fig:m}. \label{fig:cond}}{Comparing $r$ from two different methods.}

\FloatBarrier

In the above, different method to find optimize $r$ has been discussed which is a good alternative to the one explained in Section~\ref{sec:mont} and the results are similar. It could be more useful, not requiring an random number technique to find the best $r$. 


% \subsubsection{Optimized r  Revisited Based on Current Response Time}\label{sec:r_currentResponse}
% It was found that there is very slow coil current rise time while applying perturbation. To get rid of that problem, first and foremost, the fastest sampling frequency (see section [\ref{sec:filter}, \ref{sec:freq}]) is needed. Then, the next step of the problem can be solved via two ways with individual having own limitations. First way is tuning the value of P and I term of PI loop as explained in Eq.~(\ref{eq:I} and Section~\ref{sec:tune}. But with increasing the value of P, the current start oscillating after certain values as shown by the top and middle current graph on Fig.~\ref{fig:crnt} which is a problem. 

% \fig{Images/crnt}{width = \textwidth}{Coil current in one of the coil side for optimized r=2.8 with P=0 and I=1.0 (top) and with P=0 and I=1.5 (middle) and for best r considering noise with P=0 and I=1.0 (bottom). \label{fig:crnt}}



% The alternative way is to change the value of optimized r (see section [\ref{sec:mont}, \ref{sec:cond}]) which in turns increase noise in the prototype. But with inclusion of some current fluctuations, it was found that the coil current response time was increased heavily  as shown by the bottom current graph in Fig.~\ref{fig:crnt}. Now, the best compromised value of r was chosen by observing the 'rise time vs r' and 'fluctuations vs r' as shown in Fig.~\ref{fig:riseT}.



%  \doublefig{Images/riseT}{width =\textwidth, height= 8 cm}{Rise Time vs r \label{fig:rise}}{Images/fluc}{width = \textwidth, height= 8 cm}{Fluctuations\label{fig:fluc}}{{(a) shows the Rise Time vs r (b) shows the Fluctuations } \label{fig:riseT}}
% % \fig{Images/bt}{width = \textwidth}{Magnetic Field Compensation \label{fig:bt}}






% \section{Fluxgate Placements and Impact of Shields}\label{sec:flux_place}


% \begin{itemize}
% \item We still had no clue so we just started changing things randomly.
% \item We changed fluxgate positions to try to make cond number smaller to see if it had an effect.  Could not change the cond number or the problem significantly.
% \item We were confused if the slow current response might be due to magnetic responses that we slow.  So we tried removing the shields.  No change.
% \item consider a graph where there is current drifting but with no shield.
% \item And then we realized this had nothing to do with it.
% \end{itemize}

% In earlier section we have shown results using 12 fluxgate sensors placed at positions as given in horizontal axis of Fig.~\ref{fig:m} and there exact positions are described by the Fig.~\ref{fig: coil}. We were facing serious problem with coil current not being settle properly although the field seems to be compensated on time. So, we have tried different positions mainly in corners and center of each of the coil faces. Even, we have removed the outermost shields also. In this Section, the results of those studied will be shown and discussed. 

% First the study on the fluxgate placement will be discussed.

% \subsection{Fluxgate Placements}

% We have not got clear idea about the placements of the fluxgates from the previous studies and our coil current also did not settle down the way we thought is should. We have started taking data with 3*4=12 sensors at slowest sampling frequency (see Table~\ref{table:index}) in the corners but the coil currents were not properly settle down. Then we thought maybe increasing sensors will eliminate the problems. So, we bought new fluxgate sensors and build another breakout box (see Section~\ref{sec:sensor}) but still the results were not good in the coil currents. Then we thought may be if we place in the center of each of the faces of the coils the results will be better but unlucky us. Then we decided to remove the outermost shield (see Section~\ref{sec:shield}) to see the effect but still no luck. After that we decided to use the fastest sampling frequency of our ADC for which we have to build the filters. But due to time limitations and cost concern we have only build 12 filters to support 12 fluxgate sensors. Now, due to this the current response time has increased but that was not the solution of the current unsettle problem. Finally, we have realized that in addition to fastest response we have to also consider the matrix condition number and if the condition number is large then we have to lower the value of optimized $r$ (see Section~\ref{sec:new_study_r}). Here, we will talk about the the studies we have done on fluxgate placements.

% \begin{table} [htb!]
%     \centering
%     \begin{tabular} { |c|c|c|c|c|c|} 
%         \hline
%         \makecell{Fluxgates \\Position} & \makecell{$\bm{M}$\\ Condition No.} &\makecell{$\bm{M^{-1}}$\\ Condition No.} & $r$ & $r'$\\
%         \hline\hline
%         1, 3, 6 and 8 & 33.25 & 2.63 & 2.97 & 2.95\\ 
%         \hline
%         2, 4, 5 and 7 & 28.55 & 1.98 & 3.04 & 3.06 \\ 
%         \hline
%         \makecell{Center \\($C_x^\pm$, $C_y^\pm$ and $C_z^\pm$)} & 36.24 & 1.8 & 2.47 & 2.49 \\ 
%         \hline
%         \makecell{Center-6cm \\($C_x^\pm$, $C_y^\pm$ and $C_z^\pm$)} & 98.61 & 3.05 & 2.36 & 2.34 \\ 
%         \hline
%         \makecell{Center+6cm \\($C_x^\pm$, $C_y^\pm$ and $C_z^\pm$)} & 80.74 & 1.49 & \textcolor{red}{2.26} & \textcolor{red}{2.02} \\ 
%         \hline
%         \makecell{1, 2, 3, 4, \\5, 6, 7 and 8} & 28.30 & 1.94 & 3.17 & 3.19 \\ 
%         \hline
%         \makecell{1, 2, 3, 4, \\5, 6, 7, 8 and \\Center ($C_x^\pm$, $C_y^\pm$ and $C_z^\pm$)}  & 21.84 & 1.8 & \textcolor{red}{3.23} & \textcolor{red}{3.37} \\ 
%         \hline

%     \end{tabular}
%     % \vspace{4mm}
%     \caption[Properties of different no of fluxgate sensors for different positions.]{Properties of different no of fluxgate sensors for different positions. For the positions of the fluxgates see Fig.~\ref{fig: coil} }\label{table:flux-pos}
% \end{table}

% Mainly the corner positions and the center of each of the coil faces  have been tested. Also, they have been analyzed by moving slightly in different positions. The results for different no of sensors in different positions are given in Table~\ref{table:flux-pos}. Position of the fluxgates are defined by the numbers while they are in corners and when they are in the center of each of the coil faces they are termed as 'Center ($C_x^\pm$, $C_y^\pm$ and $C_z^\pm$)'. Center-6cm means all the sensors in the center of the coil faces have been brought 6cm towards the origin from the center and center+6cm means they have brught 6cm away outside the center. For, the full picture of the positions see Fig.~\ref{fig: coil}. the It is seen that that matrix condition number is from 22-36 for the fluxgates being placed either in corners or in center. But if they are slightly moved within $\pm$6cm of center then the matrix condition number becomes very large. Only considering the matrix condition number, it seems that having fluxgates in all the corners and the center of each of the coil faces should be the best choice as its giving the lowest condition number which is 22. But that is not the whole story! To quantify more we have taken the help of regenerating Fig.~\ref{fig:Isim} and Fig.~\ref{fig:fluc-sim}. From Fig.~\ref{fig:Isim}, we have recorded the maximum  $\Delta I_c^{\text{simRMS}}$ (mA) for 30 different sets of $B_s^{\text{rand}}$. For maximum compensation we have generated the Fig.\ref{fig:fluc-sim} for same for 30 different sets of $B_s^{\text{rand}}$ from which we have recorded lowest the remaining fluctuation F. For example- for fluxgate positions 1, 3, 6 and 8, the lowest F goes to 0.3. That means the maximum compensation due to the field produced by $\Delta I_c^{\text{sim}}$to counteract $B_s^{\text{rand}}$ is(1-0.3$\times$100$\%$) =70$\%$. For more details see Section~\ref{sec:mont}. It seen that the current goes crazy if the fluxgates are placed in the center of the coil faces and the maximum $\Delta I_c^{\text{simRMS}}$ ranges from 98 to 192 mA. But the maximum $\Delta I_c^{\text{simRMS}}$ is 12 mA when the center fluxgates are used with the corners one. But on that time the maximum compensation due to the field produced by $\Delta I_c^{\text{sim}}$to counteract $B_s^{\text{rand}}$ is only 15$\%$. So, anything with center seems to give crazy results. Then by looking at the all the three parameters e.g. matrix condition number, maximum $\Delta I_c^{\text{simRMS}}$ and maximum compensation for the 3 different sets of corner positions only, it is seen that the results are more balanced and surprisingly the compensation with 4*3-axis sensors are better than 8*3-axis sensors in exchange of more maximum $\Delta I_c^{\text{simRMS}}$.


% \FloatBarrier


% % The increase in matrix condition number for the center of the coil sides is noteworthy.  So, a study has been done to see the current response effect while sensors are in the middle of the coil sides as shown in Fig.~\ref{fig:cb_center}. The Fig.~\ref{fig:cb_center_p} represents the current response on all the six coil sides with their effect on the $z$-axis in the origin as shown in the right with only choosing proportional (P) term. But just adding a small fraction of integral resest (I) term makes the current response unstable as shown in Fig.~\ref{fig:cb_center_pi}. So only P controller is suitable if fluxgates are placed in the middle of the coil sides but in the case of corner positions either P only or I only or PI controller option is available. In a summary, the more the sensors the more is the matrix condition number. Corner positions are better in terms of different freedom of controlling.

% % \doublefig{Images/cB_t_center_p}{width =\textwidth, height= 6 cm}{Position=1,2,3,4,6,8\label{fig:cb_center_p}}{Images/cB_t_center_pi}{width = \textwidth, height= 6 cm}{Position=1,2,3,4,6,8\label{fig:cb_center_pi}}{{PI Active Magnetic Field Compensation Results by both Experiment and Simulation.} \label{fig:cb_center}}

% % \doublefig{Images/bt6}{width =\textwidth, height= 7 cm}{Position=1,2,3,4,6,8\label{fig:bt6}}{Images/sf6}{width = \textwidth, height= 7 cm}{Position=1,2,3,4,6,8\label{fig:sf6}}{{PI Active Magnetic Field Compensation Results by both Experiment and Simulation.} \label{fig:btSF6}}

% % \doublefig{Images/bt8}{width =\textwidth, height= 7 cm}{Position=1,2,3,4,5,6,7,8\label{fig:bt8}}{Images/sf8}{width = \textwidth, height= 7 cm}{Position=1,2,3,4,5,6,7,8\label{fig:sf8}}{{PI Active Magnetic Field Compensation Results by both Experiment and Simulation.} \label{fig:btSF8}}
% The above discussion of the Table~\ref{table:flux-pos} shows that current goes crazy if the fluxgates are placed in the center of each of the coil faces and crazier if they are slightly moved inside or outside of the center. The corners position fluxgates shows normal behaviour compare to the center positions and surprisingly less sensors shows better compensation in the corners in exchange more more currents.

% \FloatBarrier
% \subsection{Different Shields}

% There are four layers of shields have been used for passive shielding in this prototype (see Section~\ref{sec:shield}). The active compensation effect has been seen using outermost shield and no shields. Both of the configurations produce similar results.

% \begin{itemize}
%      \item message from Fig.~\ref{fig:shield_noShield}??
%  \end{itemize}

% % except for the case of shield, the matrix condition number is less. The Fig.~\ref{fig:btSF8_s} shows the AMC compensation (Fig.~\ref{fig:bt8_s}) and the corresponding allan deviation and shielding factor ( Fig.~\ref{fig:sf8_s}) with outermost layer of shield. The matrix condition number is found to be 19. Similarly, the Fig.~\ref{fig:btSF8} shows the AMC compensation (Fig.~\ref{fig:bt8}) and the corresponding allan deviation and shielding factor ( Fig.~\ref{fig:sf8}) without any shield. The matrix condition number in this case is 132. The one advantage of having shielding is that , the shielding factor for all the sensors remain $geq$1 but in case of no shield, the shielding factor may be very good at certain point but in some point it can go below 1.

% \fig{Images/shield_noShield}{width = \textwidth,height=14cm}{Currents (left vertical axis) in all six coil sides ($C_x^\pm$, $C_y^\pm$ and $C_z^\pm$) with drift $\Delta$B (right vertical axis) at 12 sensor positions with fluxgate positions 1, 3, 6 and 8 for $k_c^p$=0.0, $k_c^i$=1.0 (see Eq.~(\ref{eq:I})) and $r$=2.8. For position of coils and fluxgates see Fig.~\ref{fig: coil}. All grey color curves in $\Delta$B graph denote the drift in signal that would have been without the compensation while all other color curves denote the actual drift in signal at all 12 sensor positions found by Eq.~(\ref{eq:del_B}). The 'ON' and 'OFF' vertical dashed lines indicate the time of the perturbation coil being turned 'ON' and 'OFF' respectively. The current supplied on perturbation coil was 25  mA. The resolution index 1 (see Table~\ref{table:t7freq}) with 50 no of averages per measurement used. The loop sampling frequency was found to be $\sim$6.5 Hz.\label{fig:shield_noShield}}{Coil currents with drift $\Delta$B for $k_c^p$=0.0, $k_c^i$=1.0 and $r$=2.8}

% % \doublefig{Images/bt8_shield}{width =\textwidth, height= 7 cm}{Position=1,2,3,4,6,8\label{fig:bt8_s}}{Images/sf8_shield}{width = \textwidth, height= 7 cm}{Position=1,2,3,4,6,8\label{fig:sf8_s}}{{Shield} \label{fig:btSF8_s}}{short}

% % \doublefig{Images/bt8}{width =\textwidth, height= 7 cm}{Position=1,2,3,4,5,6,7,8\label{fig:bt8}}{Images/sf8}{width = \textwidth, height= 7 cm}{Position=1,2,3,4,5,6,7,8\label{fig:sf8}}{{No SHield} \label{fig:btSF8}}{short}



\section{Style of PI (Old/New)}\label{sec:style_pi}

\begin{itemize}
\item We were led astray by M. Rawlik in late 2017 (nEDM2017).  It is also written in his thesis.
\item The claim is that the PI algorithm is not necessary.  Just use this magic new algorithm and you don't need PI any more.
\item Describe the algorithm and its difference compared to typical PI (our understanding).  Our implementation included P and I terms also.  The main difference seemed to be that I0 wasn't used but rather In in order to determine In+1.
\item In our implementation we noticed experimentally that kp=1 in the new system corresponded to ki=1 in the old system.
\item Eventually we figured out why (derivation).
\item Main conclusion:  the ``New PID'' is the ``old PID'' but with a very particular choice ki=1 and kp=0.  It is clearly better to allow more freedom in the algorithm than this.  Old PID is therefore more general.  Although indeed ki=1 and kp=0 generally do seem to be close to optimal, we think this is a coincidence rather than physics.  We also note the time delays inserted into Rawlik's algorithm are themselves a form of tuning.
\item So we use regular PI and treat the parameters as TBD by system tuning.
\end{itemize}


We have talked about the tuning method in Section~\ref{sec:tune} and
later in Section~\ref{sec:pi_behave} we have shown the the effect of
the P and I term where the effectiveness of the P term and the tuning
method arises. Those were all based on the works from
Ref.\cite{bea}. In early 2018, another author on his PhD
thesis \cite{rawlik} claimed that there is no need of PI!! As we have
done most of our work following the Ref.\cite{bea}, so we wanted to
verify that the claim of no PI by Ref.\cite{rawlik} is actually true
or not. So, this Section is all about comparing the work from two
Ref.\cite{bea} and Ref.\cite{rawlik} and verify the claim.

Here, the PI control algorithm as discussed by the Eq.~(\ref{eq:I}) which taken from Ref.~\cite{bea}) is termed as the 'Old PI' control. and no PI claim by Ref.~\cite{rawlik} is termed as 'New PI' control and the new current can be found by -
\begin{equation}\label{eq:I_raw}
    I^{n+1}= I^n+M^{-1} (B_{setpoint}-B_{measure}^n)=I^n+M^{-1} \Delta B^n
\end{equation}
or with a delay like -
\begin{equation}\label{eq:I_raw_delay}
    I^{n+1}= I^{n-2}+M^{-1} (B_{setpoint}-B_{measure}^n)=I^{n-2}+M^{-1} \Delta B^n
\end{equation}
But is it really possible ? 

To verify that we run the prototype two times. First time we set the value of $k_c^p$=0.0 and $k_c^i$=1.0 to find the new current by Eq.~(\ref{eq:I}) from 'Old PI' and another with $k_c^p$=1.0 and $k_c^i$=0.0 to find the new current determined by Eq.~(\ref{eq:I_raw}) of 'New PI'. The results from 'Old PI' are given in Fig.~\ref{fig:style_of_pi}\textcolor{blue}{(a)} and in Fig.~\ref{fig:style_of_pi}\textcolor{blue}{(b)} and that of 'New PI' in Fig.~\ref{fig:style_of_pi}\textcolor{blue}{(c)} and in Fig.~\ref{fig:style_of_pi}\textcolor{blue}{(d)} . It is seen that the results in both of the figures are same. That is P and I term of 'Old PI' has been transformed into I and P term of 'New PI' current equation.
\begin{itemize}
    \item message from the Fig.~\ref{fig:style_of_pi} ?
\end{itemize}
% For simplicity instead of showing all the drift $\Delta$B for all the fluxgate sensors for the positions given in the horizonatal axis of Fig.~\ref{fig:m}, only '1x' is shown on the right of the figures. And same as earlier the currents  that are being sent to the coils ($C_x^\pm$, $C_y^\pm$ and $C_z^\pm$) shown on the left of the same figures. 

\fig{Images/pi_comp}{width = \textwidth,height=10cm}{Currents (left vertical axis) in coil side $C_z^+$ with drift $\Delta$B (right vertical axis) at sensor position '1x' for combine different values of $k_c^i$ and $k_c^p$ by applying Eq.~(\ref{eq:I}) at (a) and Eq.~(\ref{eq:I_raw}) at (b) . Green color curve denotes the actual drift in signal at position '1x' found by Eq.~(\ref{eq:del_B}), while the red curve denotes the drift that would have been without the compensation. The 'ON' and 'OFF' vertical dashed lines indicate the time of the perturbation coil being turned 'ON' and 'OFF' respectively. For position of coils and fluxgate sensor see Fig.~\ref{fig: coil}. The current supplied on perturbation coil was 10  mA. The resolution index 1 (see Table~\ref{table:t7freq}) with 50 no of averages per measurement was used. The loop sampling frequency was found to be $\sim$6.5 Hz. \label{fig:style_of_pi}}{Currents in coil side $C_z^+$ with drift $\Delta$B .}


% \doublefig{Images/i100_old}{width =\textwidth, height= 6.5 cm}{at $k_c^p$=0.0 and $k_c^i$=1.0 for 'Old PI' \label{fig:i100_old}}{Images/p100_new}{width = \textwidth, height= 6.5 cm}{at $k_c^p$=1.0 and $k_c^i$=0.0 for 'New PI'\label{fig:p100_new}}{{Currents (left vertical axis) in all six coil sides ($C_x^\pm$, $C_y^\pm$ and $C_z^\pm$) with drift $\Delta$B (right vertical axis) at sensor position '1x' for combine different values of $k_c^i$ and $k_c^p$ by applying Eq.~(\ref{eq:I}) at (a) and Eq.~(\ref{eq:I_raw}) at (b) . Blue color curve denotes the actual drift in signal at position '1x' found by Eq.~(\ref{eq:del_B}), while the red curve denotes the drift that would have been without the compensation. The 'ON' and 'OFF' vertical dashed lines indicate the time of the perturbation coil being turned 'ON' and 'OFF' respectively. For position of coils and fluxgate sensor see Fig.~\ref{fig: coil}..} \label{fig:style_of_pi}}{short}


\FloatBarrier
But again is it really true about what we are claiming ?

To verify that the equation has been studied deeply. The mathematical equivalent of Eq.~(\ref{eq:I_raw}) is- 
\begin{equation}\label{eq:I_raw_eq}
    I^{n+1}= \sum_{i=0}^n M^{-1} (B_{setpoint}-B_{ambient}^i)
\end{equation}

The equivalency can be proved by mathematical induction. That is, for n$\geq$0, let $P_n$ be the following statement -
\begin{equation}\label{eq:I_raw_eq_both}
   I^n+M^{-1} (B_{setpoint}-B_{measure}^n)= \sum_{i=0}^n M^{-1} (B_{setpoint}-B_{ambient}^i)
\end{equation}

For n=0, the statement $P_0$ is-

\begin{align*}
    \begin{split}
      LHS &=I^0+M^{-1} (B_{setpoint}-B_{measure}^0) \\
        &=I^0+M^{-1} (B_{setpoint}-B_{ambient}^0 -M I^0) \\
        &=I^0+M^{-1} (B_{setpoint}-B_{ambient}^0) - MM^{-1} I^0) \\
        &=I^0+M^{-1} (B_{setpoint}-B_{ambient}^0) - I^0 \\
        &=M^{-1} (B_{setpoint}-B_{ambient}^0)
    \end{split}
    \\
    \begin{split}
      RHS &=M^{-1} (B_{setpoint}-B_{ambient}^0)\\
          &= LHS
    \end{split}
\end{align*}
$\therefore$ The Eq.~(\ref{eq:I_raw_eq_both}) is true for n=0.\newline
Suppose it is also true for k$\geq$0. That is, the following statement $P_k$ holds -
\begin{equation}\label{eq:I_raw_99}
   I^k+M^{-1} (B_{setpoint}-B_{measure}^k)= \sum_{i=0}^k M^{-1} (B_{setpoint}-B_{ambient}^i)
\end{equation}

It is to be showed that $P_{k+1}$ also holds, that is-

\begin{equation}\label{eq:I_raw_98}
   I^{k+1}+M^{-1} (B_{setpoint}-B_{measure}^{k+1})= \sum_{i=0}^{k+1} M^{-1} (B_{setpoint}-B_{ambient}^i)
\end{equation}
\begin{align*}
    \begin{split}
      LHS &=I^{k+1}+M^{-1} (B_{setpoint}-B_{measure}^{k+1}) \\
        &=I^{k+1}+M^{-1} (B_{setpoint}-B_{ambient}^{k+1} -M I^{k+1}) \\
        &=I^{k+1}+M^{-1} (B_{setpoint}-B_{ambient}^{k+1}) - MM^{-1} I^{k+1}) \\
        &=I^{k+1}+M^{-1} (B_{setpoint}-B_{ambient}^0) - I^{k+1} \\
        &=M^{-1} (B_{setpoint}-B_{ambient}^{k+1})
    \end{split}
    \\
    \begin{split}
      RHS &=M^{-1} (B_{setpoint}-B_{ambient}^{k+1})\\
          &= LHS
    \end{split}
\end{align*}
$\therefore \: P_{k+1}$ also holds which validates the Eq.~(\ref{eq:I_raw_eq_both}) by the principle of mathematical induction. \newline
Now, the Eq.~(\ref{eq:I_raw_eq}) is equivalent to the implementation of PI -
\begin{equation}\label{eq:I_raw_pi}
    I^{n+1}= k_p E^n+ k_i \sum_{i=0}^n E^i \Delta t
\end{equation}
with $k_p=0$ and $k_i$ set to a particular value. Here, the error is , $E^i=M^{-1} (B_{setpoint}-B_{ambient}^i)$. So, the Eq.~(\ref{eq:I_raw}) is a PI with a particular tuning and termed as 'New PI' control. Similarly, the delay showed in Eq.~(\ref{eq:I_raw_delay}) is another form of tuning.

So the above discussion verifies that 'Old PI' and 'New PI' are just two different form of tuning based on the control algorithm. We can also clear our previous questions about the effectiveness of the P term and the tuning method discussed in Section~\ref{sec:pi_behave} that instead of the tuning method discussed in Section~\ref{sec:tune}, we are using different from of tuning with changing the value of $k_c^i$ keeping $k_c^p$=0. In upcoming Section we then studied the effect of  matrix condition number in P and I term. and propose a new method to find regularization parameter using matrix condition number. 





\section{Coil Configuration}\label{sec:coil_config}


\begin{itemize}
\item coil cube with matrix calculation in python by Jeff shows the condition number to be infinity.
\item exploits the diagonal matrix (Section~\ref{sec:m}) and found one of them is giving zero
\item wire the two coils to work as one and matrix condition number is hugely improved. close to one!!
\item Reason: Maxwell's equation and current mode in all the six coils.
\item Yes, fine, but does this actually fix the current drifting problem?
\end{itemize}




For the prototype, two different configuration of compensation coils have been exploited. In the first case, all the six coils have been used in the six faces surrounding the compensation area and for the later one, two coils have been wired together so that they can act as one making total five instead of six coils. The main reason for using the second configuration is the condition number of $\bm{M}$. It was found to be $\sim$4.7 for the second one as compared to $\sim$51 in the first.

\begin{itemize}
     \item message from Fig.~\ref{fig:v2}, Fig.~\ref{fig:wt}, Fig.~\ref{fig:v5c} and Fig.~\ref{fig:wt5c}??
 \end{itemize}

\fig{Images/v2}{width = \textwidth}{Square root of eigenvalues of $\bm{M^T}\bm{M}$ and $\bm{M}\bm{M^T}$  in sensors$\times$coils dimension for 6 coils. \label{fig:v2}}{Square root of eigenvalues of $\bm{M^T}\bm{M}$ and $\bm{M}\bm{M^T}$ for 6 coils.}

\fig{Images/wt}{width = \textwidth}{Orthonormal eigenvectors of $\bm{M^T}\bm{M}$ in coils$\times$coils dimension for 6 coils. \label{fig:wt}}{Orthonormal eigenvectors of $\bm{M^T}\bm{M}$ for 6 coils.}

\fig{Images/v5c}{width = \textwidth}{Square root of eigenvalues of $\bm{M^T}\bm{M}$ and $\bm{M}\bm{M^T}$  in sensors$\times$coils dimension for 5 coils. \label{fig:v5c}}{Square root of eigenvalues of $\bm{M^T}\bm{M}$ and $\bm{M}\bm{M^T}$ for 5 coils.}

\fig{Images/wt5c}{width = \textwidth}{Orthonormal eigenvectors of $\bm{M^T}\bm{M}$ in coils$\times$coils dimension for 5 coils. \label{fig:wt5c}}{Orthonormal eigenvectors of $\bm{M^T}\bm{M}$ for 5 coils.}

\FloatBarrier

\begin{itemize}
     \item message from Fig.~\ref{fig:coil_reg}??
 \end{itemize}

% \fig{Images/coil_reg}{width = \textwidth,height=12cm}{Currents (left vertical axis) in all six coil sides ($C_x^\pm$, $C_y^\pm$ and $C_z^\pm$) with drift $\Delta$B (right vertical axis) at 12 sensor positions with fluxgate positions 1, 3, 6 and 8 for $k_c^p$=0.0, $k_c^i$=1.0 (see Eq.~(\ref{eq:I})). The $r$=2.9 in (a) and $r$=3.39 in (b). For position of coils and fluxgates see Fig.~\ref{fig: coil}. All grey color curves in $\Delta$B graph denote the drift in signal that would have been without the compensation while all other color curves denote the actual drift in signal at all 12 sensor positions found by Eq.~(\ref{eq:del_B}). The 'ON' and 'OFF' vertical dashed lines indicate the time of the perturbation coil being turned 'ON' and 'OFF' respectively. The current supplied on perturbation coil was 10  mA. The resolution index 1 (see Table~\ref{table:t7freq}) with 50 no of averages per measurement used. The loop sampling frequency was found to be $\sim$6.5 Hz.\label{fig:coil_reg}}{Coil currents with drift $\Delta$B for different coil configuration in regularized $\bm{M^{-1}}$}
\fig{Images/coil_reg2}{width = \textwidth,height=12cm}{Currents (left vertical axis) in all six coil sides ($C_x^\pm$, $C_y^\pm$ and $C_z^\pm$) with drift $\Delta$B (right vertical axis) at 12 sensor positions with fluxgate positions 1, 3, 6 and 8 for $k_c^p$=0.0, $k_c^i$=1.0 (see Eq.~(\ref{eq:I})). The matrix condition number is 26.37 in (a) which while inverse for $r$=3.04 gives 1.71 and The matrix condition number is 2.26 in (b) which while inverse for $r$=3.6 gives 1.08. For position of coils and fluxgates see Fig.~\ref{fig: coil}. All grey color curves in $\Delta$B graph denote the drift in signal that would have been without the compensation while all other color curves denote the actual drift in signal at all 12 sensor positions found by Eq.~(\ref{eq:del_B}). The 'ON' and 'OFF' vertical dashed lines indicate the time of the perturbation coil being turned 'ON' and 'OFF' respectively. The current supplied on perturbation coil was 10  mA. The resolution index 1 (see Table~\ref{table:t7freq}) with 50 no of averages per measurement used. The loop sampling frequency was found to be $\sim$6.5 Hz.\label{fig:coil_reg}}{Coil currents with drift $\Delta$B for different coil configuration in regularized $\bm{M^{-1}}$}

\FloatBarrier
\begin{itemize}
     \item message from Fig.~\ref{fig:coil_pseudo}??
 \end{itemize}
\FloatBarrier
% \fig{Images/coil_pseudo}{width = \textwidth,height=12cm}{Currents (left vertical axis) in all six coil sides ($C_x^\pm$, $C_y^\pm$ and $C_z^\pm$) with drift $\Delta$B (right vertical axis) at 12 sensor positions with fluxgate positions 1, 3, 6 and 8 for $k_c^p$=0.0, $k_c^i$=1.0 (see Eq.~(\ref{eq:I})). Pseudoinverse was used in both case. For position of coils and fluxgates see Fig.~\ref{fig: coil}. All grey color curves in $\Delta$B graph denote the drift in signal that would have been without the compensation while all other color curves denote the actual drift in signal at all 12 sensor positions found by Eq.~(\ref{eq:del_B}). The 'ON' and 'OFF' vertical dashed lines indicate the time of the perturbation coil being turned 'ON' and 'OFF' respectively. The current supplied on perturbation coil was 10  mA. The resolution index 1 (see Table~\ref{table:t7freq}) with 50 no of averages per measurement used. The loop sampling frequency was found to be $\sim$6.5 Hz.\label{fig:coil_pseudo}}{Coil currents with drift $\Delta$B for different coil configuration in non-regularized $\bm{M^{-1}}$}

\fig{Images/coil_pseudo2}{width = \textwidth,height=12cm}{Currents (left vertical axis) in all six coil sides ($C_x^\pm$, $C_y^\pm$ and $C_z^\pm$) with drift $\Delta$B (right vertical axis) at 12 sensor positions with fluxgate positions 1, 3, 6 and 8 for $k_c^p$=0.0, $k_c^i$=1.0 (see Eq.~(\ref{eq:I})). Pseudoinverse was used in both case. For position of coils and fluxgates see Fig.~\ref{fig: coil}. All grey color curves in $\Delta$B graph denote the drift in signal that would have been without the compensation while all other color curves denote the actual drift in signal at all 12 sensor positions found by Eq.~(\ref{eq:del_B}). The 'ON' and 'OFF' vertical dashed lines indicate the time of the perturbation coil being turned 'ON' and 'OFF' respectively. The current supplied on perturbation coil was 10  mA. The resolution index 1 (see Table~\ref{table:t7freq}) with 50 no of averages per measurement used. The loop sampling frequency was found to be $\sim$6.5 Hz.\label{fig:coil_pseudo}}{Coil currents with drift $\Delta$B for different coil configuration in non-regularized $\bm{M^{-1}}$}

 \FloatBarrier
%  As previously discussed in Section~\ref{sec:prototype}, there are six compensation coils ($\bm{C_x^\pm}$, $\bm{C_y^\pm}$ and $\bm{C_z^\pm}$) and one perturbation coil ($\bm{P_z^+}$). 
 It can be as explained by Fig.~\ref{fig:cDir} where one of the coil current mode was shown for all six coils. The total current contribution is found to be zero due to current contributions from $C_x^{\pm}$ in Fig.~\ref{fig:c1}, $C_y^{\pm}$ in Fig.~\ref{fig:c3} and $C_z^{\pm}$ in Fig.~\ref{fig:c5}. To break the mode, two out of the six coils have been wired together so that they can act as one resulting total compensation coils be five. 
%It was found that due to this, the condition number is decreased drastically to 2.
\begin{figure}
    \begin{subfigure}{.5\linewidth}
        \centering
        \includegraphics[scale=.28]{Images/c1}
        \caption{Current Direction in $C_x^{\pm}$}
        \label{fig:c1}
    \end{subfigure}%
    \begin{subfigure}{.5\linewidth}
        \centering
        \includegraphics[scale=.28]{Images/c3}
        \caption{Current Direction in $C_y^{\pm}$}
        \label{fig:c3}
    \end{subfigure}\\[1ex]
    \begin{subfigure}{\linewidth}
        \centering
        \includegraphics[scale=.33]{Images/c5}
        \caption{Current Direction in $C_z^{\pm}$}
        \label{fig:c5}
    \end{subfigure}
    \caption[short]{Current direction in $C_x^{\pm}$, $C_y^{\pm}$ and $C_z^{\pm}$ . The net current is zero while adding the current contribution from all the six coils for this orientation. $C_z^{\pm}$ coils have been wired together to break the configuration which results in significant decrease in matrix condition number. }
    \label{fig:cDir}
\end{figure}

\begin{table} [htb!]
    \centering
    \begin{tabular} { |c|c|c|c|c|c|} 
        \hline
        Coils & \makecell{Matrix \\Condition Number} &\makecell{Inverse Matrix \\ Condition Number} & \makecell{Regularization \\Parameter, $r$}\\
        \hline\hline
        \makecell{$C_x^-$, $C_x^+$, $C_y^-$,\\ $C_y^+$, $C_z^-$ and $C_z^+$ } & 26.37 & 1.71 & 3.04 \\ 
        \hline
        \makecell{$C_x^{\pm}$, $C_y^-$, $C_y^+$,\\ $C_z^-$ and $C_z^+$ } & 2.26 & 1.08 & 3.6 \\         
        \hline
        \makecell{$C_x^-$, $C_x^+$, $C_y^\pm$,\\ $C_z^-$ and $C_z^+$ } & 2.27 & 1.09 & 3.2 \\
        \hline
        \makecell{$C_x^-$, $C_x^+$, $C_y^-$,\\ $C_y^+$ and $C_z^\pm$ } & 2.41 & 1.10 & 3.56 \\
        \hline

    \end{tabular}
    % \vspace{4mm}
    \caption[Properties for different coil configurations]{Properties for different coil configurations for fluxgate sensors at 1, 3, 6 and 8. For the positions of the fluxgates and coils see Fig.~\ref{fig: coil}}\label{table:mcond_coil}
\end{table}

%
% \begin{table} [htb!]
%     \centering
%     \begin{tabular} { |c|c|c|c|c|c|} 
%         \hline
%         Coils & \makecell{Matrix \\Condition Number} &\makecell{Inverse Matrix \\ Condition Number} & \makecell{Regularization \\Parameter, $r$}\\
%         \hline\hline
%         \makecell{$C_x^-$, $C_x^+$, $C_y^-$,\\ $C_y^+$, $C_z^-$ and $C_z^+$ } & 26.37 & 1.71 & 3.04 \\ 
%         \hline
%         \makecell{$C_x^{\pm}$, $C_y^-$, $C_y^+$,\\ $C_z^-$ and $C_z^+$ } & 2.26 & 1.08 & 3.6 \\         
%         \hline
%         \makecell{$C_x^-$, $C_x^+$, $C_y^\pm$,\\ $C_z^-$ and $C_z^+$ } & 2.27 & 1.09 & 3.2 \\
%         \hline
%         \makecell{$C_x^-$, $C_x^+$, $C_y^-$,\\ $C_y^+$ and $C_z^\pm$ } & 2.41 & 1.10 & 3.56 \\
%         \hline
%         \makecell{$C_x^-$, $C_x^+$+$C_y^-$,\\ $C_y^+$, $C_z^-$ and $C_z^+$ } & 2.97 & 1.14 & 3.6 \\
%         \hline
%         \makecell{$C_x^-$, $C_x^+$,$C_y^-$,\\ $C_y^+$+$C_z^-$ and $C_z^+$ } & 2.8 & 1.14 & 3.62 \\
%         \hline
%         \makecell{$C_x^-$+$C_y^-$, $C_x^+$,\\ $C_y^+$,$C_z^-$ and $C_z^+$ } & 25 & 1.69 & 3.06 \\
%         \hline
%         \makecell{$C_x^-$, $C_x^+$+$C_y^+$,\\ $C_y^-$, $C_z^-$ and $C_z^+$ } & 25 & 1.70 & 3.05 \\
%         \hline
%         \makecell{$C_x^-$, $C_x^+$, $C_y^+$,\\ $C_y^-$+$C_z^-$ and $C_z^+$ } & 26.14 & 1.65 & 3.06 \\
%         \hline
%         \makecell{$C_x^-$+$C_x^+$, $C_y^-$+ $C_y^+$,\\$C_z^-$ and $C_z^+$ } & 440.57 & 1.79 & 2.54 \\
%         \hline
%         \makecell{$C_x^-$+$C_x^+$, $C_y^-$+ $C_y^+$,\\ and $C_z^-$+$C_z^+$ } & 1.27 & 1.0 & 3.73 \\
%         \hline

%     \end{tabular}
%     % \vspace{4mm}
%     \caption[short]{Properties of different no of fluxgate sensors for different positions. Max $\Delta I_c^{\text{simRMS}}$ column has been taken for each of the positions defined by generating Fig.~\ref{fig:Isim}. Max compensation column has been determined for for each of the positions defined by generating Fig.~\ref{fig:fluc-sim} and the description is given in text. For the positions of the fluxgates see Fig.~\ref{fig: coil} }\label{table:mcond_coil}
% \end{table}

\section{Future Section ??}\label{sec:metrics_res}

\subsection{Experimental Current Error and Remaining Fluctuation}

\begin{itemize}
\item Basic message of Fig. 5.4 is that experiment and simulation kind of agree.
% \item In simulation a set of twelve $B_s^\text{rand}$ has been generated by normal distribution around zero with standard deviation being 1.5 nT.
% % \item In experiment  set of twelve $B_s^\text{rand}$ has been generated by magnetic field signals from fluxgate positions at 1, 2 and 7 using Eq.~(\ref{eq:del_B}). For position of the fluxgates please see Fig.~\ref{fig: coil}.
% \item $r$ has been varied from 0 to 6 in an increment of 0.29 and for each $r$, PI control algorithm has been implemented for n measurements and for every measurements, signals from fluxgates at positions 1,2 and 7 are stored as $B_{\text{meas}}$ and uncompensated B field has been stored as $B_{\text{uncomp}}$ using  $B_{\text{meas}}-\bm{M}(I_c^n-I_c^0)$ where $I_c^n$ can be found using Eq.~(\ref{eq:I}). From those stored values,  Eq.~(\ref{eq:B_coils-sim}) has been found by $B_{\text{meas}}^{\text{max}}(r)-B_{\text{meas}}^{\text{min}}(r)$ and $B_s^\text{rand}$ has been found by $B_{\text{uncomp}}^{\text{max}}(r)-B_{\text{uncomp}}^{\text{min}}(r)$ and $\Delta I_c^{\text{exp}}(r)$ has been found using same as Eq.(\ref{eq:del_Is}). 
\end{itemize}

\fig{Images/mont_exp}{width = \textwidth}{The effect of $r$ on (a) current error  and (b) remaining field fluctuations in experimental setup for $k_c^p=1.0$ for fluxgate signals from 1, 2 and 7. Here, 100 mA current has been supplied in the perturbation coil. For position of the fluxgates  and coil see Fig.~\ref{fig: coil}.\label{fig:mont_comp}}{The effect of $r$ for experimental setup.}

The effect of $r$ on the current error and remaining field fluctuations in experimental setup for fluxgate signals from 1, 2 and 7 are shown in Fig.~\ref{fig:mont_comp}. For position of the sensors please see Fig.~\ref{fig: coil}. For data measurement, for certain loops, PI control algorithm with $k_c^p=$1.0 and $k_c^i=$0.0 has been implemented for $r$ ranges from 0 to 6 in an increment of 0.29. For every measurements, signals from fluxgates at positions 1,2 and 7 are stored as $B_{\text{meas}}$ and uncompensated B field has been stored as $B_{\text{uncomp}}$ which is predicted by Eq.~\ref{eq:Buncomp}. The difference between maximum value of $B_{\text{uncomp}}$ and $B_{\text{uncomp}}$ will give the field fluctuation due to 100 mA supplied in rhe perturbation coil. Moreover, the difference between maximum value of $B_{\text{meas}}$ and $B_{\text{meas}}$ will give the amount of compensation of the field which when equal to field fluctuation indicates no compensation. The ratio of rms value of this to the field fluctuation will generate the remaining fluctuation $F$ which is shown for different $r$ in Fig.~\ref{fig:mont_comp}\textcolor{blue}{(b)}. It is seen that for a particular filed perturbation, the remaining fluctuation is $\approx$0.45 which indicates that the maximum correction is $\mathbf{(1-0.45)\times100\%=55\%}$. Moreover, $\Delta I_c^{\text{exp}}(r)$ has been found using Eq.(\ref{eq:del_Is}) and the rms has been determined which is shown in  Fig.~\ref{fig:mont_comp}\textcolor{blue}{(a)}. It is seen that around 100 mA rms of current array is required to compensate the perturbation which eventually vanishes with $r$ increment as expected.
% We have studied the coil current response and magnetic fluctuation reduction in

% In Section~\ref{sec:inv}, we have already talked about how the regularization parameter $r$ came into effect. Then, in Section~\ref{sec:mont}, regularization by random magnetic distribution method has been described which has been taken from Ref.~\cite{bea} where a suitable value of $r$ has been found which will give the best compromise between the magnetic field fluctuation and coil current fluctuation. We have generated a fluctuation by our perturbation coil. $r$ has been varied from 0 to 6 in an increment of 0.29 and for each $r$, PI control algorithm has been implemented for n measurements and for every measurements, signals from fluxgates at positions 1,2 and 7 are stored as $B_{\text{meas}}$ and uncompensated B field has been stored as $B_{\text{uncomp}}$ using  $B_{\text{meas}}-\bm{M}(I_c^n-I_c^0)$ where $I_c^n$ can be found using Eq.~(\ref{eq:I}). From those stored values,  Eq.~(\ref{eq:B_coils-sim}) has been found by $B_{\text{meas}}^{\text{max}}(r)-B_{\text{meas}}^{\text{min}}(r)$ and $B_s^\text{rand}$ has been found by $B_{\text{uncomp}}^{\text{max}}(r)-B_{\text{uncomp}}^{\text{min}}(r)$ and $\Delta I_c^{\text{exp}}(r)$ has been found using same as Eq.(\ref{eq:del_Is}). 

% The root mean square (RMS) of $\Delta I_c^{\text{exp}}(r)$ is calculated as
% \begin{equation}\label{eq:delta_Iexp_rms}
%      \Delta I_{\text{RMS}}^{\text{exp}}(r)= \sqrt{\frac{1}{6}\sum_{c=1}^6 (\Delta I_c^{\text{exp}}(r))^2}
% \end{equation}
% The remaining fluctuation is calculated as
% \begin{equation}\label{eq:fluc}
%     F(r)=\frac{\sqrt{\frac{1}{12} \sum_{s=1}^{12} (B_s^{\text{sim}}(r))^2}}{\sqrt{\frac{1}{12} \sum_{s=1}^{12} (B_s^{\text{rand}}(r))^2}}
% \end{equation}
% The field without compensation is predicted by 
% \begin{equation}
    
% \end{equation}
% In this Section, we have found Eq.~(\ref{eq:delta_Isim_rms}) and Eq.~(\ref{eq:fluc}) in experimental setup.
% $B_s^{\text{rand}}$ will be replaced by
% The fluctuation caused by perturbation coil is
% \begin{equation}\label{eq:B_coils-exp}
%     B_s^{\text{fluc}}(r) =B_{\text{uncomp}}^{\text{max}}(r)-B_{\text{uncomp}}^{\text{min}}(r)
% \end{equation}
% To compensate the fluctuation
% \begin{equation}\label{eq:B_coils-exp}
%     B_s^{\text{exp}}(r) =B_{\text{meas}}^{\text{max}}(r)-B_{\text{meas}}^{\text{min}}(r)
% \end{equation}


% \begin{equation}\label{eq:fluc_exp}
%     F(r)=\frac{\sqrt{\sum_s \frac{1}{s}(B_{\text{meas}}^{\text{max}}(r)-B_{\text{meas}}^{\text{min}}(r))^2}}{\sqrt{\sum_s \frac{1}{s}(B_{\text{uncomp}}^{\text{max}}(r)-B_{\text{uncomp}}^{\text{min}}(r))^2}}
% \end{equation}

% In Section~\ref{sec:mont}, 30 different sets of random magnetic field ($B_s^{\text{rand}}$) values have been chosen with center of distribution being 0 and standard deviation 5 nT for 12 different fluxgate sensor positions whose labels are given in the horizontal axis of Fig.~\ref{fig:m} and the positions are specified in Fig.~\ref{fig: coil}. Here, we have chosen one out of the 30 different sets of $B_s^{\text{rand}}$ and go through the steps as in Section~\ref{sec:mont} to generate similar figures like in Fig.~\ref{fig:Isim}, Fig.~\ref{fig:fluc-sim} and Fig.~\ref{fig:I-fluc}. For the experimental setup, we have done the similar steps to generate those figures except instead of generating $B_s^{\text{rand}}$ randomly, we have applied current in the perturbation coil to generate the required drift ( see Eq.~(\ref{eq:del_B}) ) which will be treated as $B_s^{\text{rand}}$. The results of the simulation as well as experiment are given in Fig.~\ref{fig:mont_comp}, where Fig.~\ref{fig:mont_comp}\textcolor{blue}{(a)}, Fig.~\ref{fig:mont_comp}\textcolor{blue}{(b)} and Fig.~\ref{fig:mont_comp}\textcolor{blue}{(c)} are the results found from simulation and Fig.~\ref{fig:mont_comp}\textcolor{blue}{(d)}, Fig.~\ref{fig:mont_comp}\textcolor{blue}{(e)} and Fig.~\ref{fig:mont_comp}\textcolor{blue}{(f)} are the results found from experiment. They are described in detail in Fig.~\ref{fig:Isim}, Fig.~\ref{fig:fluc-sim} and Fig.~\ref{fig:I-fluc}. It is seen that the experimental counterpart of each of the simulation that is Fig.~\ref{fig:mont_comp}\textcolor{blue}{(a)} of simulation is comparable to Fig.~\ref{fig:mont_comp}\textcolor{blue}{(d)} of experiment. Similarly, Fig.~\ref{fig:mont_comp}\textcolor{blue}{(b)} of simulation with Fig.~\ref{fig:mont_comp}\textcolor{blue}{(e)} of experiment and Fig.~\ref{fig:mont_comp}\textcolor{blue}{(c)} of simulation with Fig.~\ref{fig:mont_comp}\textcolor{blue}{(f)} of experiment are comparable and they produce similar results. The similar results of the simulation with experiment justifies the the simulation model described in Section~\ref{sec:mont}.

% \fig{Images/mont_comp3}{width = \textwidth,height =10cm}{Simulation result for one set of $B_s^{\text{rand}}$ for the Fig.~\ref{fig:Isim}, Fig.~\ref{fig:fluc-sim} and Fig.~\ref{fig:I-fluc} are shown in (a), (b) and (c) and corresponding experimental results are in (d), (e) and (f). For the description of the figures see the text in Fig.~\ref{fig:Isim}, Fig.~\ref{fig:fluc-sim} and Fig.~\ref{fig:I-fluc}. For position of fluxgate sensors see Fig.~\ref{fig: coil}.\label{fig:mont_comp}}{Short}


\FloatBarrier
% The above results in Fig.~\ref{fig:mont_comp}, where simulation results are similar to experimental one justify the simulation method described in Section~\ref{sec:mont}. The upcoming Section is all about P and I term (see Eq.~(\ref{eq:I}) ) behaviour. 
% \fig{Images/sf6p}{width = \textwidth}{p127, 100mA stimulus,p0.7,r3.5,Simulation result for one set of $B_s^{\text{rand}}$ for the Fig.~\ref{fig:Isim}, Fig.~\ref{fig:fluc-sim} and Fig.~\ref{fig:I-fluc} are shown in (a), (b) and (c) and corresponding experimental results are in (d), (e) and (f). For the description of the figures see the text in Fig.~\ref{fig:Isim}, Fig.~\ref{fig:fluc-sim} and Fig.~\ref{fig:I-fluc}. For position of fluxgate sensors see Fig.~\ref{fig: coil}.\label{fig:adev}}{Short}
\fig{Images/sf6sr}{width = \textwidth}{Allan deviation and shielding factor for fluxgate positions 1, 3, 6 and 8. For position of fluxgate sensors see Fig.~\ref{fig: coil}.\label{fig:adev}}{Allan deviation and shielding factor.}

\fig{Images/center}{width = \textwidth}{The effect on the control sensors signal fluctuations for PI control algorithm at fluxgate positions 1, 3, 6 and 8. For position of fluxgate sensors see Fig.~\ref{fig: coil}.\label{fig:center}}{The effect on the control sensors signal fluctuations.}