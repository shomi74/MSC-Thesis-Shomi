%your front matter - fill in your personal details here!
%%
\title{Active Magnetic Compensation Prototype for a Neutron Electric Dipole Moment Experiment}
\author{Shomi Ahmed}

\degreemonth{April} % month final submission occurs.
\degreeyear{2019}%
\degree{Master of Science}%
\department{Physics & Astronomy}%
\advisor{Dr. Jeff Martin and Dr. Chris Bidinosti} %



%\copyrightpage
% Insert a blank page for two-sided
%  \newpage
%  \thispagestyle{empty}
%  \hbox{}
%  \newpage


\maketitle


\begin{abstract}
The existence of a non-zero neutron electric dipole moment (nEDM) would violate parity and time-reversal symmetry.  Extensions to the Standard Model predict the nEDM to be $10^{-26}$ -- $10^{-28}$~e$\cdot$cm.  The current best upper limit set by Sussex/RAL/ILL nEDM experiment is $3.0 \times 10^{-26}$~e$\cdot$cm\cite{bestLim_1,bestLim_2}.  The nEDM experiment at TRIUMF is aiming at the $10^{-27}$~e$\cdot$cm sensitivity level.  We are developing the world's highest density source of ultracold neutrons (UCN).  The experiment requires a very stable ($<$~pT) and homogeneous ($<$~nT/m) magnetic field ($B_0$) within the measurement cell.  

My involvement in the nEDM experiment is the development of active magnetic shielding to stabilize the external magnetic field by compensation coils. I have optimized a prototype active magnetic shield at the University of Winnipeg. I have also simulated the behaviour of the coils in the presence of the passive magnetic shields  using finite element analysis (FEA), and made comparisons with  experimental results  to test the successfulness of the control system. A major challenge of the active 
compensation system is its  slow current response.  This is now understood and several recommendations are made to improve the performance in future realizations of such a system.

%  This thesis discussed my journey to understand the system and recommendations for future researchers in active compensation field.



% .Moreover, the magnetic environment at TRIUMF is more challenging than in our lab in Winnipeg, because of the closeness of the experiment to the TRIUMF cyclotron (B $\sim 350 - 400$ $\mu$T 'which is almost one order of magnitude larger than usual background fields') and the changing environment with iron.  Studies of the implementation at TRIUMF will also be reported.

\end{abstract}

\newpage
\tableofcontents
\addcontentsline{toc}{section}{Table of Contents}
%comment these out if you don't want a detailed list of figures and tables!
\listoffigures
\listoftables

\begin{acknowledgments}
\vspace{2em}

Being an engineering student, I found the alleyways were topsy-turvy during the tenure of my Masters journey. I stepped in with a high hope, stumbled sometimes in the middle but always found the right guidance and motivation for pursuing my objective. 

Firstly, I must recognize the contribution of my mentor and guide Dr.~Jeff Martin for taking the risk, keeping faith on my ability and most importantly allowing me the opportunity to experience the package under him. More so, Dr.~Martin has given me unparalleled freedom to feel a foreign land as my second home despite located 12000~km away from my motherland. He has always guided me whenever I got stuck in my research. He has given me the opportunity to attend several conferences all over Canada beyond the imagination of a Masters student. His contributions to my research as well as while writing this thesis are phenomenal. His mentoring and passion for exploring the mysteries of this project motivated me in every step. He has been influential in my growth as a student as well as a better researcher.  


% By the grace of almighty Allah, I am finishing my Masters in Physics. It was a very tough decision to switch from engineering into the world of particle physics. I am grateful to my supervisor Dr.~Jeff Martin for believing in me, and always be kind throughout my Masters journey. 

Next important person who deserves special mention is David Ostapchuk. I had absolutely no idea how and from where to begin my research. I am grateful to many whom I, formally or informally, have discussed, and consulted in connection to my research, but amongst all Dave is the most resourceful person. He has helped me graciously to settle in, write codes as well as build apparatus for my research project. In addition to my supervisor, whenever I faced problems, I knew that I could always rely on Dave for solution.


I would like to thank my co-supervisor Dr.~Chris Bidinosti for his generous help and support during my Masters journey. I would also like to thank Dr.~Peter Blunden and Dr.~Cyrus Shafai for serving to my committee. In addition to my supervisor, co-supervisor, and committee members, I want to thank Dr.~Takashi Higuchi for reviewing my thesis. I am grateful to Dr.~Russell Mammei for his contribution in OPERA simulation. I am also grateful to Dr.~Beatrice Franke and Dr.~Blair Jamieson for their valuable suggestions during my research work. 


The graduate students particularly Dr.~Taraneh Andalib, Sakib Rahman, Michael Lang, Moushumi Das, Ray Dwaipayan, Sean Hansen-Romu and Wolfgang Klassen were especially cooperative to me. I am thankful to each of them. 

Last but not the least, I want to thank my family specially my mother and also my friends who have supported me always. Above all, I am grateful to the almighty Allah for His endless mercy.

It has been an honor to be part of this wonderful project and be around extraordinary scholars.

\vspace{2em}
Thank you all.

-Shomi

\end{acknowledgments}

%
%\quotation
%\begin{quote}
%\hsp \em Is it the God's will or the Lotus flower's intention,  when it blooms in the mud?
%\end{quote}



\dedication
\vspace*{\fill}
\begin{center}
\begin{quote}
\hfil \hsp \Large \em Dedicated to - My father whom I lost during my Masters journey. May Allah rest his soul and place him in heaven.\hfil
\end{quote}
\vspace*{\fill}
\end{center}
\newpage
\cleardoublepage
\thispagestyle{empty}
\startarabicpagination
%Use 'startarabicpagination' to use both numbers and roman letters for pages
%%% end
