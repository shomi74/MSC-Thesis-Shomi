%your front matter - fill in your personal details here!
%%

\title{Active Magnetic Compensation Prototype for Neutron Electric Dipole Moment Experiment}
\author{Shomi Ahmed}

\degreemonth{April} % month final submission occurs.
\degreeyear{2019}%
\degree{Master of Science}%
\department{Physics & Astronomy}%
\advisor{Dr. Jeff Martin and Dr. Chris Bidinosti} %



%\copyrightpage
% Insert a blank page for two-sided
%  \newpage
%  \thispagestyle{empty}
%  \hbox{}
%  \newpage


\maketitle

\begin{abstract}
The existence of a non-zero neutron electric dipole moment (nEDM) would violate parity and time-reversal symmetry.  Extensions to the Standard Model predict the nEDM to be $10^{-26}$ -- $10^{-28}$~e$\cdot$cm.  The current best upper limit set by Sussex/RAL/ILL nEDM experiment is $3.0 \times 10^{-26}$~e$\cdot$cm\cite{bestLim_1,bestLim_2}.  The nEDM experiment at TRIUMF is aiming at the $10^{-27}$~e$\cdot$cm sensitivity level.  We are developing the world's highest density source of UCN.  The experiment requires a very stable ($<$~pT) and homogeneous ($<$~nT/m) magnetic field ($B_0$) within the measurement cell.  

My involvement in the nEDM experiment is the development of active magnetic shielding to stabilize the external magnetic field by compensation coils. I have optimized a prototype active magnetic shield at The University of Winnipeg. I have also simulated the behaviour of the coils in the presence of the passive magnetic shields  using finite element analysis (FEA), and made comparisons with  experimental results  to test the successfulness of the control system. A major challenge of the active 
compensation system is its  slow current response.  This is now understood and several recommendations are made to improve the performance in future realizations of such a system.

%  This thesis discussed my journey to understand the system and recommendations for future researchers in active compensation field.



% .Moreover, the magnetic environment at TRIUMF is more challenging than in our lab in Winnipeg, because of the closeness of the experiment to the TRIUMF cyclotron (B $\sim 350 - 400$ $\mu$T 'which is almost one order of magnitude larger than usual background fields') and the changing environment with iron.  Studies of the implementation at TRIUMF will also be reported.

\end{abstract}

\newpage
\tableofcontents
\addcontentsline{toc}{section}{Table of Contents}
%comment these out if you don't want a detailed list of figures and tables!
\listoffigures
\listoftables

\begin{acknowledgments}
\vspace{2em}

By the grace of almighty Allah, I am finishing my Masters in Physics although I have an Engineering background. I am grateful to my supervisor Dr. Jeff Martin for taking the risk, for believing in me, giving me the chance and always be kind with me throughout my Masters journey. He has given me unparalleled freedom to feel a foreign land which is $\sim12000$~km away from my motherland as my second home. I must say that all of my friends are jealous of my super kind supervisor. He has always given me ideas whenever I have stucked in the research and helped me solve the problem. He has given me the opportunity to discover all over Canada attending conferences beyond the imagination of a Masters student. His contribution throughout my thesis writing journey is phenomenal. I have learned a lot from him.

Next important person who made my Masters life smoother is David Ostapchuk. He is a very talented person with enormous patience. I had absolutely no idea how and from where to begin my research. Dave helped me throughout my first summer to settle in, understand, write codes, and build apparatus for my research project. In addition to my supervisor, whenever I faced problems, I knew that I could always rely on Dave for solution.

I would also like to thank my co-supervisor Dr.~Chris Bidinosti, for his generous help and support during my graduate studies. I am grateful to Dr.~Russell Mammei to help me understand OPERA simulation and allow me to use his codes. I am also grateful to Dr.~Beatrice Franke and Dr.~Blair Jamieson for their valuable suggestions during my research work. 

The graduate students specially Dr.~Taraneh Andalib, Sakib Rahman, Michael Lang, Moushumi Das, Ray Dwaipayan, Sean Hansen-Romu and Wolfgang Klassen have helped me a lot. I am thankful to each of them. 

Last but the least, I want to thank my family and friends who have supported me always. 

\vspace{2em}
Thank You All.

-Shomi

\end{acknowledgments}

%
%\quotation
%\begin{quote}
%\hsp \em Is it the God's will or the Lotus flower's intention,  when it blooms in the mud?
%\end{quote}



\dedication
\vspace*{\fill}
\begin{center}
\begin{quote}
\hfil \hsp \Large \em Dedicated to - My father whom I lost during my Masters journey. May Allah rest his soul and place him in heaven.\hfil
\end{quote}
\vspace*{\fill}
\end{center}
\newpage
\cleardoublepage
\thispagestyle{empty}
\startarabicpagination
%Use 'startarabicpagination' to use both numbers and roman letters for pages
%%% end
