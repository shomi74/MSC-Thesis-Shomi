%change to one side if you want single sided.  Two side will properly
%alternate margins and page numbers for two-sided binding.
\documentclass[letterpaper,12pt,oneside,final]{thesis}
%\documentclass[11pt,one side,final]{Thesis}
\usepackage[section]{placeins}
\usepackage{epsfig,bm,epsf,float}
\usepackage{cite}
\usepackage{layouts}
\usepackage[font=singlespacing]{caption}
%%%%%%%%%%%%%%%%%%%%%%%%%%%%%%%%%%%%%%%%%%%%%%%%%%
%% Image Macros
% Single figure macro
\newcommand{\fig}[4]{\begin{figure}[!htb]
\centering
\includegraphics[{#2}]{{#1}}
\caption[#4]{#3}
%\caption{#3}
\end{figure}}

% Double figure, side by side macro
\newcommand{\doublefig}[8]{\begin{figure}[!htb]
\centering
\begin{subfigure}{0.49\textwidth}
  \centering
  \includegraphics[{#2}]{#1}
  \caption{#3}
\end{subfigure}%
\begin{subfigure}{0.49\textwidth}
  \centering
  \includegraphics[{#5}]{#4}
  \caption{#6}
\end{subfigure}
% \caption{#7}
\caption[#8]{#7}
\end{figure}}

%%%%%%%%%%%%%%%%%%%%%%%%%%%%%%%%%%%%%%%%%%%%%%%%%%%%%%%%%%%%%%%%%%%%%
\usepackage[pdfpagemode={UseOutlines},bookmarks=true,bookmarksopen=true,
   bookmarksopenlevel=0,bookmarksnumbered=true,hypertexnames=false,filecolor=magenta,colorlinks=true,linkcolor={blue},citecolor={cyan},urlcolor={red},
   pdfstartview={FitV},unicode,breaklinks=true]{hyperref}
   
%hyperref is nice, I find - it turns all your references (to citations, figures, etc) into hyperlinks,
% so someone reading electronically can jump around your document.  It makes no visible marks on a
   
%'makecell' For making newline within a box in a table
\usepackage{makecell}
%%%%%%%%%%%%
\usepackage{amsmath,paralist,amssymb,multirow}
\usepackage{amsthm,framed,enumitem}
\usepackage{lineno,amssymb,graphicx}
%%%%%%%%%%%%
\usepackage{bm}% bold math

\newcommand{\muo}{ \mu_{\scriptscriptstyle 0}}
%%%%%%%%%%%%%%%%%%%%%%
\usepackage{pgfplots}
\usepackage{algorithmic}
\usepackage{subcaption}

\usepackage{xcolor,colortbl,hhline}
\usepackage[ruled,vlined,linesnumbered]{algorithm2e}

\usepackage{pifont}% http://ctan.org/pkg/pifont


% Table of contents max depth listed:
% 1 = section, 2 = subsection, 3 = subsubsection
\setcounter{tocdepth}{3}
\pgfplotsset{compat=newest}

\begin{document}

% Single spacing: takes place of `draft' mode, without losing figures.
%  \ssp
% makes double-spaced
 \dsp

%your front matter - fill in your personal details here!
%%

\title{Active Magnetic Compensation Prototype for Neutron Electric Dipole Moment Experiment}
\author{Shomi Ahmed}

\degreemonth{March} % month final submission occurs.
\degreeyear{2019}%
\degree{Master of Science}%
\department{Physics & Astronomy}%
\advisor{Dr. Jeff Martin and Dr. Chris Bidinosti} %



%\copyrightpage
% Insert a blank page for two-sided
%  \newpage
%  \thispagestyle{empty}
%  \hbox{}
%  \newpage


\maketitle

\begin{abstract}
The existence of a non-zero neutron electric dipole moment (nEDM) would violate parity and time-reversal symmetry.  Extensions to the Standard Model predict the nEDM to be $10^{-26}$ -- $10^{-28}$ e-cm.  The current best upper limit set by Sussex/RAL/ILL nEDM experiment is $3.0 \times 10^{-26}$ e-cm\cite{bestLim_1,bestLim_2}.  The nEDM experiment at TRIUMF is aiming at the $10^{-27}$ e-cm sensitivity level.  We are developing the world's highest density source of UCN.  The experiment requires a very stable ($<$~pT) and homogeneous ($<$~nT/m) magnetic field (B0) within the measurement cell.  My involvement in the nEDM experiment is the development of active magnetic shielding to stabilize the external magnetic field by compensation coils.  I have optimized a prototype active magnetic shield at The University of Winnipeg. I have also simulated the shield and coils using Finite Element Analysis (FEA). I have compared my experimental results with the simulation results to test the successfulness of the control system. I have faced problems in terms of slow coil current response for correcting the magnetic field change. I have successfully understand the slow current response problem. This thesis discussed my journey to understand the system and recommendations for future researchers in active compensation field.

% .Moreover, the magnetic environment at TRIUMF is more challenging than in our lab in Winnipeg, because of the closeness of the experiment to the TRIUMF cyclotron (B $\sim 350 - 400$ $\mu$T 'which is almost one order of magnitude larger than usual background fields') and the changing environment with iron.  Studies of the implementation at TRIUMF will also be reported.

\end{abstract}

\newpage
\tableofcontents
\addcontentsline{toc}{section}{Table of Contents}
%comment these out if you don't want a detailed list of figures and tables!
\listoffigures
\listoftables

\begin{acknowledgments}
\vspace{5em}

\end{acknowledgments}

%
%\quotation
%\begin{quote}
%\hsp \em Is it the God's will or the Lotus flower's intention,  when it blooms in the mud?
%\end{quote}



%\dedication
%\vspace*{\fill}
%%\begin{center}
%\begin{quote}
%\hfil \hsp \Large \em Dedicated to all the sacrifices ...\hfil
%\end{quote}
%\vspace*{\fill}
%\end{center}
%\newpage
%\cleardoublepage
%\thispagestyle{empty}
\startarabicpagination
%Use 'startarabicpagination' to use both numbers and roman letters for pages
%%% end


% Chapters: list all the chapter latex files you want included here, in order

 \chapter{Motivation Behind the nEDM Experiment}
\lhead{\emph{Motivation Behind the nEDM Experiment}}\label{ch:motivation} 
This chapter highlights the scientific interest in a new precise measurement of the neutron dipole moment (nEDM). Then, the measurement principle of the nEDM experiment is discussed with the importance of the magnetic environment for the successfulness of the experiment. Finally, this chapter ends with describing the TRIUMF Ultra Cold Advenced Neutron (TUCAN) nEDM experiment.

\section{Baryon Asymmetry}

\fig{Images/SM_Pic2}{width = 0.8\textwidth}{Three generations of matter aligned column-wise and classified into two groups : quarks and leptons. The figure is modified from the original figure of standard model of elementary particles of Ref.~\cite{SM_Pic}.\label{fig:SM_Pic}}{Three generations of matter.}

The universe is created from some fundamental particles which are governed by the four fundamental forces (electromagnetic, weak interaction, strong interaction and gravitational force). The Standard Model (SM) has been created as a theory to classify the particles and relation among three forces excluding gravity. Six of each quarks and leptons are the fundamental particles which classify the matters based on the quantum numbers namely baryon carried by quarks and lepton number carried by leptons. They are shown in Fig.~\ref{fig:SM_Pic}. In the early universe, there were equal particle numbers of matter and antimatter. But the universe today contains almost only matter. By experimental observations of the cosmic microwave background radiation (CMBR) the baryon asymmetry can be deduced~\cite{expBar}.

\begin{equation}\label{eq:baryons}
    \eta =\frac{n_b-\bar{n}_b}{n_\gamma}\simeq6.0 \times 10^{-10}
\end{equation}
where $n_b$, $\bar{n}_b$ and ${n_\gamma}$ are the number of baryons, anti-baryons and photons respectively. 

% So, this $6.0 \times 10^{-10}$ excess baryons over anti baryons per photon is called the baryon asymmetry of the universe. Scientists are continuously searching for the answer of this mystery. The breakthrough is possible by finding the non zero value of nEDM. 

The relation of the non zero nEDM with the baryon asymmetry has been discussed next.

\section{Sakharov Criteria and nEDM}
Baryogenesis is the process of creation of a baryon asymmetry from an initially symmetric state. In 1967, A.D. Sakharov described following three conditions on particle theories aiming to explain barayogenesis~\cite{Sakharov:1967dj}
\begin{itemize}
    \item Baryon (B) number violation.
    \item Charge (C) and Charge-Parity (CP) symmetry violation.
    \item Departure from thermal equilibrium.
\end{itemize}

Any theory which creates net baryon number obviously requires B-violating processes. Particle and antiparticle reaction must be different otherwise net baryon creation would be balanced by net anti-baryon creation. In thermal equation, forward reaction rates would balance backwards rates and no net baryon number could be produced. A departure from thermal equation is therefore required. The Big Bang and subsequent cooling of the universe offers a way to provide the departure from thermal equation~\cite{sakharov_3rd_cond}. Electroweak baryogenesis is one scenario which uses SM processes and the electroweak phase transition to explain baryogenesis. One drawback of the electroweak baryogensis is that there is no enough CP violation in the SM. This motivates new sources of CP violation near the weak scale.

% The first criterion can be explained by the second and third conditions. During the expansion of the universe immediately after the big bang, most of the particles were departed from the thermal equilibrium ( defined by the temperature of the universe between $10^{2}$ GeV and $10^{12}$ GeV ) at $\sim 10^{19}$ GeV temperature as suggested by Sakharov \cite{sakharov_3rd_cond} resulting in net non zero B number. The non zero B number also exists due to C and CP asymmetry. Because in the absence of those violations, excess baryon reactions can be nullified by the excess anti-baryon reactions \cite{sakharov_1st_cond}. Though the standard model of particle physics possesses CP violation, there is an insufficient amount of CP violation to explain the observed baryon asymmetry.  Theories of new physics beyond standard model, motivated in part by the baryon asymmetry, predict new sources of CP violation.

A fundamental symmetry of quantum field theories is CPT symmetry, which implies that time-reversal (T) violation is equivalent to CP violation.  An electric dipole moment (EDM) is a measure of separation of oppositely charged particles within a system. To have a nonzero EDM, the system should not violate both parity (P) and time-reversal (T) inversion~\cite{edm_reason}. So, a search for a non-zero EDM represents a search for new physics that violates CP symmetry. The neutron may have an EDM with its magnitude depending on the nature and origin of the T violation~\cite{nEDM_reason}. A precise measurement of the nEDM could be a very important metric for explaining the baryon asymmetry on particle physics beyond the known interaction in the SM. 

% \section{Experimental Efforts So Far}\label{sec:lim}
% \fig{Images/lim}{width = \textwidth}{Experimental nEDM upperlimit over the years \cite{1_lim,2_lim,3_lim,4_lim,5_lim,6_lim,7_lim,8_lim,9_lim,10_lim,11_lim,12_lim,13_lim,14_lim,15_lim,16_lim,17_lim} along with theoretical predictions \cite{theory_lim_1, theory_lim_2, theory_lim_3}.\label{fig:lim}}

% Additional sources of CP violation beyond the standard model predict the nEDM to be in the range of $10^{-27}$ -- $10^{-28}$  e$\cdot$cm as compared to $10^{-33}$ -- $10^{-31}$ e$\cdot$cm predicted by the standard model \cite{theory_lim_1, theory_lim_2, theory_lim_3}. There are several experiments aiming at improving the uncertainty on the nEDM. The Fig. \ref{fig:lim} summarize all the results. Since the first measurement data published in 1957 \cite{1_lim}, the upper limit set on nEDM has been reduced by eight orders of magnitude over the last six decades. At 1980, ultra cold neutron (UCN) experiment overtook over neutron beam experiments on precision. The current best upper limit set by ILL/Sussex/RAL nEDM experiment is $3.0 \times 10^{-26}$ e$\cdot$cm (90 \% C.L) or $3.6 \times 10^{-26}$ e$\cdot$cm (96 \% C.L)  \cite{bestLim_1,bestLim_2}. The experiment has been done at the Institut Laue-Langevin (ILL) which is situated at Grenoble, France. A new generation of UCN source known as superthermal UCN source which uses a new method of cooling by transferring energy to quantum excitations in a material have recently come online. To use such a UCN source, the nEDM appa

% The apparatus for nEDM experiment has been moved from ILL to a superthermal UCN source at Paul Scherrer Institut (PSI) which is situated at Villigen, Switzerland. UCN superthermal source use the cooling technique . A new UCN source generation have recently come online. They use a technique from condensed
% matter physics involving cooling by transferring energy to quantum excitations in a material.
% UCN sources that employ this method of cooling are known as superthermal sources and they are
% beginning to transform the landscape of fundamental neutron physics at various facilities in the
% world.
% The nEDM apparatus from ILL was moved to such a UCN source at Paul Scherrer Institut
% (PSI, Villigen, Switzerland). The apparatus was also improved and upgraded in several respects.
% Data-taking for this new nEDM experiment was completed recently and the analysis of the data is
% ongoing [9]. The expectation in the community is that this new result will improve the previous
% best by a factor of about three to four.
% Next generation UCN EDM experiments are now in preparation at a variety of sites aiming to
% improve the result by an order of magnitude or more. Experiments are either ongoing or planned
% at ILL [10, 11], PSI [12], the Gatchina reactor [10], the Forchungsreaktor Munchen II (FRM2)
% reactor [13], Los Alamos National Laboratory (LANL, Los Alamos, NM, USA) [14], the Spallation
% Neutron Source (SNS, Oak Ridge, TN, USA) [15], and our eort at TRIUMF [16, 17]. We discuss
% our relationship to these experiments in Section 4. Our goal of dn < 10��27 ecm within the next
% 6-7 years (running until 2024-25, as stated earlier) is competitive with these eorts. One of the key
% factors is our unique UCN source, which we are upgrading. We envision achieving UCN counting
% rates over 100 times larger than the previous best nEDM experiment and the recently completed
% experiment at PSI, and similar to or surpassing the plans of other experiments.


% The nEDM experiment at TRIUMF is aiming to constrain the uncertainty on the nEDM at the $10^{-27}$ e$\cdot$cm sensitivity level. 


\section{Ultracold Neutrons}
Ultracold neutrons (UCN) are used to measure the nEDM. They have small kinetic energies ($<$ 300 neV). They can be confined in a material bottle as they are reflected at any angle of incidence off suitable material walls \cite{ucn_storage}. This provides long time frame (the mean lifetime of neutron is $\tau_n=881.5$ s \cite{mike}) for observation marking them ideal media for nEDM experiment. Their interaction with the neutron optical potential of the walls through the strong force enables to trap them. First UCN were produced in 2017 for TUCAN nEDM experiment using superfluid-helium at 0.9 K as the UCN production medium~\cite{TRIUMF_UCN,taraneh_theis}.

\section{Measurement Principle of nEDM}\label{sec:nEDM}

For the extraction of nEDM, Ramsey's method of separated oscillatory fields~\cite{ramsey} will be used. UCN will be polarized by passage through a strong magnetic field and guided to the nEDM cell. 

\fig{Images/ramsey}{width =\textwidth}{Ramsey's method of separated oscillatory fields, as applied to the measurement of the nEDM. $\bm{E_o}$ field parallel to $\bm{B_o}$ field in the left where in the right they are antiparallel. \label{fig:ramsey}}{Ramsey's method of separated oscillatory fields.}

%\FloatBarrier

The Ramsey's method of separated oscillatory fields for nEDM measurement has been shown in detail in Fig.~\ref{fig:ramsey}. It is seen that at the beginning and end of free precession short $\pi$/2 pulses are applied and polarized neutron detection after the pulse sequence is used to measure the free spin precession frequency $v$ (the Larmor frequency). In the first instance (left one in Fig.~\ref{fig:ramsey}), an electric field ($\bm{E_o}$) parallel to the magnetic field ($\bm{B_o}$) will be applied giving a spin-precession frequency as
\begin{equation}\label{eq:up_freq}
    h v_{\Uparrow \Uparrow}=2\mu_n\bm{B_o}+2 d_n\bm{E_o}
\end{equation}
where, $\mu_n$ and $d_n$ are the magnetic moment and electric dipole moment respectively, $h$ is the Planck's constant and arrows indicate parallel orientation of $\bm{E_o}$ and $\bm{B_o}$.
Now the same experiment is repeated with anti-parallel $\bm{E_o}$ (right one in Fig.~\ref{fig:ramsey})which gives the spin-precession frequency
\begin{equation}\label{eq:down_freq}
    h v_{\Uparrow \Downarrow}=2\mu_n\bm{B_o}-2 d_n\bm{E_o}
\end{equation}
where, the arrows indicate anti-parallel orientation of $\bm{E_o}$ and $\bm{B_o}$.
The measured change in the precession frequency (using Eq.~(\ref{eq:up_freq}) and Eq.~(\ref{eq:down_freq})) can be used to deduce the nEDM via
\begin{equation}\label{eq:nEDM}
    d_n=\frac{h (v_{\Uparrow \Uparrow}-v_{\Uparrow \Downarrow})}{4\bm{E_o}}
\end{equation}

Since the NMR frequency is proportional to the magnetic field in the nEDM cell, the requirement is to have a very stable and homogeneous $\bm{B_o}$ field within the cell.


\section{Experimental Efforts So Far}\label{sec:lim}

Additional sources of CP violation beyond the standard model predict the nEDM to be in the range of $10^{-27}$ -- $10^{-28}$  e$\cdot$cm as compared to $10^{-33}$ -- $10^{-31}$ e$\cdot$cm predicted by the standard model \cite{theory_lim_1, theory_lim_2, theory_lim_3}. There are several experiments aiming at improving the uncertainty on the nEDM. 

\fig{Images/lim}{width = 0.9\textwidth}{Experimental nEDM upperlimit over the years \cite{1_lim,2_lim,3_lim,4_lim,5_lim,6_lim,7_lim,8_lim,9_lim,10_lim,11_lim,12_lim,13_lim,14_lim,15_lim,16_lim,17_lim} along with theoretical predictions \cite{theory_lim_1, theory_lim_2, theory_lim_3}. The vertical dashed line indicates the introduction of UCN. The light green region indicates the the nEDM limit in SUSY, M-theory and others while he light red region indicates for SM. \label{fig:lim}}{Experimental nEDM upperlimit over the years}

Figure~\ref{fig:lim} summarize all the results. Since the first measurement data published in 1957 \cite{1_lim}, the upper limit set on nEDM has been reduced by eight orders of magnitude over the last six decades. At 1980, ultra cold neutron (UCN) experiment overtook over neutron beam experiments on precision. The current best upper limit set by ILL/Sussex/RAL nEDM experiment is $3.0 \times 10^{-26}$ e$\cdot$cm (90 \% C.L) or $3.6 \times 10^{-26}$ e$\cdot$cm (96 \% C.L) \cite{bestLim_1,bestLim_2} and the experiment is performed at Institut Laue-Langevin (ILL, Grenoble, France). The new $^{199}\mathrm{Hg}$ EDM measurement constrains the nEDM better than direct nEDM measurements, $d_n$ $<$   $\mathrm{1.6\times10^{-26}}$ e$\cdot$cm, although subject to uncertainty from Schiff screening~\cite{schiff_screen}. The TUCAN nEDM experiment is aiming to constrain the uncertainty on the nEDM at the $10^{-27}$ e$\cdot$cm sensitivity level. 

The Paul Scherrer Institut (PSI, Villigen, Switzerland) nEDM experiment uses an improved version of the former ILL/Sussex/RAL single-cell apparatus. Several innovations have been made at PSI, including a new SD2 spallation-driven UCN source. The experiment employs several Cs magnetometers outside the EDM cell, and a $^{199}\mathrm{Hg}$ comagnetometer. Active magnetic shielding and other environmental controls have been improved. A new detector that can simultaneously count both spin states of UCN has also been implemented. The final sensitivity expected is $\mathrm{10^{-26}}$ e$\cdot$cm~\cite{psi}. Some of the chief improvements made at PSI recently have been in the area of nearby alkali atom (Cs) magnetometry, Hg comagnetometry, and neutron magnetometry. A recent achievement at PSI is the understanding of the Cs magnetometer signals in terms of magnetic field gradients internal to the magnetic shielding. This has led to a detailed understanding of the false EDM of the Hg comagnetometer~\cite{psi_falseEDM}. Another recent achievement is in using the neutrons themselves to measure gradients~\cite{psi_n_gradient}. PSI also aims to improve their magnetometry with $^3\mathrm{He}$ magnetometers inside the electrodes of the double EDM measurement cells for their future n2EDM effort. They have performed R$\&$D using Cs magnetometers to sense the free-induction decay signal from $^3\mathrm{He}$, which resulted in a new high-precision magnetometer possessing excellent long-term stability\cite{psi_magnetometer}. The precision goal for n2EDM is $5 \times 10^{-28}$ e$\cdot$cm~\cite{psi_n2edm_nEDM-workshop,psi_n2edm_PPNS-workshop}.

The nEDM collaboration at Spallation Neutron Source (SNS, Oak Ridge, TN, USA) plans to measure $d_n\approx$ $2\times10^{-28}$ e$\cdot$cm, two orders of magnitude improvement from the current limit~\cite{sns_lim}. They plan to use a unique experimental technique. A Cold Neutron (CN) beam from the SNS will impinge upon a volume of superfluid $^4\mathrm{He}$ creating UCN. The nEDM measurement will also be conducted in the superfluid. A small amount of polarized $^3\mathrm{He}$ introduced into the superfluid $^4\mathrm{He}$ will act as both a comagnetometer and spin analyzer for the UCN. The $^3\mathrm{He}$ neutron capture rate is strongly spin dependent, and will beat at the difference of the Larmor precession frequencies of the neutrons and $^3\mathrm{He}$. A non-zero EDM would change the beat frequency with E-reversal. Scintillation light produced in the superfluid will be used to detect the capture products. The target precision is $10^{-28}$ e$\cdot$cm The false EDM of the $^3\mathrm{He}$ comagnetometer may be reduced by collisions in the surrounding $^4\mathrm{He}$~\cite{sns_false_edm}. The group aims to commission the experiment at SNS by 2020.

A new room-temperature nEDM experiment will be conducted using an upgraded
Los Alamos National Laboratory (LANL, Los Alamos, NM, USA) UCN source~\cite{lanl_nEDM-workshop}. The aim of the project is to increase the UCN density by a factor of five to ten, which could then be used to carry out a $\sim$ $10^{-27}$ e$\cdot$cm determination of the nEDM. The experiment aims for completion of a $10^{-27}$ level result, to be completed in the years prior to the SNS nEDM experiment, which shares a number of collaborators. Two other room temperature nEDM experiments are being pursued at the Forchungsreaktor Muncheon II (FRM2) reactor in Munich~\cite{frm2} and Gatchina reactor at ILL~\cite{PNPI}. Both experiments feature double measurement cells and Cs magnetometers internal to the innermost magnetic shield. The Munich effort features an impressive new effort in active and passive magnetic shielding~\cite{msr_design,shield_pnpi,shield_pnpi2}, and uses $^{199}\mathrm{Hg}$ comagnetometer. The ILL/Gatchina experiment has produced results at ILL~\cite{PNPI}. This could be improved in further runs at ILL in the EDM position, or in runs using the superfluid He UCN source at ILL, where a statistical sensitivity of $3.5 \times 10^{-27}$ e$\cdot$cm could be obtained~\cite{pnpi_nEDM-workshop}. The group will build a UCN source at the WWR-M reactor in Gatchina in order to increase the UCN flux.

\section{TUCAN nEDM Experiment}


\fig{Images/tucan}{width =\textwidth}{Conceptual design of the proposed TUCAN source and nEDM Experiment. The major portion of the biological shielding is not shown. Protons strike a tungsten spallation target. Neutrons are moderated in the $\mathrm{LD_2}$ cryostat and become UCN in a super fluid $^4\mathrm{He}$ bottle within, which is cooled by the superfluid $^4\mathrm{He}$ cryostat. UCN pass through guides and the superconducting magnet (SCM) to reach the nEDM experiment located within a magnetically shielded room (MSR). Simultaneous spin analyzers (SSA's) detect the UCN at the end of each nEDM experimental cycle. \label{fig:tucan}}{Conceptual design of the proposed TUCAN source and nEDM Experiment.}

The proposed TUCAN facility with designated place for nEDM experiment is shown in Fig.~\ref{fig:tucan}. A proton beam at 480 MeV and 40 $\mu$A from the cyclotron impinges upon the tungsten spallation target liberating fast neutrons. Above the target is a neutron moderator system containing liquid deuterium ($\mathrm{LD_2}$) which creates a large flux of cold neutrons (CN). The CN enter a bottle surrounded by the $\mathrm{LD_2}$ which contains superfluid $^4\mathrm{He}$ at below 1 K. In the superfluid, the CN excite phonon and roton transitions, losing virtually all their kinetic energy to become ultracold. 

Once a sufficient density of UCN has built up, a UCN valve opens. The UCN are transported out of the source by reflection on the surfaces of UCN guides. A superconducting magnet (SCM) transmits one neutron spin orientation in the magnetic field, giving near-unity UCN polarization and facilitating transmission through a vacuum-isolation foil at room temperature. The UCN are then transported to the nEDM experiment by additional guides. At the end of each nEDM experiment cycle, simultaneous spin analyzers (SSA's) detect the UCN.


In the upcoming chapter, the magnetic field system around the EDM measurement cell has been discussed in detail.



% that are slowed in in a room-temperature neutron moderator composed of lead, graphite and ${D_2}$O and converted to cold neutron (CN) by L${D_2}$ in a bottle having superfluid He-II below 1K. Finally, they are reduced to ultra cold speeds by phonon scattering in superfluid helium. After generating sufficient density of UCN, the proton beam is turned off opening a cryogenic UCN valve. The UCN then transported by reflection on the surfaces UCN guides and to accelerate polarized UCN through barrier foils to a vacuum volume at room temperature, a superconducting magnet (SCM) has been used. At last , UCN's are transported to the nEDM experiment via additional guides. At the end of each nEDM experiment cycle, simultaneous spin analyzers (SSA's) detect the UCN. 






 \thispagestyle{plain}
 \lhead{\emph{Overview of Magnetic Field Systems}}
\chapter{Overview of Magnetic Field Systems}\label{ch:magnetics}

% \begin{itemize}
%     \item Literature Review
%     \item Literature Review of nEDM Experiments
%     \item Literature Review of Current Approaches to Magnetic Field
%     \item Magnetic Environment at TRIUMF
%     \item Magnetic Requirements for TRIUMF
%     \item Design Concepts for TRIUMF
% \end{itemize}


In Chapter~\ref{ch:motivation}, the motivation behind nEDM experiment alongwith measurement principle of nEDM have been described. The TUCAN nEDM experiment has been also introduced there. This chapter describes about the magnetic requirement for the TUCAN nEDM experiment. It also describe the magnetic subsystems which are required to maintain the label of precision for the experiment to be successful. Moreover, the chapter also described the current nEDM status worldwide and also the use of active shielding for nEDM measurement. Finally, the chapter ends with  describing the main objective for this thesis. 

 \section{Magnetic field requirements for the TUCAN nEDM experiment}\label{sec:msr}
% To achieve $10^{-27}$ e$\cdot$cm sensitivity level for TUCAN nEDM experiment, it must be ensured to have a magnetic field of $\sim$1 pT stability i.e. $<$pT drift in one nEDM measurement cycle and $\sim$1 nT/m homogeneity. So, the magnetic field should be precisely monitored and controlled which imply that the coil generating the field must be better than that level. The nEDM experiment itself along with passive shielding will be under a Magnetically Shielded Room (MSR) and around which compensation coils will be placed for active shielding which needs to be designed as shown in Fig. \ref{fig:msr}. 
% A stable and homogeneous magnetic field $\bm{B_o}$ ($\sim$1 $\mu$T) is required to achieve the desired $10^{-27}$ e$\cdot$cm sensitivity level for TUCAN nEDM experiment. 

A stable and homogeneous magnetic field is required for the TUCAN nEDM experiment. The magnetic field should be precisely monitored and controlled which imply that the coil generating the field must be better than that level. The magnetic stability upper limit is $\sim$1 pT i.e. $<$pT drift in one nEDM measurement cycle and the magnetic homogeneity upper limit is $\sim$1 nT/m. For compensating $\bm{B_0}$ field fluctuations,  $^{199}Hg$ co-magnetometer will be used. To nullify the uncontrolled and time-varying external fields both active and passive shielding will be utilized. The nEDM experiment itself along with passive shielding will be under a Magnetically Shielded Room (MSR) and around which compensation coils will be placed for active shielding which needs to be designed. The schematic diagram of the magnetic components of the
TUCAN nEDM experiment is shown in Fig.~\ref{fig:msr}. Each of the magnetic component is explained below. 


\fig{Images/msr_pic}{width = \textwidth}{Schematic diagram for the TUCAN nEDM magnetics. From inside out: The UCN and the co-magnetometers followed by the internal coil system (  $\bm{B_0}$ and B $\bm{B_1}$ coils) to generate the magnetic fields for the Ramsey cycle. Outside to that there are four layers of passive shielding which made the magnetically shielded room (msr). The active compensation system which needs to be designed is just outside the msr.\label{fig:msr}}{Short}
 
% As explained in Section~\ref{sec:lim} and Section~\ref{sec:nEDM} that to achieve $10^{-27}$ e$\cdot$cm sensitivity level for TUCAN nEDM experiment, it must be ensured to have a magnetic field of $\sim$1 pT stability and $\sim$1 nT/m homogeneity. So, the magnetic field should be precisely monitored and controlled which imply that the coil generating the field must be better than that level. The nEDM experiment itself along with passive shielding will be under a Magnetically Shielded Room (MSR) and around which compensation coils will be placed for active shielding which needs to be designed as shown in Fig. \ref{fig:msr}. 
%  \fig{Images/exp2}{width = 0.8\textwidth}{Schematic diagram of prototype active compensation system at University of Winnipeg.\label{fig: active}}



% \doublefig{Images/msr_pic}{width =\textwidth,height =7cm}{Schematic diagram \label{fig:msr_pic}}{Images/msr_sketch}{width = \textwidth,height =7cm}{Photograph\label{fig:msr_sketch}}{{Schematic diagram for the TUCAN nEDM magnetics. From outside in: The
% active compensation system followed by several layers of magnetically shielded room and
% passive shields nullify the environmental magnetic eld. The magnetometers inside the
% active shielding monitor the changes in the magnetic eld internal to that region. The
% internal coil system (B0 and B1 coils) generate the magnetic elds for the Ramsey cycle.
% The UCN and the co-magnetometers are internal to the coils. } \label{fig:msr}}

\FloatBarrier


%  Furthermore, a multi-layer passive shielding system must reduce and stabilize the field around the internal coil. Self-shielded $\bm{B_0}$ coils and shim coils are the options for internal coils around the nEDM cells due to their immunization capability from the field perturbations that is induced by the changes in the magnetic permeability of the passive shields arising from temperature fluctuations \cite{Andalib_temp}. The passive shielding's magnetization must be tailored with degaussing and the shields themselves stabilized mechanically and thermally.  Finally, the magnetic environment of the passive shielding system must be stabilized.  The active shielding will control the magnetic field immediately outside the outermost passive shielding layer.   This thesis came into play for possible designing concepts of active shielding which will be discussed next.


% to the $\pm$ 1 nT level in the 10 Hz to DC frequency range. 


\subsection{Co-magnetometer}
Comagnetometry enables a measurement of the magnetic field inside the nEDM cell while the nEDM measurement is being conducted. It is the only way to correct for possible false EDM's caused by time-varying leakage currents. 

A comagnetometer is used to measure and correct for  $\bm{B_o}$ field drifts. The comagnetometer is an atomic species (usually $^{199}\text{Hg}$). In the comagnetometer, optical pumping is used to polarize a vapor of mercury atoms  which are then intoduced into the
nEDM cell at the same time as the neutrons, and the spin-precession frequencies of both species are measured simultaneously. Comagnetometer and neutrons motion in the EDM cell in the presence of a magnetic gradient causes frequency shifts that reverse sign with $\bm{E}$ reversal \cite{comag_1,comag_2,comag_3}. Both the comagnetometer and UCN are affected, but comagnetometer effects tend to dominate due to the higher (thermal) velocity of the atoms. So, precession frequencies of the comagnetometer can be used to normalize the magnetic field drifts. Normally, drifts of 1-10 pT in $\bm{B_o}$ field may be corrected using the comagnetometer technique in a typical nEDM experiment. The design of the $^{199}\mathrm{Hg}$ comagnetometer in the TUCAN ndem experiment will be similar to that employed in the previous ILL nEDM experiment \cite{bestLim_1,comag_4}.


Besides that a number of atomic magnetometers are placed just outside the
nEDM measurement cell. They are used to characterize magnetic homogeneity and stability. The chief purpose is to characterize gradients, in order to characterize the leading contributions to false EDM's arising from Hg and UCN motion in the EDM cell.

\subsection{Internal Coils}

A static $\bm{B_0}$ field ($\sim$1 $\mu$T) as well as an oscillating $\bm{B_1}$ field ($\sim$30 Hz) are required in the Ramsey Resonance method for $\pi$/2 spin reorientation. In addition, shim coils which carry a relatively small current are required to null remnant transverse fields and gradients over the nEDM measurement cell

$\bm{B_0}$ coils will be based on self-shielded coils which prevent eddy currents (loops of electrical current induced by a changing magnetic field) from being generated in the first place. The self-shielded $\bm{B_0}$ coils and shim are considered to be installed in the msr due to their immunization capability from the field perturbations that is induced by the changes in the magnetic permeability of the passive shields arising from temperature fluctuations \cite{Andalib_temp}. High-precision current supplies ($\sim$ 1 ppm) will be used to drive all internal coils, regardless of design. AC coils will apply $\pi$/2 pulses for the UCN and comagnetometer species, to initiate free spin precession.



\subsection{Passive Shield}\label{sec:passive}
The task of the passive magnetic shielding system is to provide a magnetically stable environment to perform the precision low field NMR spectroscopy on the neutrons. The TUCAN nEDM experiment will employ a magnetically shielded room (MSR) consisting of four nested mu-metal enclosures as its passive shielding, conceptually similar to Ref.~\cite{msr_design}.  The passive shield is designed so that the magnetic fields inside are stable enough to be be measured by
our precision comagnetometer and magnetometers discussed earlier. 

% It includes a degaussing/idealization system used to reduce remnant magnetization in the shield layers which further improves the stability.

During quiet times at TRIUMF, fuctuations in the magnetic feld are $\sim$100 nT. An MSR with a quasi-static shielding factor of $\sim$ 100,000 is sufficient to reduce these fuctuations to the $\sim$ pT level, beyond which a comagnetometer must be used to correct the field to the $\sim$ 10 fT level. A four-layer MSR with an inner cubic space of side-length 1.8 m and outer side-length 2.8 m produces this shielding factor, with mu-metal wall thicknesses 2 mm, 6 mm, 4 mm, 4 mm (inner to outer), equally spaced.


\subsection{Active Magnetic Field Compensation (AMC)}\label{sec:amc}

The environment of the TUCAN nEDM experiment needs to be thermally and vibrationally isolated and controlled, primarily to avoid small changes and slow drifts in the properties of the passive magnetic
shielding. 

A particularly challenging aspect is that the TUCAN nEDM experiment will be in fringe field of TRIUMF cyclotron, at a location where the fringe could be as large as eight times the magnetic field of the earth ($\sim\;400\;\mu T$) with  fluctuations $\sim\;100\;nT$ due to external magnetic sources such as nearby beam line magnets or the displacement of large magnetic objects (e.g.,
the crane is the Meson Hall).

Fluctuations can be compensated using a system of magnetic sensors and coils to characterize and control the environment around the magnetic shields in a feedback loop. The TUCAN's plan is to reduce the static field to less than 50 $\mu$T using dedicated compensation coils and constant-current supplies, with a readily achievable stability of $\mathrm{10^{-3}}$ and to reduce the remaining static field and fluctuations by up to a factor of 100 through a separate set of compensation coils and current supplies, using fluxgate magnetometers
for magnetic feedback. The fluxgate sensors will be placed in the region
between the compensation coils and the passive shields. The PSI~\cite{bea}
and Munich~\cite{lins} groups have both commissioned and characterized such systems. The PSI group has also presented designs of their planned system for the n2EDM upgrade~\cite{rawlik}.

The active shielding will control the magnetic field immediately outside the outermost passive shielding layer. Overall, the active shielding system should be able to reduce the net background magnetic field to the level of tens of nT over the volume of the nEDM cell. This thesis came into play for possible designing concepts of active shielding which will be discussed in the following Chapters.



% At large fields, saturation of the passive magnetic shielding system can be a concern, which would seriously impact its effectiveness. Furthermore, when accessing the experiment, the door to the MSR must be opened. If presented with a large external field, the innermost layer of the passive shielding system could themselves become magnetized, necessitating degaussing and additional experimental down time with these factors in mind. The proposed plan is to nullify and stabilize the magnetic field environment at TRIUMF to ($\sim\;1\;\mu T$) using dedicated large bucking coils and also to reduce the fluctuations upto a factor of 100 using a separate set of coils by supplying currents to them  where the fluctuations will be measured by fluxgate sensors in a continuous feedback loop. Moreover,
% every channel  going through the feedback loop must be sampled faster than goal correction rate which is 6 Hz. The goal correction rate is set by :

% \begin{itemize}
%     \item  Precision frequency of the species in the experiment which is Hg-199 larmor frequency at 1 $\mu$ T.
%     \item  Induction of correction coils sets fundamental settling time.
%     \item  Correction for changes due to the sources in Meson Hall of TRIUMF where the actual experiment will take place .
%     \item  Change of earth's field with time.
% \end{itemize}

\section{Experimental Efforts So Far}\label{sec:lim}
\fig{Images/lim}{width = \textwidth}{Experimental nEDM upperlimit over the years \cite{1_lim,2_lim,3_lim,4_lim,5_lim,6_lim,7_lim,8_lim,9_lim,10_lim,11_lim,12_lim,13_lim,14_lim,15_lim,16_lim,17_lim} along with theoretical predictions \cite{theory_lim_1, theory_lim_2, theory_lim_3}. The vertical dashed line indicates the introduction of UCN. The light green region indicates the the nEDM limit in SUSY, M-theory and others while he light red region indicates for SM. \label{fig:lim}}{Short}

Additional sources of CP violation beyond the standard model predict the nEDM to be in the range of $10^{-27}$ -- $10^{-28}$  e$\cdot$cm as compared to $10^{-33}$ -- $10^{-31}$ e$\cdot$cm predicted by the standard model \cite{theory_lim_1, theory_lim_2, theory_lim_3}. There are several experiments aiming at improving the uncertainty on the nEDM. The Fig. \ref{fig:lim} summarize all the results. Since the first measurement data published in 1957 \cite{1_lim}, the upper limit set on nEDM has been reduced by eight orders of magnitude over the last six decades. At 1980, ultra cold neutron (UCN) experiment overtook over neutron beam experiments on precision. The current best upper limit set by ILL/Sussex/RAL nEDM experiment is $3.0 \times 10^{-26}$ e-cm (90 \% C.L) or $3.6 \times 10^{-26}$ e$\cdot$cm (96 \% C.L)  \cite{bestLim_1,bestLim_2} and the experiment is performed at Institut Laue-Langevin (ILL, Grenoble, France). The new $^{199}\mathrm{Hg}$ EDM measurement constrains the nEDM better than direct nEDM measurements, $d_n$ $<$   $\mathrm{1.6\times10^{-26}}$ e$\cdot$cm, although subject to uncertainty from
Schiff screening~\cite{schiff_screen}. The TUCAN nEDM experiment is aiming to constrain the uncertainty on the nEDM at the $10^{-27}$ e$\cdot$cm sensitivity level. 

The Paul Scherrer Institut
(PSI, Villigen, Switzerland) nEDM experiment uses an improved version of the former 
ILL/Sussex/RAL single-cell apparatus. Several innovations have been made at PSI, including a new SD2 spallation-driven UCN source. The experiment employs several Cs magnetometers outside the EDM cell, and a $^{199}\mathrm{Hg}$ comagnetometer. Active magnetic shielding and other environmental controls have been improved. A new detector that can simultaneously count both spin states of UCN has also been implemented. The final sensitivity expected
is $\mathrm{10^{-26}}$ e$\cdot$cm~\cite{psi}. Some of the chief improvements made at PSI recently have been in the area of nearby alkali atom (Cs) magnetometry, Hg comagnetometry, and neutron magnetometry. A recent achievement at PSI is the understanding of the Cs magnetometer signals in terms of magnetic field gradients internal to the magnetic shielding. This has led to a detailed understanding of the false EDM of the Hg comagnetometer~\cite{psi_falseEDM}. Another recent achievement is in using the neutrons themselves to measure gradients~\cite{psi_n_gradient}. PSI also aims to improve their magnetometry with $^3\mathrm{He}$ magnetometers inside the electrodes of the double EDM measurement cells for their future n2EDM effort. They have performed R$\&$D using Cs magnetometers to sense the free-induction decay signal from $^3\mathrm{He}$, which resulted in a new high-precision magnetometer possessing excellent long-term stability\cite{psi_magnetometer}. The precision goal for n2EDM is $5 \times 10^{-28}$ e$\cdot$cm~\cite{psi_n2edm_nEDM-workshop,psi_n2edm_PPNS-workshop}.



The nEDM collaboration at Spallation
Neutron Source (SNS, Oak Ridge, TN, USA) plans to measure $d_n\approx$ $2\times10^{-28}$ e$\cdot$cm, two orders
of magnitude improvement from the current limit~\cite{sns_lim}. They plan to use a unique experimental technique. A Cold Neutron (CN) beam from the SNS will impinge upon a volume of superfluid $^4\mathrm{He}$ creating UCN. The nEDM measurement will also be conducted in the superfluid. A small amount of polarized $^3\mathrm{He}$ introduced into the superfluid $^4\mathrm{He}$ will
act as both a comagnetometer and spin analyzer for the UCN. The $^3\mathrm{He}$ neutron capture rate is strongly spin dependent, and will beat at the difference of the Larmor precession frequencies of the neutrons and $^3\mathrm{He}$. A non-zero EDM would change the beat frequency with E-reversal. Scintillation light produced in the superfluid will be used to detect the capture products. The target precision is $10^{-28}$ e$\cdot$cm The false EDM of the $^3\mathrm{He}$ comagnetometer may be reduced by collisions in the surrounding $^4\mathrm{He}$~\cite{sns_false_edm}. The group aims to
commission the experiment at SNS by 2020.

A new room-temperature nEDM experiment will be conducted using an upgraded
Los Alamos National Laboratory (LANL, Los Alamos, NM, USA) UCN source~\cite{lanl_nEDM-workshop}. The aim of the project is to increase the UCN density by a factor of five to ten, which could then be used to carry out a $\sim$ $10^{-27}$ e$\cdot$cm determination
of the nEDM. The experiment aims for completion of a $10^{-27}$ level result, to be completed in the years prior to the SNS nEDM experiment, which shares a number of collaborators. Two other room temperature nEDM experiments are being pursued at the Forchungsreaktor Muncheon II (FRM2) reactor in Munich~\cite{frm2} and Gatchina reactor at ILL~\cite{PNPI}. Both experiments feature double measurement cells and Cs magnetometers internal to the innermost magnetic shield. The Munich effort features an impressive new effort in active and passive magnetic shielding~\cite{msr_design,shield_pnpi,shield_pnpi2}, and uses $^{199}\mathrm{Hg}$ comagnetometer. The ILL/Gatchina experiment has produced results at ILL~\cite{PNPI}. This could be improved in further runs at ILL in the EDM position, or in runs using the superfluid He UCN source at ILL, where a statistical sensitivity of $3.5 \times 10^{-27}$ e$\cdot$cm could be obtained~\cite{pnpi_nEDM-workshop}. The group will build a UCN source at the WWR-M reactor in Gatchina in order to increase the UCN 
flux.

\section{Review of the AMC System Worldwide for nEDM Measurement}

The active magnetic magnetic compensation system has been first commissioned and characterized by the PSI group where the detail process has been written in the PhD thesis of the Ref.~\cite{bea} in 2013. The PSI's nEDM experiment is shielded from the outside magnetic field via a four-layer cylindrical shield
of Mu-metal. To reduce and stabilize the outer magnetic field to keep the magnetization of the Mu-metal in a stable state, they build a AMC system known as surrounding field compensation (SFC) system. There they use 
six 6m$\times$8m rectangular coils creating three orthogonal pairs which will act as compensation coils. The currents in these coils
can be controlled dynamically via a proportional-integral feedback loop. A regularized, pseudoinverse matrix of proportionality factors, which correlate magnetic field changes at given sensor positions to current changes in the SFC coils, is incorporated in the feedback loop. This approach is superior to simpler feedback controls, and the magnetic field can be stabilized
by roughly one order of magnitude within a large control volume and not only at single points. The system that they described is pioneer for other AMC systems doe nEDM measurement. But the system doesn't provide clear solution to the possible placements of the fluxgates. They claim that having more sensors than coils produce better compensation but they have not the clearly uses the matrix condition nymber or coil design. In 2018, the flaws of that system have been pointed out by another student of the same group in his PhD thesis~\cite{rawlik}. There, a well conditioned matrix has been discussed by new coil design. The author in Ref.~\cite{rawlik} also claims that there is no need of Proportional (P)-Integral(I) or simply PI feedback algorithm which turns out to be false.

Besides that in 2016, the Munich group has also characterized such a system where the where the detail process has been written in the PhD thesis of the Ref.~\cite{lins}. The work described their was similar to that in Ref.~\cite{bea} in terms of matrix inversion. They have also talked about the different coil designs. They had 24 coils and 180 field probes for active shield.

Moreover, before the work in this thesis came into play, another thesis from our collaboration (TUCAN) had been written by Ref.~\cite{mike} in 2013. He introduced the development of active magnetic shielding for TUCAN nEDM measurement. He talked about the PID control while there was no shield present.


% The experiment has been done at the Institut Laue-Langevin (ILL) which is situated at Grenoble, France. A new generation of UCN source known as superthermal UCN source which uses a new method of cooling by transferring energy to quantum excitations in a material have recently come online. To use such a UCN source, the nEDM appa

% The apparatus for nEDM experiment has been moved from ILL to a superthermal UCN source at Paul Scherrer Institut (PSI) which is situated at Villigen, Switzerland. UCN superthermal source use the cooling technique . A new UCN source generation have recently come online. They use a technique from condensed
% matter physics involving cooling by transferring energy to quantum excitations in a material.
% UCN sources that employ this method of cooling are known as superthermal sources and they are
% beginning to transform the landscape of fundamental neutron physics at various facilities in the
% world.
% The nEDM apparatus from ILL was moved to such a UCN source at Paul Scherrer Institut
% (PSI, Villigen, Switzerland). The apparatus was also improved and upgraded in several respects.
% Data-taking for this new nEDM experiment was completed recently and the analysis of the data is
% ongoing [9]. The expectation in the community is that this new result will improve the previous
% best by a factor of about three to four.
% Next generation UCN EDM experiments are now in preparation at a variety of sites aiming to
% improve the result by an order of magnitude or more. Experiments are either ongoing or planned
% at ILL [10, 11], PSI [12], the Gatchina reactor [10], the Forchungsreaktor Munchen II (FRM2)
% reactor [13], Los Alamos National Laboratory (LANL, Los Alamos, NM, USA) [14], the Spallation
% Neutron Source (SNS, Oak Ridge, TN, USA) [15], and our eort at TRIUMF [16, 17]. We discuss
% our relationship to these experiments in Section 4. Our goal of dn < 10��27 ecm within the next
% 6-7 years (running until 2024-25, as stated earlier) is competitive with these eorts. One of the key
% factors is our unique UCN source, which we are upgrading. We envision achieving UCN counting
% rates over 100 times larger than the previous best nEDM experiment and the recently completed
% experiment at PSI, and similar to or surpassing the plans of other experiments.





\section{Overview of The Thesis}
The goals to design an AMC system as explained in section \ref{sec:amc}  are the following:
\begin{itemize}
    \item  To stablize the magnetic field surrounding  MSR $\leq\;100\;nT$ and for that sample every channel in the feedback loop more than 6 Hz.
    \item  To reduce $\sim\;400\;\mu T$ background field to avoid saturation
    \item  To able to open the door without magnetizing internal layers.
\end{itemize}

The overall objective of this thesis is to focus on the the development and testing of AMC prototype so that the above mentioned goals can be implemented for the actual experiment at TRIUMF. The AMC prototype has been set up at University of Winnipeg and tested at a highest possible way within the last three years time frame and expalined in Chapter \ref{ch:amcP}. The main focus has been given on the stablization of the surrounding field fluctuations and understanding every problems that have been faced with possible solutions. Mainly, the work has been done by closely following the work of PhD thesis \cite{bea,lins,rawlik} as explained in Chapter \ref{ch:operation}. However, while studying those theses, some faults have been discovered which are considered as the achievements for this thesis as explained in Chapter \ref{ch:quantification}. Finally, Chapter \ref{ch:conclusion} talks about the future work that could be done and how the thesis will be milestone for future studies before summarizing everything.


% \section{Concept of Active Compensation}

% \fig{Images/exp2}{width = 0.8\textwidth}{Schematic diagram of prototype active compensation system at University of Winnipeg.\label{fig: active}}


% \newcommand{\fig}[4]{\begin{figure}[h]
% \centering
% \includegraphics[{#2}]{{#1}}
% \caption{#3}
% \end{figure}}

% \begin{figure}
%     \centering
%     \includegraphics[width=0.4 \textwidth]{Images/exp}
%     \caption[width=0.4 \textwidth]{Schematic diagram of prototype active compensation system at University of Winnipeg . }
%     \label{fig:active shielding}
% \end{figure}
% \begin{figure}
%     \centering
%     \includegraphics[width=0.5 \textwidth]{Images/ss}
%     \caption[width=0.4 \textwidth]{PI loop in flow chart. First measurement from the fluxgates will act as set-point. Then the repeated measurements from the fluxgates are taken. For each measurement difference with the setpoint are noted. On the basis of the difference, the required amount of current for the coils surrounding the outermost layer of the shileding are determined and sent. Those coil currents generate required amount of magnteic flux to compensate for the differences.}
%     \label{fig:active shielding}
% \end{figure}
% The four layer Mu-metal cylinder enclosing the experimental area, which will be used for passive shielding has surrounded by six coils on six faces , each of having 1 mm$^2$ area. The outermost cylinder with coils surrounding it has been shown in the Fig.\ref{fig: active}. The magnetic environment is sensed by the fluxgates placed in different positions on the surface of the coils. The fluxgates that have been used are 3-axis. So, a breakout box has been built to separate each of x,y $and$ z axis in respective direction. Then, a fourth order low pass butterworh filter with cutoff frequency at 10 Hz has been built  to get rid of high frequency noise. After filtering,  the signals are transmitted to the computer via analog to digital converter (ADC) of LabJack T7 Pro for controlling via proportional integral (PI) feedback loop. 


 \lhead{\emph{Active Magnetic Field Compensation Prototype}}
\chapter{Active Magnetic Field Compensation Prototype}\label{ch:amcP}
% \printinunitsof{pt}\prntlen{\textwidth}

In Chapter~\ref{ch:magnetics}, the requirements of the magnetic field for nEDM experiment have been discussed. It also includes the active magnetic compensation to stabilize the external magnetic field. This thesis is based on the development of Active Compensation system for the TUCAN nEDM experiment. For that purpose a prototype has been built and tested at University of Winnipeg. This chapter describes the different components needed to make the prototype successfully running. It starts with giving an overall overview of the prototype and then move into describing every components in details. Finally, the chapter ends with giving an idea about the surrounding field fluctuations around the prototype experimental setup. In Chapter~\ref{ch:operation}, we then move into discussing the feedback algorithm needed to make the prototype working as active compensation.

The first section in this chapter gives an overview of the prototype which is presented next.


% The chapter describes the Active Magnetic Field Compensation (AMC) prototype that has been built at the University of Winnipeg. All the apparatus that made the AMC prototype will be discussed in detail.

\section{Overview of AMC Prototype}\label{sec:amcp_overview}
This Section gives an overview of the AMC prototype that has been built at the University of Winnipeg in the development process of implementing the original one at TUCAN nEDM experiment.

The magnetic background fluctuation has been found to be $\sim$ 100 nT (see section \ref{sec:field}) over a full day. The prototype has been designed to compensate that by introducing six current carrying closed loop coils on six faces of a cube surrounding the outermost mu-metal shield of a four layer cylindrical passive shielding enclosing the imaginary experimental area. To see how well the prototype response for a constant perturbation, an electro-magnet coil has also been used. The properties of mu-metal shields have been discussed more in Section~\ref{sec:shield} and the coil cube in Section~\ref{sec:cube}. The full prototype with different apparatus has shown in terms of schematic diagram in Fig.~\ref{fig:active}\textcolor{blue}{(a)} and also a glimpse of the actual experiment was shown in Fig.~\ref{fig:active}\textcolor{blue}{(b)}.

There are four 3-axis fluxgate sensors placed at different positions within the compensation coils to measure field for compensation and one 3-axis fluxgate sensor placed at the center of the origin of the prototype for quantification of the prototype. A breakout box was built to separate each of $x$, $y$ and $z$ axis in respective direction and provide power. More about the fluxgate sensors have been discussed in Section~\ref{sec:sensor}.

\doublefig{Images/exp3}{width =\textwidth,height =8cm}{Schematic diagram \label{fig:exp}}{Images/exp_photo}{width = \textwidth,height =8cm}{Photograph\label{fig:exp_photo}}{{Active Magnetic Field Compensation (AMC) Prototype at University of Winnipeg. The schematic diagram with different apparatus is shown in (a) and the actual experiment photograph is shown in (b). Surrounding the outermost mu-metal passive shielding layer, there are six coils on six faces. There is another elctro-magnet coil working as perturbation coil. The magnetic environment is sensed by the 3-axis fluxgates placed in different positions on the surface of the coils where $x$, $y$ and $z$ axis have been separated by a breakout box. A fourth order low pass Butterworh filter has been built  to get rid of high frequency noise. For signals transferring to and from computer, analog to digital converter (ADC) and digital to analog converter (DAC) of LabJack T7 Pro have been used. Six current sinks circuits generate the required currents that are sent to the six coils. } \label{fig:active}}{Active Magnetic Field Compensation (AMC) Prototype at University of Winnipeg}


\FloatBarrier
\clearpage
The fluxgates suffer from environmental noise in terms of 60 Hz electrical noise due to grid line transmitted from power station and others. Therefore, a $4^{th}$ order low pass Butterworth filter with corner frequency at 10 Hz has been built to try to reduce the noise. The filter details are in Section~\ref{sec:filter}. After filtering, the signals are transmitted to the computer via analog to digital converter (ADC) of LabJack T7 Pro where required currents are calculated via proportional integral (PI)for control algorithm. The information for required currents are sent from computer to 6 voltage controlled current source circuits termed as current sinks. Finally, the current generated by the current sinks are sent to the six compensation coils for reducing the drift in the fluxgate signals.  via digital to analog converter (DAC).

% \fig{Images/exp2}{width = 0.8\textwidth}{Schematic diagram of prototype active compensation system at University of Winnipeg.\label{fig: active}}
The above discussion  has given an idea about the different apparatus that made the prototype. Next each of them will be discussed in details starting with  the mu-metal passive shields.

\section{Mu-Metal Shields}\label{sec:shield}
Generally, the passive magnetic shielding as explained in Section \ref{sec:passive} is composed of a thin multi-layer shields with materials having high magnetic permeability such as mu-metal. The outer layers are usually cylindrical \cite{mu_cyl_1,mu_cyl_2} but they can also take the same forms as the magnetic shielded room (msr)~\cite{mu_msr_1,mu_msr_2}. The inner layer is designed based on the coil to achieve required homogeneity \cite{mu_inner_1,mu_inner_2}. 

\fig{Images/passive}{width = \textwidth}{Prototype passive shielding at University of Winnipeg. Inner cylinder is shown in (a), (b) shows 4 mu-metal cylindrical shields of passive shielding by OPERA simulation software and (c) shows photograph of 3 layers of prototype passive shielding. The outer layer is not shown in (c).\label{fig: passive}}{Prototype passive shielding at University of Winnipeg}


However, in the prototype, there are four layers of cylindrical shields enclosing the imaginary experimental area with end-caps on them. The prototype passive shielding at University of Winnipeg is shown in Fig.~\ref{fig: passive}. The inner cylinder with end caps is shown in Fig.~\ref{fig: passive}\textcolor{blue}{(a)}. All the 4 layers are shown in Fig.~\ref{fig: passive}\textcolor{blue}{(b)} by OPERA simulation software and the software itself has been explained in detail in Section~\ref{fig:opera_setup}. A photograph of 3 layers excluding the outermost layer of prototype passive shielding is shown in Fig.~\ref{fig: passive}\textcolor{blue}{(c)}. Amumetal (Magnifer 7904) of thickness 0.0625 inch has been used for the layers,  which is an 80\% Nickel-Iron alloy and was fabricated and annealed by Amuneal Manufacturing Corp \cite{mu-metal}.  There are two end-caps in each cylinder with a 7.5 cm diameter in the central hole. And to minimize the leakage of the external fields into the progressively shielded inner volumes, a stove-pipe of length 5.5 cm is placed on each hole. One of the stove-pipe of the inner cylinder is seen in Fig.~\ref{fig: passive}\textcolor{blue}{(a)}. The radius and length of the four layers including the stove pipes are shown in Table \ref{table:mu-metal}. The center of the enclosed area is the center of the origin of the prototype. A combined DC shielding factor of the order of $\mathrm{10^6}$ is expected. A discussion about another prototype of similar design but smaller in size can be found in Ref. \cite{baby_shield}.

\begin{table} [!htb]
    \centering
    \begin{tabular} { |c|c|c|c|c|c|} 
        \hline
        Parameters & Innermost Layer & $\mathrm{2^{nd}}$ Layer & $\mathrm{3^{rd}}$ Layer & Outermost Layer & Stove Pipe\\
        \hline\hline
        Radius (cm) & 18.5 & 23.5 & 30 & 38 & 3.7 \\ 
        \hline
        Length (cm) & 37 & 55 & 71 & 90 & 5.5 \\ 
         \hline
    \end{tabular}
    % \vspace{4mm}
    \caption{Prototype passive shielding layers including stove pipes radius and length.}\label{table:mu-metal}
\end{table}

\FloatBarrier
The above discussion gives an idea about the dimensions and properties of the prototype passive shielding at the University of Winnipeg. Next the coil cube will be discussed in details.

\section{Coil Cube}\label{sec:cube}
As discussed in Section ~\ref{sec:amcp_overview}, the prototype has been designed to compensate $\sim 100$ nT (see section \ref{sec:field}) magnetic background fluctuations over a full day by introducing six current carrying closed loop coils on six faces of a cube surrounding the outermost mu-metal shield. The coils are termed as $C_x^\pm$, $C_y^\pm$ and $C_z^\pm$ respectively and known as compensation coils as they are responsible for compensating the magnetic fluctuations. The opposite face coils are separated by 1.15 m taking relatively similar shape as  Helmholtz coils. In addition to those, there is also a perturbation electro-magnet coil namely $P_z^+$. The coil configuration is shown in Fig.\ref{fig: coil}. The numbers 1-8 indicating the postions of the fluxgate sensors which will be discussed in the next section.

\fig{Images/coil}{width = \textwidth}{Schematic diagram indicating the coil   configuration and the position of the fluxgate sensors.\label{fig: coil}}{Schematic diagram indicating the coil configuration and the position of the fluxgate sensors.}


\FloatBarrier
The compensation coils have been chosen to be single turn to have small resistance ($\sim$ 0.15 $\Omega$) and small inductive reactance that oppose the reduction of the current flow in them. The have the dimensions of 1.15 m$\times$1.15 m. The perturbation electro-magnet coil has also the same dimension as the compensation coils but it has 77 no. of turns. The perturbation coil was typically placed in the same face as $C_z^+$ and separated from it by 1.06 m. The coils properties are shown by Table \ref{table:coil}. 

\begin{table} [!htb]
    \centering
    \begin{tabular} { |c|c|c| } 
        \hline
        Parameters & \makecell{Compensation Coils \\ ($C_x^\pm$, $C_y^\pm$ and $C_z^\pm$)} & \makecell{Perturbation Coil \\ ($P_z^+$)} \\
        \hline\hline
        Dimension (m$\times$m) & 1.15$\times$1.15 & 1.15$\times$1.15\\ 
        \hline
        No. of Turns & 1 & 77\\ 
        \hline
        Resistance ($\Omega$) & 0.15 & 11.55\\ 
        \hline
        Inductance (mH) & 0.32 & 24.62\\
        % \hline
        % Current ($A$) & 0 & 1\\
         \hline
    \end{tabular}
    % \vspace{4mm}
    \caption{Coils dimension and properties.}\label{table:coil}
\end{table}

\FloatBarrier
The above discussion gives an idea about the coil configuration, dimensions and properties. Next, the apparatus used to measure magnetic field will be discussed.


\section{Fluxgate Sensors}\label{sec:sensor}

A fluxgate sensor is a piece of magnetic material, within a sensing winding used for measuring the magnitude and direction of DC or low-frequency ac magnetic fields. The magnetometer$'$s sensitivity direction is the axis of the sensing winding. The magnetic material is periodically saturated by a source of excitation power. It is the transition of this material into magnetic saturation that creates the measurement signal.

For the prototype, we have used four 3-axis fluxgate sensors (2 Bartington Mag-03 and 2 Bartington Mag690) placed at different positions within the compensation coils to measure field for compensation. Beside those, one 3-axis fluxgate sensor (Bartington Mag690) has been placed at the center of the origin of the prototype for quantification of the prototype. The numbering 1-8 in Fig.~\ref{fig: coil} indicate the positions of the possible placements of the fluxgates from where the magnetic fields have been measured. Note that, we have also tried more positions other than that are shown on that figure which will be discussed in Chapter~\ref{ch:quantification}.  The measured data from the fluxgate sensors are transferred to the computer via anlog to digital converter (ADC) of LabJack T7 Pro which will be discussed in the next section. But before coming to the ADC, the 3-axis of the fluxgates have to be separated and for that purpose we have build  2 breakout boxes. Each of the breakout boxes can separate four 3-axis fluxgate sensors that means each can provide the data for 3$\times$4=12 sensors. The the 3-axis directions of the fluxgate sensors are shown in Fig.~\ref{fig: coil} by the label X, Y and Z which represent the $x$, $y$ and $z$ axis of a fluxgate.  The properties of the fluxgate sensors that we have used for the prototype are shown in Table~\ref{tablE:sensor} and collected from the Bartington's (manufacturer) website~\cite{flux}. 

\begin{table} [!htb]
    \centering
    \begin{tabular} { |c|c|c|c|c|c|} 
        \hline
        Parameters & Mag-03 & Mag690 \\
        \hline\hline
        Measure Range ($\mu$T) & $\pm$70 & $\pm$100 \\ 
        \hline
        \makecell{Noise Level \\($\mathrm{pT_{rms}\;/\sqrt{Hz}}$ at 1 Hz)} & $<$6 & 10$-<$20 \\ 
        \hline
        Bandwidth (kHz) & 3 & 1 \\ 
        \hline

    \end{tabular}
    % \vspace{4mm}
    \caption{Properties of the fluxgate sensors used for the prototype.}\label{tablE:sensor}
\end{table}

\FloatBarrier
According to the manufacturer (Bartington), the typical noise levels are from $\mathrm{10}$ to $\mathrm{<20\;pT_{rms}\;/\sqrt{Hz}}$ and $\mathrm{<6 \; pT_{rms} \;/\sqrt{Hz}}$ at $\mathrm{1}$ Hz for Mag690 and Mag-03 respectively. The noise levels of the fluxgates had been tested  and were found to match the manufacturer standard. For that, data was taken by placing Mag690 and Mag03 fluxgates inside the shield at different times. Then the data was processed in Mathematica \cite{Mathematica} to estimate the power spectral density using Discrete Fourier Transform (DFT) \cite{dft}. The Fig.~\ref{fig:noise} shows that the noise level is $\sim$ $\mathrm{16\;pT_{rms}\;/\sqrt{Hz}}$ at $\mathrm{1}$ Hz for Mag690. It confirms that ADC doesn't present noise beyond fundamental noise of fluxgate.

\fig{Images/noise}{width =  \textwidth}{Average Spectral Density for Mag690. \label{fig:noise}}{Average Spectral Density for Mag690.}


The above discussion gives an idea about the definition of the fluxgate sensor and the properties of the fluxgate sensors that we hav used. Next in the list according to Fig.~\ref{fig:active}\textcolor{blue}{(a)}, we should discuss the filter. But the necessity of the filter came later and so the data acquisition process will be discussed  where the analog to digital converter (ADC) and digital to analog converter (DAC) are the main components.


\section{Data Acquisition (DAQ) Module}\label{sec:DAQ}

The signals that we get from the fluxgates are all analog which we have to sent to the computer to process the signals by PI control algorithm. But the computer doesn$'$t understand analog signal rather it works on digital signal. So, we need a converter that will transform our analog signal to the digital one. Again, the siganls that are processed by PI control algorithm in computer must send to the coils. So, we also need a converter that will transform the digital data of the compute to the analog data which can be sent to the coils. In short, we need analog to digital converter (ADC) to transfer information in the computer and digital to analog converter (DAC) to transfer information out of the computer. For that purpose, we have used LabJack T7 Pro Data Acquisition (DAQ) device which is build and supplied by LabJack Corporation. Next we will discuss the LabJack T7 Pro DAQ device in terms of ADC first and then in terms of DAC.

\subsubsection{ADC of LabJack T7 Pro}

The signals that are measured by fluxgate sensors are separated into $x$, $y$ and $z$ axis via breakout boxes and then with filter (discussed later)/ without filter are transmitted to the computer via analog to digital converter (ADC) of LabJack T7 Pro. The ADC records the analog voltage in terms of ADC counts. An ADC count is the smallest change in the voltage for a single change in ADC value. The resolution of ADC defines the no. of discrete voltages which can be represented for a input range. For eaxple, an ADC with 16 bit resolution can record $\mathrm{2^16=65536}$ discrete voltages and for an input range of 10 V, the smallest change in the voltages will be $\mathrm{10}$ V$\mathrm{/2^16=0.153}$ mV. But it is just theoretical prediction considering no channel noise. But in reality, in addition to noise of the ADC itself, there are noises from the power source and the fluxgates which contribute to the channel resolution and that resolution is called effective resolution which is lower than ideal resolution.

In addition to the standard 16-bit ADC, the LabJack T7-Pro is equipped with a 24 bit delta-sigma ADC. The effective resolution can be varied from 16 to 19.1 bits when analog conversions occur on the 16-bit ADC and for 24 bit it is from 19.6 to 21.8 bits with gain 1 (i.e. the ratio between output to input is 1) \cite{T7}. So, for testing that some successive voltage readings, using a short jumper between the test channel and ground/battery was collected and the difference between successive readings was stored. Then the mean and root mean square (rms) of the mean was calculated. Now the, $\mathrm{mean_{rms}}$ has been converted into ADC counts depend on the 16 bit or the 24 bit ADC with its input range. For example, for a 16 bit ADC with $\pm$10 V range, after finding the $\mathrm{mean_{rms}}$, the ADC counts of that $\mathrm{mean_{rms}}$ will be
\begin{equation}\label{eq:adc_mean}
    \mathrm{mean_{rms} (in\;ADC \;counts)=\frac{mean_{rms}\times 2^{16}}{20}}
\end{equation}
where, 16-bit ADC with $\pm$10 V range is used in the example.

After finding out the ADC counts of $\mathrm{mean_{rms}}$, the corresponding bits has been found by multiplying with $\mathrm{log_2}$.  Finally, the effective resolution was found by subtracting that amount of bit form either the 16 bit or 24 bit ADC depends on which one is used. In short the formula \cite{T7} for 16-bit ADC will be
\begin{equation}
    \mathrm{Effective\;Resolution=16\;bits - [log_2 \times mean_{rms} (in\;ADC \;counts)]\;bits}
\end{equation}
And for 24 bit delta-sigma ADC will be
\begin{equation}
    \mathrm{Effective\;Resolution=24\;bits - [log_2 \times mean_{rms} (in\;ADC \;counts)]\;bits}
\end{equation}
where, $\mathrm{mean_{rms} (in\;ADC \;counts)]}$ has been described in Eq.~(\ref{eq:adc_mean}).

\fig{Images/res_T7adc}{width = \textwidth}{Effective resolution for different resolution index for gain 1 and $\pm 10\:V$ range. A higher resolution index will result in lower noise and higher effective resolution but increases sample times. \label{fig: res}}{Effective resolution for different resolution index for gain 1 and $\pm 10\:V$ range.}

The end result was shown in Fig~\ref{fig: res} where the manufacturer \cite{T7} result (dashed red) has been compared with the measured result (solid green) for different resolution index. The resolution index higher means lower noise and higher effective effective resolution but also increases the time require to make single analog to digital conversion.. It is seen from the figure that the results are quite similar. Note that for resolution index 1-8, T7 Pro uses 16-bit ADC and for 9-12 resolution index it uses 24-bit delta-sigma ADC. 

After having idea about the LabJack T7 Pro ADCs, now sampling frequency will be discussed. Note from the above that, the time required for the ADC hardware to make a single analog to digital conversion on any channel is ADC sampling time which doesn$'$t include command/response and overhead time associated with the host computer/application~\cite{T7}. And the inverse of this ADC sampling time is called the sampling frequency. In normal mode, the ADC can sample as low as 6.3 Hz to 25 kHz per sample as shown in Table \ref{table:t7freq}. Sampling at 6.3 Hz, the ADC itself is capable of avoiding all sorts of noises due its long average time. But the more channels means less sample frequency which makes the system more slower.

\begin{table} [!htb]
    \centering
    \begin{tabular} { |c|c|c|c|c| } 
        \hline
        \thead{Res. Index} & \makecell{Bandwidth (Hz) \\ (Gain/Range: \\ 1/$\pm$ 10V)} & \makecell{Bandwidth (Hz) \\ (Gain/Range: \\ 10/$\pm$ 1V)} & \makecell{Bandwidth (Hz) \\ (Gain/Range: \\ 100/$\pm$ 0.1V)} & \makecell{Bandwidth (Hz) \\ (Gain/Range: \\ 1000/$\pm$ 0.01V)}\\
        \hline\hline
        1 & 25000.0 & 4347.8 & 970.9 & 198.8\\ 
        \hline
        2 & 25000.0 & 4347.8 & 492.6 & 100.0\\ 
        \hline
        3 & 16666.7 & 1818.2 & 198.0 & 99.0\\ 
        \hline
        4 & 11111.1 & 1724.1 & 196.9 & 99.0\\ 
        \hline
        5 & 6250.0 & 869.6 & 194.2 & 98.0\\ 
         \hline
        6 & 3448.3 & 438.6 & 97.3 & 97.1\\ 
        \hline
        7 & 1785.7 & 392.2 & 94.8 & 94.3\\ 
        \hline
        8 & 917.4 & 324.7 & 90.3 & 90.1\\ 
         \hline
        9 & 285.7 & 285.7 & 285.7 & 285.7\\ 
        \hline
        10 & 74.6 & 74.6 & 74.6 & 74.6\\ 
        \hline
        11 & 15.1 & 15.1 & 15.1 & 15.1\\ 
         \hline
        12 & 6.3 & 6.3 & 6.3 & 6.3\\ 
         \hline
         
    \end{tabular}
    % \vspace{4mm}
    \caption{T7 Pro sample frequency for different resolution index, gain and voltage range.  A higher resolution index will result in lower noise and higher effective resolution but increases sample times.}\label{table:t7freq}
\end{table}
\FloatBarrier

The above discussion shows that the effective resolution that the LabJack corporation has given in their website are pretty similar to our result (see Fig.~\ref{fig: res}) and if we want to sample faster we need a lower resolution index ADC but that will increase noise. To get rid of that noise we can use filter which will be discussed in the latter part of this chapter.

\subsubsection{LabJack T7 Pro with LJTickDAC}
The signals from fluxgates via breakout boxes came to the computer and converted to digital one by ADC. Now the signal has been processed in the PI control algorithm within the computer and ready to sent to the compensation coils. The very first step is to convert those digital signals into analog one and that can be done via digital to analog converter (DAC). There is an expansion module from LabJack called the LJTick-DAC (LJTDAC) that provides a pair of 14-bit analog outputs with a range of $\pm$10 volts. The LJTick-DAC plugs into any digital I/O block of T7 Pro. Each of the LJTick-DAC has two 14-bit DAC in its two channel. We have got seven coils including the perturbation one. So, we have used 4 LJTick-DAC and connected to LabJack T7 Pro via CB15 terminal. 

% There are 3 LJTickDACs each with two channels thus total six for six compensation coils. They are shown in Fig. \ref{fig:tick}.

% \fig{Images/dac}{width =  \textwidth, height =6 cm}{LJTickDAC \label{fig:tick}}

% The DACs have 14 bits of resolution over $\pm$10 V range. For the prototype, there should be a current source that can convert DACs voltage to current so that 100 nT cna be generated. So, a current sink has been designed in the half of the range of DACs voltage. The currents required can be found by -

% \begin{equation}\label{eq:iSink}
%     I_{out}^{sink}=V\times nT^{-1}/R
% \end{equation}
% So, for 100 nT the required current according to Eq. \ref{eq:iSink} is -
% \begin{equation}\label{eq:iSink_100}
%     I_{out}^{sink}=10\times 100^{-1}/50=200\:mA
% \end{equation}

This ends our discussion of DAQ for now. Next, we will test the resolution of the DAC as claimed by the LabJack by doing experiment on it after describing the current sink circuit.


\section{Current Sink}\label{sec:sink}


In previous section, we have discussed about the LJTick-DAC. For the prototype, we are using 4 LJTick-DACs each having two channels (chnnel a and chhanel b) to provide 14 bits anlog outputs with a range of $\pm$10 V by converting the signal processed in the computer via PI control algorithm. Now, those analog outputs are all voltage signals which we needed to convert in currents to sent to the coils. For that purpose, we have built a 8 channels current sink device which will be discussed in this section. 

% To generate the designed $\sim 200\:nT$ magnetic field by the coils (see section \ref{sec:cube}), it was found that $\sim 200\:mA$ current must be sent to them. So, a current sink  device as shown in Fig. \ref{fig: currenSink} with 8 channels has been built.

The 8 channels are actually 8 different voltage controlled current source circuits whose output currents are controlled by their input voltages. The circuit actively monitors and regulates the voltage drop across its sense resistor ($\mathrm{R_{sense}}$) until that is equal to the incoming voltage (in this case the output voltage of the LJTickDACs). Now, the the coil is kept in series with this resistor and thus, whatever current flows through the sense resistor must also flow through the coil. The device is termed as current sink device as current flows into the coils. The current sinks were designed to control 0-200 mA current from a 10V signal range. Because, it was found that the field generated by one of the coils at the center of the cube was $\sim$ roughly 200 nT, when there was $\sim$ 200 mA current flowing through the coil (How? I am confused. Can you help me in this line Jeff ??). And it was already discussed earlier that the environmental magnetic fluctuation for a full day is $\sim$100 nT, so 200 mA current sink was good enough to design. The size of the sense resistor comes from Ohm's law i.e. as 200 mA current has to be generated over 10V range, the resistor must be $\mathrm{R_{sense}}=\frac{10}{0.2}=50 \Omega$. Out of the eight channels, six have been connected to the six compensation coils and the first prototype current sink has been connected to the perturbation coil. Other two remaining channels are not instrumented. The device is powered by 24 V at a limit of $\sim 1.3 \: A$ current. 

\fig{Images/currentSink2}{width = \textwidth}{Current Sink Device. Circuit diagram of one of the voltage controlled current sink device shows in (a) and the pictorial topview of all the assembled current sink circuits shows in (b). The description is given in the text. \label{fig: currenSink}}{Current Sink Device}


The circuit diagram of one of the voltage controlled curent sink is shown in Fig.~\ref{fig: currenSink}\textcolor{blue}{(a)}. It is seen that the output voltage from LJTick-DAC$\_$1a (channel a of first LJTick-DAC) is coming to the non-inverting input of an operational-amplifier (op-amp) and inverting input of that op-amp is connected to the sense resistor. The output of the op-amp is connected pin 1 (also known as gate) of a power metal-oxide-semiconductor field-effect transistor (MOSFET) designed for low voltage, high
speed switching applications. So, voltage from LJTick-DAC$\_$1a ($\mathrm{V_{in}}$) is coming to gate of MOSFET and also current is generated as $\mathrm{I_{out}=V_{in}/R_{sense}}=V_{in}/50$ where, $\mathrm{Vin}$ ranges from 0 to 10 V. Now, when voltage between pin 1 (gate G) and pin 3 (source S) of MOSFET i.e.  $\mathrm{V_{GS}>0}$, then $\mathrm{I_{out}}$ flows from pin 3 (source S) to pin 2 (drain D) and eventually flows towards the coil $C_x^-$. It is also seen that a Schottky diode has been connected in parallel with the output where coil $C_x^-$ is connected for reverse current protection. It should be noted that outputs can not be left open. All the current sink circuits are assembled in a box and a picture from top is shown in Fig.~\ref{fig: currenSink}\textcolor{blue}{(b)}.

For calibrating current sink device, currents in the range of $0-200$ mA was requested and on the basis of that DACs generate the required voltages. Then the DACs generated voltage (input voltage) and the current that has been requested have been measured by 34970A data acquisition device. The readings of the input voltage and measured current have been noted. A linear fit of the input voltage (y-axis) based on LabJack 24-bit delta-sigma ADC specifications and measured current (x-axis) was made. The slope and offset of that fit is the ultimate gain and offset of the current sink channels. The gain found was 50 as expected because $\mathrm{R_{sense}}$ is 50 $\Omega$.  

Although, the LJTickDACs support $\pm$10 V range, the current sink device has been built to be in 0-10 V range. The reason is that it is relatively easy to make a circuit that can source or sink current based on a unipolar voltage and unipolar current. However, to generate a circuit that can do both without any distortion near $0\:V$ is a bit harder but certainly doable. But losing half ranges was not an issue for the prototype design and thus without wasting time, the device was built. Now, at this point, if it is required to utilize the full resolution of the DACs,  it$'$s probably easier to just modify the LJTick-DACs $\pm$10 V range to a smaller range prior to the generation of the bipolar output instead of making a new voltage controlled current supply. The first LJTick-DAC had been done modified in this way. 

For testing the resolution of the DACs, 500 values in the range of  0-1 mA was requested with 0.001 mA increment each time and the currents have been measured by 34970A data acquisition device. The readings of measured current had been noted.Then, the differences between the successive measured currents (mA)  were calculated from which the average difference was determined. Finally, from that average difference the resolution of the DACs had been calculated as
\begin{equation}
\mathrm{Resolution\;of\;DAC= log_2\left(\frac{200}{Measured\;Current\;Average\;Difference}\right)\;bits}
\end{equation}
where, 200 came from the fact that for full range of LJTick-DAC, the current sink produced 0-200 mA current.

\fig{Images/dacRes2}{width =  \textwidth}{Resolution of LJTick-DACs. Currents measured (vertical-axis) with 0.001 mA increment each time  for requested current (horizontal axis) from 0 to 1 mA to find the resolution of the DACs. Resolution of channel b of modified first LJTick$\_$DAC is hown in (a) and resolution of channel b of unmodified second LJTick$\_$DAC is shown in (b). The measurement process is described in text.\label{fig:dac}}{Resolution of LJTick-DACs}

The resolution of channel b of modified first LJTick$\_$DAC is shown in Fig.~\ref{fig:dac}\textcolor{blue}{(a)}. It is seen that we have found 14 bits resolution as expected from the modified one. Again, the resolution of channel b of unmodified second LJTick$\_$DAC is shown in Fig.~\ref{fig:dac}\textcolor{blue}{(b)}. It is seen that the resolution is 13 bits in this case.


The above discussion is all about the current sink that we have built. Next we will talk about a filter which is also built by us.





\section{Filtering}\label{sec:filter}
In Section~\ref{sec:DAQ}, it is discussed that the ADC itself is capable of avoiding all sorts of noises for highest effective resolution or least sampling frequency (see Table~\ref{table:t7freq}) for a single channel which also decreases with increasing number of channels. So, the system response is very slow when we use the highest resolution index for all the 14 channels (see Section~\ref{sec:sensor}) that we have. But if we use lower resolution index, the system response will be higher but also the fluxgates will then suffer from environmental noise in terms of 60 Hz electrical noise due to grid line transmitted from power station and others. Again, we have also shown that sampling frequency plays a pivotal role for coil current settling time in Section~\ref{sec:freq}. So, if we want to make our system response time as fastest as can be possible by our LabJack T7 Pro ADC (see Section~\ref{sec:DAQ}), we need a device that can filter out the noises explained earlier. In this section, we will discuss a filter that we have built to sort out the problems.

We have built a $4^{th}$ order low pass Butterworth filter with corner frequency at 10 Hz with an aim to reduce noise while increasing sample rate. The filter has been designed online in analog filter wizard \cite{fWizard}. The specification is shown in Table~\ref{table:butter}. We have chosen the voltage range to be $\pm$12 V which is the same for our fluxgate sensors. So, we can use the same power supply for both the filter and the fluxgates.

%  The Fig.\ref{fig:f} shows the difference between the same measurements done by with/without filter.

\begin{table} [!htb]
    \centering
    \begin{tabular} { |c|c| } 
        \hline
        % \thead{Parameters} & \makecell{Compensation Coils \\ ($C_x^\pm$, $C_y^\pm$ and $C_z^\pm$)} & \makecell{Perturbation Coil \\ ($P_z^+$)} \\
        % \hline\hline
        Voltage Range (V) & $\pm$12\\ 
        \hline
        Gain (V/V) & 1 \\ 
        \hline
        Passband & -3 dB at 10 Hz\\ 
        \hline
        Stopband & -3 dB at 100 Hz\\
        % \hline
        % Current ($A$) & 0 & 1\\
         \hline
    \end{tabular}
    % \vspace{4mm}
    \caption{$4^{th}$ Order Low Pass Butterworth Filter Specification}\label{table:butter}
\end{table}

\fig{Images/mag_dB-Freq}{width =  \textwidth}{Comparison of simulation with measured in terms of Magnitude (Volts per Volt) vs Frequency (left) and Magnitude (dB) vs Frequency (right). \label{fig:butter}}
To quantify the filter, it was connected to a lock-in amplifier from Stanford Research Systems where the frequency had been changed from 1 Hz$-$15 kHz with constant amplitude. While changing the frequency, rms and theta from lock-in amplifier had been acquired. From those value, a comparison has been made between simulation and measured values as shown in Fig. \ref{fig:butter}.

% % But the more channels means less sample frequncy which makes the system more slower with go below the goal correction rate. On the otherhand, the fastest sampling time gives huge noises in terms of 60 Hz electrical noise due to grid line transmitted from power station and others. 
% So, a filter is indispensable. Based on all conditions,

% \doublefig{Images/nf}{width =\textwidth, height= 4 cm}{No Filter \label{fig:nf}}{Images/f}{width = \textwidth, height= 4 cm}{With filter\label{fig:f}}{{(a) shows the B over time without any filter (b) B over time with 10 Hz LPF  } \label{fig:f}}









\section{Field Fluctuations Surrounding the Prototype}\label{sec:field}

The field fluctuations over a typical day surrounding the prototype has been measured using the fluxgate sensors. Then the fluctuation has been found using-
\begin{equation}
    \Delta B_{fluc}(t) = B_{meas}(0) - B_{meas}(t)
\end{equation}

Then that fluctuation has been compared with the data from Brandon space weather station \cite{weather_station} which is $\sim$215 km away from the prototype and that is the nearest one. According to them-
\begin{itemize}
    \item X component is the northward magnetic field.
    \item Y component is the eastward magnetic field.
    \item Z component is the vertical downward. magnetic field
\end{itemize}
The fluctuation has been found to be $\sim \pm 100$ nT and the results from the prototype are similar as of the Brandon weather station as shown in Fig. \ref{fig:fluc_24hrs}. The data has been taken at 1pm on August 29, 2018 to 1pm on August 30, 2018.

\fig{Images/bmeas_24hrs}{width =  \textwidth}{$\Delta\;B$ over 24 hrs measured at UofW with that from Barndon, MB space weather station. \label{fig:fluc_24hrs}}




% \begin{table} [h!]
%     \centering
%     \begin{tabular} { |c|c|c| } 
%         \hline
%         \thead{Parameters} & Current Sink Device \\
%         \hline\hline
%         Dimension ($m \times m$) & 1.15$\times$1.15 & 1.15$\times$1.15\\ 
%         \hline
%         No. of Turns & 1 & 77\\ 
%         \hline
%         Resistance ($\Omega$) & 0.2 & 15.4\\ 
%         \hline
%         Inductance ($mH$) & 1 & 0\\
%         % \hline
%         % Current ($A$) & 0 & 1\\
%          \hline
%     \end{tabular}
%     % \vspace{4mm}
%     \caption{Coils dimension and properties.}\label{table:coil}
% \end{table}


%%%%%%%%%%%%%%%%%%%%

% \newcommand{\fig}[4]{\begin{figure}[h]
% \centering
% \includegraphics[{#2}]{{#1}}
% \caption{#3}
% \end{figure}}

% \begin{figure}
%     \centering
%     \includegraphics[width=0.4 \textwidth]{Images/exp}
%     \caption[width=0.4 \textwidth]{Schematic diagram of prototype active compensation system at University of Winnipeg . }
%     \label{fig:active shielding}
% \end{figure}
% \begin{figure}
%     \centering
%     \includegraphics[width=0.5 \textwidth]{Images/ss}
%     \caption[width=0.4 \textwidth]{PI loop in flow chart. First measurement from the fluxgates will act as set-point. Then the repeated measurements from the fluxgates are taken. For each measurement difference with the setpoint are noted. On the basis of the difference, the required amount of current for the coils surrounding the outermost layer of the shileding are determined and sent. Those coil currents generate required amount of magnteic flux to compensate for the differences.}
%     \label{fig:active shielding}
% \end{figure}
 \lhead{\emph{Basic Operation}}

\chapter{AMC Prototype Feedback Control Algorithm and its Simulation}\label{ch:operation}

% In Chapter~\ref{ch:amcP}, the overall system has been described.  Each component of the system was discussed in detail.

% This chapter is about
% \begin{itemize}
%     \item how the system works as a unit to provide magnetic field control.  This includes the algorithm which generates the control and some typical results from its operation.  It also includes the methods used in operating the system, such as methods used in tuning the PI control system.
%     \item simulation methods used to understand system performance, and some basic results from the simulation which are compared with data
%     \item definition of metrics that will be used to quantify further system performance.  These will be applied in Chapter~\ref{ch:quantification} where further results will be presented.
% \end{itemize}



% Much of this work follows the work of others, especially Refs.~\cite{bea,rawlik,lins}.  New work that builds on these results is presented in Chapter~\ref{ch:quantification}.

%This chapter describes the main tool that generates the required currents that should be fed to the coils for compensation.
%The algorithm is based on the control theory and similar to as discussed in Ref. \cite{bea}.

In Chapter~\ref{ch:amcP}, the overall system has been described.  Each component of the system was discussed in detail. This chapter is about how the system works as a unit to provide magnetic field control.  This includes the algorithm which generates the control and some typical results from its operation.  It also includes the methods used in operating the system, such as methods used in tuning the control algorithm for the system. In addition to that, it will cover the simulation methods used to understand the system performance and some basic results from the simulation which are compared with the data. Finally, it will end with the definition of metrics that will be used to further quantify the system performance.  These will be applied in Chapter~\ref{ch:quantification} where more results will be presented. Much of this work follows the work of others, especially Refs.~\cite{bea,rawlik,lins}.  New work that builds on these results is presented in Chapter~\ref{ch:quantification}.

\section{Principle of Operation\label{sec:process}}

% The goal of this Section is to tell the general idea of how the system works, and to introduce the principles of operation.  It also introduces a few of the key issues faced when operating the system.

% \begin{itemize}
%     \item fluxgates (hopefully described in some Section in Chapter~\ref{ch:amcP}) measure the field
%     \item a setpoint for each fluxgate axis is decided
%     \item when the fluxgate signal drifts from the setpoint, the error grows
%     \item how the fluxgates respond to changes in the current is described by a matrix
%     \item the inverse of the matrix describes how to correct the currents based on fluxgate signals
%     \item PI control is used to decide the corrected currents based on the input errors from the fluxgate readings
% \end{itemize}

% Details:
% \begin{itemize}
%     \item the matrix isn't square and its inverse has to be defined
%     \item the problem can be ill-conditioned so that the matrix must be regularized in order that small changes in currents do not generate large, uncontrolled fluctuations in field.  The regularization itself is nontrivial.
%     \item the PI loop must be tuned
% \end{itemize}


% \fig{Images/feedback}{width = 0.6\textwidth}{PI loop in flow chart.\label{fig: feedback}}
% The first measurement from the sensors after filtering (see section \ref{sec:f}) will act as setpoint ($B_{setpoint}$)


% The basic idea is that there will be a goal/setpoint with whom the consecutive measurement will be compared and any deviation from the setpoint will be minimized by the Proportional Integral (PI) feedback algorithm. For this prototype, the first measurement from the sensors after filtering (see section \ref{sec:filter}) will act as setpoint. Then the repeated measurements from the sensors have been taken. For each measurement, difference with the setpoint is  noted as -
% \begin{equation*}
%     \Delta B = B(setpoint) - B(measure)
% \end{equation*}
% % The current is related to magnetic field by- 
% % \begin{equation*}
% %     B=M \times I
% % \end{equation*}
% % Where, $\bm{M}$ is the matrix of proportionality factor in $nT/A$. So, the difference in current can easily be found by -
% % \begin{equation*}
% %     \Delta I=M^{-1} \times \Delta B
% % \end{equation*}
% On the basis of that, PI feedback algorithm will generate the required current that should be fed to the coils to compensate $\Delta B$. After certain period, a perturbation is also applied using an eletro-magnet coil to test how good the system is performing.  The above process is repeated as long as the compensation is running. 

The goal of this Section is to tell the general idea of how the system works, and to introduce the principles of operation.  It also introduces a few of the key issues faced when operating the system. 

First of all , fluxgates (see Section~\ref{sec:sensor}) is used to measure the magnetic field by placing them in different position within the coil cube (see Section~\ref{sec:cube}) and a setpoint for each fluxgate axis is decided after filtering (see Section~\ref{sec:filter}). When the fluxgate signal drifts from the setpoint, the error grows and how the fluxgates respond to changes in the current is described by a matrix. Moreover, to make the errors even more, a perturbation is also applied using an eletro-magnet coil as descibed in Section~\ref{sec:cube} to test how good the system is performing. The inverse of the matrix describes how to correct the currents based on the errors from the consecutive fluxgate signals. Propotional Integral or simply PI control algorithm  is used to decide the corrected currents based on the input errors from the fluxgate readings. As, the matrix isn't square and its inverse has to be defined. The problem arises as the matrix can be ill-conditioned which must be regularized in order that small changes in currents do not generate large, uncontrolled fluctuations in the field.  The regularization itself is nontrivial. After fixing those, the PI control loop must be tuned also.

In the upcoming Sections, the mathematical definitions required to explain system operation will be discussed. Issues encountered in ill-conditioning and regularization including the mathematical principles of regularization will be covered and some basic results of the measurements after tuning the PI parameters will be shown. Simulations and metrics of the system performance are also discussed. Chapter~\ref{ch:quantification} relates to use of these tools to characterize system performance, focusing on the novel aspects of any work.

\section{Matrix of Proportionality Factors}\label{sec:m}
% where subscript indices $s$, $c$ and $n$ has been used to indicate sensors, coil and no. of measurement respectively.

This Section is about the matrix which relates fluxgate readings to coil currents. As previously discussed in Sections~\ref{sec:cube} and~\ref{sec:sensor}, there are total 14 sensors and 7 coils for the prototype. Among them, 12 sensors and 6 coils are used for compensation and others are used for quantification. The matrix relates the currents in the six coils to the magnetic field readings in the 12 sensors. The magnetic field readings change linearly with current, represented by a constant matrix $\bm{M}$.  The matrix is a constant if we ignore hysteresis.  This is generally a good approximation because the magnetic permeability of the nearby magnetic shields is very large.

The relationship between the relative sensor readings and the coil currents is
\begin{equation}\label{eq:B_coils}
    B_s^n(\mathrm{coils})=\sum_{c=1}^{6} M_{sc} I_c^n
\end{equation}
where, $M_{sc}$ are the elements of the matrix $\bm{M}$ and $I_c^n$ is the current set on the coils for a particular measurements where $n$ runs from 0 to N for N being total no. of measurements. The sum is over coils $c$ where $c$ runs from 1 to 6 and the sensor index $s$ runs from 1 to 12. They have been defined in Table~\ref{table:index}. The field $B_s^n(coils)$ refers only to the relative field, in the sense that it is the field generated at sensor $s$ by the coil set. There could be other fields from the environment, which are the fields we seek ultimately to correct and will be discussed in the following Section~\ref{sec:pi}.


\begin{table} [htb!]
    \centering
    \begin{tabular} { |c|c|c|c|c|c|} 
        \hline
        Index & Range & Labels & Definition\\
        \hline\hline
        $c$ & 1-6  & $C_x^\pm$, $C_y^\pm$ and $C_z^\pm$  & Define specific coil \\ 
        \hline
        $S$ & 1-14  & \makecell{1$x$, 1$y$, 1$z$ or \\3$x$, 3$y$, 3$z$ or\\ center-$x$ ... center-$z$ \\ etc.}  & \makecell{Define the $x$, $y$ and $z$ \\of a specific position \\ (Numbering based \\on positions). \\Control sensors used\\ on the center\\ of the prototype} \\ 
        \hline
        $s$ & 1-12  & \makecell{Same as labels of $S$}  & \makecell{Subset of $S$ excluding\\ the 2 control sensors} \\ 
        \hline
        $n$ &  0$<$N$<\infty$ & 1,2,3,..,N  & \makecell{Define no. of \\PI loop iteration} \\ 
        \hline        
    \end{tabular}
    % \vspace{4mm}
    \caption[Definition of different indices to indicate sensor, coil and no. of iteration in the feedback loop]{Definition of different indices to indicate sensor, coil and no. of iteration in the feedback loop where $s$ or $S$ (if control sensors are included) can be taken different labels based on how the sensor position has been specified. For example- it could start with center-$x$ or 3$x$ which means the first sensor is at position center or 3 as specified in Section~ \ref{sec:cube} in $x$-axis respectively. Similarly, $c$ started with $C_x^-$ means the first coil is $C_x^-$ as defined again in Section~ \ref{sec:cube}. The index $n$ denotes the iteration number of the PI feedback control algorithm. }\label{table:index}
\end{table}

\FloatBarrier


The matrix defined in this way is easy to measure using the system.  Each coil $c$ is set to a current, with all other currents set to zero.  The change in the sensor reading $s$ then gives the matrix element $M_{sc}$. 


\fig{Images/Mdiff3_31}{width = \textwidth}{Color map of $\bm{M}$ measured using the scheme indicated in the text.  Horizontal axis indicates the various sensors, which are counted using the index $s$.  Vertical axis indicates the various coils, which are counted using the index $c$.  The color axis ($z$-axis) indicates the value of the matrix element.  Red elements indicate positive values while blue elements indicate negative values.  Elements that appear white are near zero in the matrix element.\label{fig:m}}{Color map of $\bm{M}$}

\FloatBarrier
The color map of a matrix $\bm{M}$ measured in this fashion is shown in Fig.~\ref{fig:m}.  The results are reasonable considering the design of the system.  The system was designed to compensate magnetic field of order 100~nT for currents of 200~mA, which would imply the matrix elements should be of scale 500~nT/A.  Generally this agrees with the color scale seen in Fig.~\ref{fig:m}.  Furthermore, the strongest matrix elements are those where the coil is closest to the sensor in question.  For example, 8y is closest to coils Y- and X+.  It also makes sense that 8y would have strong matrix elements here because it is near the corner of the magnetic shield which causes {\it e.g.} the X+ coil to be converted into a strong y component.  The red and blue colors are a result of the definitions of positive/negative current in relation to coil in question's orientation and the sensor axis definition.



\section{Implementation of PI Control Algorithm}\label{sec:pi}
 In my algorithm, the magnetic field is measured using the fluxgates and used to define the setpoints. As will be shown, when the field changes, the error in the field can be translated into an error in current based on inverting Eq.~(\ref{eq:B_coils}). New current values are then calculated that must be fed to the coils completing the PI control loop. 

% The measurement from the fluxgate sensors will have  the contribution from both the environment and  the field generated at sensors by the coil set of specific current value as presented in Eq.~(\ref{eq:B_coils} and can be related by 
% \begin{equation}\label{eq:B}
%     B_s^n (measure) = B_s^n(environment)+ B_s^n(coils)
% \end{equation}

The typical magnetic field of the surroundings has been measured using a fluxgate and is reported in Table~\ref{table:Benvironment}. The values are similar in scale to Earth$'$s magnetic field which is $\sim$45 $\mu$T in vertical downward direction. These are the typical scale of the setpoints.

\begin{table} [htb!]
    \centering
    \begin{tabular} { |c|c|c|c|c|c|} 
        \hline
        Axis & \makecell{Typical B field \\($\mu$ T)}\\
        \hline\hline
        $x$ & 10 \\ 
        \hline
        $y$ & 42 \\ 
        \hline
        $z$ & -5 \\ 
        \hline
    \end{tabular}
    % \vspace{4mm}
    \caption[Typical magnetic fields surrounding the prototype]{Typical magnetic field values of the environment surrounding the prototype obtained from fluxgate measurements when the $x$, $y$ and $z$ axis represent the northward, vertical downward, and westward direction respectively. }\label{table:Benvironment}
\end{table}

\FloatBarrier
For PI control, the setpoint is measured using the fluxgate array by applying 100 mA current to the coils. Our current sink (see Section~\ref{sec:sink}) can generate 0-200 mA current. So, the 100 mA current has been chosen to utilize the full capacity of the current sink to compensate the field fluctuations in either direction. After finding the setpoint from the first measurement of the fluxgates, the change in the fluxgate signal is
\begin{equation}\label{eq:del_B}
    \Delta B_s^n = B_s(setpoint) - B_s^n(measure)
\end{equation}

For iteration $n=1$, Eq~(\ref{eq:del_B}) will give zero value as the first measurement is acting as to be setpoint, as explained earlier. The consecutive measurements are used in the control algorithm. 

Using the relationship between the sensor readings and the coil currents as explained in Eq.~(\ref{eq:B_coils}) and the $\Delta B_s^n$ obtained from Eq.~(\ref{eq:del_B}), the error in terms of the current  will be
\begin{equation}\label{eq:del_I}
    \Delta I_c^n =\sum_{s=1}^{12} M^{-1}_{cs} \Delta B_s^n
\end{equation}
Here, $M^{-1}_{cs}$ are the elements of the inverse of the matrix $\bm{M}$. The inversion of the non-square matrix is a subtle problem and is discussed in Section~\ref{sec:inv}. The sum in Eq.~(\ref{eq:del_I}) is over sensors $s$ and the coil index $c$ runs from 1 to 6 as defined in Table~\ref{table:index}. $\Delta I_c^n$ is the error for $n^{\mathrm{th}}$ measurement on the basis of which the new set of currents will be calculated using a PI control algorithm. In the algorithm, the new current~\cite{bea} is
\begin{equation}\label{eq:I}
    I^n_c=I^0_c+k^p_c \Delta I_c^n+k^i_c\sum_n \Delta I_c^n
\end{equation}
where, $I^0_c$ indicates the initial value of current for the set of the coils when the setpoint has been set (100 mA). $\Delta I_c^n$ is the error found using Eq.~(\ref{eq:del_I}) and the sum is over no. of iterations $n$ where $n$ runs from 1 to $N$ as explained in Table~\ref{table:index}. The term $k^p_c$ is the proportional gain (P) and $k^i_c$ is the integral reset (I) for a particular coil. P and I that is $k^p_c$ and $k^i_c$ terms need to tune properly before they calculate the new current which will be explained later in Section~\ref{sec:tune}. The index $c$ in $k^p_c$ and $k^i_c$ terms represent the P and I value for coils $c$ where $c$ runs from 1 to 6. As only P and I terms have been used, so it is called Proportional Integral or simply PI controller~\cite{pid}. We limit ourselves from using derivative term (D) because of the amplification of noise while using it.


\section{Inversion of Matrix}\label{sec:inv}

This Section describes in detail the inversion process of the non-square matrix $\bm{M}$ and issues encountered. It also discusses the solution to inversion which uses regularization by random fluctuations, a strategy pursued previously by other groups. 


\fig{Images/1s1c3zc1}{width = \textwidth}{The compensation effect on sensor position 3z due to coil $C_x^-$. Vertical axis shows the $\Delta$ B found using Eq.~(\ref{eq:del_B}), where the green color indicates $\Delta$ B due to compensation and the red indicates $\Delta$ B without compensation. The red color dashed line indicates the time when the perturbation coil is turned on and the green color dashed line indicates when that is turned off. For position of the sensor and coils see the Fig.~\ref{fig:coil}.\label{fig:1s1c3zc1} }{The compensation effect on single sensor due to single coil.}

We have tested the one dimensional control of a single fluxgate sensor using a single coil current which is shown in Fig.~\ref{fig:1s1c3zc1}. It is seen that the magnetic field on sensor position 1z due to coil current $C_x^-$ has been stablized very well indicated by the green color as compared to the field that would have been without compensation indicated by the red color. 

\fig{Images/1s2c3zc1c62}{width = \textwidth}{The compensation effect on sensor position 1z due to coil $C_x^-$ and coil $C_z^+$. Vertical axis in left shows the coil currents that have been sent to coils $C_x^-$ (blue color) and $C_z^+$ (green color). $C_x^-$ and $C_z^+$ have been separated by 40 mA for showing them together in the same figure. Vertical axis in right shows the $\Delta$ B found using Eq.~(\ref{eq:del_B}), where the blue color indicates $\Delta$ B due to compensation and the red indicates $\Delta$ B without compensation. In both figures, the red color dashed line indicates the time when the perturbation coil is turned on and the green color dashed line indicates when that is turned off. For position of the sensor and coils see the Fig.~\ref{fig:coil}.\label{fig:1s2c3zc1c6}}{The compensation effect on single sensor due to two coils.}


Similar effect has been found if we use two coils current to control one sensor as shown in Fig.~\ref{fig:1s2c3zc1c6}. It is seen that the magnetic field on sensor position 1z (right) due to coils current $C_x^-$ and $C_z^+$ (left) has been stablized very well indicated by the blue color (right) as compared to the field that would have been without compensation indicated by the red color. 

\fig{Images/2s1c3y3zc12}{width = \textwidth}{The compensation effect on sensor position 1y and 1z due to coil $C_x^-$. Vertical axis in left shows the coil currents that have been sent to coil $C_x^-$. Vertical axes in middle and right show the $\Delta$ B found using Eq.~(\ref{eq:del_B}), where the blue (middle) and green (right) colors indicate $\Delta$ B due to compensation and the red in both indicates $\Delta$ B without compensation. In all three figures, the red color dashed line indicates the time when the perturbation coil is turned on and the green color dashed line indicates when that is turned off. For position of the sensors and coil see the Fig.~\ref{fig:coil}.\label{fig:2s1c3y3zc1}}{The compensation effect on two sensors due to one coils.}


We have also tested the magnetic compensation effect on two sensor positions due to single coil current as shown in Fig.~\ref{fig:2s1c3y3zc1}. It is seen that the magnetic field fluctuations ($\Delta$ B found using Eq.~(\ref{eq:del_B})) on sensor positions 1y (middle) and 1z (right) due to coil current $C_x^-$ (left) are similar/more while compensation indicated by blue color (middle) and green color (right) as compared to without compensation indicated by the red color. So, if we don$'$t have a matrix to deal with, this system seems to work fine.














% \begin{figure}[!htb]
%     \begin{subfigure}{.5\linewidth}
%         \centering
%         \includegraphics[width=\linewidth, height= 3 cm]{Images/1s1c3y}
%         \caption{for 3y and $C_x^-$ }
%         \label{fig:1s1c3y}
%     \end{subfigure}%
%     \begin{subfigure}{.5\linewidth}
%         \centering
%         \includegraphics[width=\linewidth, height= 3 cm]{Images/1s1c3z}
%         \caption{for 3z and $C_x^-$}
%         \label{fig:1s1c3z}
%     \end{subfigure}\\[1ex]
%     \begin{subfigure}{.5\linewidth}
%         \centering
%         \includegraphics[width=\linewidth, height= 3 cm]{Images/1s2c3yc1c6}
%         \caption{for 3y, 3z, $C_x^-$ and $C_z^+$}
%         \label{fig:1s2c3yc1c6}
%     \end{subfigure}%
%         \begin{subfigure}{.5\linewidth}
%         \centering
%         \includegraphics[width=\linewidth, height= 3 cm]{Images/1s2c3zc1c6}
%         \caption{for 3y, 3z, $C_x^-$ and $C_z^+$}
%         \label{fig:1s2c3zc1c6}
%     \end{subfigure}\\[1ex]
%     \begin{subfigure}{1\linewidth}
%         \centering
%         \includegraphics[width=\linewidth, height= 3 cm]{Images/2s1c3y3zc1}
%         \caption{for 3y, 3z and $C_x^-$ }
%         \label{fig:2s1c3y3zc1}
%     \end{subfigure}

%     \caption{The compensation effect due to individual coil. Vertical axis in Fig.~\ref{fig:1s1c3y} and in Fig~\ref{fig:1s1c3z} shows the $\Delta$ B found using Eq.~\ref{eq:del_B} over time for two different sensors where the green indicates $\Delta$ B due to compensation and the red indicates $\Delta$ B without compensation. Vertical axis in left of both Fig.~\ref{fig:1s2c3yc1c6} and  Fig.~\ref{fig:1s2c3zc1c6} indicates the coil current that has been sent to coils $C_x^-$ and $C_z^+$ to satblize 3y and 3z as shoen in right of those respectively. Initially both $C_x^-$ and $C_z^+$ were in the same level but for showing them together in the same figure they have been separated by some constant.  Vertical axis in left of Fig.~\ref{fig:2s1c3y3zc1} also indicates the current but only for $C_x^-$. Vertical axis in right of Fig.~\ref{fig:1s2c3yc1c6} and  Fig.~\ref{fig:1s2c3zc1c6} and in middle and right of Fig.~\ref{fig:2s1c3y3zc1} have the same description as in in Fig~\ref{fig:1s1c3z} }
%     \label{fig:1d}
% \end{figure}

\FloatBarrier
The author in Ref.~\cite{bea} in Section \textcolor{blue}{7.2.3} of her PhD thesis claimed that more sensors than coils with some optimization give better compensation forming a non-square matrix which needs to be inverted for calculating the error using Eq.~\ref{eq:del_I}. The Moore--Penrose pseudoinverse can be used for the inversion process. A tutorial review of the pseudoinverse can be found in Ref.~\cite{pseudo}. The pseudoinverse of a matrix can be computed using singular value decompostion (SVD). SVD is a factorization or diagonalization of a matrix. The matrix can be real or complex square or rectangular matrix. The SVD~\cite{svd2,svd3} of a real rectangular matrix $\bm{M}$ with dimension $m \times n$ where $m<n$ is 
\begin{equation}\label{eq:m}
        \bm{M} = \bm{U} \bm{\Sigma} \bm{V^T}
\end{equation}
% where, $\bm{U}\:\epsilon\:\mathbb{R}^{m \times m}$, $\bm{V^*}\:\epsilon\:\mathbb{R}^{n \times n}$ 
where, $\bm{U}$ and $\bm{V}$ are the orthogonal matrices with dimensions $m \times m$ and $n \times n$ respectively and $\bm{V^T}$ is the transpose of $\bm{V}$. $\bm{\Sigma}$ is a real non-negative diagonal matrix with same dimension as  $\bm{M}$ and can be written as

\begin{equation*}
\bm{\Sigma} = \begin{pmatrix} 
\Sigma_{11} &  &  & 0 \\
  & . &  &  \\
  &   & . &  \\

0 &  &  & \Sigma_{nn} \\
 \end{pmatrix}
\end{equation*}
where, $\Sigma_{11},..,\Sigma_{nn}=\text{diag}(\Sigma)$ with $\Sigma_{11}\gg... \Sigma_{nn}\geq0$ and are called singular values of $\bm{M}$  and they are the positive square roots of the non-negative eigenvalues of $\bm{M^T}\bm{M}$.

The point of SVD is that this is easy to invert because the transpose of an orthogonal matrix is equal to its invert. That is   $\bm{U^T}=\bm{U^{-1}}$ and $\bm{V^T}=\bm{V^{-1}}$ and
\begin{equation}\label{eq:sigma_inv}
\frac{1}{\bm{\Sigma}} = \begin{pmatrix} 
\frac{1}{\Sigma_{1}} &  &  & 0 \\
  & . &  &  \\
  &   & . &  \\

0 &  &  & \frac{1}{\Sigma_{n}} \\
 \end{pmatrix}
\end{equation}

The pseudoinverse of $\bm{M}$ will be then the inverse of $\bm{U}$, $\bm{\Sigma}$ and $\bm{V^T}$ and can be written as
\begin{equation}\label{eq:psMinv}
    \bm{M^{-1}} = \bm{V \Sigma^{-1} U^T}
\end{equation}
% ${\rm I\!R}$

\fig{Images/6c_U_Mcond26_p1368}{width = \textwidth}{Color map of $\bm{U}$ for transpose of $\bm{M}$ shown in Fig.~\ref{fig:m}. $\bm{U}$ is orthonormal eigenvectors of $\bm{M}\bm{M^T}$ in sensors$\times$sensors dimension for 12 sensors signals (nT). Red elements indicate positive values while blue elements indicate near zero values. \label{fig:6c_U}}{Color map of $\bm{U}$.}

\fig{Images/6c_V_Mcond26_p1368}{width = \textwidth}{Color map of $\bm{\Sigma}$ for transpose of $\bm{M}$ shown in Fig.~\ref{fig:m}. $\bm{\Sigma}$ is positive square roots of the non-negative eigenvalues of $\bm{M^T}\bm{M}$ in sensors$\times$coils dimension for 12 sensors and 6 coils in nT/A. Red elements indicate positive values while blue elements indicate near zero values. \label{fig:v}}{Color map of $\bm{\Sigma}$}

\fig{Images/6c_Wt_Mcond26_p1368}{width = \textwidth}{Color map of $\bm{V^T}$ for transpose of $\bm{M}$ shown in Fig.~\ref{fig:m}. $\bm{V^T}$ is orthonormal eigenvectors of $\bm{M^T}\bm{M}$ in coils$\times$coils dimension for 6 coils currents (A). Red elements indicate positive values while blue elements indicate near zero values. \label{fig:6c_Vt}}{Color map of $\bm{V^T}$.}

In earlier Section we have discussed a matrix $\bm{M}$ which is shown in Fig.~\ref{fig:m}. $\bm{M^T}$ is the transpose of that matrix. $\bm{M^T}$ has been decomposed via SVD in $\bm{U}$ , $\bm{\Sigma}$ and $\bm{V^T}$ which are shown in Fig.~\ref{fig:6c_U}, Fig.~\ref{fig:v} and Fig.~\ref{fig:6c_Vt} respectively. The $\bm{\Sigma}$ in Fig.~\ref{fig:v} shows that except diagonal values all are zero with the diagonal values are arranged from $\Sigma_{11}$ to $\Sigma_{nn}$ in a decreasing order each having non-negative values. From those values the condition number of $\bm{M^T}$  can be found. The condition number of a matrix measures the change in the output for a small change in input. Large condition number indicates ill-conditioned matrix while small condition number indicates well-conditioned matrix. The condition number of $\bm{M^T}$ can be determined from the diagonal matrix $\bm{\Sigma}$ as 
 \begin{equation}\label{eq:cond}
     cond(\bm{M})=\frac{max(\sigma_n)}{min(\sigma_n)}
 \end{equation}
 
Using Eq.~(\ref{eq:cond}), the condition number of $\bm{M^T}$=5676/215=26.4 which indicates an ill-conditioned matrix.

Elements of $\bm{\Sigma}$ represent eigenvalues. Since no of coils ($c$) is less than no. of fluxgate sensors ($s$), it is easily thought of as represent modes of the coil set. A large $\sigma_1$ represents a coil mode where a small change in current gives rise to a large change in field. If $\sigma_c<<\sigma_1$ it means that mode corresponding to  $\sigma_c$ requires a much larger current in order to generate the same scale of magnetic field as for the $\sigma_1$ mode. The condition number of $\bm{M}$ being large therefore indicates that problematic modes exist which will have practically no control on the magnetic field. In Section~\ref{sec:coil_config} we analyzed this effect further in both simulation and experiment. For now, we present the strategy of Ref.~\cite{bea} which we followed initially and it involved regularization.

Tikhonov regularization~\cite{tikhonov2013numerical,tikhonov_book,svd,svd3} is one of the most commonly used regularization methods which modifies the diagonal elements of $\Sigma^{-1}$ from Eq.~(\ref{eq:psMinv}) as

\begin{equation}\label{eq:minvR}
    \frac{1}{\sigma_n} \rightarrow \frac{\sigma_n}{\sigma_n^2+\alpha^2} 
\end{equation}
Tikhonov regularization has been further modified by defining $\alpha=10^r$ as in the previous study \cite{bea}) where, $r$ is called the regularization parameter. If  $r \rightarrow - \infty$, Eq.~(\ref{eq:minvR}) becomes the same as Eq.~(\ref{eq:sigma_inv}) hence no regularization whereas $r \rightarrow + \infty$ results in $\bm{M^{-1}} \rightarrow 0$ i.e. no control. Generally, $r$ should be of order $\mathbf{log}(\sigma_{d})$ in order for regularization to have its desired effect of making the diagonal values of $\bm{\Sigma^{-1}}$ more equal. Example : suppose $\sigma_1$=100 and $\sigma_n$=1 and $\alpha$=10.

\begin{equation*}
    \frac{1}{\sigma_1} \rightarrow \frac{100}{100^2+10^2}=\frac{1}{101} 
\end{equation*}
\begin{equation*}
    \frac{1}{\sigma_n} \rightarrow \frac{1}{1^2+10^2}=\frac{1}{101} 
\end{equation*}
Therefore, the singular values become equalized.

The value of $r$ may be selected using several iterative methods. The obtained $r$ can be directly used in feedback algorithm to determine $\bm{M^{-1}}$. Because, $\bm{M^{-1}}$ appears in the definition of the error in current (Eq.~(\ref{eq:del_I})). It has an effect also on the PI parameters. The PI parameters can be tuned in concept with $r$ by observing the effect on current response and this was studied further in Section~\ref{sec:r_pi}.


An iterative method has been discussed in  the previous study~\cite{bea} to find $r$. The concept has been applied to the prototype which will be discussed next.

% \FloatBarrier

% \begin{equation}\label{eq:sigma}
%     \frac{1}{\sigma_d} \rightarrow \frac{\sigma_d}{\sigma_d^2+(10^r)^2}
% \end{equation}

% So, the Eq.~(\ref{eq:psMinv}) will be modified due to the modified form of Tikhnov regularization as
% \begin{equation}\label{eq:minvR}
%     \bm{M^{-1}}(r) = \bm{V}\begin{pmatrix} 
% \frac{\sigma_1}{\sigma_1^2+(10^r)^2} & 0 \\
% 0 & \frac{\sigma_d}{\sigma_d^2+(10^r)^2}
% \end{pmatrix} \bm{U^T}
% \end{equation}



% \fig{Images/v}{width = \textwidth}{Color map of $\Sigma$ for a test $\bm{M}$ found using Python programming language. Horizontal axis indicates the various coils, which are counted using the index $c$. Vertical axis indicates the various sensors, which are counted using the index $s$. Red elements indicate positive values while blue elements indicate near zero values. \label{fig:v}}{Color map of $\Sigma$}

% \FloatBarrier

% The problem can visualized from Fig.~\ref{fig:m} where for example X-(1) has negligible sensitivity for 3x which means that particular matrix element (in nT/A) is very small. So, the pseudoinverse of $\bm{M}$ using Eq.~\ref{eq:psMinv} will produce very large element (in A/nT) which means that the amount of current will be huge there which will make the prototype unstable. To minimize the effect, the matrix must be regularized which means that all ill-conditioned places must be replaced  by well-conditioned ones.  



\subsection{Regularization by Random Fluctuation}\label{sec:mont}

The method of a previous study~\cite{bea} was reproduced for my prototype to determine a value of $r$ by study its effect on the ability to cancel random field fluctuations without generating unacceptably large current fluctuations.

For the method, a different sets of reasonable random magnetic fields ($B_s^{\text{rand}}$) are generated according to the normal distribution with a central value of 0 and standard deviation 1.5 nT. As in Ref.~\cite{bea}, the reasonable value for the standard deviation was determined by the scale of the fluctuations seen from second to second by the sensor array. The exact value of the standard deviation will turn out to be unimportant in the way Ref.~\cite{bea} finally deduces $r$ using normalized field and current fluctuations.

Using the setpoint as zero, according to Eq.~(\ref{eq:del_B}) the change in the B field is  
\begin{equation}\label{eq:del_Bs}
    \Delta B_s^{\text{sim}} = 0 - B_s^{\text{rand}}=-B_s^{\text{rand}}
\end{equation}

The array of the current errors  as function of $r$ due to the the change in field $\Delta B_s^{\text{sim}}$ is then calculated using the regularized pseudoinverse using Eq.~(\ref{eq:del_I}) as
\begin{equation}\label{eq:del_Is}
    \Delta I_c^{\text{sim}}(r) =\sum_{s=1}^{12} M^{-1}_{cs}(r) \Delta B_s^{\text{sim}}=\sum_{s=1}^{12} M^{-1}_{cs}(r) (-B_s^{\text{rand}})
\end{equation}

To estimate the overall response from the array of the current, the root mean square (RMS) of $\Delta I_c^{\text{sim}}(r)$ is calculated as
\begin{equation}\label{eq:delta_Isim_rms}
     \Delta I_{\text{RMS}}^{\text{sim}}(r)= \sqrt{\frac{1}{6}\sum_{c=1}^6 (\Delta I_c^{\text{sim}}(r))^2}
\end{equation}

Since, $\Delta I_c^{\text{sim}}(r)$ depends on both $r$ and $B_s^{\text{rand}}$, $ \Delta I_{\text{RMS}}^{\text{sim}}(r)$  is calculated as a function of $r$ for different sets of $B_s^{\text{rand}}$ (Fig.~\ref{fig:Isim}). It is noticeable that with the increase of $r$, the current fluctuations $\Delta I_{\text{RMS}}^{\text{sim}}$ vanish as expected.


% $ \Delta I_{\text{RMS}}^{\text{sim}}(r)$ as indicated by vertical axis over $r$ as indicated by horizontal axis has been shown for 30 different sets of $B_s^{\text{rand}}$. The distribution $B_s^{\text{rand}}$ is randomly chosen with center of distribution and standard deviation is discussed earlier. 

\fig{Images/6c_I}{width = \textwidth,height =9.5cm}{The effect of $r$ on the coil currents for 30 different sets of $B_s^{\text{rand}}$ generated according to the normal distribution with a central value of 0 and standard deviation 1.5 nT. The horizontal axis represents $r$ while the vertical axis shows $\Delta I_{\text{RMS}}^{\text{sim}}$. \label{fig:Isim}}{The effect of $r$ on the coil currents.}

\FloatBarrier
The field produced by $ \Delta I_c^{\text{sim}}$ can be calculated using Eq.~(\ref{eq:B_coils}). Thus the total field at each sensor $s$ will be the superposition of $B_s^{\text{rand}}$ and the response produced by $ \Delta I_c^{\text{sim}}(r)$ 
\begin{equation}\label{eq:B_coils-sim}
    B_s^{\text{sim}}(r) =\sum_{c=1}^6 M_{sc} \Delta I_c^{\text{sim}}(r) + B_s^{\text{rand}}
\end{equation}
For a perfectly compensated system, the field produced by $ \Delta I_c^{\text{sim}}$ would equal $- B_s^{\text{rand}}$ which in turn would make the $B_s^{\text{sim}}(r)$ in Eq.~(\ref{eq:B_coils-sim}) identically zero. In practise, this is rarely the case. To quantify the effectiveness of the compensation,  the ratio of RMS of $B_s^{\text{sim}}(r)$ to the RMS of $B_s^{\text{rand}}$ is calculated as
\begin{equation}\label{eq:fluc}
    F(r)=\frac{\sqrt{\frac{1}{12} \sum_{s=1}^{12} (B_s^{\text{sim}}(r))^2}}{\sqrt{\frac{1}{12} \sum_{s=1}^{12} (B_s^{\text{rand}}(r))^2}}
\end{equation}
The function $F(r)$ would be zero for a perfectly compensated system and unity for an uncompensated system. In Ref.~\cite{bea}, $F(r)$ is called the "remaining noise". The values of $F(r)$ is shown in Fig.~\ref{fig:fluc-sim} for same sets of $B_s^{\text{rand}}$ used in Fig.~\ref{fig:Isim}. It is seen that with the increase of $r$, the field produced due to $\Delta I_c^{\text{sim}}$ to compensate $B_s^{\text{rand}}$  are increasing i.e. more field fluctuations. It is also noticeable that the system can not be fully compensated due to the field produced by $ \Delta I_c^{\text{sim}}(r)$ as $F(r)$ never goes to zero. The lowest $F(r)$ is 0.45, indicating the system will not be terribly successful at correcting random fluctuations. We think this is mainly due to the purposely limited design used for this prototype, which was designed instead to focus on issues in multi-dimensional PI control.

\fig{Images/6c_f}{width = \textwidth,height =9.5cm}{The effect of $r$ on horizontal axis on the compensation due to the field produced by $ \Delta I_c^{\text{sim}}(r)$ to counteract $B_s^{\text{rand}}$ indicated by $F(r)$ on vertical axis. The different curves indicate 30 different sets of $B_s^{\text{rand}}$ with the distribution describes earlier. \label{fig:fluc-sim}}{The effect of $r$ on remaining fluctuations.}

\FloatBarrier
It is seen from the Fig.~\ref{fig:Isim} and Fig.~\ref{fig:fluc-sim} that with the increase of $r$, current fluctuations are decreasing but field fluctuations are increasing step. Ref.~\cite{bea} suggested a compromise between them be struck to determine the value of $r$, $\it i.e.$ that reducing current fluctuations ($r \rightarrow + \infty$) be traded off against reducing magnetic field fluctuations ($r \rightarrow - \infty$). to decide the value of $r$, $\Delta I_{\text{RMS}}^{\text{sim}}(r)$ and $F(r)$ were normalized as

\begin{equation}\label{eq:Inorm}
    \overline{\Delta I_{\text{RMS}}^{\text{sim}}}(r)=\frac{\Delta I_{\text{RMS}}^{\text{sim}}(r)}{\Delta I_{\text{RMS}}^{\text{sim}}(r\rightarrow - \infty)} \;\mathbf{,\;and}
\end{equation}
\begin{equation}\label{eq:flucNorm}
    \overline{F}(r)=\frac{F(r)- F(r\rightarrow - \infty)}{F(r\rightarrow \infty)- F(r\rightarrow - \infty)}
\end{equation}

\fig{Images/6c_I-fluc2}{width = \textwidth,height=9cm}{The effect of $r$ indicated by horizontal axis on $\overline{\Delta I_{\text{RMS}}^{\text{sim}}}$ indicated by left vertical axis and $\overline{F}$ indicated by right vertical. The different curves on both vertical axes indicate 30 different sets of $B_s^{\text{rand}}$ with the distribution describes earlier. The red line indicates the 0.5 level in the figure.  \label{fig:I-fluc}}{The effect of $r$ on normalized current and remaining fluctuations.}

\FloatBarrier
The effect of $r$ on $\overline{\Delta I_{\text{RMS}}^{\text{sim}}}$ and $\overline{F}$ is shown in Fig.~\ref{fig:I-fluc}. With an increase in $r$, $\overline{\Delta I_{\text{RMS}}^{\text{sim}}}$ decreases and $\overline{F}$ increases as expected from the discussion earlier.
Two values of $r$ were then found for each $B_s^{\text{rand}}$ by alternatively setting $r$ on $\overline{\Delta I_{\text{RMS}}^{\text{sim}}(r)}$ and $\overline{F(r)}$=0.5. This is indicated schematically by the horizontal line in Fig.~\ref{fig:I-fluc}. The values of $r$ so determined are averaged and in turn the averages are averaged over a large number of $B_s^{\text{rand}}$ ($>>$30). The optimized $r$ is thus found to be 2.87.

The calculation to find $r$ gives insight about the effect of regularization on current and field fluctuations, resulting in a compromise that adequately reduces both. With $r$ in hand, $\bm{M^{-1}}$ may be determined and the current error  ?? which is used in the PI algorithm.

\section{Tuning of PI Parameters}\label{sec:tune}
This section describes the tuning of the proportional (P) gain $k_c^p$ and integral (I) reset $k_c^i$ terms appearing in Eq.~(\ref{eq:I}) to compensate the changes in magnetic fields measured by the fluxgate sensors.

According to the type of system and the way the control loop for a particular system has been chosen, the P and I can be tuned using various methods~\cite{tuning}. A possible tuning method is Ziegler-Nichols closed tuning method~\cite{tuning_ZN}. We usually used this method as an initial guess for the PI parameters. Here, the tuning will be described in its application to our prototype. The control variables in our case are the currents in the coils. The errors are also currents, those deduced from matrix inversion and magnetic sensors deviations from setpoints. In Ziegler-Nichols, first, $k_c^i$ is set to zero and $k_c^p$ is increased until the currents in the different coils start oscillating. The $k_c^p$ value for which the current in the coils start oscillating is noted as the ultimate gain $G_{u}$ and the period of the oscillation is noted as ultimate oscillation period $T_u$. Now the the value of $k_c^p$ and $k_c^i$ are chosen based on the PI row of Table~\ref{table:tuning} modified from the Table~\textcolor{blue}{4} in Ref.~\cite{tuning_formula} where integral time $T_i=T_u/1.2$ is given. 

\begin{table} [htb!]
    \centering
    \begin{tabular} { |c|c|c|c|c|c|} 
        \hline
        Controller & Gain ($k_c^p$) & Reset ($k_c^i$)\\
        \hline\hline
         P & 0.5 $G_u$ & 0 \\ 
        \hline
         PI & 0.45 $G_u$ & $\left(\frac{\text{0.54} G_u}{T_u}\right)\Delta t$ \\ 
        \hline
    \end{tabular}
    % \vspace{4mm}
    \caption{Ziegler-Nichols tuning method for P and PI controllers.}\label{table:tuning}
\end{table}

\FloatBarrier
In Ref.~\cite{tuning_formula}, it is also mentioned that $k_c^i=(k_c^p/T_i)$ which for our case will be $k_c^i=(k_c^p/T_i)\Delta t$, where, $\Delta t$ is the time difference between two consecutive feedback loop. So, $k_c^i$ will be
\begin{equation}
    k_c^i=\left(\frac{k_c^p}{T_u/1.2}\right)\Delta t=\left(\frac{1.2\times0.45 G_u}{T_u}\right)\Delta t=\left(\frac{0.54 G_u}{T_u}\right)\Delta t
\end{equation}

\fig{Images/p134c1}{width = \textwidth}{Zoomed current behaviour in coil $C_x^+$ with $k_c^p$ =1.34 and $k_c^i$=0. Vertical axis represent the currents in coil $C_x^+$ with initial current being 100 mA. For position of the coil see Fig.~\ref{fig:coil}. \label{fig:tuning}}{Zoomed current behaviour in coil $C_x^+$.}
% in response to the perturbation coils
\FloatBarrier
Fig.~\ref{fig:tuning} shows the first step in the tuning process for the prototype. At $k_c^p$=1.34 and $k_c^i$=0, the current in the coils oscillates allowing us to identify $G_u$=1.34. For simplicity, zoomed version of the current in coil $C_x^+$ for $G_u$=1.34 is shown only. The ultimate period $T_u$ is obtained to be $T_u$=0.287 s. Now, according to Table~\ref{table:tuning}, the proportional gain and integral reset are

\begin{equation}
    k_c^p=0.45\times1.34=0.60\;\;\text{and}\;\; k_c^i=\left(\frac{0.54 \times1.34}{0.287}\right)\times0.146=0.37
\end{equation}
They can be further tuned for the individual coil currents if necessary. In general, we treated them as free parameters and studied the impact of changing them on system response. More about PI tuning will be discussed in the next chapter with compensation results. Next, a simulation model will be discussed to quantify the prototype.



% \doublefig{Images/p97}{width =\textwidth,height=8cm}{at $k_c^p$=0.97. \label{fig:tuningNmod}}{Images/p97mod}{width = \textwidth,height=8cm}{$k_c^p$=0.97 with zoomed $C_y^-$.\label{fig:tuningMod}}{{The current behaviour in all six coils with $k_c^p$ =0.97 and $k_c^i$=0. Vertical axes represent the currents in all six coils having different colors. The 'ON' and 'OFF' vertical dashed lines indicate the time of the perturbation coil being turned 'ON' and 'OFF' respectively. Both the figures are same except the current in $C_z^+$ coil has been zoomed in left figure.  } \label{fig:tuning}}{Tuning by observing coil currents}



% Moreover, to double check, the experimentally obtained $\bm{M}$ has been compared with a simulation done using finite element analysis (FEA) via Opera 3D software as shown in Fig.\ref{fig:Mdiff}.


%  \fig{Images/Mdiff}{width =  \textwidth}{Comparison of Experimental M with Simulation \label{fig:Mdiff}}
 
 
% 

% Fig.\ref{fig:bt} shows the compensation over time with applied perturbation. \newpage
\FloatBarrier




\section{Quantitative Measures of Magnetic Compensation Performance\label{sec:metrics}}
For every system, there are some metrics, which quantitatively determine the system's performance. The success of the prototype has been claimed similarly. Mainly, four areas have been looked into as derived from the previous studies \cite{bea,lins,rawlik}. They are

\begin{itemize}
    \item Condition Number
    \item Reduction of Magnetic Field Fluctuation
    \item PI Behavior to Stimulus
    \item Allan Deviations and Shielding Factor

\end{itemize}
\subsection{Condition Number}
The condition number of a matrix and its relation to the regularization parameter have already been discussed in Section~\ref{sec:inv}. Recall that the condition number of a matrix is $\geq$1, where low condition number indicates a well-conditioned matrix while large indicates a ill-conditioned matrix.  The condition number can be influenced by the dimension of a matrix. But, the change in condition number in case of  well-conditioned matrix is incomparably smaller than that in a ill-conditioned matrix~\cite{cond_m_size}. However, there is no upper bound that indicates the limit of well-conditioned matrix. So, there is no definite answer about how small is considered as well-conditioned matrix. For the active compensation system, the goal is to have matrix condition number as close to 1 as possible so that compensating a certain scale of change in magnetic field requires less change in current and also the compensation system does not loose its effect on all the sensors while making the condition number less. The results with well and ill-conditioned matrix are discussed in Chapter~\ref{ch:quantification}.

% The main aim is to make the prototype well-conditioned by determining the best compromise between current and field fluctuations which will lead us to the optimized $r$. It has been also explored to determine whether the optimized $r$ is giving the best result or not. If not, additional parameters have been tuned to get the best possible result.

\subsection{Reduction of Magnetic Field Fluctuation}

Active magnetic field compensation come into play in the first place to make the surrounding magnetic fluctuations as small as possible. So, prototype response for magnetic fluctuations has been studied in terms of its capabilities as reduction or amplification. We have already shown in Fig.~\ref{fig:fluc-sim} that the lowest remaining fluctuations due to several random fluctuations is 0.45. Ideally, the remaining fluctuations should be zero. The system is completely unsuccessful if the remaining fluctuation is found to be 1. Our goal for the prototype is to get remaining fluctuation at least 0.5.


\subsection{PI Behavior to Stimulus}
Tuning is one of the main factor of PI control feedback algorithm. After going through all the steps, it needs to have a suitable value of PI parameters which can be obtained by tuning the prototype as discussed in section \ref{sec:tune}. Moreover, PI behavior has been also monitored for step response in the perturb electromagnet coil. As the main concern is about the experiment which will be carried inside the passive shielding, the response has been specially studied on the central sensors. Some of the responses have been shown in Fig. \ref{fig:bt_mod}. It is seen that without the compensation system (red colors) there are $\sim$+20 nT step responses in  1x and 1y sensor positions and $\sim$-30 nT step responses in 1z, 8x and 8y sensor positions. The magnetic field without the compensation effect i.e. uncompensated magnetic field is predicted by
\begin{equation}\label{eq:Buncomp}
     B_s^n(\text{uncompensated})=B_s^n(\text{measured})- B_s^n(\text{coils})
\end{equation}
where, $B_s^n(\text{measured})$ is the actual measurement from coming the fluxgate sensors with $n$=(1,...N) is the no of measurements taken and $B_s^n(coils)$ is determined using Eq.~(\ref{eq:B_coils}). The active compensation can fully eliminate the step response in 1x sensor position and to some levels in other positions indicated by blue, green and cyan colors. Moreover, the last one named 'center-Z' predicts the response in the center of the prototype in $z$ axis without actually being in the feedback loop and there is also improvement in terms of step response level. We are considering the good results if the system response time is fastest without any overshoot and also reduces the step response levels as much as possible. The results are considered bad if the there is very slow response of the system.

\fig{Images/bt-mod}{width =\textwidth,height=10cm}{Magnetic field reductions from some of the sensor positions closer to the perturbation coil. For positions of the sensors please see Fig.~\ref{fig:coil}. Vertical axis represents the $\Delta B_s$ (see Eq~(\ref{eq:del_B})) and the horizontal axis represent the time. All the colors has been discussed in the text. The 'ON' and 'OFF' vertical dashed lines indicate the time of the perturbation coil being turned 'ON' and 'OFF' respectively. \label{fig:bt_mod}}{Magnetic field reductions from some of the sensor positions.}
\FloatBarrier
 
 \subsection{Allan Deviations and Shielding Factor}
 
Allan standard deviation \cite{allan} is usually applied to the time series to determine the time stability in clocks, amplifiers and oscillators. The same concept can be applied to find the magnetic field stability in time~\cite{bea} as
\begin{equation}\label{eq:adev}
    \sigma_{Adev} (\tau)=\sqrt{\frac{1}{2(N-1)}\sum_{l=1}^{N-1} \left(B_{l+1}(\tau)-B_l(\tau)\right)^2}
\end{equation}
where, $B_l(\tau)$ is the average magnetic field for a sub sample $l$ over integration time $\tau$ and $\tau = \frac{T}{N}$ with $T$ is the total measurement time and $N$ is total number of sub samples. The largest integration time for total measurement time $T$ is $\tau_{\text{max}}=T/2$ and for that the total number of sub sample is $N = \frac{T}{\tau_{\text{max}}}=\frac{T}{T/2}=2$. So, $\sigma_{Adev}(\tau_{\text{max}})$ is negligible since there is only one difference value of sub sample as seen by Eq.~(\ref{eq:adev}). Moreover, $\sigma_{Adev}\approx$1/$\sqrt{\tau}$ indicates random noise, $\sigma_{Adev}\approx \sqrt{\tau}$ indicates random walk and $\sigma_{Adev}\approx \tau$ indicates linear drift. 

The Allan deviation calculation of $\Delta$B found by measuring 8x without any feedback loop for tau=100 s has been shown in Fig.~\ref{fig:sf_mod}\textcolor{blue}{(a)}. The Fig.~\ref{fig:sf_mod}\textcolor{blue}{(b)} shows the Allan deviation of the same $\Delta$B for different taus.  It is seen that $\sigma_{Adev}\approx \sqrt{\tau}$ which indicates random walk when there is no compensation system is running. For the prototype, $\sigma_{Adev}\approx$1/$\sqrt{\tau}$ considers as good result while $\sigma_{Adev}\approx \sqrt{\tau}$ or $\approx \tau$ consider to be bad. That is the Allan deviation of the magnetic field due to compensation should be always less than the Allan deviation of the uncompensated magnetic field which indicates that the magnetic field is less fluctuated i.e. more stable during compensation than without compensation in time. 
% \fig{Images/sf_mod}{width =\textwidth,height=12cm}{The Allan deviation (top) and shielding factor (bottom) for signals from some fluxgate sensors. Top vertical axis indicates the Allan deviation while bottom indicates shielding factor. Horizontal axis indicates the integration time. The different colors in top and bottom have been discussed in text.  \label{fig:sf_mod}}{The Allan deviation.}

\fig{Images/sf_b}{width =\textwidth,height=12cm}{(a)$\Delta$B without feedback loop in sensor position 8x and (b) The Allan standard deviation of that $\Delta$B. Red horizontal lines in (a) indicate the Allan deviations for tau=100 s of 10 sub samples of $\Delta$B deviation and vertical black dashed lines indicate the upper limit of each sub sample. The Allan deviation of $\Delta$B in (a) for different taus are shown in (b). For position of sensor please see Fig.~\ref{fig:coil}. \label{fig:sf_mod}}{The Allan deviation.}


Moreover, to determine the factor by which the magnetic field is more stable in compensation than uncompensated at a given $\tau$, the shielding factor~\cite{bea} has been calculated as
\begin{equation}\label{eq:sf}
    \text{sf} (\tau)=\frac{\sigma_{\text{Adev}}^{\text{uncompensated}}(\tau)}{\sigma_{\text{Adev}}^{\text{measured}}(\tau)}
\end{equation}
So, the shielding factor in any of the fluxgate sensorat a given $\tau$  $>$ 1 indicates that the magnetic field is more stable on compensation than uncompensated on that fluxgate sensor on that $\tau$. Shielding factor is a vital factor to observe the long term stability under no stimulus condition. For the prototype, $\text{sf} (\tau)<$1 indicates terrible compensation, $\text{sf} (\tau)=$1 indicates bad compensation and $\text{sf} (\tau)>$1 indicates good result.
 

%\fig{Images/sf}{width = 0.5 \textwidth}{Allan Deviation and Shielding Factor \label{fig:sf}}
 
 
 

 
\lhead{\emph{Quantification of AMC Prototype}}
%\part{Privacy Preserving Approximation of Edit Distance on Genomic Data}

\chapter{Quantification of AMC Prototype}\label{ch:quantification}
% \label{chap:image}

In Chapter~\ref{ch:operation}, the magnetic control process and the simulation methods to understand the process have been described. In addition to that for further quantification of the system some metrics have been defined. This chapter starts with the results following those metrics and side by side it also shows the results in terms of simulation methods. The chapter also includes some unique points which aren't explicitly present on previous studies on active compensation. Finally, it will end with some suggestions on fluxgate placements and effect of different shields. The new directions for future studies set by the unique points and the fluxgates placement's suggestions are presented in Chapter~\ref{ch:conclusion}.

% There are vast number of parameters that can be varied and studied for active compensation. For quantifying the the prototype, five among them are chosen based on importance, time limitation, resources available etc. and studied via both experiment and simulation. Those will create a huge impact for the future studies on active compensation. Some unique points have been discussed which aren't explicitly present on previous studies on active compensation. Their successfulness have been discussed using some metrics (see Section~\ref{sec:metrics}). In this chapter, studies on the parameter as well as the metrics to justify them will be discussed in terms of results.

\section{Sampling Frequency and Filtering}\label{sec:freq}


\begin{itemize}
\item We built some analog filters which were discussed in Chapter 3
\item Goal of the filter was to remove high-frequency noise (low-pass Butterworth with 10~Hz corner frequency)
\item This would allow the ADC to operate with less averaging, reducing its effective sampling time (denoted by the ``resolution index'')  (refer to Ch. 3)
\item We show the effectiveness in this section.
\end{itemize}

This Section talks about the sampling frequency and the effect of
filtering in sampling frequency. It starts with defining the sampling
frequency and how we first consider building an analog filter in the
first place. Then it shows the effect improving the response time on
the prototype. Finally, the Section wraps up with some comparison in
current response time for different sampling frequency.


Sampling frequency is the number of samples per second in a signal. For the prototype, by sampling frequency we meant that the number of signals of the fluxgate sensors that are going to the feedback algorithm via analog to digital converter (ADC) per second per measurement. But we are recording the time for a complete iteration which we call it as loop sampling frequency which includes the ADC sampling frequency plus rest of the loop frequency. In the early days of the prototype experiment, we have used the slowest sampling rate which is corresponding to resolution index 12 in Table~\ref{table:t7freq}.  The problem with that was the response time for the current was very slow and so does the compensation of the magnetic field. Then we have decided to use the fastest sampling frequency that the ADC can offer which is corresponding to resolution index 1 in Table~\ref{table:t7freq}. But new problem has arisen in terms of noise especially the 60 Hz electrical noise from the power source which has motivated us to build the 4\textsuperscript{th} order low pass Butterworth filter (see Section~\ref{sec:filter}) to get rid of unwanted high frequency noise. This Section talks about the results that we got to solve those problems by increasing the sampling frequency.

\subsubsection{Effect on Filtering to Use the Fastest Sampling Frequency}

\begin{itemize}
\item Fix Fig. 5.1 (scales) -- ideally plot all of them on the same graph.
\item Noise reduced by factor of {\bf X}
\item This is resolution index 1
\item Without filter noise is dominated by 60 Hz and higher.
\item Compensation is off
\item Also compared with SCU... it agreed well.  Our filters gave slightly better performance (lower noise) likely due to slight difference in design.
\item Loop frequency is 100~Hz.  This is larger than 25~kHz$/12\approx 2$~kHz reported in Chapter 3 (ref) because of polling time.  Typ.~1~ms/channel read.  When read in this way, can increase resolution index to 7 (and averaging time) before significant delays are noted.  (when ADC effective sample time approaches 1~ms)
\item At times we did this alternate way:  increase resolution index.
\item Another problem:  current drifting during compensation although field would not change.  This is discussed further in Sections~\ref{drifting1} and \ref{drifting2}.
\item We thought maybe this was due to sequential reads (higher resol index $\sim$ 7) so in general we aimed for resol index 1 which makes the reads distributed in time.  (Explain better... it is leading up to Fig. 5.2 and 5.3.)  In such cases when using lower resol index (1) we did software averaging until the design loop rate was met.  This will be discussed further in the next sections.
\item Another point:  in Fig. 5.1 is that there is a time lag (slewing?) for the filter... consistent with expectation.
\item Conclusion:  the filter works and allows us to go to higher effective sample rate.
\item We would be stuck at resolution index 11 or 12 otherwise.  12 has 6 Hz effective sample rate $\approx$ 10 PLC averaging.  If doing this for 12 channels it it means $<0.5$~Hz.  Impossible to do compensation at 6~Hz this way (fails to meet design goal).
\end{itemize}

As we have decided to use the fastest sampling frequency that our ADC
can offer as discussed above , we have faced problems in terms of
noise there. So, we have build a 4\textsuperscript{th} order low pass
Butterworth filter (see Section~\ref{sec:filter}) to get rid of
unwanted high frequency noise.

\fig{Images/filtering}{width = \textwidth}{Filtering effect on the data taken for magnetic field compensation. Vertical axis represent $\Delta$B (see Eq.~(\ref{eq:del_B})) due to '1y' . The left one indicates data measurement using a filter and right one without a filter. They are described more in the text. The 'ON' and 'OFF' vertical dashed lines indicate the time of the perturbation coil being turned 'ON' and 'OFF' respectively. \label{fig:filtering}}{Short}

The Fig.~\ref{fig:filtering} shows the importance of using the filter discussed above. The data was taken by measuring the drift in the signal $\Delta$B (see Eq.~(\ref{eq:del_B})) by '1y' sensor position two times where one with filter and another one without filter. On the part of data measurement, current has been applied in the perturbation coil at $\sim$ 0.18s which is indicated by the vertical red dashed line and is termed as ON. At $\sim$0.58s current supply in the perturbation coil has cut off which is indicated by the green vertical dash line and termed as OFF. The total loop sampling frequency is found to be 100 Hz. Now for comparison the data taken with filer has been shown in Fig.~\ref{fig:filtering}\textcolor{blue}{(a)} and that without filter in in Fig.~\ref{fig:filtering}\textcolor{blue}{(b)}. It is seen that the signal is very less attenuated while using filter and very noisy without any filter.

So, for faster sampling it needs to have filter to avoid high frequency noise. Next, after applying filter let's see how the sampling frequency has impact on response time of the magnetic field compensation.

%\FloatBarrier
\subsubsection{Effect on Response Time of Magnetic Field Compensation}

\begin{itemize}
\item Fig. 5.2.  Left: with filter, resol index 1, 50 software averages per point, compensation ON, loop cycle rate (correction rate) 6.57 Hz indicated in the Fig.  Right:  no filter, resolution index 12, no other averaging (other than what the ADC oes), graph is on longer timescale because loop rate is slower (0.45~Hz).  Corrected/uncorrected (projected) values are shown.
\item Main point:  loop cycle rate is good!  $\sim$ 15x faster.  Even plenty of time for additional averaging to reduce noise further.  Likely the same result if somewhat higher resolution index used, possibly even with reduced software averaging.
\end{itemize}




So after we have our filter ready to use, now let's see what I meant by increasing the sampling frequency and its effect on the magnetic field  compensation response time.

\fig{Images/samp_freq_iteration}{width = \textwidth}{Active magnetic field compensation by sensor position '1y'. Vertical axis represent $\Delta$B (see Eq.~(\ref{eq:del_B})) due to '1y' with red represents uncompensated $\Delta$B. The left one indicates the compensation by '1y' at the fastest sampling frequency and right one for slowest one. They are described more in the text. The 'ON' and 'OFF' vertical dashed lines indicate the time of the perturbation coil being turned 'ON' and 'OFF' respectively. \label{fig:samp_freq_iteration}}{Short}

The Fig.~\ref{fig:samp_freq_iteration} shows the comparison between using highest sampling frequency (left) and the slowest one (right). The results are part of the compensation by all the 12 sensors placed just like explained by the horizontal axis in Fig.~\ref{fig:m} using regularized pseudinverse and PI tuning in the feedback algorithm explained in Chapter~\ref{ch:operation}. The figure only shows compensation for position '1y for simplicity. Note that the left and right are two different compensation measurements wih highest and lowest sampling frequency of our ADC (see Section~\ref{sec:DAQ}) having same number of feedback loop in each case. It is seen that for same number of measurements in the feedback algorithm it takes $\sim$45 s for the fastest one (left) compared to $\sim$670 s for the slowest one (right). The loop sampling  frequency for the fastest one (left) is 6.57 Hz compared to 0.45 Hz (left) for the slowest one (right). 

So the fastest sampling frequency of our ADC can offer 670/45$\approx$15 times faster response time than the slowest one. Next let's reveal the effect of sampling frequency on coil current response time.

%\FloatBarrier
\subsubsection{Effect on Coil Current Response Time }

\begin{itemize}
\item Fig. 5.3.  Always filtered.  Different resolution indices indicated on the figure along with their loop cycle rates.  Always 50 software averages done.
\item By resolution index 8 or 9, the loop cycle rate for the 50 software averages slows to 1.1 and 0.39 Hz respectively.  The results look similar to resolution index 12 (without filter) but much less noisy, smoother.
\item 
\end{itemize}


Form above discussion it is found that the more the sampling frequency is the less it takes for the compensation to respond. The effect of the sampling frequency has been revisited here again to show its effect in coil current response time.

\FloatBarrier
\fig{Images/samp_freq_example}{width = \textwidth}{$C_x^-$ coil current (left) and active magnetic field compensation (right) by sensor position '8x' for different resolution index. The resolution indices are (see Table~\ref{fig: res} and Sec.~\ref{sec:DAQ}) represented by 1, 6, 7, 8 and 9 while loop sampling frequencies are given in the parentheses respectively. They are describe more in the text. Vertical axis of the left one represents $C_x^-$ coil current graph and the right one indicates   $\Delta$B (see Eq.~(\ref{eq:del_B})) due to '8x' with red represents uncompensated $\Delta$B for different resolution index respectively. The 'ON' and 'OFF' vertical dashed lines indicate the time of the perturbation coil being turned 'ON' and 'OFF' respectively. \label{fig:samp_freq_example_multi}}{Short}

 The effect in coil current response time has been shown by Fig.~\ref{fig:samp_freq_example_multi}.  The results are again part of the compensation by all the 12 sensors placed just like explained earlier. Only difference here is that, now the different measurements have been taken for a same time frame i.e. 180s in this case compared to earlier where time was independent for different ones. For simplicity only $C_x^-$ coil current (left) and magnetic compensation (right) on fluxgate sensor position '8x' for different loop sampling frequencies have been shown instead of showing all current coils and sensor positions. For a complete list of positions see the color map of $\bm{M}$ in Fig.~\ref{fig:m}. As seen from the $C_x^-$ coil current (left) and $\Delta$B (right) due to '8x' graph that with decrease of loop sampling frequency, the current response time increases. For example, it takes $\sim$ 7s for $C_x^-$ coil current to settle down after the perturbation coil being turned ON when the loop sampling frequency is 6.35 Hz indicated by blue color curve compared to $\sim$ 52s when the loop sampling frequency is 0.39 Hz indicated by magenta color curve. So, the current response time has been faster by 52/7$\approx$7.5 times and we are not even considering the slowest one.

So for better response time in the coil current and eventually in magnetic field compensation system, there should an ADC with higher sampling frequency and a filtering method to get rid of higher frequencies. In the upcoming Section PI, regularization parameter and condition number will be revisited with results.

\section{Regularization Parameter and PI Tuning Revisited}
This Section starts with the reconstruction of the Fig.~\ref{fig:Isim}, Fig.~\ref{fig:fluc-sim} and Fig.~\ref{fig:I-fluc} in the experimental environment to justify the Monte Carlo simulation to find regularization parameter 'r' discussed on Section~\ref{sec:mont}. In Section~\ref{sec:tune}, tuning method of proportional gain (P) or $k_c^p$ term  and integral reset (I) or $k_c^i$ term (
see Eq.~(\ref{eq:I}) ) which is in short PI tuning has been discussed. After justifying 'r', in this Section, the discussion about the effect of changing P and I term individually will be made and then ended with discussing the effect of applying them combinely. The results of each of the effects will be presented and discussed.


\subsection{Justification of Monte Carlo Simulation to Find r}

In Section~\ref{sec:inv}, we have already talked about how the regularization parameter 'r' came into effect. Then, in Section~\ref{sec:mont}, a Monte Carlo simulation method has been described which has been taken from Ref.~\cite{bea} where a suitable value of 'r' has been found which will give the best compromise between the magnetic field fluctuation and coil current fluctuation. In this Section, we have run the model in experimental setup in order to verify the simulation method.

In Section~\ref{sec:mont}, 30 different sets of random magnetic field ($B_s^{\text{rand}}$) values have been chosen with center of distribution being 0 and standard deviation 5 nT for 12 different fluxgate sensor positions whose labels are given in the horizontal axis of Fig.~\ref{fig:m} and the positions are specified in Fig.~\ref{fig: coil}. Here, we have chosen one out of the 30 different sets of $B_s^{\text{rand}}$ and go through the steps as in Section~\ref{sec:mont} to generate similar figures like in Fig.~\ref{fig:Isim}, Fig.~\ref{fig:fluc-sim} and Fig.~\ref{fig:I-fluc}. For the experimental setup, we have done the similar steps to generate those figures except instead of generating $B_s^{\text{rand}}$ randomly, we have applied current in the perturbation coil to generate the required drift ( see Eq.~(\ref{eq:del_B}) ) which will be treated as $B_s^{\text{rand}}$. The results of the simulation as well as experiment are given in Fig.~\ref{fig:mont_comp}, where Fig.~\ref{fig:mont_comp}\textcolor{blue}{(a)}, Fig.~\ref{fig:mont_comp}\textcolor{blue}{(b)} and Fig.~\ref{fig:mont_comp}\textcolor{blue}{(c)} are the results found from simulation and Fig.~\ref{fig:mont_comp}\textcolor{blue}{(d)}, Fig.~\ref{fig:mont_comp}\textcolor{blue}{(e)} and Fig.~\ref{fig:mont_comp}\textcolor{blue}{(f)} are the results found from experiment. They are described in detail in Fig.~\ref{fig:Isim}, Fig.~\ref{fig:fluc-sim} and Fig.~\ref{fig:I-fluc}. It is seen that the experimental counterpart of each of the simulation that is Fig.~\ref{fig:mont_comp}\textcolor{blue}{(a)} of simulation is comparable to Fig.~\ref{fig:mont_comp}\textcolor{blue}{(d)} of experiment. Similarly, Fig.~\ref{fig:mont_comp}\textcolor{blue}{(b)} of simulation with Fig.~\ref{fig:mont_comp}\textcolor{blue}{(e)} of experiment and Fig.~\ref{fig:mont_comp}\textcolor{blue}{(c)} of simulation with Fig.~\ref{fig:mont_comp}\textcolor{blue}{(f)} of experiment are comparable and they produce similar results. The similar results of the simulation with experiment justifies the the simulation model described in Section~\ref{sec:mont}.

\fig{Images/mont_comp2}{width = \textwidth,height =10cm}{Simulation result for one set of $B_s^{\text{rend}}$ for the Fig.~\ref{fig:Isim}, Fig.~\ref{fig:fluc-sim} and Fig.~\ref{fig:I-fluc} are shown in (a), (b) and (c) and corresponding experimental results are in (d), (e) and (f). For the description of the figures see the text in Fig.~\ref{fig:Isim}, Fig.~\ref{fig:fluc-sim} and Fig.~\ref{fig:I-fluc}. For position of fluxgate sensors see Fig.~\ref{fig: coil}.\label{fig:mont_comp}}{Short}

\FloatBarrier
The above results in Fig.~\ref{fig:mont_comp}, where simulation results are similar to experimental one justify the simulation method described in Section~\ref{sec:mont}. The upcoming Section is all about P and I term (see Eq.~(\ref{eq:I}) ) behaviour. 

\subsection{PI Tuning General Behavior}\label{sec:pi_behave}
Here, the individual effect of P and I term will be discussed first and then concludes with the effect of using PI together.

\subsubsection{Effect of changing only P term}
Here, the effect of changing proportional gain term (P) or $k_c^p$ of Eq.~(\ref{eq:I}) will be discussed.

P term is proportionally multiplying the error (the difference between setpoint and actual measurement) with a constant gain. For the prototype it is

\begin{equation}
    P_{\text{PI}}=k_c^p \Delta I_c^n
\end{equation}
where, $k_c^p$ is the proportional gain and $\Delta I_c^n$ is explained in Eq.~(\ref{eq:del_I}).

Depending on the value $k_c^p$, it tries to minimize the error level between the setpoint and the actual measurement with passage of several measurements. A large value of $k_c^p$ will result large output change for a particular error and eventually it reaches a threshold point above which the system becomes unstable. 

% \begin{figure}[!htb]
%     \begin{subfigure}{.5\linewidth}
%         \centering
%         \includegraphics[width=\linewidth, height= 6.5 cm]{Images/p25}
%         \caption{at $k_c^p$=0.25}
%         \label{fig:p25}
%     \end{subfigure}%
%     \begin{subfigure}{.5\linewidth}
%         \centering
%         \includegraphics[width=\linewidth, height= 6.5 cm]{Images/p50}
%         \caption{at $k_c^p$=0.50}
%         \label{fig:p50}
%     \end{subfigure}\\[1ex]
%     \begin{subfigure}{.5\linewidth}
%         \centering
%         \includegraphics[width=\linewidth, height= 6.5 cm]{Images/p75}
%         \caption{at $k_c^p$=0.75}
%         \label{fig:p75}
%     \end{subfigure}%
%         \begin{subfigure}{.5\linewidth}
%         \centering
%         \includegraphics[width=\linewidth, height= 6.5 cm]{Images/p100}
%         \caption{at $k_c^p$=1.0}
%         \label{fig:p100}
%     \end{subfigure}

%     \caption{Currents (left vertical axis) in all six coil sides ($C_x^\pm$, $C_y^\pm$ and $C_z^\pm$) with drift $\Delta$B (right vertical axis) at sensor position '1x' for different values of $k_c^p$ with $k_c^i$ in Eq.~(\ref{eq:I}) being zero. Blue color curve denotes the actual drift in signal at position '1x' found by Eq.~(\ref{eq:del_B}) while the red curve denotes the drift that would have been without the compensation. The 'ON' and 'OFF' vertical dashed lines indicate the time of the perturbation coil being turned 'ON' and 'OFF' respectively. For position of coils and sensors see Fig.~\ref{fig: coil}. }
%     \label{fig:p_pi}
% \end{figure}
\begin{figure}[!htb]
    \begin{subfigure}{.5\linewidth}
        \centering
        \includegraphics[width=\linewidth, height= 6.5 cm]{Images/p25_33}
        \caption{at $k_c^p$=0.25}
        \label{fig:p25}
    \end{subfigure}%
    \begin{subfigure}{.5\linewidth}
        \centering
        \includegraphics[width=\linewidth, height= 6.5 cm]{Images/p50_33}
        \caption{at $k_c^p$=0.50}
        \label{fig:p50}
    \end{subfigure}\\[1ex]
    \begin{subfigure}{.5\linewidth}
        \centering
        \includegraphics[width=\linewidth, height= 6.5 cm]{Images/p75_33}
        \caption{at $k_c^p$=0.75}
        \label{fig:p75}
    \end{subfigure}%
        \begin{subfigure}{.5\linewidth}
        \centering
        \includegraphics[width=\linewidth, height= 6.5 cm]{Images/p100_33}
        \caption{at $k_c^p$=1.0}
        \label{fig:p100}
    \end{subfigure}

    \caption[short]{Currents (left vertical axis) in all six coil sides ($C_x^\pm$, $C_y^\pm$ and $C_z^\pm$) with drift $\Delta$B (right vertical axis) at sensor position '1x' for different values of $k_c^p$ with $k_c^i$ in Eq.~(\ref{eq:I}) being zero. Blue color curve denotes the actual drift in signal at position '1x' found by Eq.~(\ref{eq:del_B}) while the red curve denotes the drift that would have been without the compensation. The 'ON' and 'OFF' vertical dashed lines indicate the time of the perturbation coil being turned 'ON' and 'OFF' respectively. For position of coils and sensors see Fig.~\ref{fig: coil}. }
    \label{fig:p_pi}
\end{figure}

The effect of changing $k_c^p$ has been shown in Fig.~\ref{fig:p_pi} where the currents (left) that are being sent to the coils ($C_x^\pm$, $C_y^\pm$ and $C_z^\pm$) for drift $\Delta$B found by Eq.~(\ref{eq:del_B}) in sensor position '1x'.  It is seen that $\Delta$B=17.5 nT, 15.5 nT and 13.5 nT for $k_c^p$ = 0.25, 0.5 and 0.75 respectively (see Fig.~\ref{fig:p_pi}\textcolor{blue}{(a)}, Fig.~\ref{fig:p_pi}\textcolor{blue}{(b)}, Fig.~\ref{fig:p_pi}\textcolor{blue}{(c)}). That is, with the increase of $k_c^p$, $\Delta$B magnetic field decreases. But, it has a limit after which with the increase of $k_c^p$, the systems becomes unstable and starts oscillating which can be seen from Fig.~\ref{fig:p_pi}\textcolor{blue}{(d)}) where the currents (left) are oscillating and the drift itself also at $\Delta$B=12.5 nT (right). So, the error is reduced maximum by (20.5-12.5)/20.5 * 100$\%\approx$37$\%$ from the initial drift of $\Delta$B=20.5 nT denoted by the red curve at position '1x'. 

\FloatBarrier
The above results confirm that the difference between the setpoint and the actual measurements of the magnetic field can be reduced upto a certain point. So, only having the P term is no the solution for the prototype. Next, we will discuss about the effect of only I term.

\subsubsection{Effect of changing only I term}
Here, the effect of changing integral reset term (I) or $k_c^i$ of Eq.~(\ref{eq:I}) will be discussed.

The error (the difference between setpoint and actual measurement) is accumulated for the length of measurements and I term is multiplying that accumulated error  with a constant gain. For the prototype it is

\begin{equation}
    I_{\text{PI}}=k_c^i \sum_n \Delta I_c^n
\end{equation}
where, $k_c^i$ is the integral gain and $\Delta I_c^n$ is explained in Eq.~(\ref{eq:del_I}).

Accumulated error keep tracks of the offsets that should be corrected previously. I term takes care of the offset which are not corrected by the P term and thus accelerates reducing the error level. Depending on the value $k_c^i$, how fast the feedback loop will response to the drift in the signal will be determined. A large value of $k_c^p$ will result large faster response to reducing the error level and eventually it reaches a threshold point above which the actual measurement will overshoot i.e. exceed the setpoint. 
% The main downfall of this is that the time required for the coil current to be settle in after reducing the error level may be very slow or never ever settle in.



% As like the effect on P, the effect of changing I has been shown in Fig.~\ref{fig:i_pi} where the change in current in all six coil sides with  $\Delta$B on a particular sensor position have been observed for $k_c^i$=0.25, 0.5, 0.75 and 1.0 . It is seen that with increase of I the level of compensation of the magnetic field is almost similar but the main difference occurs on how fast the system response in an expense of increasing current in all the coil sides (see Fig.~\ref{fig:i_pi}\textcolor{blue}{(a)}, Fig.~\ref{fig:i_pi}\textcolor{blue}{(b)}, Fig.~\ref{fig:i_pi}\textcolor{blue}{(c)} and Fig.~\ref{fig:i_pi}\textcolor{blue}{(d)}). The main problem with changing only I term is that it creates a very slow current response time. But, in terms of compensation only changing I gives very good result. The slow current response can be minimized by decreasing the value of optimized 'r' (see Section~\ref{sec:r_pi} and Section~\ref{sec:r_currentResponse} ).

% \begin{figure}[!htb]
%     \begin{subfigure}{.5\linewidth}
%         \centering
%         \includegraphics[width=\linewidth, height= 6.5 cm]{Images/i25}
%         \caption{at $k_c^i$=0.25}
%         \label{fig:i25}
%     \end{subfigure}%
%     \begin{subfigure}{.5\linewidth}
%         \centering
%         \includegraphics[width=\linewidth, height= 6.5 cm]{Images/i50}
%         \caption{at $k_c^i$=0.5}
%         \label{fig:i50}
%     \end{subfigure}\\[1ex]
%     \begin{subfigure}{.5\linewidth}
%         \centering
%         \includegraphics[width=\linewidth, height= 6.5 cm]{Images/i75}
%         \caption{at $k_c^i$=0.75}
%         \label{fig:i75}
%     \end{subfigure}%
%         \begin{subfigure}{.5\linewidth}
%         \centering
%         \includegraphics[width=\linewidth, height= 6.5 cm]{Images/i100}
%         \caption{at $k_c^i$=1.0}
%         \label{fig:i100}
%     \end{subfigure}

%     \caption{Currents (left vertical axis) in all six coil sides ($C_x^\pm$, $C_y^\pm$ and $C_z^\pm$) with drift $\Delta$B (right vertical axis) at sensor position '1x' for different values of $k_c^i$ with $k_c^p$ ( see Eq.~(\ref{eq:I}) ) being zero. Blue color curve denotes the actual drift in signal at position '1x' found by Eq.~(\ref{eq:del_B}) while the red curve denotes the drift that would have been without the compensation. The 'ON' and 'OFF' vertical dashed lines indicate the time of the perturbation coil being turned 'ON' and 'OFF' respectively. For position of coils and sensors see Fig.~\ref{fig: coil}.}
%     \label{fig:i_pi}
% \end{figure}
\begin{figure}[!htb]
    \begin{subfigure}{.5\linewidth}
        \centering
        \includegraphics[width=\linewidth, height= 6.5 cm]{Images/i25_33}
        \caption{at $k_c^i$=0.25}
        \label{fig:i25}
    \end{subfigure}%
    \begin{subfigure}{.5\linewidth}
        \centering
        \includegraphics[width=\linewidth, height= 6.5 cm]{Images/i75_33}
        \caption{at $k_c^i$=0.75}
        \label{fig:i50}
    \end{subfigure}\\[1ex]
    \begin{subfigure}{.5\linewidth}
        \centering
        \includegraphics[width=\linewidth, height= 6.5 cm]{Images/i100_33}
        \caption{at $k_c^i$=1.0}
        \label{fig:i75}
    \end{subfigure}%
        \begin{subfigure}{.5\linewidth}
        \centering
        \includegraphics[width=\linewidth, height= 6.5 cm]{Images/i125_33}
        \caption{at $k_c^i$=1.25}
        \label{fig:i100}
    \end{subfigure}

    \caption[short]{Currents (left vertical axis) in all six coil sides ($C_x^\pm$, $C_y^\pm$ and $C_z^\pm$) with drift $\Delta$B (right vertical axis) at sensor position '1x' for different values of $k_c^i$ with $k_c^p$ ( see Eq.~(\ref{eq:I}) ) being zero. Blue color curve denotes the actual drift in signal at position '1x' found by Eq.~(\ref{eq:del_B}) while the red curve denotes the drift that would have been without the compensation. The 'ON' and 'OFF' vertical dashed lines indicate the time of the perturbation coil being turned 'ON' and 'OFF' respectively. For position of coils and sensors see Fig.~\ref{fig: coil}.}
    \label{fig:i_pi}
\end{figure}

\FloatBarrier
The effect of changing $k_c^i$ has been shown in Fig.~\ref{fig:i_pi} where the currents (left) that are being sent to the coils ($C_x^\pm$, $C_y^\pm$ and $C_z^\pm$) for drift $\Delta$B found by Eq.~(\ref{eq:del_B}) in sensor position '1x'.  It is seen from Fig.~\ref{fig:i_pi}\textcolor{blue}{(a)}, Fig.~\ref{fig:i_pi}\textcolor{blue}{(b)}, Fig.~\ref{fig:i_pi}\textcolor{blue}{(c)} and Fig.~\ref{fig:i_pi}\textcolor{blue}{(d)} which are correspond to $k_c^i$ = 0.25, 0.75, 1.0 and 1.25 respectively that the error level (right) seems to be $\sim$3.5 nT in every case and the coil currents(left) settle  faster for increasing value of $k_c^i$. The figures are neither helpful to understand the system response time nor the overshoot effect in the $\Delta$B graph (right). So for understating those effects, the $\Delta$B graphs (right) have been zoomed in and shown in Fig.~\ref{fig:i_pi_zoom}. Now it is easily seen that for the same duration of perturbation (here 'ON' at 10s and 'OFF' at 50s and so total 40s), the system still correcting the error level for $k_c^i$=0.25 in Fig.~\ref{fig:i_pi_zoom}\textcolor{blue}{(a)}. The error level is $\sim$3.23 nT  at 35-10=25s (the perturbation is turned 'ON' at 10s) for $k_c^i$=0.75 as seen in Fig.~\ref{fig:i_pi_zoom}\textcolor{blue}{(b)} and 25-10=15s for $k_c^i$=1.0  as seen in Fig.~\ref{fig:i_pi_zoom}\textcolor{blue}{(c)} and it will be less for increasing value of $k_c^i$. It is also seen from the Fig.~\ref{fig:i_pi_zoom}\textcolor{blue}{(d)} that there is an overshoot in the error level before it settles in.

\begin{figure}[!htb]
    \begin{subfigure}{.5\linewidth}
        \centering
        \includegraphics[width=\linewidth, height= 6.5 cm]{Images/i25_33_zoom.png}
        \caption{at $k_c^i$=0.25}
        \label{fig:i25zoom}
    \end{subfigure}%
    \begin{subfigure}{.5\linewidth}
        \centering
        \includegraphics[width=\linewidth, height= 6.5 cm]{Images/i75_33_zoom.png}
        \caption{at $k_c^i$=0.75}
        \label{fig:i75zoom}
    \end{subfigure}\\[1ex]
    \begin{subfigure}{.5\linewidth}
        \centering
        \includegraphics[width=\linewidth, height= 6.5 cm]{Images/i100_33_zoom.png}
        \caption{at $k_c^i$=1.0}
        \label{fig:i100zoom}
    \end{subfigure}%
        \begin{subfigure}{.5\linewidth}
        \centering
        \includegraphics[width=\linewidth, height= 6.5 cm]{Images/i125_33_zoom.png}
        \caption{at $k_c^i$=1.25}
        \label{fig:i125zoom}
    \end{subfigure}

    \caption[short]{Zoomed in version of the drift $\Delta$B shown in right side of Fig.~\ref{fig:i_pi}\textcolor{blue}{(a)}, Fig.~\ref{fig:i_pi}\textcolor{blue}{(b)}, Fig.~\ref{fig:i_pi}\textcolor{blue}{(c)} and Fig.~\ref{fig:i_pi}\textcolor{blue}{(d)} respectively at sensor position '1x' for different values of $k_c^i$ with $k_c^p$ ( see Eq.~(\ref{eq:I}) ) being zero. The red vertical dashed line indicates the time of the perturbation coil being turned 'ON'. For position of coils and sensors see Fig.~\ref{fig: coil}.\label{fig:i_pi_zoom}}
\end{figure}

% \fig{Images/i_pi_zoom}{width = \textwidth,height =10cm}{Zoomed in version of the drift $\Delta$B shown in right side of Fig.~\ref{fig:i_pi}\textcolor{blue}{(a)}, Fig.~\ref{fig:i_pi}\textcolor{blue}{(b)}, Fig.~\ref{fig:i_pi}\textcolor{blue}{(c)} and Fig.~\ref{fig:i_pi}\textcolor{blue}{(d)} respectively at sensor position '1x' for different values of $k_c^i$ with $k_c^p$ ( see Eq.~(\ref{eq:I}) ) being zero. The red vertical dashed line indicates the time of the perturbation coil being turned 'ON'. For position of coils and sensors see Fig.~\ref{fig: coil}.\label{fig:i_pi_zoom}}


\FloatBarrier
The above results confirm that to get rid of the offsets that could not be reduce by the P term, an I term is a must. Next the effect of applying both of them after tuning (see Section~\ref{sec:tune}) will be discussed.
% \subsection{r vs. Condition No.}\label{sec:cond}
% Instead of going through all the steps that are discussed in Section~\ref{sec:inv}, the concept of condition number of a matrix can be used. The condition number of $\bm{M}$ can be determined from the diagonal matrix $\bm{\Sigma}$ as given in eq.\ref{eq:m} by -
%  \begin{equation}
%      cond(\bm{M})=\frac{max(\sigma_d)}{min(\sigma_d)}
%  \end{equation}
 

\subsubsection{Effect of changing PI term Combinely}
Finally, the Section~\ref{sec:pi_behave} will be ended here with the discussion of the effect of changing P and I term at a time which will complete the Eq.~(\ref{eq:I}).

Here, first the P and I term have been tuned following the discussion on Section~\ref{sec:tune} which has generated $k_c^p$=0.43 and $k_c^i$=0.52 . The results by applying $k_c^p$ and $k_c^i$ as those tuned values are shown in Fig.~\ref{fig:tuned_vs_i}\textcolor{blue}{(a)}. For simplicity instead of showing all the drift $\Delta$B for all the fluxgate sensors for the positions given in the horizonatal axis of Fig.~\ref{fig:m}, only '1x' is shown on the right of the figure. And same as earlier the currents  that are being sent to the coils ($C_x^\pm$, $C_y^\pm$ and $C_z^\pm$) shown on the left of the same figure. But, we couldn't determine the effect of having both of them at a time. So, keeping $k_c^i$ as 0.52 and excluding P term i.e. $k_c^p$=0.0 we run the same measurement again and the results are shwon in Fig.~\ref{fig:tuned_vs_i}\textcolor{blue}{(b)}. As as matter of surprise, there is hardly any difference between the results in Fig.~\ref{fig:tuned_vs_i}\textcolor{blue}{(a)} and Fig.~\ref{fig:tuned_vs_i}\textcolor{blue}{(b)}. Why is that so ? For the moment, the Fig.~\ref{fig:tuned_vs_i} suggests that may be we don't need P term at all or maybe we need different tuning methods. So, applying P and I term at a time raises question of the necessity of the P term or importance of the tuning method describe in Section~\ref{sec:tune}. Due to lack of time, we did not further go into other tuning methods. Rather we have tried to discover the differences in the work between Ref.~\cite{bea} and Ref.~\cite{rawlik}.
\doublefig{Images/p43i52_33}{width =\textwidth,height =8cm}{at $k_c^p$=0.43 and $k_c^i$=0.52. \label{fig:pi_tuned}}{Images/i52_33}{width = \textwidth,height =8cm}{at $k_c^p$=0.0 and $k_c^i$=0.52..\label{fig:i52}}{{Currents (left vertical axis) in all six coil sides ($C_x^\pm$, $C_y^\pm$ and $C_z^\pm$) with drift $\Delta$B (right vertical axis) at sensor position '1x' for combine different values of $k_c^i$ and $k_c^p$ ( see Eq.~(\ref{eq:I}) ). Blue color curve denotes the actual drift in signal at position '1x' found by Eq.~(\ref{eq:del_B}), while the red curve denotes the drift that would have been without the compensation. The 'ON' and 'OFF' vertical dashed lines indicate the time of the perturbation coil being turned 'ON' and 'OFF' respectively. For position of coils and fluxgate sensor see Fig.~\ref{fig: coil}.} \label{fig:tuned_vs_i}}{short}

\FloatBarrier
The above results give us confusion on the effectiveness of the P term and also the tuning method. Instead of looking more deep into tuning method, we moved our focused on the PI control equation explained by Ref.~\cite{bea} and Ref.~\cite{rawlik}.

\section{Style of PI (Old/New)}
We have talked about the tuning method in Section~\ref{sec:tune} and later in Section~\ref{sec:pi_behave} we have shown the the effect of the P and I term where the effectiveness of the P term and the tuning method arises. Those  were all based on the works from Ref.\cite{bea}. In early 2018, another author on his PhD thesis \cite{rawlik} claimed that there is no need of PI!! As we have done most of our work following the Ref.\cite{bea}, so we wanted to verify that the claim of no PI by Ref.\cite{rawlik} is actually true or not. So, this Section is all about comparing the work from two Ref.\cite{bea} and Ref.\cite{rawlik} and verify the claim. 

Here, the PI control algorithm as discussed by the Eq.~(\ref{eq:I}) which taken from Ref.~\cite{bea}) is termed as the 'Old PI' control. and no PI claim by Ref.~\cite{rawlik} is termed as 'New PI' control and the new current can be found by -
\begin{equation}\label{eq:I_raw}
    I^{n+1}= I^n+M^{-1} (B_{setpoint}-B_{measure}^n)=I^n+M^{-1} \Delta B^n
\end{equation}
or with a delay like -
\begin{equation}\label{eq:I_raw_delay}
    I^{n+1}= I^{n-2}+M^{-1} (B_{setpoint}-B_{measure}^n)=I^{n-2}+M^{-1} \Delta B^n
\end{equation}
But is it really possible ? 

To verify that we run the prototype two times. First time we set the value of $k_c^p$=0.0 and $k_c^i$=1.0 to find the new current by Eq.~(\ref{eq:I}) from 'Old PI' and another with $k_c^p$=1.0 and $k_c^i$=0.0 to find the new current determined by Eq.~(\ref{eq:I_raw}) of 'New PI'. The results from 'Old PI' are given in Fig.~\ref{fig:style_of_pi}\textcolor{blue}{(a)} and that of 'New PI' in Fig.~\ref{fig:style_of_pi}\textcolor{blue}{(b)}. For simplicity instead of showing all the drift $\Delta$B for all the fluxgate sensors for the positions given in the horizonatal axis of Fig.~\ref{fig:m}, only '1x' is shown on the right of the figures. And same as earlier the currents  that are being sent to the coils ($C_x^\pm$, $C_y^\pm$ and $C_z^\pm$) shown on the left of the same figures. It is seen that the results in both of the figures are same. That is P and I term of 'Old PI' has been transformed into I and P term of 'New PI' current equation.



\doublefig{Images/i100_old}{width =\textwidth, height= 6.5 cm}{at $k_c^p$=0.0 and $k_c^i$=1.0 for 'Old PI' \label{fig:i100_old}}{Images/p100_new}{width = \textwidth, height= 6.5 cm}{at $k_c^p$=1.0 and $k_c^i$=0.0 for 'New PI'\label{fig:p100_new}}{{Currents (left vertical axis) in all six coil sides ($C_x^\pm$, $C_y^\pm$ and $C_z^\pm$) with drift $\Delta$B (right vertical axis) at sensor position '1x' for combine different values of $k_c^i$ and $k_c^p$ by applying Eq.~(\ref{eq:I}) at (a) and Eq.~(\ref{eq:I_raw}) at (b) . Blue color curve denotes the actual drift in signal at position '1x' found by Eq.~(\ref{eq:del_B}), while the red curve denotes the drift that would have been without the compensation. The 'ON' and 'OFF' vertical dashed lines indicate the time of the perturbation coil being turned 'ON' and 'OFF' respectively. For position of coils and fluxgate sensor see Fig.~\ref{fig: coil}..} \label{fig:style_of_pi}}{short}


\FloatBarrier
But again is it really true about what we are claiming ?

To verify that the equation has been studied deeply. The mathematical equivalent of Eq.~(\ref{eq:I_raw}) is- 
\begin{equation}\label{eq:I_raw_eq}
    I^{n+1}= \sum_{i=0}^n M^{-1} (B_{setpoint}-B_{ambient}^i)
\end{equation}

The equivalency can be proved by mathematical induction. That is, for n$\geq$0, let $P_n$ be the following statement -
\begin{equation}\label{eq:I_raw_eq_both}
   I^n+M^{-1} (B_{setpoint}-B_{measure}^n)= \sum_{i=0}^n M^{-1} (B_{setpoint}-B_{ambient}^i)
\end{equation}

For n=0, the statement $P_0$ is-

\begin{align*}
    \begin{split}
      LHS &=I^0+M^{-1} (B_{setpoint}-B_{measure}^0) \\
        &=I^0+M^{-1} (B_{setpoint}-B_{ambient}^0 -M I^0) \\
        &=I^0+M^{-1} (B_{setpoint}-B_{ambient}^0) - MM^{-1} I^0) \\
        &=I^0+M^{-1} (B_{setpoint}-B_{ambient}^0) - I^0 \\
        &=M^{-1} (B_{setpoint}-B_{ambient}^0)
    \end{split}
    \\
    \begin{split}
      RHS &=M^{-1} (B_{setpoint}-B_{ambient}^0)\\
          &= LHS
    \end{split}
\end{align*}
$\therefore$ The Eq.~(\ref{eq:I_raw_eq_both}) is true for n=0.\newline
Suppose it is also true for k$\geq$0. That is, the following statement $P_k$ holds -
\begin{equation}\label{eq:I_raw_99}
   I^k+M^{-1} (B_{setpoint}-B_{measure}^k)= \sum_{i=0}^k M^{-1} (B_{setpoint}-B_{ambient}^i)
\end{equation}

It is to be showed that $P_{k+1}$ also holds, that is-

\begin{equation}\label{eq:I_raw_98}
   I^{k+1}+M^{-1} (B_{setpoint}-B_{measure}^{k+1})= \sum_{i=0}^{k+1} M^{-1} (B_{setpoint}-B_{ambient}^i)
\end{equation}
\begin{align*}
    \begin{split}
      LHS &=I^{k+1}+M^{-1} (B_{setpoint}-B_{measure}^{k+1}) \\
        &=I^{k+1}+M^{-1} (B_{setpoint}-B_{ambient}^{k+1} -M I^{k+1}) \\
        &=I^{k+1}+M^{-1} (B_{setpoint}-B_{ambient}^{k+1}) - MM^{-1} I^{k+1}) \\
        &=I^{k+1}+M^{-1} (B_{setpoint}-B_{ambient}^0) - I^{k+1} \\
        &=M^{-1} (B_{setpoint}-B_{ambient}^{k+1})
    \end{split}
    \\
    \begin{split}
      RHS &=M^{-1} (B_{setpoint}-B_{ambient}^{k+1})\\
          &= LHS
    \end{split}
\end{align*}
$\therefore \: P_{k+1}$ also holds which validates the Eq.~(\ref{eq:I_raw_eq_both}) by the principle of mathematical induction. \newline
Now, the Eq.~(\ref{eq:I_raw_eq}) is equivalent to the implementation of PI -
\begin{equation}\label{eq:I_raw_pi}
    I^{n+1}= k_p E^n+ k_i \sum_{i=0}^n E^i \Delta t
\end{equation}
with $k_p=0$ and $k_i$ set to a particular value. Here, the error is , $E^i=M^{-1} (B_{setpoint}-B_{ambient}^i)$. So, the Eq.~(\ref{eq:I_raw}) is a PI with a particular tuning and termed as 'New PI' control. Similarly, the delay showed in Eq.~(\ref{eq:I_raw_delay}) is another form of tuning.

So the above discussion verifies that 'Old PI' and 'New PI' are just two different form of tuning based on the control algorithm. We can also clear our previous questions about the effectiveness of the P term and the tuning method discussed in Section~\ref{sec:pi_behave} that instead of the tuning method discussed in Section~\ref{sec:tune}, we are using different from of tuning with changing the value of $k_c^i$ keeping $k_c^p$=0. In upcoming Section we then studied the effect of  matrix condition number in P and I term. and propose a new method to find regularization parameter using matrix condition number. 



\section{New Studies on Regularization Parameter}\label{sec:new_study_r}
In Section~\ref{sec:inv} we have introduced the regularization parameter 'r' and then discussed a simulation model in Section~\ref{sec:mont} which later has been justified by comparing with experimental setup in Section~\ref{fig:mont_comp}. We have also talked about the PI behaviour in Section~\ref{sec:pi_behave}. Here, we will first show the effect of matrix condition number on PI behaviour and then propose a new method to find 'r' based on condition number of matrix. 

So, first the importance of matrix condition number on PI will be discussed.

\subsection{Importance of Matrix Condition Number on PI}\label{sec:pi_behave_m}
Matrix Condition number ( see Eq.~(\ref{eq:cond} ) have been introduced in Section.~\ref{sec:m} while discussing the inversion of the matrix $\bm{M}$ . An ill-conditioned matrix has large condition number and a matrix with small condition number is said to be well-conditioned. Here, the effect of matrix condition number on PI will be discussed in terms of a very large condition number that is 129.
 
\subsubsection{Condition Number Effect on changing only P term}
Here, the effect of changing proportional gain term (P) or $k_c^p$ of Eq.~(\ref{eq:I}) will be discussed for a matrix condition number of 129 which is very ill-conditioned compared to 33 discussed in Section~\ref{sec:pi_behave}.

P term is proportionally multiplying the error (the difference between setpoint and actual measurement) with a constant gain and discussed more in Section~\ref{sec:pi_behave}.


\begin{figure}[!htb]
    \begin{subfigure}{.5\linewidth}
        \centering
        \includegraphics[width=\linewidth, height= 6.5 cm]{Images/p25}
        \caption{at $k_c^p$=0.25}
        \label{fig:p25m}
    \end{subfigure}%
    \begin{subfigure}{.5\linewidth}
        \centering
        \includegraphics[width=\linewidth, height= 6.5 cm]{Images/p50}
        \caption{at $k_c^p$=0.50}
        \label{fig:p50m}
    \end{subfigure}\\[1ex]
    \begin{subfigure}{.5\linewidth}
        \centering
        \includegraphics[width=\linewidth, height= 6.5 cm]{Images/p75}
        \caption{at $k_c^p$=0.75}
        \label{fig:p75m}
    \end{subfigure}%
        \begin{subfigure}{.5\linewidth}
        \centering
        \includegraphics[width=\linewidth, height= 6.5 cm]{Images/p100}
        \caption{at $k_c^p$=1.0}
        \label{fig:p100m}
    \end{subfigure}

    \caption[short]{Currents (left vertical axis) in all six coil sides ($C_x^\pm$, $C_y^\pm$ and $C_z^\pm$) with drift $\Delta$B (right vertical axis) at sensor position '1x' for different values of $k_c^p$ with $k_c^i$ in Eq.~(\ref{eq:I}) being zero. Blue color curve denotes the actual drift in signal at position '1x' found by Eq.~(\ref{eq:del_B}) while the red curve denotes the drift that would have been without the compensation. The 'ON' and 'OFF' vertical dashed lines indicate the time of the perturbation coil being turned 'ON' and 'OFF' respectively. For position of coils and sensors see Fig.~\ref{fig: coil}. }
    \label{fig:p_pi_m}
\end{figure}

The effect of changing $k_c^p$ has been shown in Fig.~\ref{fig:p_pi_m} where the currents (left) that are being sent to the coils ($C_x^\pm$, $C_y^\pm$ and $C_z^\pm$) for drift $\Delta$B found by Eq.~(\ref{eq:del_B}) in sensor position '1x'.  It is seen that $\Delta$B=23 nT, 18.5 nT and 16.5 nT for $k_c^p$ = 0.25, 0.5 and 0.75 respectively (see Fig.~\ref{fig:p_pi_m}\textcolor{blue}{(a)}, Fig.~\ref{fig:p_pi_m}\textcolor{blue}{(b)}, Fig.~\ref{fig:p_pi_m}\textcolor{blue}{(c)}). That is, with the increase of $k_c^p$, $\Delta$B magnetic field decreases. But, it has a limit after which with the increase of $k_c^p$, the systems becomes unstable and starts oscillating which can be seen from Fig.~\ref{fig:p_pi_m}\textcolor{blue}{(d)}) where the currents (left) are oscillating and the drift itself also at $\Delta$B=14.5 nT (right). So, the error is reduced maximum by (29.5-14.5)/29.5 * 100$\%\approx$51$\%$ from the initial drift of $\Delta$B=29.5 nT denoted by the red curve at position '1x'. 

\FloatBarrier
The above results confirm that matrix condition number has minimal effect on P term by comparing the effect shown in Section~\ref{sec:pi_behave}. Next, we will discuss about the effect on only I term.

\subsubsection{Condition Number Effect on Changing only I term}
Here, the effect matrix condition number effect on changing integral reset term (I) or $k_c^i$ of Eq.~(\ref{eq:I}) will be discussed.

The error (the difference between setpoint and actual measurement) is accumulated for the length of measurements and I term is multiplying that accumulated error  with a constant gain as discussed in Section~\ref{sec:pi_behave}.




As like the effect on P, the effect of changing $k_c^i$ has been shown in Fig.~\ref{fig:i_pi_m} where the currents (left) that are being sent to the coils ($C_x^\pm$, $C_y^\pm$ and $C_z^\pm$) for drift $\Delta$B found by Eq.~(\ref{eq:del_B}) in sensor position '1x' for $k_c^i$=0.25, 0.5, 0.75 and 1.0.  It is seen from Fig.~\ref{fig:i_pi_m}\textcolor{blue}{(a)}, Fig.~\ref{fig:i_pi_m}\textcolor{blue}{(b)}, Fig.~\ref{fig:i_pi_m}\textcolor{blue}{(c)} and Fig.~\ref{fig:i_pi_m}\textcolor{blue}{(d)}). which are correspond to $k_c^i$ = 0.25, 0.75, 1.0 and 1.25 respectively that the error level (right) is $\sim$3.5 nT in every case, the coil currents(left) never settle in any of them which is not he case for low matrix condition number as seen from Fig.~\ref{fig:i_pi}\textcolor{blue}{(a)}, Fig.~\ref{fig:i_pi}\textcolor{blue}{(b)}, Fig.~\ref{fig:i_pi}\textcolor{blue}{(c)} and Fig.~\ref{fig:i_pi}\textcolor{blue}{(d)}) where the condition number is 33 compared to 129 here.

\begin{figure}[!htb]
    \begin{subfigure}{.5\linewidth}
        \centering
        \includegraphics[width=\linewidth, height= 6.5 cm]{Images/i25}
        \caption{at $k_c^i$=0.25}
        \label{fig:i25m}
    \end{subfigure}%
    \begin{subfigure}{.5\linewidth}
        \centering
        \includegraphics[width=\linewidth, height= 6.5 cm]{Images/i50}
        \caption{at $k_c^i$=0.5}
        \label{fig:i50m}
    \end{subfigure}\\[1ex]
    \begin{subfigure}{.5\linewidth}
        \centering
        \includegraphics[width=\linewidth, height= 6.5 cm]{Images/i75}
        \caption{at $k_c^i$=0.75}
        \label{fig:i75m}
    \end{subfigure}%
        \begin{subfigure}{.5\linewidth}
        \centering
        \includegraphics[width=\linewidth, height= 6.5 cm]{Images/i100}
        \caption{at $k_c^i$=1.0}
        \label{fig:i100m}
    \end{subfigure}

    \caption[short]{Currents (left vertical axis) in all six coil sides ($C_x^\pm$, $C_y^\pm$ and $C_z^\pm$) with drift $\Delta$B (right vertical axis) at sensor position '1x' for different values of $k_c^i$ with $k_c^p$ ( see Eq.~(\ref{eq:I}) ) being zero. Blue color curve denotes the actual drift in signal at position '1x' found by Eq.~(\ref{eq:del_B}) while the red curve denotes the drift that would have been without the compensation. The 'ON' and 'OFF' vertical dashed lines indicate the time of the perturbation coil being turned 'ON' and 'OFF' respectively. For position of coils and sensors see Fig.~\ref{fig: coil}.}
    \label{fig:i_pi_m}
\end{figure}



% \fig{Images/i_pi_zoom}{width = \textwidth,height =10cm}{Zoomed in version of the drift $\Delta$B shown in right side of Fig.~\ref{fig:i_pi}\textcolor{blue}{(a)}, Fig.~\ref{fig:i_pi}\textcolor{blue}{(b)}, Fig.~\ref{fig:i_pi}\textcolor{blue}{(c)} and Fig.~\ref{fig:i_pi}\textcolor{blue}{(d)} respectively at sensor position '1x' for different values of $k_c^i$ with $k_c^p$ ( see Eq.~(\ref{eq:I}) ) being zero. The red vertical dashed line indicates the time of the perturbation coil being turned 'ON'. For position of coils and sensors see Fig.~\ref{fig: coil}.\label{fig:i_pi_zoom}}


\FloatBarrier
The above results confirm the currents in the coils never settles in if there is a ill-conditioned matrix present.. Next the condition number effect on both both of them after tuning (see Section~\ref{sec:tune}) will be discussed and may be the currents settle there!!


% \subsection{r vs. Condition No.}\label{sec:cond}
% Instead of going through all the steps that are discussed in Section~\ref{sec:inv}, the concept of condition number of a matrix can be used. The condition number of $\bm{M}$ can be determined from the diagonal matrix $\bm{\Sigma}$ as given in eq.\ref{eq:m} by -
%  \begin{equation}
%      cond(\bm{M})=\frac{max(\sigma_d)}{min(\sigma_d)}
%  \end{equation}
 

\subsubsection{Condition Number Effect of changing PI term Combinely}
Finally, the Section~\ref{sec:pi_behave_m} will be ended here with the discussion of the condition number effect on changing P and I term at a time which will complete the Eq.~(\ref{eq:I}).

Here, first the P and I term have been tuned following the discussion on Section~\ref{sec:tune} which has generated $k_c^p$=0.43 and $k_c^i$=0.52 . The results by applying $k_c^p$ and $k_c^i$ as those tuned values are shown in Fig.~\ref{fig:tuned_vs_i}\textcolor{blue}{(a)}. For simplicity instead of showing all the drift $\Delta$B for all the fluxgate sensors for the positions given in the horizonatal axis of Fig.~\ref{fig:m}, only '1x' is shown on the right of the figure. And same as earlier the currents  that are being sent to the coils ($C_x^\pm$, $C_y^\pm$ and $C_z^\pm$) shown on the left of the same figure. But, we couldn't determine the effect of having both of them at a time. So, keeping $k_c^i$ as 0.52 and excluding P term i.e. $k_c^p$=0.0 we run the same measurement again and the results are shwon in Fig.~\ref{fig:tuned_vs_i_m}\textcolor{blue}{(b)}. The results of Fig.~\ref{fig:tuned_vs_i_m}\textcolor{blue}{(a)} are similar to that of Fig.~\ref{fig:tuned_vs_i_m}\textcolor{blue}{(b)} which is also true for low condition number as discussed in Section~\ref{sec:pi_behave}. 

\doublefig{Images/p43i52}{width =\textwidth,height =8cm}{at $k_c^p$=0.43 and $k_c^i$=0.52. \label{fig:pi_tuned_m}}{Images/i52}{width = \textwidth,height =8cm}{at $k_c^p$=0.0 and $k_c^i$=0.52..\label{fig:i52m}}{{Currents (left vertical axis) in all six coil sides ($C_x^\pm$, $C_y^\pm$ and $C_z^\pm$) with drift $\Delta$B (right vertical axis) at sensor position '1x' for combine different values of $k_c^i$ and $k_c^p$ ( see Eq.~(\ref{eq:I}) ). Blue color curve denotes the actual drift in signal at position '1x' found by Eq.~(\ref{eq:del_B}), while the red curve denotes the drift that would have been without the compensation. The 'ON' and 'OFF' vertical dashed lines indicate the time of the perturbation coil being turned 'ON' and 'OFF' respectively. For position of coils and fluxgate sensor see Fig.~\ref{fig: coil}.} \label{fig:tuned_vs_i_m}}{short}

\FloatBarrier
The above results rather confirm that matrix condition number does not bring any change while using P and  term at a time. But it has got huge impact on I term for which the coil current never settles in case of ill conditioned matrix. Next we will try to improve the current settling problem by changing the regularization parameter.

\subsection{Effect of 'r' on PI Tuning}\label{sec:r_pi}

The discussion on Section~\ref{sec:pi_behave_m} suggests that I term is necessary for fast system response due to any drift in the magnetic signal as it takes care of the offset problems which is unsolvable by using only P term. But, in doing so, it also creates problems in terms of the coil currents which never settle in for the duration of the perturbation in case of ill-conditioned matrix. The tuning method described in Section~\ref{sec:tune} should have take care of this but we have realized that having tuned P and I has similar effect like having only I term. So, tuning method doesn't give us the solution. Rather looking for different tuning method , we have focused on the effect of 'r' on PI tuning. This Section will discuss that effect with possible outcome.

The experimental setup is same as discussed in Section~\ref{sec:pi_behave_m}. But in this case we have chosen $k_c^p$=0 and $k_c^i$=0.52. Among those $k_c^i$=0.52 has been found due to PI tuning (see Section~\ref{sec:tune}) and instead of choosing  $k_c^p$=0.43 we have made this zero as from the earlier discussion we saw that it barely has any effect while we use the I term. So, these values of $k_c^p$ and $k_c^i$ will be applied on Eq.~\ref{eq:I} to find the currents to be sent to the coils ($C_x^\pm$, $C_y^\pm$ and $C_z^\pm$) for drift $\Delta$B found by Eq.~(\ref{eq:del_B}) in the sensor positions given in the horizontal axis of Fig.~\ref{fig:m}. So, keeping those fixed, we will try to change the value of 'r' which will modify the Eq.~(\ref{eq:minvR}) for each change of 'r' value.

The effect of changing 'r' with $k_c^p$=0 and $k_c^i$=0.52 has been shown in Fig.~\ref{fig:r_pi} where the currents (left) that are being sent to the coils ($C_x^\pm$, $C_y^\pm$ and $C_z^\pm$) for drift $\Delta$B found by Eq.~(\ref{eq:del_B}) in sensor position '1x'.  It is seen from Fig.~\ref{fig:r_pi}\textcolor{blue}{(a)}, Fig.~\ref{fig:r_pi}\textcolor{blue}{(b)}, Fig.~\ref{fig:r_pi}\textcolor{blue}{(c)} and Fig.~\ref{fig:r_pi}\textcolor{blue}{(d)} which are correspond to 'r' = 2.0, 2.4, 2.8 and 3.2 respectively that the changing 'r' has significant effect on the coil current graph and barely any effect on the system response time for reducing the drift in the signal. That is at 'r'=2.0, the coil current graph has the fastest settling time where the current settles within 3 s after the perturbation has been applied. At 'r'=2.4, it takes 10s for the coil currents to settle in. But at 'r'=2.8, it seems like the coil current never settles in which is again improved at 'r'=3.2. Note that the here seriously ill conditioned matrix with condtion number 129 has been used and optimized 'r' found by the simulation model is $\sim$2.9 which tells us that the coil settling of the current graph seems to have issue with the that optimized 'r for ill-conditioned matrix. So, instead of taking the optimized 'r' that has been found by the simulation model we may have to choose the lower value of 'r'. 

\begin{figure}[!htb]
    \begin{subfigure}{.5\linewidth}
        \centering
        \includegraphics[width=\linewidth, height= 6.5 cm]{Images/r20}
        \caption{at r=2.0}
        \label{fig:r20}
    \end{subfigure}%
    \begin{subfigure}{.5\linewidth}
        \centering
        \includegraphics[width=\linewidth, height= 6.5 cm]{Images/r24}
        \caption{at r=2.4}
        \label{fig:r24}
    \end{subfigure}\\[1ex]
    \begin{subfigure}{.5\linewidth}
        \centering
        \includegraphics[width=\linewidth, height= 6.5 cm]{Images/r28}
        \caption{at r=2.8}
        \label{fig:r28}
    \end{subfigure}%
        \begin{subfigure}{.5\linewidth}
        \centering
        \includegraphics[width=\linewidth, height= 6.5 cm]{Images/r32}
        \caption{at r=3.2}
        \label{fig:r32}
    \end{subfigure}


    \caption[short]{Currents (left vertical axis) in all six coil sides ($C_x^\pm$, $C_y^\pm$ and $C_z^\pm$) with drift $\Delta$B (right vertical axis) at sensor position '1x' for combine different values of $k_c^i$ and $k_c^p$ ( see Eq.~(\ref{eq:I}) ). Blue color curve denotes the actual drift in signal at position '1x' found by Eq.~(\ref{eq:del_B}), while the red curve denotes the drift that would have been without the compensation. The 'ON' and 'OFF' vertical dashed lines indicate the time of the perturbation coil being turned 'ON' and 'OFF' respectively. For position of coils and fluxgate sensor see Fig.~\ref{fig: coil}.}
    \label{fig:r_pi}
\end{figure}

\FloatBarrier
Now the question arises about what if 'r' value is chosen more than the optimized 'r'. For answering that question, we have also studied the effect for more values of 'r' with same setup which are shown in Fig.~\ref{fig:r_pi_more}. It is seen from Fig.~\ref{fig:r_pi_more}\textcolor{blue}{(a)}, Fig.~\ref{fig:r_pi_more}\textcolor{blue}{(b)}, Fig.~\ref{fig:r_pi_more}\textcolor{blue}{(c)} and Fig.~\ref{fig:r_pi_more}\textcolor{blue}{(d)} which are correspond to 'r' = 3.5, 3.6, 3.7 and 3.9 respectively that the coil current graph seems to be settle in for larger value of 'r' before it starts showing less responsive for example at 'r'=3.9.   

\begin{figure}[!htb]
    \begin{subfigure}{.5\linewidth}
        \centering
        \includegraphics[width=\linewidth, height= 6.5 cm]{Images/r35}
        \caption{at r=3.5}
        \label{fig:r35}
    \end{subfigure}%
    \begin{subfigure}{.5\linewidth}
        \centering
        \includegraphics[width=\linewidth, height= 6.5 cm]{Images/r36}
        \caption{at r=3.6}
        \label{fig:r36}
    \end{subfigure}\\[1ex]
    \begin{subfigure}{.5\linewidth}
        \centering
        \includegraphics[width=\linewidth, height= 6.5 cm]{Images/r37}
        \caption{at r=3.7}
        \label{fig:r37}
    \end{subfigure}%
        \begin{subfigure}{.5\linewidth}
        \centering
        \includegraphics[width=\linewidth, height= 6.5 cm]{Images/r39}
        \caption{at r=3.9}
        \label{fig:r39}
    \end{subfigure}


    \caption[short]{Currents (left vertical axis) in all six coil sides ($C_x^\pm$, $C_y^\pm$ and $C_z^\pm$) with drift $\Delta$B (right vertical axis) at sensor position '1x' for combine different values of $k_c^i$ and $k_c^p$ ( see Eq.~(\ref{eq:I}) ). Blue color curve denotes the actual drift in signal at position '1x' found by Eq.~(\ref{eq:del_B}), while the red curve denotes the drift that would have been without the compensation. The 'ON' and 'OFF' vertical dashed lines indicate the time of the perturbation coil being turned 'ON' and 'OFF' respectively. For position of coils and fluxgate sensor see Fig.~\ref{fig: coil}.}
    \label{fig:r_pi_more}
\end{figure}

\FloatBarrier

The above results confirm that for a matrix with large condition number, the value of 'r' has to be tuned alongside the PI tuning. Next we will talk about a new method to find 'r'.

\subsection{Regularization by Matrix Condition Number Method  }\label{sec:cond}
We have talked about the importance of matrix condition number on PI tuning. Matrix Condition number ( see Eq.~(\ref{eq:cond} ) and regularization parameter 'r' ( see Eq.~(\ref{eq:minvR} ) have been introduced in Section.~\ref{sec:m} while discussing the inversion of the matrix $\bm{M}$ . Moreover, in Section~\ref{sec:mont}, a method of regularization by random fluctuation has been discussed. Here, we will propose another method of regularization using the concept of matrix condition number.
 
 Recall from Section~\ref{sec:inv}, regularization is needed in the first place while inverse of the matrix $\bm{M}$ because $\bm{M}$ itself is ill-conditioned matrix. That means the $\bm{M}$ has a large condition number which while inverse would produce large currents in some ill-positioned places that will make the system unstable. So, it is required to have a well-conditioned  $\bm{M^{-1}}$ which implies that the condition number of $\bm{M^{-1}}$ should be small and that's what regularization has been doing. So, we introduce Eq.~\ref{eq:minvR} with various values of 'r' and each time the condition number of $\bm{M^{-1}}$ is stored. Then the optimized 'r'  has been determined by selecting the 'r' for which the condition number of $\bm{M^{-1}}$ is the minimum.
 
 The condition number of $\bm{M^{-1}}$ for different values of 'r' has been shown in Fig.~\ref{fig:cond}. It is seen that for 'r'=0, the condition number of $\bm{M^{-1}}$=$\sim$40 that is same as the condition number of $\bm{M}$ itself. So, without regularization that is the condition number of pseudo-inverse of $\bm{M}$ would also give =$\sim$40. In regularization method, several 'r' is tried ( see Eq.~(\ref{eq:minvR}) ) and each time the condition number has been stored which are shown in the vertical axis. The red diamond symbol indicates that for 'r'=2.94, the condition number of $\bm{M^{-1}}$ is minimum and that is 3.1. That by using 'r'=2.94 in Eq.~(\ref{eq:minvR}), the condition number decrease from 40 to 3.1 which is 40/3.1$\approx$13 times of decrements. The Fig.~\ref{fig:I-fluc} shows that the 'r'=2.87 compared to 'r'=2.94 that we found here. So, both method shows comparable result. This method will always produce fixed optimized 'r' for a particular  $\bm{M}$ but the method by random fluctuation (see Section~\ref{sec:mont}) will produce different optimize 'r' for different run as it because depends on the random field.

 
\fig{Images/6c_Mcond}{width = \textwidth,height =10cm}{Condition Number of $\bm{M^{-1}}$ (vertical axis) for different values of 'r. The matrix is same as described by the Fig.~\ref{fig:m}.  \label{fig:cond}}{short}

\FloatBarrier

In the above, different method to find optimize 'r' has been discussed which is a good alternative to the one explained in Section~\ref{sec:mont} and the results are similar. In upcoming Section we will discuss fluxgate placement and impact of different shields.


% \subsubsection{Optimized r  Revisited Based on Current Response Time}\label{sec:r_currentResponse}
% It was found that there is very slow coil current rise time while applying perturbation. To get rid of that problem, first and foremost, the fastest sampling frequency (see section [\ref{sec:filter}, \ref{sec:freq}]) is needed. Then, the next step of the problem can be solved via two ways with individual having own limitations. First way is tuning the value of P and I term of PI loop as explained in Eq.~(\ref{eq:I} and Section~\ref{sec:tune}. But with increasing the value of P, the current start oscillating after certain values as shown by the top and middle current graph on Fig.~\ref{fig:crnt} which is a problem. 

% \fig{Images/crnt}{width = \textwidth}{Coil current in one of the coil side for optimized r=2.8 with P=0 and I=1.0 (top) and with P=0 and I=1.5 (middle) and for best r considering noise with P=0 and I=1.0 (bottom). \label{fig:crnt}}



% The alternative way is to change the value of optimized r (see section [\ref{sec:mont}, \ref{sec:cond}]) which in turns increase noise in the prototype. But with inclusion of some current fluctuations, it was found that the coil current response time was increased heavily  as shown by the bottom current graph in Fig.~\ref{fig:crnt}. Now, the best compromised value of r was chosen by observing the 'rise time vs r' and 'fluctuations vs r' as shown in Fig.~\ref{fig:riseT}.



%  \doublefig{Images/riseT}{width =\textwidth, height= 8 cm}{Rise Time vs r \label{fig:rise}}{Images/fluc}{width = \textwidth, height= 8 cm}{Fluctuations\label{fig:fluc}}{{(a) shows the Rise Time vs r (b) shows the Fluctuations } \label{fig:riseT}}
% % \fig{Images/bt}{width = \textwidth}{Magnetic Field Compensation \label{fig:bt}}






\section{Fluxgate Placements and Impact of Shields}


In earlier section we have shown results using 12 fluxgate sensors placed at positions as given in horizonatal axis of Fig.~\ref{fig:m} and there exact positions are described by the Fig.~\ref{fig: coil}. We were facing serious problem with coil current not being settle properly although the field seems to be compensated on time. So, we have tried different posoitions mainly in corners and center of each of the coil faces. Eeven, we have removed the outermost shields also. In this Section, the results of those studied will be shown and discussed. 

First the study on the fluxgate placement will be discussed.

\subsection{Fluxgate Placements}

We have not got clear idea about the placements of the fluxgates from the previous studies and our coil current also did not settle down the way we thought is should. We have started taking data with 3*4=12 sensors at slowest sampling frequency (see Table~\ref{table:index}) in the corners but the coil currents were not properly settle down. Then we thought maybe increasing sensors will eliminate the problems. So, we bought new fluxgate sensors and build another breakout box (see Section~\ref{sec:sensor}) but still the results were not good in the coil currents. Then we thought may be if we place in the center of each of the faces of the coils the results will be better but unlucky us. Then we decided to remove the outermost shield (see Section~\ref{sec:shield}) to see the effect but still no luck. After that we decided to use the fastest sampling frequency of our ADC for which we have to build the filters. But due to time limitations and cost concern we have only build 12 filters to support 12 fluxgate sensors. Now, due to this the current response time has increased but that was not the solution of the current unsettle problem. Finally, we have realized that in addition to fastest response we have to also consider the matrix condition number and if the condition number is large then we have to lower the value of optimized $r$ (see Section~\ref{sec:new_study_r}). Here, we will talk about the the studies we have done on fluxgate placements.

\begin{table} [htb!]
    \centering
    \begin{tabular} { |c|c|c|c|c|c|} 
        \hline
        Positions & \makecell{Matrix \\Condition Number} &\makecell{Max $\Delta I_c^{\text{simRMS}}$\\ (mA)} & Max Compensation\\
        \hline\hline
        1, 3, 6 and 8 & 33 & 33 & 70$\%$ \\ 
        \hline
        2, 4, 5 and 7 & 29 & 21 & 62$\%$ \\ 
        \hline
        \makecell{Center \\($C_x^\pm$, $C_y^\pm$ and $C_z^\pm$)} & 36 & 98 & 50$\%$ \\ 
        \hline
        \makecell{Center-6cm \\($C_x^\pm$, $C_y^\pm$ and $C_z^\pm$)} & 99 & 172 & 58$\%$ \\ 
        \hline
        \makecell{Center+6cm \\($C_x^\pm$, $C_y^\pm$ and $C_z^\pm$)} & 81 & 192 & 39$\%$ \\ 
        \hline
        \makecell{1, 2, 3, 4, \\5, 6, 7 and 8} & 28 & 13 & 35$\%$ \\ 
        \hline
        \makecell{1, 2, 3, 4, \\5, 6, 7, 8 and \\Center ($C_x^\pm$, $C_y^\pm$ and $C_z^\pm$)}  & 22 & 12 & 15$\%$ \\ 
        \hline

    \end{tabular}
    % \vspace{4mm}
    \caption[short]{Properties of different no of fluxgate sensors for different positions. Max $\Delta I_c^{\text{simRMS}}$ column has been taken for each of the positions defined by generating Fig.~\ref{fig:Isim}. Max compensation column has been determined for for each of the positions defined by generating Fig.~\ref{fig:fluc-sim} and the description is given in text. For the positions of the fluxgates see Fig.~\ref{fig: coil} }\label{table:flux-pos}
\end{table}

Mainly the corner positions and the center of each of the coil faces  have been tested. Also, they have been analyzed by moving slightly in different positions. The results for different no of sensors in different positions are given in Table~\ref{table:flux-pos}. Position of the fluxgates are defined by the numbers while they are in coreners and when they are in the center of each of the coil faces they are termed as 'Center ($C_x^\pm$, $C_y^\pm$ and $C_z^\pm$)'. Cenetr-6cm means all the senosrs in the center of the coil faces have been brought 6cm towards the origin from the center and center+6cm means they have brught 6cm away outside the center. For, the full picture of the postions see Fig.~\ref{fig: coil}. the It is seen that that matrix condition number is from 22-36 for the fluxgates being placed either in corners or in center. But if they are slightly moved within $\pm$6cm of center then the matrix condition number becomes very large. Only considering the matrix condition number, it seems that having fluxgates in all the corners and the center of each of the coil faces should be the best choice as its giving the lowest condition number which is 22. But that is not the whole story! To quatify more we have taken the help of regenerating Fig.~\ref{fig:Isim} and Fig.~\ref{fig:fluc-sim}. From Fig.~\ref{fig:Isim}, we have recorded the maximum  $\Delta I_c^{\text{simRMS}}$ (mA) for 30 different sets of $B_s^{\text{rand}}$. For maximum compensation we have generated the Fig.\ref{fig:fluc-sim} for same for 30 different sets of $B_s^{\text{rand}}$ from which we have recorded lowest the remaining fluctuation F. For example- for fluxgate positions 1, 3, 6 and 8, the lowest F goes to 0.3. That means the maximum compensation due to the field produced by $\Delta I_c^{\text{sim}}$to counteract $B_s^{\text{rand}}$ is(1-0.3$\times$100$\%$) =70$\%$. For more details see Section~\ref{sec:mont}. It seen that the current goes crazy if the fluxgates are placed in the center of the coil faces and the maximum $\Delta I_c^{\text{simRMS}}$ ranges from 98 to 192 mA. But the maximum $\Delta I_c^{\text{simRMS}}$ is 12 mA when the center fluxgates are used with the corners one. But on that time the maximum compensation due to the field produced by $\Delta I_c^{\text{sim}}$to counteract $B_s^{\text{rand}}$ is only 15$\%$. So, anything with center seems to give crazy results. Then by looking at the all the three parameters e.g. matrix condition number, maximum $\Delta I_c^{\text{simRMS}}$ and maximum compensation for the 3 different sets of corner positions only, it is seen that the results are more balanced and surprisingly the compensation with 4*3-axis sensors are better than 8*3-axis sensors in exchange of more maximum $\Delta I_c^{\text{simRMS}}$.


\FloatBarrier


% The increase in matrix condition number for the center of the coil sides is noteworthy.  So, a study has been done to see the current response effect while sensors are in the middle of the coil sides as shown in Fig.~\ref{fig:cb_center}. The Fig.~\ref{fig:cb_center_p} represents the current response on all the six coil sides with their effect on the $z$-axis in the origin as shown in the right with only choosing proportional (P) term. But just adding a small fraction of integral resest (I) term makes the current response unstable as shown in Fig.~\ref{fig:cb_center_pi}. So only P controller is suitable if fluxgates are placed in the middle of the coil sides but in the case of corner positions either P only or I only or PI controller option is available. In a summary, the more the sensors the more is the matrix condition number. Corner positions are better in terms of different freedom of controlling.

% \doublefig{Images/cB_t_center_p}{width =\textwidth, height= 6 cm}{Position=1,2,3,4,6,8\label{fig:cb_center_p}}{Images/cB_t_center_pi}{width = \textwidth, height= 6 cm}{Position=1,2,3,4,6,8\label{fig:cb_center_pi}}{{PI Active Magnetic Field Compensation Results by both Experiment and Simulation.} \label{fig:cb_center}}

% \doublefig{Images/bt6}{width =\textwidth, height= 7 cm}{Position=1,2,3,4,6,8\label{fig:bt6}}{Images/sf6}{width = \textwidth, height= 7 cm}{Position=1,2,3,4,6,8\label{fig:sf6}}{{PI Active Magnetic Field Compensation Results by both Experiment and Simulation.} \label{fig:btSF6}}

% \doublefig{Images/bt8}{width =\textwidth, height= 7 cm}{Position=1,2,3,4,5,6,7,8\label{fig:bt8}}{Images/sf8}{width = \textwidth, height= 7 cm}{Position=1,2,3,4,5,6,7,8\label{fig:sf8}}{{PI Active Magnetic Field Compensation Results by both Experiment and Simulation.} \label{fig:btSF8}}
The above discussion of the Table~\ref{table:flux-pos} shows that current goes crazy if the fluxgates are placed in the center of each of the coil faces and crazier if they are slightly moved inside or outside of the center. The corners position fluxgates shows normal behaviour compare to the center positions and surprisingly less sensors shows better compensation in the corners in exchange more more currents.

\FloatBarrier
\subsection{Different Shields}

There are four layers of shields have been used for passive shielding in this prototype (see Section~\ref{sec:shield}). The active compensation effect has been seen using outermost shield and no shields. Both of the configurations produce similar results except for the case of shield, the matrix condition number is less. The Fig.~\ref{fig:btSF8_s} shows the AMC compensation (Fig.~\ref{fig:bt8_s}) and the corresponding allan deviation and shielding factor ( Fig.~\ref{fig:sf8_s}) with outermost layer of shield. The matrix condition number is found to be 19. Similarly, the Fig.~\ref{fig:btSF8} shows the AMC compensation (Fig.~\ref{fig:bt8}) and the corresponding allan deviation and shielding factor ( Fig.~\ref{fig:sf8}) without any shield. The matrix condition number in this case is 132. The one advantage of having shielding is that , the shielding factor for all the sensors remain $geq$1 but in case of no shield, the shielding factor may be very good at certain point but in some point it can go below 1.

\doublefig{Images/bt8_shield}{width =\textwidth, height= 7 cm}{Position=1,2,3,4,6,8\label{fig:bt8_s}}{Images/sf8_shield}{width = \textwidth, height= 7 cm}{Position=1,2,3,4,6,8\label{fig:sf8_s}}{{Shield} \label{fig:btSF8_s}}{short}

\doublefig{Images/bt8}{width =\textwidth, height= 7 cm}{Position=1,2,3,4,5,6,7,8\label{fig:bt8}}{Images/sf8}{width = \textwidth, height= 7 cm}{Position=1,2,3,4,5,6,7,8\label{fig:sf8}}{{No SHield} \label{fig:btSF8}}{short}


\section{Coil Configuration}

\begin{itemize}
\item coil cube with matrix calculation in python by Jeff shows the condition number to be infinity.
\item exploits the diagonal matrix (Section~\ref{sec:m}) and found one of them is giving zero
\item wire the two coils to work as one and matrix condition number is hugely improved. close to one!!
\item Reason : Maxwell's equation and current mode in all the six coils.
\end{itemize}




For the prototype, two different configuration of compensation coils have been exploited. In the first case, all the six coils have been used in the six faces surrounding the compensation area and for the later one, two coils have been wired together so that they can act as one making total five instead of six coils. The main reason for using the second configuration is the condition number of $\bm{M}$. It was found to be around 2 for the second one as compared to 40 in the first.

\fig{Images/v2}{width = \textwidth}{Square root of eigenvalues of $\bm{M^T}\bm{M}$ and $\bm{M}\bm{M^T}$  in sensors$\times$coils dimension. \label{fig:v2}}{short}

\fig{Images/wt}{width = \textwidth}{Orthonormal eigenvectors of $\bm{M^T}\bm{M}$ in coils$\times$coils dimension. \label{fig:wt}}{short}

\fig{Images/v5c}{width = \textwidth}{Square root of eigenvalues of $\bm{M^T}\bm{M}$ and $\bm{M}\bm{M^T}$  in sensors$\times$coils dimension. \label{fig:v5c}}{short}

\fig{Images/wt5c}{width = \textwidth}{Orthonormal eigenvectors of $\bm{M^T}\bm{M}$ in coils$\times$coils dimension. \label{fig:wt5c}}{short}

\begin{figure}
    \begin{subfigure}{.5\linewidth}
        \centering
        \includegraphics[scale=.28]{Images/c1}
        \caption{Current Direction in $C_x^{\pm}$}
        \label{fig:c1}
    \end{subfigure}%
    \begin{subfigure}{.5\linewidth}
        \centering
        \includegraphics[scale=.28]{Images/c3}
        \caption{Current Direction in $C_y^{\pm}$}
        \label{fig:c3}
    \end{subfigure}\\[1ex]
    \begin{subfigure}{\linewidth}
        \centering
        \includegraphics[scale=.33]{Images/c5}
        \caption{Current Direction in $C_z^{\pm}$}
        \label{fig:c5}
    \end{subfigure}
    \caption[short]{Current direction in $C_x^{\pm}$, $C_y^{\pm}$ and $C_z^{\pm}$ . The net current is zero while adding the current contribution from all the six coils for this orientation. $C_z^{\pm}$ coils have been wired together to break the configuration which results in significant decrease in matrix condition number. }
    \label{fig:cDir}
\end{figure}

% As previously discussed in Section~\ref{sec:prototype}, there are six compensation coils ($\bm{C_x^\pm}$, $\bm{C_y^\pm}$ and $\bm{C_z^\pm}$) and one perturbation coil ($\bm{P_z^+}$). 
 It can be as explained by Fig.~\ref{fig:cDir} where one of the coil current mode was shown for all six coils. The total current contribution is found to be zero due to current contributions from $C_x^{\pm}$ in Fig.~\ref{fig:c1}, $C_y^{\pm}$ in Fig.~\ref{fig:c3} and $C_z^{\pm}$ in Fig.~\ref{fig:c5}. To break the mode, two out of the six coils have been wired together so that they can act as one resulting total compensation coils be five. 
%It was found that due to this, the condition number is decreased drastically to 2.



 \lhead{\emph{Conclusion}}
\chapter{Conclusion}\label{ch:conclusion}



\section{Key Findings}

%\begin{itemize}
%\item I was given a kind-of working prototype and a copy of Bea's thesis, and I was told to ``make it work''.  It works, now.  Really, really well.  So, the project was a complete success.
%\item Along the way, I discovered a whole bunch of new problems.  I solved each problem and improved the system considerably each time.
%\item We need to list here the main important improvements and results.  It should mimic the list at the start of Chapter 5, probably.
%\end{itemize}

My MSc thesis work was initiated with the goal of establishing a
working multi-dimensional PID control system for magnetic fields,
based on the work of Refs.~\cite{bea,lins}.  This goal was
successfully achieved and the work went beyond the goal in several
respects.  In the process of developing the system, I discovered a host
of new challenges, to which I found innovative solutions.  Below I
list the key improvements made to the system and the results of those
improvements.  They are the following:
\begin{enumerate}
\item {\bf $\mathbf{4^{th}}$ order low pass Butterworth filters.}  

I designed 12 active filters which are excellent in reducing high frequency noise, even slightly better than the low pass filter (LPF) of Bartington's SCU1. The filters will be important for future studies facing high frequency ($>10$~Hz) noise issues. The filters could be improved further by designing them with variable gain and offset, which is an advantage of the SCU1. This would make it easier to adjust the range of values passed to the DAQ module, without much additional noise. This is one reason that in my case, to acquire the fluxgate signals placed inside the shield within the coil cube. In such cases, I used the SCU1 with gain 100 so that it can easily be read by the ADC.  

\item {\bf Finite Element Analysis and Multi-dimensional PI control simulation.}  

The simulation of the prototype active compensation system was vital to demonstrate a full understanding of the experimental results. Finite element analysis (FEA) was used to generate both the matrix $\bm{M}$ and the field change $\Delta B$ due to the perturbation coil for any number of the sensors placed within the coil cube. The FEA results were then used in a time-dependent PI feedback algorithm implemented in Python. This resulted in a real time PI control simulation which gave agreement with experiment. Problems observed in the data, for example the current drifting problem, were correctly reproduced by the simulation. Even a simulation conducted in free space based on analytical magnetic field calculations (not requiring OPERA) showed many of the same issues. The strong message for future work is to use this kind of simulation as a tool for testing the entire system before it is built.

% This will be a very handy tool for the future students as they will have the freedom to use thousands of sensors within their respective cube dimension and most importantly they do not have to worry about placing the sensors correctly in their system to study in detail about the effect of $\bm{M}$ for any set of sensors. Simulation of a certain system integrating feedback algorithm is a unique work that has not been still published in any thesis related to active compensation

%\begin{itemize}
%\item OPERA part (static)
%\item PI part (dynamic)
%\item Putting them together gives very good results compared to experiment.
%\item Main proposal for future work is to use this kind of simulation as a tool for testing the entire system before it is built.
%\item Many problems observed in the data, e.g. the current drifting problem, could have been found before ever using the system if this simulation had existed a priori.
%\item Even simulation in free space (not requiring OPERA) would have shown many of the same issues.
%\end{itemize}

\item {\bf Better understanding of matrix inversion, PI parameters, and tuning.} 

The author in Ref.\cite{bea} proposed matrix inversion with Tikhonov regularization, which I followed initially. But I realized later that there was more to the story. I noticed there a relationship between the regularization parameter $r$ and the PI parameters. I came to the conclusion that no amount of PI tuning could reduce the current drifting problem even though $r$ was optimized according to Ref.\cite{bea}.  However, I found that the current drifting problem could be reduced by treating $r$ more-or-less another free parameter at the cost of introducing high frequency noise. Moreover, I proposed a new method to find $r$ based on the condition number of the matrix which I argue is more robust than the method of Ref.\cite{bea}. Eventually, I realized that regularized matrix inversion was nonoptimal. It can be avoided by carefully designing a well-conditioned system. This conclusion is consistent with Ref.~\cite{rawlikpriv}. 

Reference~\cite{rawlik,rawlikpriv}, further proposed a new feedback algorithm. I showed that this was equivalent to a PI system restricted to one particular choice of tuning parameters. It is clearly better not to use a restricted set of tuning parameters.


%\begin{itemize}
%\item Bea did matrix inversion.  I realized there's a relationship between the matrix regularization and PI parameters and studied this in more details.
%\item Rawlik proposed a new feedback algorithm apparently not based on PI.  I showed that this is equivalent to a PI system restricted to one particular choice of tuning.  It is clearly better to allow for more flexibility in the tuning than proposed by Rawlik.
%\end{itemize}

\item {\bf Coil current modes based on coil configuration.}  

Solving the current drifting problem was one breakthrough of this thesis. After many experimental tests, I found that the simulation in free space was very effective in improving the understanding of the problem. I discovered that the 6-coil feedback algorithm always had one mode which generated zero field no matter what the current. This resulted in one singular value that was always near zero. Tikhonov regularization tries to force this mode to be treated on an equal footing with the others. I found a superior solution which connects two coils in series (a 5-coil feedback algorithm) thus preventing the bad mode from occurring. It was then I realized I did not need to regularize the system if the system is well conditioned in the first place. In the end, I agreed with Ref.~\cite{rawlik} that using a low condition number (near unity) as a measure of good system design. As further coil design was beyond the scope of this thesis, it suggests future work studying coil design in the context of the condition number to study the ill-conditioning problem.

%  Because while searching for the solution, I have built very efficient filters and realized that passive shield has no impact on final active compensation result, and regularization is a waste of time, made an useful tool {\it i.e.} simulation to test the system and finally able to solve by understanding the coil current pattern for different coil configuration. 
%\begin{itemize}
%\item Agreement with Rawlik on low condition number as a measure of good system performance.  This will be another recommendation for future work on coil design:  to make sure the condition number is reasonable.
%\end{itemize}
\end{enumerate}



\section{Recommendations on the Active Magnetic Compensation System Design Process for TUCAN}

%\begin{enumerate}
%\item Take the attitude that you are supervising the next person to work on this.  What is the first thing they should do?  The next thing?  How to bring the design process to a conclusion so that the system could be built.
%\item discussion on better ideas beyond regularization, such as spherical harmonics, patch coils, and how this can lead to a properly regularized system that is more flexible
%\item Designing the system for the known perturbations... at TRIUMF, what are they?  This can be used as an input to the design process.  Relate back to the $C_x^\pm$ data.
%\item Small condition number is not the most important thing.  We can get small condition number if only three coils are used.  But this system will only be able to cancel uniform fields.
%\item Use spherical harmonics to the desired order and use coils or restricted combinations of coils to mimic those spherical harmonics.  The restricted set of coil currents should then be well-conditioned.  Rawlik went beyond this and suggested rewiring patch coils to generate spherical harmonics more efficiently.
%\item Discussion on simulation and simulating the system ``fully'' (including the PI loop) using the tools you developed before beginning to build it.  Should make the point that we now know enough to do a good job a priori.
%\end{enumerate}

In this Section, I make a few general recommendations on how I would proceed if designing the ultimate active magnetic compensation system solution for the TUCAN nEDM experiment.


\subsection{Test designs based on known perturbations.\label{sec:sources}}

When using my 6-coil feedback algorithm, a key observation was the current-drifting problem.  I used a process of developing hypothesis followed by experimentation, which took a long time to determine the source of the problem.

One of the key observations that suggested my system was finally working properly was that when the perturbation coil was turned on, the system would respond dominantly by turning on only the nearest coil that generates a field in the same axis as that coil (in the 5-coil system). In the 6-coil system, this was not the case.  All 6 coils would eventually engage.  Eventually I determined that this was simply an error induced by inappropriate constraints being placed on the system, which were covered up by the process of matrix regularization. What I wish I had done early would have been to recognized that this is a sign of a failing treatment of the matrix.  This would have helped me to focus in on the real problem and solve it faster.

Another recommendation related to this one is to carefully measure the perturbations expected, or to simulate the planned perturbations if driven by a perturbation coil. The plan of measurement of the perturbations is being developed and some experiments are ongoing at this time~\cite{beapriv}. On the other hand, a perturbation coil has already been built at TRIUMF~\cite{smith,cudmore}. These should be used, in simulation, to test any planned multi-dimensional PI system.  Based on the discussion above, if I had conducted such analysis in simulation first, I would have discovered the solution to the current drifting problem much earlier.


\subsection{To regularize or not to regularize.}
 
My recommendation is to develop a system with sufficient degrees of freedom and with condition number as close to unity as possible. 

In my case, the matrix regularization tended to stabilize the magnetic fields properly, but gave a very slow response in the currents. Eventually this was found to be due to a poorly constrained system with too many degrees of freedom. Tikhonov regularization made the system work, after a fashion, but could not solve the current-drifting problem. The reason is that Tikhonov regularization forced the ``zero field'' mode (singular value zero) to contain some current. There is no real purpose to have this mode exist at all.

My expectation is that this problem can be solved in an alternate way. In my case of a 6-coil system, it was easy. I simply reduced the number of degrees of freedom of the system without loss of generality, so that there were only five independent currents (5-coil system). I expect this solution can be generalized to any arbitrary number of coils, and I suggest a few possible methods to do this in Section~\ref{sec:spherical} below.

It is important to consider the condition number of the matrix when designing the coil system.  If the condition number is reduced by a change, it could be a step in the right direction.  It is also important to consider the right singular vectors (coil modes), which could reveal why a certain singular value is small.


\subsection{Condition number is not everything.}

One clear way to reduce the condition number is to simply use Helmholtz coils for everything (three independent sets with three independent currents).  In fact such a solution was pursued in the prototype system of Ref.~\cite{rawlik} constructed at ETH Z\"urich. However, this is clearly a bad strategy because such a system will not have as many degrees of freedom as six independent coils.  For example, it would never be able to compensate magnetic gradients. What is not trivial is why five independent coils is sufficient.  But if thinking in terms of spherical harmonics applied to the magnetic scalar potential, and the kinds of fields that can be generated by Helmholtz coils, this becomes more obvious.  This leads to my next recommendation.


\subsection{What to do instead of regularizing.\label{sec:spherical}}

If I have to begin designing coils tomorrow, I would try the following two general strategies.
\begin{enumerate}
    \item Coils that generate spherical harmonics.
    \item Excising the zero-field mode.
\end{enumerate}

\subsubsection{Coils that generate spherical harmonics}

A decomposition of the magnetic field into a desired order of spherical harmonics can be conducted. The order could be constrained by the potential sources (discussed in Section~\ref{sec:sources}) that are desired to be compensated.  Decomposing the magnetic scalar potential in this way allows one to design coils, each of which generates a spherical harmonic.  Alternately, patch coils could be used to generate each spherical harmonic. Both strategies were discussed in Refs.~\cite{rawlik,rawlik_paper_coil}. Since this method will prevent the (bad) zero-field mode from occurring, this should result in a properly conditioned system.  This can easily be tested in a coil simulation.

\subsubsection{Excising the zero-field mode}

Another alternate solution could be to wind patch coils on a convenient square frame and initially to allow all possible modes to be excited in the coil system.  One of these modes will then correspond to the zero-field mode.  Fortunately, this mode can easily be identified because it will have a singular value that is zero, or at least considerably smaller than the other modes.  Once this mode has been identified, it can be removed from the singular matrix and the dimension of the matrix reduced by one.  The remaining right singular vectors (coil modes) can then be used as the degrees of freedom of the system.

If any other modes should appear with singular values that are small, they too could be removed in a similar fashion until the condition number is small enough.  This would prevent the need for matrix regularization, since it provides another method to limit the number of degrees of freedom.

Both methods could easily be implemented in a coil simulation. I would even recommend that the calculation be done in free space initially so that FEA need not be used. Then the system could be designed much more quickly and a reduced set of simulation be done in FEA once the appropriate number of degrees of freedom has been decided.

\subsection{Develop a full FEA plus feedback system simulation now.}

A key achievement of my thesis was the application of FEA results to the PI simulation. It was the simulation of the current-drifting problem in my 6-coil system that eventually convinced me that this must be a problem inherent to the multi-dimensional control system. 

If either free-space and/or FEA calculations are available for both the perturbations and for the coil system, it is easy to implement these into a single time-dependent PI simulation. This kind of simulation generally reproduces all the experimental results as I have shown in my thesis. If I had done this simulation first, I might even have been able to discover the current-drifting problem in advance of every conducting the experiments.


%As discussed in Chapter~\ref{ch:magnetics}, the TUCAN nEDM experiment is located next to TRIUMF cyclotron with $\sim400~\mu$T background, $\sim100$~nT fluctuations, and $100~\mu$T/m gradients. Moreover, the 50 ton crane in the experimental site can also induces very large changes. The perturbations must be quantified properly. I recommend to build a well-conditioned system where the perturbation can be used as an input to the design process. I discourage to justify the design process by considering having small condition number only. Because, small condition number can easily be generated if only three coils are used but then the system having three coils will be restricted to cancel only uniform fields ignoring gradients. Hence, I rather suggest to use spherical harmonics to the desired order and use coils or restricted combinations of coils to mimic those spherical harmonics. The restricted set of coil currents should then be well-conditioned. I have already showed the restricted set of coil currents in Section~\ref{sec:coil_config} which decreases the condition number for my system significantly. The author in Ref.~\cite{rawlik} went beyond this and suggested rewiring patch coils to generate spherical harmonics more efficiently and described a coil design method in Ref.~\cite{rawlik_paper_coil}. A well conditioned system is more flexible as it does not have to worry about regularization and it exactly knows the value coil currents to eliminate a certain field in certain direction.

% \textcolor{red}{Designing the system for the known perturbations... at TRIUMF, what are they?  This can be used as an input to the design process.  Relate back to the $C_x^\pm$ data.  }

%Having the system defined properly by improved coil design based on spherical harmonics and known perturbation, I recommend to make the simulation of the system integrating the PI feedback algorithm using the tools that I developed before beginning to build it. The simulation will give the full picture of what is expected and what the system is offering and if that are not aligned make sure to correct the design and confirm again via simulation before building the system. Making a simulation beforehand should make the point that thes system has been understood and tested enough to do a good job a priori.
% Discussion on simulation and simulating the system ``fully'' (including the PI loop) using the tools you developed before beginning to build it.  Should make the point that we now know enough to do a good job a priori.







\section{Implementation in the TUCAN nEDM Experiment}\label{sec:implemantation}

%\begin{itemize}
%\item Requirements... we need 'em.  Cite PSI conference proceeding which says we ``might'' implement such a system in n2EDM.  Trade-offs of active vs.~passive shielding and the decision on the dividing line between the two.  History of the PSI system.
%\item Engineering statements...  we have to be able to build it and make it fit
%\item Other applications of active shield:
%\begin{itemize}
%\item Saturation?
%\item Providing a somewhat smaller field when the door to the room is opened?
%\item Somewhat smaller field to prepare components for the room?
%\end{itemize}
%\end{itemize}

In the previous section, I discussed a few specific ideas on how I would proceed to design an active compensation system. Most of these relate to the development of simulation tools which can guide the design.

Aside from this, there are a large number of other factors that must be considered when designing such a system for TUCAN.

The first and foremost question is to be answered is whether the nEDM experiment needs an active compensation system or not. 

Historically, such systems were not used in the Sussex-RAL-ILL nEDM experiments.
The PSI group was the first to implement such an active compensation system in an nEDM experiment~\cite{bea_paper}. The system was developed mainly to improve experiment up-time. With the PSI experiment being located closer to facilities generating strong magnetic fields, the experiment would have to spend longer periods of time degaussing without an active compensation system.

Their upgraded experiment n2EDM will be located in the same area. One of the main improvements will be to use a magnetically shielded room (MSR). Even in this situation, it is unclear whether any active magnetic compensation system is necessary.  %, but they ``might be left with measurable changes in the magnetic field of the experiment'' as quoted in Ref.~\cite{psi_n2edm_PPNS-workshop}.
In Ref.~\cite{rawlik}, drawings of a potential system were shown. In a recent conference proceeding~\cite{psi_n2edm_PPNS-workshop}, it was indicated an active compensation system was being developed as an additional shielding layer and that it ``might be installed after initial characterization measurements.'' 

\fig{Images/active_scheme}{width = 0.7\textwidth}{Schematic diagram of TUCAN nEDM magnetic field subsystems. From inside out: UCN and the comagnetometer, followed by the internal coil system ($B_0$ and $B_1$ coils), four layers of passive shielding comprising the magnetically shielded room (MSR), and the active compensation system which needs to be designed.\label{fig:msr_zoomed}}{Schematic diagram of TUCAN nEDM magnetic field subsystems.}

%  At large fields, saturation of the passive magnetic shielding system can be a concern, which would seriously impact its effectiveness. Furthermore, when accessing the experiment, the door to the MSR must be opened. If presented with a large external field, the innermost layer of the passive shielding system could themselves become magnetized, necessitating degaussing and additional experimental down time with these factors in mind. The proposed plan is to nullify and stabilize the magnetic field environment at TRIUMF to ($\sim\;1\;\mu T$) using dedicated large bucking coils and also to reduce the fluctuations upto a factor of 100 using a separate set of coils by supplying currents to them  where the fluctuations will be measured by fluxgate sensors in a continuous feedback loop.



To decide on the active compensation strategy for the TUCAN nEDM experiment (Fig.~\ref{fig:msr_zoomed}), I recommend to consider the following factors:
\begin{enumerate}

    \item {\bf Which fields the active compensation system should correct, and why.}

    The MSR is likely to be designed with shielding factor $10^5$ on the basis that external 100~nT fluctuations be reduced to the pT level.  At this level they are within the typical level of magnetic noise and drift arising from changes in the remnant magnetization of the innermost shield layer after degaussing (idealization).

    The active compensation system might be able to correct 1000~nT fluctuations to the 100~nT level as an aggressive but potentially realistic goal. This might make it possible to run the system with worse exterior fluctuations. The question at this point is whether there are any such 1000~nT fluctuations present in Meson Hall, which is relatively unknown.
    
    It is known that crane motion can amount to a 10000~nT or larger perturbation. It is unlikely a compensation system could be design that could compensate this level of fluctuations. The best we could then hope for is that the system would be used more as its design goal at PSI: to reduce downtime by minimizing the amount of degaussing required after such an excursion.

    \item {\bf Trade-offs of active vs.~passive shielding and the decision on the dividing line between the two.} 
    
    %Both the active and passive shielding are expensive, but it’s not clear as of yet which might cost more, an additional layer of mu-metal or an active magnetic compensation system.
    
    %Advantages of coils:   potentially allows to compensate larger DC magnetic fields which improve the performance of MSR and long term stability of $\bm{B_0}$ field inside the MSR;
    
    %JWM:  no, it does not.  Not if it does not saturate.  Or rather, how much larger and why?
    
    %disadvantage: complex system which requires a lot of development and constant monitoring and analysis until proven to operate stably as good as possible. It is noteworthy to mention that active shields do not replace passive ones. For very low frequencies {\it i.e. $\mathrm{<~Hz}$}, the shielding factor of passive shields degrades~\cite{active_raw_app_0,active_raw_app_1}. Meanwhile, the performance of active shields are best best at DC and reach up to $\mathrm{KHz}$. A stable magnetic field over the whole range of frequencies is expected from the combination of the two shielding methods~\cite{active_raw_app_0,active_raw_app_1,active_raw_app_2,active_raw_app_6}.

    The main question here might be about possible 1000~nT fluctuations in Meson Hall. If they are continuous and negate running the experiment, the budget, personnel, and schedule question would be whether it is superior to develop an active compensation system or to add one layer of passive magnetic shielding.

    Clearly, the MSR is designed to handle 100~nT fluctuations, and crane motion is likely rare during nEDM running. So, the real question is if there are any unnaturally large fluctuations 1000~nT. There is presently insufficient information on magnetic fields in Meson Hall to say whether this is worthwhile to consider or not.


    \item {\bf Saturation of the outermost layer of the MSR.} 
    
    At large DC fields such as the 400,000~nT scale experienced in Meson Hall, saturation of the passive magnetic shielding system comprising the MSR can be a concern, which would seriously impact its effectiveness.
    
    As long as the outermost magnetic shield layer does not saturate, and the exterior fluctuations to be compensated by the MSR are at the 100~nT scale, then this is no longer a concern:  the MSR will certainly perform adequately without any active magnetic compensation system.
    
    The question becomes if it is worthwhile to consider compensating for changes in the cyclotron field which drives the magnetic environment of the area, or if saturation has any realistic chance of occurring.  The active compensation system could be used to counter such effects.
    
    %The companies that offer MSR will not give a written specification that the equipment delivered by them will perform the same when in a field significantly larger than Earth fields.
    
    %translation:  because the company is stupid, we don't know what to do.  I.e. we are stupider than the company.
    
    %The performance of MSR must be tested first and we should be satisfied fully that the MSR's performance will not be hampered in the field conditions more larger than the MSR has been specified for.
    
    % So looking at it from a legal point of view, it would probably be a good idea for the case we ever wanted to complain about the performance of the MSR that we can show that we operate it in the conditions it has been specified for.

    \item {\bf Magnetically shielded access to the MSR.} 
    
    When accessing the experiment, the door to the MSR must be opened. If presented with a large external field, the inner layers of the passive shielding system could themselves become magnetized, necessitating degaussing and additional experimental down-time.  Furthermore, it is also useful to have an area just outside the door with a somewhat smaller magnetic field where components can be prepared for installation.  An active compensation could provide such a region as a side goal.

    \item {\bf Engineering, space, and access requirements.} 
    
    The compensation system would need to fit into the experimental area and not limit access to important parts of the experiment.  The interfaces to other subsystems needs to be taken into account.
\end{enumerate}

 I expect that based on these factors, an active magnetic compensation system will eventually be implemented into the TUCAN nEDM experiment. My work serves as a useful study of a prototype system. Several new challenges were uncovered and solved along the way, which should help guide the design of the system at TRIUMF.
 

% \subsubsection{Effect of changing only P term}
% Here, the effect of changing proportional gain term (P) or $k_c^p$ of Eq.~(\ref{eq:I}) will be discussed.

% P term is proportionally multiplying the error (the difference between setpoint and actual measurement) with a constant gain. For the prototype it is

% \begin{equation}
%     P_{\text{PI}}=k_c^p \Delta I_c^n
% \end{equation}
% where, $k_c^p$ is the proportional gain and $\Delta I_c^n$ is explained in Eq.~(\ref{eq:del_I}).

% Depending on the value $k_c^p$, it tries to minimize the error level between the setpoint and the actual measurement with passage of several measurements. A large value of $k_c^p$ will result large output change for a particular error and eventually it reaches a threshold point above which the system becomes unstable. 

% % \begin{figure}[!htb]
% %     \begin{subfigure}{.5\linewidth}
% %         \centering
% %         \includegraphics[width=\linewidth, height= 6.5 cm]{Images/p25}
% %         \caption{at $k_c^p$=0.25}
% %         \label{fig:p25}
% %     \end{subfigure}%
% %     \begin{subfigure}{.5\linewidth}
% %         \centering
% %         \includegraphics[width=\linewidth, height= 6.5 cm]{Images/p50}
% %         \caption{at $k_c^p$=0.50}
% %         \label{fig:p50}
% %     \end{subfigure}\\[1ex]
% %     \begin{subfigure}{.5\linewidth}
% %         \centering
% %         \includegraphics[width=\linewidth, height= 6.5 cm]{Images/p75}
% %         \caption{at $k_c^p$=0.75}
% %         \label{fig:p75}
% %     \end{subfigure}%
% %         \begin{subfigure}{.5\linewidth}
% %         \centering
% %         \includegraphics[width=\linewidth, height= 6.5 cm]{Images/p100}
% %         \caption{at $k_c^p$=1.0}
% %         \label{fig:p100}
% %     \end{subfigure}

% %     \caption{Currents (left vertical axis) in all six coil sides ($C_x^\pm$, $C_y^\pm$ and $C_z^\pm$) with drift $\Delta$B (right vertical axis) at sensor position '1x' for different values of $k_c^p$ with $k_c^i$ in Eq.~(\ref{eq:I}) being zero. Blue color curve denotes the actual drift in signal at position '1x' found by Eq.~(\ref{eq:del_B}) while the red curve denotes the drift that would have been without the compensation. The 'ON' and 'OFF' vertical dashed lines indicate the time of the perturbation coil being turned 'ON' and 'OFF' respectively. For position of coils and sensors see Fig.~\ref{fig:coil}. }
% %     \label{fig:p_pi}
% % \end{figure}
% \begin{figure}[!htb]
%     \begin{subfigure}{.5\linewidth}
%         \centering
%         \includegraphics[width=\linewidth, height= 6.5 cm]{Images/p25_33}
%         \caption{at $k_c^p$=0.25}
%         \label{fig:p25}
%     \end{subfigure}%
%     \begin{subfigure}{.5\linewidth}
%         \centering
%         \includegraphics[width=\linewidth, height= 6.5 cm]{Images/p50_33}
%         \caption{at $k_c^p$=0.50}
%         \label{fig:p50}
%     \end{subfigure}\\[1ex]
%     \begin{subfigure}{.5\linewidth}
%         \centering
%         \includegraphics[width=\linewidth, height= 6.5 cm]{Images/p75_33}
%         \caption{at $k_c^p$=0.75}
%         \label{fig:p75}
%     \end{subfigure}%
%         \begin{subfigure}{.5\linewidth}
%         \centering
%         \includegraphics[width=\linewidth, height= 6.5 cm]{Images/p100_33}
%         \caption{at $k_c^p$=1.0}
%         \label{fig:p100}
%     \end{subfigure}

%     \caption{Currents (left vertical axis) in all six coil sides ($C_x^\pm$, $C_y^\pm$ and $C_z^\pm$) with drift $\Delta$B (right vertical axis) at sensor position '1x' for different values of $k_c^p$ with $k_c^i$ in Eq.~(\ref{eq:I}) being zero. Blue color curve denotes the actual drift in signal at position '1x' found by Eq.~(\ref{eq:del_B}) while the red curve denotes the drift that would have been without the compensation. The 'ON' and 'OFF' vertical dashed lines indicate the time of the perturbation coil being turned 'ON' and 'OFF' respectively. For position of coils and sensors see Fig.~\ref{fig:coil}. }
%     \label{fig:p_pi}
% \end{figure}

% The effect of changing $k_c^p$ has been shown in Fig.~\ref{fig:p_pi} where the currents (left) that are being sent to the coils ($C_x^\pm$, $C_y^\pm$ and $C_z^\pm$) for drift $\Delta$B found by Eq.~(\ref{eq:del_B}) in sensor position '1x'.  It is seen that $\Delta$B=17.5 nT, 15.5 nT and 13.5 nT for $k_c^p$ = 0.25, 0.5 and 0.75 respectively (see Fig.~\ref{fig:p_pi}\textcolor{blue}{(a)}, Fig.~\ref{fig:p_pi}\textcolor{blue}{(b)}, Fig.~\ref{fig:p_pi}\textcolor{blue}{(c)}). That is, with the increase of $k_c^p$, $\Delta$B magnetic field decreases. But, it has a limit after which with the increase of $k_c^p$, the systems becomes unstable and starts oscillating which can be seen from Fig.~\ref{fig:p_pi}\textcolor{blue}{(d)}) where the currents (left) are oscillating and the drift itself also at $\Delta$B=12.5 nT (right). So, the error is reduced maximum by (20.5-12.5)/20.5 * 100$\%\approx$37$\%$ from the initial drift of $\Delta$B=20.5 nT denoted by the red curve at position '1x'. 

% \FloatBarrier
% The above results confirm that the difference between the setpoint and the actual measurements of the magnetic field can be reduced upto a certain point. So, only having the P term is no the solution for the prototype. Next, we will discuss about the effect of only I term.

% \subsubsection{Effect of changing only I term}
% Here, the effect of changing integral reset term (I) or $k_c^i$ of Eq.~(\ref{eq:I}) will be discussed.

% The error (the difference between setpoint and actual measurement) is accumulated for the length of measurements and I term is multiplying that accumulated error  with a constant gain. For the prototype it is

% \begin{equation}
%     I_{\text{PI}}=k_c^i \sum_n \Delta I_c^n
% \end{equation}
% where, $k_c^i$ is the integral gain and $\Delta I_c^n$ is explained in Eq.~(\ref{eq:del_I}).

% Accumulated error keep tracks of the offsets that should be corrected previously. I term takes care of the offset which are not corrected by the P term and thus accelerates reducing the error level. Depending on the value $k_c^i$, how fast the feedback loop will response to the drift in the signal will be determined. A large value of $k_c^p$ will result large faster response to reducing the error level and eventually it reaches a threshold point above which the actual measurement will overshoot i.e. exceed the setpoint. 
% % The main downfall of this is that the time required for the coil current to be settle in after reducing the error level may be very slow or never ever settle in.



% % As like the effect on P, the effect of changing I has been shown in Fig.~\ref{fig:i_pi} where the change in current in all six coil sides with  $\Delta$B on a particular sensor position have been observed for $k_c^i$=0.25, 0.5, 0.75 and 1.0 . It is seen that with increase of I the level of compensation of the magnetic field is almost similar but the main difference occurs on how fast the system response in an expense of increasing current in all the coil sides (see Fig.~\ref{fig:i_pi}\textcolor{blue}{(a)}, Fig.~\ref{fig:i_pi}\textcolor{blue}{(b)}, Fig.~\ref{fig:i_pi}\textcolor{blue}{(c)} and Fig.~\ref{fig:i_pi}\textcolor{blue}{(d)}). The main problem with changing only I term is that it creates a very slow current response time. But, in terms of compensation only changing I gives very good result. The slow current response can be minimized by decreasing the value of optimized 'r' (see Section \ref{sec:r_pi} and Section \ref{sec:r_currentResponse} ).

% % \begin{figure}[!htb]
% %     \begin{subfigure}{.5\linewidth}
% %         \centering
% %         \includegraphics[width=\linewidth, height= 6.5 cm]{Images/i25}
% %         \caption{at $k_c^i$=0.25}
% %         \label{fig:i25}
% %     \end{subfigure}%
% %     \begin{subfigure}{.5\linewidth}
% %         \centering
% %         \includegraphics[width=\linewidth, height= 6.5 cm]{Images/i50}
% %         \caption{at $k_c^i$=0.5}
% %         \label{fig:i50}
% %     \end{subfigure}\\[1ex]
% %     \begin{subfigure}{.5\linewidth}
% %         \centering
% %         \includegraphics[width=\linewidth, height= 6.5 cm]{Images/i75}
% %         \caption{at $k_c^i$=0.75}
% %         \label{fig:i75}
% %     \end{subfigure}%
% %         \begin{subfigure}{.5\linewidth}
% %         \centering
% %         \includegraphics[width=\linewidth, height= 6.5 cm]{Images/i100}
% %         \caption{at $k_c^i$=1.0}
% %         \label{fig:i100}
% %     \end{subfigure}

% %     \caption{Currents (left vertical axis) in all six coil sides ($C_x^\pm$, $C_y^\pm$ and $C_z^\pm$) with drift $\Delta$B (right vertical axis) at sensor position '1x' for different values of $k_c^i$ with $k_c^p$ ( see Eq.~(\ref{eq:I}) ) being zero. Blue color curve denotes the actual drift in signal at position '1x' found by Eq.~(\ref{eq:del_B}) while the red curve denotes the drift that would have been without the compensation. The 'ON' and 'OFF' vertical dashed lines indicate the time of the perturbation coil being turned 'ON' and 'OFF' respectively. For position of coils and sensors see Fig.~\ref{fig:coil}.}
% %     \label{fig:i_pi}
% % \end{figure}
% \begin{figure}[!htb]
%     \begin{subfigure}{.5\linewidth}
%         \centering
%         \includegraphics[width=\linewidth, height= 6.5 cm]{Images/i25_33}
%         \caption{at $k_c^i$=0.25}
%         \label{fig:i25}
%     \end{subfigure}%
%     \begin{subfigure}{.5\linewidth}
%         \centering
%         \includegraphics[width=\linewidth, height= 6.5 cm]{Images/i75_33}
%         \caption{at $k_c^i$=0.75}
%         \label{fig:i50}
%     \end{subfigure}\\[1ex]
%     \begin{subfigure}{.5\linewidth}
%         \centering
%         \includegraphics[width=\linewidth, height= 6.5 cm]{Images/i100_33}
%         \caption{at $k_c^i$=1.0}
%         \label{fig:i75}
%     \end{subfigure}%
%         \begin{subfigure}{.5\linewidth}
%         \centering
%         \includegraphics[width=\linewidth, height= 6.5 cm]{Images/i125_33}
%         \caption{at $k_c^i$=1.25}
%         \label{fig:i100}
%     \end{subfigure}

%     \caption{Currents (left vertical axis) in all six coil sides ($C_x^\pm$, $C_y^\pm$ and $C_z^\pm$) with drift $\Delta$B (right vertical axis) at sensor position '1x' for different values of $k_c^i$ with $k_c^p$ ( see Eq.~(\ref{eq:I}) ) being zero. Blue color curve denotes the actual drift in signal at position '1x' found by Eq.~(\ref{eq:del_B}) while the red curve denotes the drift that would have been without the compensation. The 'ON' and 'OFF' vertical dashed lines indicate the time of the perturbation coil being turned 'ON' and 'OFF' respectively. For position of coils and sensors see Fig.~\ref{fig:coil}.}
%     \label{fig:i_pi}
% \end{figure}

% \FloatBarrier
% The effect of changing $k_c^i$ has been shown in Fig.~\ref{fig:i_pi} where the currents (left) that are being sent to the coils ($C_x^\pm$, $C_y^\pm$ and $C_z^\pm$) for drift $\Delta$B found by Eq.~(\ref{eq:del_B}) in sensor position '1x'.  It is seen from Fig.~\ref{fig:i_pi}\textcolor{blue}{(a)}, Fig.~\ref{fig:i_pi}\textcolor{blue}{(b)}, Fig.~\ref{fig:i_pi}\textcolor{blue}{(c)} and Fig.~\ref{fig:i_pi}\textcolor{blue}{(d)} which are correspond to $k_c^i$ = 0.25, 0.75, 1.0 and 1.25 respectively that the error level (right) is $\sim$3.5 nT in every case, the coil currents(left) never settle in any of them. The figures are neither helpful to understand the system response time nor the overshoot effect in the $\Delta$B graph (right). So for understating those effects, the $\Delta$B graphs (right) have been zoomed in and shown in Fig.~\ref{fig:i_pi_zoom}. Now it is easily seen that the system tries to keep the error level within $\sim$ 3 nT of the setpoint which is at 0 nT as a low as 6s for $k_c^i$=0.25, then 2.2 s for $k_c^i$=0.5, 1.5s for $k_c^i$=0.75 and as fast as 0.45s for $k_c^i$=1.0 in Fig.~\ref{fig:i_pi_zoom}\textcolor{blue}{(a)}, Fig.~\ref{fig:i_pi_zoom}\textcolor{blue}{(b)}, Fig.~\ref{fig:i_pi_zoom}\textcolor{blue}{(c)} and Fig.~\ref{fig:i_pi_zoom}\textcolor{blue}{(d)} respectively. It is also seen from the Fig.~\ref{fig:i_pi_zoom}\textcolor{blue}{(d)} that there is an overshoot in the error level before it settles in. That is the error level is exceeding the target which is $\sim$3 nT of the setpoint and then it settles in.

% \begin{figure}[!htb]
%     \begin{subfigure}{.5\linewidth}
%         \centering
%         \includegraphics[width=\linewidth, height= 6.5 cm]{Images/i25_33_zoom.png}
%         \caption{at $k_c^i$=0.25}
%         \label{fig:i25zoom}
%     \end{subfigure}%
%     \begin{subfigure}{.5\linewidth}
%         \centering
%         \includegraphics[width=\linewidth, height= 6.5 cm]{Images/i75_33_zoom.png}
%         \caption{at $k_c^i$=0.75}
%         \label{fig:i75zoom}
%     \end{subfigure}\\[1ex]
%     \begin{subfigure}{.5\linewidth}
%         \centering
%         \includegraphics[width=\linewidth, height= 6.5 cm]{Images/i100_33_zoom.png}
%         \caption{at $k_c^i$=1.0}
%         \label{fig:i100zoom}
%     \end{subfigure}%
%         \begin{subfigure}{.5\linewidth}
%         \centering
%         \includegraphics[width=\linewidth, height= 6.5 cm]{Images/i125_33_zoom.png}
%         \caption{at $k_c^i$=1.25}
%         \label{fig:i125zoom}
%     \end{subfigure}

%     \caption{Zoomed in version of the drift $\Delta$B shown in right side of Fig.~\ref{fig:i_pi}\textcolor{blue}{(a)}, Fig.~\ref{fig:i_pi}\textcolor{blue}{(b)}, Fig.~\ref{fig:i_pi}\textcolor{blue}{(c)} and Fig.~\ref{fig:i_pi}\textcolor{blue}{(d)} respectively at sensor position '1x' for different values of $k_c^i$ with $k_c^p$ ( see Eq.~(\ref{eq:I}) ) being zero. The red vertical dashed line indicates the time of the perturbation coil being turned 'ON'. For position of coils and sensors see Fig.~\ref{fig:coil}.\label{fig:i_pi_zoom}}
% \end{figure}

% % \fig{Images/i_pi_zoom}{width = \textwidth,height =10cm}{Zoomed in version of the drift $\Delta$B shown in right side of Fig.~\ref{fig:i_pi}\textcolor{blue}{(a)}, Fig.~\ref{fig:i_pi}\textcolor{blue}{(b)}, Fig.~\ref{fig:i_pi}\textcolor{blue}{(c)} and Fig.~\ref{fig:i_pi}\textcolor{blue}{(d)} respectively at sensor position '1x' for different values of $k_c^i$ with $k_c^p$ ( see Eq.~(\ref{eq:I}) ) being zero. The red vertical dashed line indicates the time of the perturbation coil being turned 'ON'. For position of coils and sensors see Fig.~\ref{fig:coil}.\label{fig:i_pi_zoom}}


% \FloatBarrier
% The above results confirm that to get rid of the offsets that could not be reduce by the P term, an I term is a must. But using I term shows that the currents in the coils never settles in. Next the effect of applying both of them after tuning (see Section~\ref{sec:tune}) will be discussed and may be the currents settle there!!
% % \subsection{r vs. Condition No.}\label{sec:cond}
% % Instead of going through all the steps that are discussed in section \ref{sec:inv}, the concept of condition number of a matrix can be used. The condition number of $\bm{M}$ can be determined from the diagonal matrix $\bm{\Sigma}$ as given in eq.\ref{eq:m} by -
% %  \begin{equation}
% %      cond(\bm{M})=\frac{max(\sigma_d)}{min(\sigma_d)}
% %  \end{equation}
 

% \subsubsection{Effect of changing PI term Combinely}
% Finally, the Section~\ref{sec:pi_behave} will be ended here with the discussion of the effect of changing P and I term at a time which will complete the Eq.~(\ref{eq:I}).

% Here, first the P and I term have been tuned following the discussion on Section~\ref{sec:tune} which has generated $k_c^p$=0.43 and $k_c^i$=0.52 . The results by applying $k_c^p$ and $k_c^i$ as those tuned values are shown in Fig.~\ref{fig:tuned_vs_i}\textcolor{blue}{(a)}. For simplicity instead of showing all the drift $\Delta$B for all the fluxgate sensors for the positions given in the horizonatal axis of Fig.~\ref{fig:m}, only '1x' is shown on the right of the figure. And same as earlier the currents  that are being sent to the coils ($C_x^\pm$, $C_y^\pm$ and $C_z^\pm$) shown on the left of the same figure. But, we could not determine the effect of having both of them at a time. So, keeping $k_c^i$ as 0.52 and excluding P term i.e. $k_c^p$=0.0 we run the same measurement again and the results are shwon in Fig.~\ref{fig:tuned_vs_i}\textcolor{blue}{(b)}. As as matter of surprise, there is hardly any difference between the results in Fig.~\ref{fig:tuned_vs_i}\textcolor{blue}{(a)} and Fig.~\ref{fig:tuned_vs_i}\textcolor{blue}{(b)}. Why is that so ? For the moment, the Fig.~\ref{fig:tuned_vs_i} suggests that may be we don't need P term at all or maybe we need different tuning methods. So, applying P and I term at a time does not solve our original problem of unsettle current, rather it raises another question of the necessity of the P term or importance of the tuning method describe in Section~\ref{sec:tune}. Due to lack of time, we did not further go into other tuning methods. Rather we have tried to correct our original problem of unsettle current and also discover the differences in the work between Ref.~\cite{bea} and Ref.~\cite{rawlik}.
% \doublefig{Images/p43i52_33}{width =\textwidth,height =8cm}{at $k_c^p$=0.43 and $k_c^i$=0.52. \label{fig:pi_tuned}}{Images/i52_33}{width = \textwidth,height =8cm}{at $k_c^p$=0.0 and $k_c^i$=0.52..\label{fig:i52}}{{Currents (left vertical axis) in all six coil sides ($C_x^\pm$, $C_y^\pm$ and $C_z^\pm$) with drift $\Delta$B (right vertical axis) at sensor position '1x' for combine different values of $k_c^i$ and $k_c^p$ ( see Eq.~(\ref{eq:I}) ). Blue color curve denotes the actual drift in signal at position '1x' found by Eq.~(\ref{eq:del_B}), while the red curve denotes the drift that would have been without the compensation. The 'ON' and 'OFF' vertical dashed lines indicate the time of the perturbation coil being turned 'ON' and 'OFF' respectively. For position of coils and fluxgate sensor see Fig.~\ref{fig:coil}.} \label{fig:tuned_vs_i}}

% \FloatBarrier
% The above results rather clearing our original acquisition of unsttle coil currents, give us more confusion on the effectiveness of the P term and also the tuning method. Instead of loooking more deep into tuning method, we moved our focused into regularization parameter to settle coil currents that will be presented in upcoming Section.


% \section{New Studies on Regularization Parameter}\label{sec:new_study_r}
% In Section~\ref{sec:inv} we have introduced the regularization parameter 'r' and then discussed a simulation model in Section~\ref{sec:mont} which later has been justified by comparing with experimental setup in Section~\ref{fig:mont_comp}. We have also talked about the tuning method in Section~\ref{sec:tune} and later in Section~\ref{sec:pi_behave} we have shown the effect of the P and I term and a lot of questions arises there. Here, we will propose a new method to find 'r' based on condition number of matrix and finally we will will wrap up the Section with further tuning of optimized 'r' which will try to solve the current unsettle problems discussed in the earlier Section and 

% So, first new method to find 'r' will be discussed.

% \subsection{Regularization by Matrix Condition Number Method  }\label{sec:cond}
% Matrix Condition number ( see Eq.~(\ref{eq:cond} ) and regularization parameter 'r' ( see Eq.~(\ref{eq:minvR} ) have been introduced in Section.~\ref{sec:m} while discussing the inversion of the matrix $\bm{M}$ . Moreover, in Section~\ref{sec:mont}, a method of regularization by random fluctuation has been discussed. Here, we will propose another method of regularization using the concept of matrix condition number.
 
%  Recall from Section~\ref{sec:inv}, regularization is needed in the first place while inverse of the matrix $\bm{M}$ because $\bm{M}$ itself is ill-conditioned matrix. That means the $\bm{M}$ has a large condition number which while inverse would produce large currents in some ill-positioned places that will make the system unstable. So, it is required to have a well-conditioned  $\bm{M^{-1}}$ which implies that the condition number of $\bm{M^{-1}}$ should be small and that's what regularization has been doing. So, we introduce Eq.~\ref{eq:minvR} with various values of 'r' and each time the condition number of $\bm{M^{-1}}$ is stored. Then the optimized 'r'  has been determined by selecting the 'r' for which the condition number of $\bm{M^{-1}}$ is the minimum.
 
%  The condition number of $\bm{M^{-1}}$ for different values of 'r' has been shown in Fig.~\ref{fig:cond}. It is seen that for 'r'=0, the condition number of $\bm{M^{-1}}$=$\sim$40 that is same as the condition number of $\bm{M}$ itself. So, without regularization that is the condition number of pseudo-inverse of $\bm{M}$ would also give =$\sim$40. In regularization method, several 'r' is tried ( see Eq.~(\ref{eq:minvR}) ) and each time the condition number has been stored which are shown in the vertical axis. The red diamond symbol indicates that for 'r'=2.94, the condition number of $\bm{M^{-1}}$ is minimum and that is 3.1. That by using 'r'=2.94 in Eq.~(\ref{eq:minvR}), the condition number decrease from 40 to 3.1 which is 40/3.1$\approx$13 times of decrements. The Fig.\ref{fig:I-fluc} shows that the 'r'=2.87 compared to 'r'=2.94 that we found here. So, both method shows comparable result. This method will always produce fixed optimized 'r' for a particular  $\bm{M}$ but the method by random fluctuation (see Section~\ref{sec:mont}) will produce different optimize 'r' for different run as it because depends on the random field.

 
% \fig{Images/6c_Mcond}{width = \textwidth,height =10cm}{Condition Number of $\bm{M^{-1}}$ (vertical axis) for different values of 'r. The matrix is same as described by the Fig.~\ref{fig:m}.  \label{fig:cond}}

% \FloatBarrier

% In the above, different method to find optimize 'r' has been discussed which is a good alternative to the one explained in Section~\ref{sec:mont} and the results are similar. But it does not also solve the current unsettle problem discussed in Section~\ref{sec:pi_behave}. So, in the next tried again the several values of 'r' to see its effect on current unsettle problem. 


% % \subsubsection{Optimized r  Revisited Based on Current Response Time}\label{sec:r_currentResponse}
% % It was found that there is very slow coil current rise time while applying perturbation. To get rid of that problem, first and foremost, the fastest sampling frequency (see section [\ref{sec:filter}, \ref{sec:freq}]) is needed. Then, the next step of the problem can be solved via two ways with individual having own limitations. First way is tuning the value of P and I term of PI loop as explained in Eq.~(\ref{eq:I} and section \ref{sec:tune}. But with increasing the value of P, the current start oscillating after certain values as shown by the top and middle current graph on Fig.~\ref{fig:crnt} which is a problem. 

% % \fig{Images/crnt}{width = \textwidth}{Coil current in one of the coil side for optimized r=2.8 with P=0 and I=1.0 (top) and with P=0 and I=1.5 (middle) and for best r considering noise with P=0 and I=1.0 (bottom). \label{fig:crnt}}



% % The alternative way is to change the value of optimized r (see section [\ref{sec:mont}, \ref{sec:cond}]) which in turns increase noise in the prototype. But with inclusion of some current fluctuations, it was found that the coil current response time was increased heavily  as shown by the bottom current graph in Fig.~\ref{fig:crnt}. Now, the best compromised value of r was chosen by observing the 'rise time vs r' and 'fluctuations vs r' as shown in Fig.~\ref{fig:riseT}.



% %  \doublefig{Images/riseT}{width =\textwidth, height= 8 cm}{Rise Time vs r \label{fig:rise}}{Images/fluc}{width = \textwidth, height= 8 cm}{Fluctuations\label{fig:fluc}}{{(a) shows the Rise Time vs r (b) shows the Fluctuations } \label{fig:riseT}}
% % % \fig{Images/bt}{width = \textwidth}{Magnetic Field Compensation \label{fig:bt}}
% \subsection{r Behavior on PI Tuning}\label{sec:r_pi}
% % \begin{figure}[!htb]
% %     \begin{subfigure}{.5\linewidth}
% %         \centering
% %         \includegraphics[width=\linewidth, height= 5 cm]{Images/r16}
% %         \caption{at r=1.6}
% %         \label{fig:r16}
% %     \end{subfigure}%
% %     \begin{subfigure}{.5\linewidth}
% %         \centering
% %         \includegraphics[width=\linewidth, height= 5 cm]{Images/r18}
% %         \caption{at r=1.8}
% %         \label{fig:r18}
% %     \end{subfigure}\\[1ex]
% %     \begin{subfigure}{.5\linewidth}
% %         \centering
% %         \includegraphics[width=\linewidth, height= 5 cm]{Images/r20}
% %         \caption{at r=2.0}
% %         \label{fig:fExp}
% %     \end{subfigure}%
% %         \begin{subfigure}{.5\linewidth}
% %         \centering
% %         \includegraphics[width=\linewidth, height= 5 cm]{Images/r22}
% %         \caption{at r=2.2}
% %         \label{fig:r22}
% %     \end{subfigure}\\[1ex]
% %     \begin{subfigure}{.5\linewidth}
% %         \centering
% %         \includegraphics[width=\linewidth, height= 5 cm]{Images/r24}
% %         \caption{at r=2.4}
% %         \label{fig:r24}
% %     \end{subfigure}%
% %     \begin{subfigure}{.5\linewidth}
% %         \centering
% %         \includegraphics[width=\linewidth, height= 5 cm]{Images/r26}
% %         \caption{at r=2.6}
% %         \label{fig:r26}
% %     \end{subfigure}\\[1ex]
% %     \begin{subfigure}{.5\linewidth}
% %         \centering
% %         \includegraphics[width=\linewidth, height= 5 cm]{Images/r28}
% %         \caption{at r=2.8}
% %         \label{fig:r28}
% %     \end{subfigure}%
% %         \begin{subfigure}{.5\linewidth}
% %         \centering
% %         \includegraphics[width=\linewidth, height= 5 cm]{Images/r30}
% %         \caption{at r=23.0}
% %         \label{fig:r30}
% %     \end{subfigure}


% %     \caption{Change in the current in all six coil sides with obtained $\Delta$B on a particular sensor position. Here, the value of I=0.25,0.5,0.75 and 1.00 with P term being zero.}
% %     \label{fig:r_pi}
% % \end{figure}
% The discussion on Section \ref{sec:pi_behave} suggests that I term is necessary for fast system response due to any drift in the magnetic signal as it takes care of the offset problems which is unsolvable by using only P term. But, in doing so, it also creates problems in terms of the coil currents which never settle in for the duration of the perturbation. The tuning method described in Section~\ref{sec:tune} should have take care of this but we have realized that having tuned P and I has similar effect like having only I term. So, tuning method does not give us the solution. Rather looking for different tuning method , we have focused on the effect of 'r' on PI tuning. This Section will discuss that effect with possible outcome.

% The experimental setup is same as discussed in Section~\ref{sec:pi_behave}. But in this case we have chosen $k_c^p$=0 and $k_c^i$=0.52. Among those $k_c^i$=0.52 has been found due to PI tuning (see Section~\ref{sec:tune}) and instead of choosing  $k_c^p$=0.43 we have made this zero as from the earlier discussion we saw that it barely has any effect while we use the I term. So, these values of $k_c^p$ and $k_c^i$ will be applied on Eq.~\ref{eq:I} to find the currents to be sent to the coils ($C_x^\pm$, $C_y^\pm$ and $C_z^\pm$) for drift $\Delta$B found by Eq.~(\ref{eq:del_B}) in the sensor positions given in the horizontal axis of Fig.~\ref{fig:m}. So, keeping those fixed, we will try to change the value of 'r' which will modify the Eq.~(\ref{eq:minvR}) for each change of 'r' value.

% The effect of changing 'r' with $k_c^p$=0 and $k_c^i$=0.52 has been shown in Fig.~\ref{fig:r_pi} where the currents (left) that are being sent to the coils ($C_x^\pm$, $C_y^\pm$ and $C_z^\pm$) for drift $\Delta$B found by Eq.~(\ref{eq:del_B}) in sensor position '1x'.  It is seen from Fig.~\ref{fig:r_pi}\textcolor{blue}{(a)}, Fig.~\ref{fig:r_pi}\textcolor{blue}{(b)}, Fig.~\ref{fig:r_pi}\textcolor{blue}{(c)} and Fig.~\ref{fig:r_pi}\textcolor{blue}{(d)} which are correspond to 'r' = 2.0, 2.4, 2.8 and 3.2 respectively that the changing 'r' has significant effect on the coil current graph and barely any effect on the system response time for reducing the drift in the signal. That is at 'r'=2.0, the coil current graph has the fastest settling time where the current settles within 3 s after the perturbation has been applied. At 'r'=2.4, it takes 10s for the coil currents to settle in. But at 'r'=2.8, it seems like the coil current never settles in which is again improved at 'r'=3.2. Note that the here seroiusly ill conditioned matrix has been usedoptimized 'r' found by the simulation model is $\sim$2.9 which tells us that the coil settling of the current graph seems to have issue with the that optimized 'r. So, instead of taking the optimized 'r' that has been found by the simulation model we may have to choose the lower value of 'r'. Then question arises about what if 'r' value is chosen more than the optimized 'r'. For answering that question, we have also studied the effect for more values of 'r' with same setup which are shown in Fig.~\ref{fig:r_pi_more}. It is seen from Fig.~\ref{fig:r_pi_more}\textcolor{blue}{(a)}, Fig.~\ref{fig:r_pi_more}\textcolor{blue}{(b)}, Fig.~\ref{fig:r_pi_more}\textcolor{blue}{(c)} and Fig.~\ref{fig:r_pi_more}\textcolor{blue}{(d)} which are correspond to 'r' = 3.5, 3.6, 3.7 and 3.9 respectively that the coil current graph seems to be settle in for larger value of 'r' before it starts showing less responsive for example at 'r'=3.9 .   





% with the increase of 'r' although the error level (right) is $\sim$3 nT in every case, the coil currents(left) never settle in any of them. The figures are neither helpful to understand the system response time nor the overshoot effect in the $\Delta$B graph (right). So for understating those effects, the $\Delta$B graphs (right) have been zoomed in and shown in Fig.~\ref{fig:i_pi_zoom}. Now it is easily seen that the system tries to keep the error level within $\sim$ 3 nT of the setpoint which is at 0 nT as a low as 6s for $k_c^i$=0.25, then 2.2 s for $k_c^i$=0.5, 1.5s for $k_c^i$=0.75 and as fast as 0.45s for $k_c^i$=1.0 in Fig.~\ref{fig:i_pi_zoom}\textcolor{blue}{(a)}, Fig.~\ref{fig:i_pi_zoom}\textcolor{blue}{(b)}, Fig.~\ref{fig:i_pi_zoom}\textcolor{blue}{(c)} and Fig.~\ref{fig:i_pi_zoom}\textcolor{blue}{(d)} respectively. It is also seen from the Fig.~\ref{fig:i_pi_zoom}\textcolor{blue}{(d)} that there is an overshoot in the error level before it settles in. That is the error level is exceeding the target which is $\sim$3 nT of the setpoint and then it settles in.


% and fast response and that also causes the current in the coil sides being higher with very slow current response time. To minimize that in addition to normal tuning of PI, the effect of 'r' on PI tuning has been also studied as shown in Fig.~\ref{fig:r_pi} where the change in current in all six coil sides with  $\Delta$B on a particular sensor position have been observed for P=0.45, I=0.27 and r=1.8, 2.2, 2.6 and 3.0. It is seen that with increase of 'r' having same P and I term, the response on the current decreases (see Fig.~\ref{fig:r20}, Fig.~\ref{fig:r24}, Fig.~\ref{fig:r28} and Fig.~\ref{fig:r32}). So, the tuned system can be more tuned up by changing 'r' (more on Section \ref{sec:r_pi}). This is an unique finding as it suggests to alternative option of tuning.

% \begin{figure}[!htb]
%     \begin{subfigure}{.5\linewidth}
%         \centering
%         \includegraphics[width=\linewidth, height= 6.5 cm]{Images/r20}
%         \caption{at r=2.0}
%         \label{fig:r20}
%     \end{subfigure}%
%     \begin{subfigure}{.5\linewidth}
%         \centering
%         \includegraphics[width=\linewidth, height= 6.5 cm]{Images/r24}
%         \caption{at r=2.4}
%         \label{fig:r24}
%     \end{subfigure}\\[1ex]
%     \begin{subfigure}{.5\linewidth}
%         \centering
%         \includegraphics[width=\linewidth, height= 6.5 cm]{Images/r28}
%         \caption{at r=2.8}
%         \label{fig:r28}
%     \end{subfigure}%
%         \begin{subfigure}{.5\linewidth}
%         \centering
%         \includegraphics[width=\linewidth, height= 6.5 cm]{Images/r32}
%         \caption{at r=3.2}
%         \label{fig:r32}
%     \end{subfigure}


%     \caption{Change in the current in all six coil sides with obtained $\Delta$B on a particular sensor position with red represents uncompensated $\Delta$B. Here, the value of P=0.45, I=0.27 and r=1.8, 2.2, 2.6 and 3.0}
%     \label{fig:r_pi}
% \end{figure}
% \FloatBarrier

% \begin{figure}[!htb]
%     \begin{subfigure}{.5\linewidth}
%         \centering
%         \includegraphics[width=\linewidth, height= 6.5 cm]{Images/r35}
%         \caption{at r=3.5}
%         \label{fig:r35}
%     \end{subfigure}%
%     \begin{subfigure}{.5\linewidth}
%         \centering
%         \includegraphics[width=\linewidth, height= 6.5 cm]{Images/r36}
%         \caption{at r=3.6}
%         \label{fig:r36}
%     \end{subfigure}\\[1ex]
%     \begin{subfigure}{.5\linewidth}
%         \centering
%         \includegraphics[width=\linewidth, height= 6.5 cm]{Images/r37}
%         \caption{at r=3.7}
%         \label{fig:r37}
%     \end{subfigure}%
%         \begin{subfigure}{.5\linewidth}
%         \centering
%         \includegraphics[width=\linewidth, height= 6.5 cm]{Images/r39}
%         \caption{at r=3.9}
%         \label{fig:r39}
%     \end{subfigure}


%     \caption{Change in the current in all six coil sides with obtained $\Delta$B on a particular sensor position with red represents uncompensated $\Delta$B. Here, the value of P=0.45, I=0.27 and r=1.8, 2.2, 2.6 and 3.0}
%     \label{fig:r_pi_more}
% \end{figure}
% \FloatBarrier



%% ----------------------------------------------------------------
% Now begin the Appendices, including them as separate files

\addtocontents{toc}{\vspace{2em}} % Add a gap in the Contents, for aesthetics

\appendix % Cue to tell LaTeX that the following 'chapters' are Appendices

%  \chapter{Fluxgate Magnetometer Basics \& Applications}
\lhead{\emph{Fluxgate Magnetometer Basics \& Applications}}
\section{Basics}
Fluxgate sensors are solid-state devices used for measuring the magnitude and direction of dc or low-frequency ac magnetic fields. Fluxgate sensors come in several configurations, the ring core type being the most commonly used. The fuxgate sensor used for magnetic field measurements in
the active shielding system developed for this thesis is a racetrack  fluxgate, which is essentially a modification of the ring core type sensor. Figure A.1 shows the main components of a ring core  fluxgate sensor. The central ring is typically made from an alloy of high magnetic permeability \(\mu\), wrapped in two sets of coil windings. The first set of windings, which are wrapped directly around the ring, are known as the drive windings or the excitation coil. The second set of windings are wrapped around the entire ring forming a pick-up, or sense coil. An ac current is driven into the excitation coil inducing a magnetic  flux \(\phi\) within the core. This field from the excitation coil causes the core to become magnetically saturated. If the core is thought of as two separate halves, then a the field generated in one half \((\phi_1)\) will be in the opposite direction as a the field generated in the other half \((\phi_2)\). If no external magnetic field is present, the two half cores go in and out of saturation simultaneously. The fields generated by the two half cores cancel, and no net change in  flux is measured by the pick-up coil. If an external field B0 is present, it will cause the saturation in each half of the core to vary relative to one another. The fields produced by the half cores no longer cancel and the pick-up coil is subjected to a net  flux. The result is an induced voltage in the pick-up coil, which provides a measure of the external field.



The sensor can be placed in a feedback loop (as shown in Figure A.2) in which another coil,known as the compensation or feedback coil, is used to drive the external field to zero resulting in no voltage output from the pick-up coil. The feedback loop maintains the zero voltage output from the pick-up coil by adjusting the current in the feedback coil. The external field measurement is then based on the current driven into the feedback coil. This is known as a feedback magnetometer, and the benefit is an increase in the dynamic range of the  fluxgate field measurement \cite{flux1}.

\section{Applications}
\begin{enumerate}
   \item Airborne Magnetics.
   \begin{itemize}
     \item 
   \end{itemize}
   \item Archaeology.
   \begin{itemize}
     \item 
   \end{itemize}
   \item Forensics.
   \begin{itemize}
     \item 
   \end{itemize}
   \item Exploration.
   \begin{itemize}
     \item 
   \end{itemize}
   \item Geosciences.
   \begin{itemize}
     \item 
   \end{itemize}
   \item Industrial.
   \begin{itemize}
     \item 
   \end{itemize}
   \item Medical.
   \begin{itemize}
     \item 
   \end{itemize}
   \item Near Surface Geophysics.
   \begin{itemize}
     \item 
   \end{itemize}
   \item Physics.
   \begin{itemize}
     \item 
   \end{itemize}
\end{enumerate}




\section{Use of Filter}
For prototype active magnetic field compensation system, it is required to use low pass filter (LPF) to ignore high frequency noises coming out from fluxgate sensors.

\begin{figure*}[t]
    \centering
    \captionsetup{justification=centering,margin=1cm}
    \includegraphics[height=3in, width=6in]{Figures/flux_r12.png}
    % \includegraphics[width=.80\textwidth]{socialDiscovery.eps}
    \caption{B field vs T for sample time 1.91 s\label{fig: fr12_1,2}}
\end{figure*}
For example, compensation can be done using 21.8 bits resolution. In that case, we don't need any filter as shown in the figure \ref{fig: fr12_1,2}.

But, for increasing the response of the feedback system, ADC resolution has to decrease which in turns introduce high frequency noise in the system that can be removed using LPF filter as shown in fig. \ref{fig: fr1_1,2} \& \ref{fig: fr8_1,2}. Thus , the sample time is vastly decreased by 96 times for fig. \ref{fig: fr8_1,2} and even more 191 times for fig. \ref{fig: fr1_1,2}.
\begin{figure*}[t]
    \centering
    \captionsetup{justification=centering,margin=1cm}
    \includegraphics[height=3in, width=6in]{Figures/flux_r8.png}
    % \includegraphics[width=.80\textwidth]{socialDiscovery.eps}
    \caption{B field vs T for sample time 0.02 s\label{fig: fr8_1,2}}
\end{figure*}
\begin{figure*}[t]
    \centering
    \captionsetup{justification=centering,margin=1cm}
    \includegraphics[height=3in, width=6in]{Figures/flux_r1.png}
    % \includegraphics[width=.80\textwidth]{socialDiscovery.eps}
    \caption{B field vs T for sample time 0.01 s\label{fig: fr1_1,2}}
\end{figure*}	% Appendix Title
   
\chapter{Magnetic field inside the bulk of a mu-metal shield}
\lhead{\emph{Magnetic field inside the bulk of a mu-metal shield}}
% \newpage
% \label{chap:appendix_bustamante}
% \section{Magnetic field inside the bulk of a mu-metal shield}
\section{Analytic solution assuming linear permeability}
\subsection{General solution for an applied zonal field}
\subsubsection{Using Scalar Potential}
% \lhead{\emph{Magnetic field inside the bulk of a mu-metal shield}}

% \subsubsection{Using Scalar Potential}
Let us, place a spherical shell of thickness, t=b-a, which is made of material of permeability $\mu_r$, in a uniform magnetic field, \(\bm B_{\rm ext}=\bm{B_o}= \muo \bm H_o\) \cite{jackson}.

We know, \(\bm{\nabla}\times\bm{H}=\bm{J}\). But, we are considering no free current. So, \(\bm{J}_f=\bm{K}_f=0\). Thus, we get , \(\bm{\nabla}\times\bm{H}=0\) everywhere which implies that there exists a magnetic scalar potential \(\Phi_m\) that is everywhere continuous and
\begin{equation}\label{H}
\bm{H}=-\bm{\nabla}\Phi_m
\end{equation} 
Again, \(\bm{B}=\mu_r\bm{H}\) and \(\bm{\nabla}\cdot\bm{B}=0\). So, we get,
\begin{equation}\label{nablaH}
\bm{\nabla}\cdot\bm{H}=0
\end{equation}
Therefore, using eq. [\ref{H},\ref{nablaH}] we get,
\begin{equation}
\bm{\nabla}\cdot\bm{H}=-\bm{\nabla}\Phi_m^2=0
\end{equation}
So, $\Phi_m$ satisfies the Laplace equation and moreover, there exists azimuthal symmetry (i.e. m=0). Therefore, we can get,
\begin{equation}
\Phi(r,\theta)=\sum_{l=0}^{\infty} [A_lr^l+B_lr^{-(l+1)}]P_l(\cos\theta)
\end{equation}

For r\textgreater b we get,
\begin{equation}\label{r>b}
\Phi_m=-Hr^lP_l(\cos\theta)+\sum_{l=0}^{\infty}\frac{\alpha}{r^{l+1}}P_l(\cos\theta)
\end{equation}

For a\textless r\textless b we get,
\begin{equation}\label{a<r<b}
\Phi(r,\theta)=\sum_{l=0}^{\infty} \left(\beta r^l+\frac{\gamma}{r^{l+1}}\right)P_l(\cos\theta)
\end{equation}

And lastly, for r\textless a we get,
\begin{equation}\label{r<a}
\Phi(r,\theta)=\sum_{l=0}^{\infty} \delta r^l P_l(\cos\theta)
\end{equation}

So, the problem reduces to finding the proper solutions in the different regions to satisfy the following boundary conditions at r=a and r=b,


\begin{equation}\label{b1}   
\bm{H_2}\times n = \bm{H_1}\times n  
\end{equation}

\begin{equation}\label{b2}   
\bm{H_2}\cdot n=\frac{\mu_{r1}}{\mu_{r2}}\bm{H_1}\cdot n
\end{equation}

$$(\bm{B_2} - \bm{B_1})\times n = \muo \bm{K_b}$$
That's,
\begin{equation}\label{b3}   
(\mu_2\bm{H_2} - \mu_1\bm{H_1})\times n = \muo \bm{K_b}  
\end{equation}

Now using Eq.[\ref{H}] and boundary condition from Eq.[\ref{b1}] we get,
$$\frac{\delta \Phi_3}{\delta \theta}(b)=\frac{\delta \Phi_2}{\delta \theta}(b)$$

Using Eq.[\ref{r>b}, \ref{a<r<b}]
$$-Hb^l\frac{\delta}{\delta \theta}(P_l(\cos\theta))+\frac{\alpha}{r^{l+1}}\frac{\delta}{\delta \theta}(P_l(\cos\theta))=\left(\beta b^l+\frac{\gamma}{b^{l+1}}\right)\frac{\delta}{\delta \theta}(P_l(\cos\theta))$$
\begin{equation}\label{e01}
-Hb^l+\frac{\alpha}{b^{l+1}}=\beta b^l+\frac{\gamma}{b^{l+1}}
\end{equation}

Similarly,
$$\frac{\delta \Phi_2}{\delta \theta}(a)=\frac{\delta \Phi_1}{\delta \theta}(a)$$
Using Eq.[\ref{a<r<b}, \ref{r<a}]
$$\left(\beta a^l+\frac{\gamma}{a^{l+1}}\right)\frac{\delta}{\delta \theta}(P_l(\cos\theta))=\delta a^l \frac{\delta}{\delta \theta}(P_l(\cos\theta))$$
\begin{equation}\label{e02}
\beta a^l+\frac{\gamma}{a^{l+1}}=\delta a^l
\end{equation}

Similarly, using Eq.[\ref{H}] and another boundary condition from Eq.[\ref{b2}] we get,
$$\muo\frac{\delta \Phi_3}{\delta r}(b)=\mu_r\frac{\delta \Phi_2}{\delta r}(b)$$
Using Eq.[\ref{r>b}, \ref{a<r<b}]
$$\muo\left[-Hlb^{l-1}+\frac{-(l+1)\alpha}{b^{l+2}}\right]P_l(\cos\theta)=\mu_r\left[\beta l b^{l-1}+\frac{-(l+1)\gamma}{b^{l+2}}\right]P_l(\cos\theta)$$
\begin{equation}\label{e03}
-Hlb^{l-1}-\frac{(l+1)\alpha}{b^{l+2}}=\frac{\mu_r}{\muo}\left[\beta l b^{l-1}-\frac{(l+1)\gamma}{b^{l+2}}\right]
\end{equation}
Again,
$$\mu_r\frac{\delta \Phi_2}{\delta r}(a)=\muo\frac{\delta \Phi_1}{\delta r}(a)$$
Using Eq.[\ref{a<r<b}, \ref{r<a}]
$$\frac{\mu_r}{\muo}\left[\beta l a^{l-1}+\frac{-(l+1)\gamma}{a^{l+2}}\right]P_l(\cos\theta)=\delta l a^{l-1} P_l(\cos\theta)$$
\begin{equation}\label{e04}
\delta l a^{l-1}=\mu_r'\left[\beta l a^{l-1}-\frac{(l+1)\gamma}{a^{l+2}}\right]
\end{equation}

From eq. [\ref{e01}] we get,
$$-Hb^l+\frac{\alpha}{b^{l+1}}=\beta b^l+\frac{\gamma}{b^{l+1}}$$
$$\frac{-Hb^{2l+1}+\alpha}{b^{l+1}}=\frac{\beta b^{2l+1}+\gamma}{b^{l+1}}$$
$$-Hb^{2l+1}+\alpha=\beta b^{2l+1}+\gamma$$
\begin{equation}
\alpha- b^{2l+1}\beta-\gamma=b^{2l+1}H
\end{equation}

From eq.[\ref{e02}] we get,
$$\beta a^l+\frac{\gamma}{a^{l+1}}=\delta a^l$$
$$\frac{\beta a^{2l+1}+\gamma}{a^{l+1}}=\delta a^l$$
\begin{equation}
 a^{2l+1}\beta+\gamma-a^{2l+1}\delta =0
\end{equation}

From eq. [\ref{e03}] we get,
$$-Hlb^{l-1}-\frac{(l+1)\alpha}{b^{l+2}}=\mu_r'\left[\beta l b^{l-1}-\frac{(l+1)\gamma}{b^{l+2}}\right]$$
$$\frac{-Hlb^{2l+1}-(l+1)\alpha}{b^{l+2}}=\mu_r'\left[\frac{\beta l b^{2l+1}-(l+1)\gamma}{b^{l+2}}\right]$$
\begin{equation}
(l+1)\alpha+\mu_r' l b^{2l+1}\beta-\mu_r'(l+1)\gamma=-lb^{2l+1}H
\end{equation}

From eq.[\ref{e04}] we get,
$$\delta l a^{l-1}=\mu_r'\left[\beta l a^{l-1}-\frac{(l+1)\gamma}{a^{l+2}}\right]$$
$$\delta l a^{l-1}=\mu_r'\left[\frac{\beta l a^{2l+1}-(l+1)\gamma}{a^{l+2}}\right]$$
$$\delta l a^{2l+1}=\mu_r'\beta l a^{2l+1}-\mu_r'(l+1)\gamma$$
\begin{equation}
\mu_r'l a^{2l+1}\beta -\mu_r'(l+1)\gamma- l a^{2l+1}\delta=0
\end{equation}

So we get,
\begin{equation}\label{e1}
\alpha- b^{2l+1}\beta-\gamma=b^{2l+1}H
\end{equation}
\begin{equation}\label{e2}
 a^{2l+1}\beta+\gamma-a^{2l+1}\delta =0
\end{equation}
\begin{equation}\label{e3}
(l+1)\alpha+\mu_r' l b^{2l+1}\beta-\mu_r'(l+1)\gamma=-lb^{2l+1}H
\end{equation}
\begin{equation}\label{e4}
\mu_r'l a^{2l+1}\beta -\mu_r'(l+1)\gamma- l a^{2l+1}\delta=0
\end{equation}

Subtracting eq. [\ref{e2}] from eq.[\ref{e4}]/$l$ we get,
$$\mu_r' a^{2l+1}\beta -\left(\frac{l+1}{l}\right)\mu_r'\gamma- a^{2l+1}\delta-a^{2l+1}\beta-\gamma+a^{2l+1}\delta=0$$
$$ \beta a^{2l+1}(\mu_r'-1) -\gamma\left[\left(\frac{l+1}{l}\right)\mu_r'+1\right]=0$$
So we get,
\begin{equation}\label{g1}
\gamma=\beta \left[\frac{a^{2l+1}(\mu_r'-1)}{\left(\frac{l+1}{l}\right)\mu_r'+1}\right]
\end{equation}

Again, subtracting eq. [\ref{e1}]$*(l+1)$ from eq.[\ref{e3}] we get,
$$(l+1)\alpha+\mu_r' l b^{2l+1}\beta-\mu_r'(l+1)\gamma+lb^{2l+1}H-(l+1)\alpha+(l+1) b^{2l+1}\beta+(l+1)\gamma+(l+1)b^{2l+1}H$$
$$(\mu_r'l+l+1) b^{2l+1}\beta+(1-\mu_r')(l+1)\gamma=-(l+1+l)b^{2l+1}H$$
Putting the value of eq.[\ref{g1}] we get,
$$(\mu_r'l+l+1) b^{2l+1}\beta+(l+1)a^{2l+1}\beta \left[\frac{(1-\mu_r')(\mu_r'-1)}{\left(\frac{l+1}{l}\right)\mu_r'+1}\right]=-(2l+1)b^{2l+1}H$$
$$\beta(\mu_r'l+l+1) -\beta(l+1)\left(\frac{a}{b}\right)^{2l+1} \left[\frac{(\mu_r'-1)^2}{\left(\frac{l+1}{l}\right)\mu_r'+1}\right]=-(2l+1)H$$
$$\beta\left[(\mu_r'l+l+1) -(l+1)\left(\frac{a}{b}\right)^{2l+1} \left[\frac{(\mu_r'-1)^2}{\left(\frac{l+1}{l}\right)\mu_r'+1}\right]\right]=-(2l+1)H$$
$$\beta\left[\frac{(\mu_r'l+l+1)\left[\left(\frac{l+1}{l}\right)\mu_r'+1\right]-(l+1)\left(\frac{a}{b}\right)^{2l+1}(\mu_r'-1)^2}{\left(\frac{l+1}{l}\right)\mu_r'+1}\right]=-(2l+1)H$$
So we get,
\begin{equation}
\beta=-\left[\frac{(2l+1)\left(\left(\frac{l+1}{l}\right)\mu_r'+1\right)}{(\mu_r'l+l+1)\left[\left(\frac{l+1}{l}\right)\mu_r'+1\right]-(l+1)\left(\frac{a}{b}\right)^{2l+1}(\mu_r'-1)^2}\right]H
\end{equation}

For \(\mu_r\gg\muo \Rightarrow \mu_r/\muo\gg 1 \Rightarrow \mu_r'\gg 1\) we get,
$$\beta\approx-\left[\frac{\left[\frac{(2l+1)(l+1)}{l}\right]\mu_r'}{(l+1)\mu_r'^2-(l+1)\left(\frac{a}{b}\right)^{2l+1}\mu_r'^2}\right]H$$
$$\beta\approx-\left[\frac{(2l+1)(l+1)\mu_r'}{l(l+1)\mu_r'^2\left[1-\left(\frac{a}{b}\right)^{2l+1}}\right]\right]H$$
So we get,
\begin{equation}\label{bt2}
\beta\approx-\left[\frac{(2l+1)\muo}{l\mu_r\left[1-\left(\frac{a}{b}\right)^{2l+1}}\right]\right]H
\end{equation}

Putting the value of Eq.[\ref{b1}] in Eq.[\ref{g1}] we get,
$$\gamma=-\left[\frac{(2l+1)\left(\left(\frac{l+1}{l}\right)\mu_r'+1\right)}{(\mu_r'l+l+1)\left[\left(\frac{l+1}{l}\right)\mu_r'+1\right]-(l+1)\left(\frac{a}{b}\right)^{2l+1}(\mu_r'-1)^2}\right] \left[\frac{a^{2l+1}(\mu_r'-1)}{\left(\frac{l+1}{l}\right)\mu_r'+1}\right]H$$
\begin{equation}
\gamma=-\left[\frac{(2l+1)a^{2l+1}(\mu_r'-1)}{(\mu_r'l+l+1)\left[\left(\frac{l+1}{l}\right)\mu_r'+1\right]-(l+1)\left(\frac{a}{b}\right)^{2l+1}(\mu_r'-1)^2}\right] H    
\end{equation}


So, for \(\mu_r'\gg 1\) we get,
$$\gamma\approx-\left[\frac{(2l+1)a^{2l+1}\mu_r'}{\mu_r'^2(l+1)-(l+1)\left(\frac{a}{b}\right)^{2l+1}\mu_r'^2}\right] H$$
$$\gamma\approx-\left[\frac{(2l+1)a^{2l+1}\mu_r'}{\mu_r'^2(l+1)\left(1-\left(\frac{a}{b}\right)^{2l+1}\right)}\right] H$$
So, we get,
\begin{equation}\label{g2}
\gamma\approx-\left[\frac{(2l+1)a^{2l+1}\muo}{\mu_r(l+1)\left(1-\left(\frac{a}{b}\right)^{2l+1}\right)}\right] H
\end{equation}
From Eq. [\ref{e2}] we get,
$$\delta =\beta+\frac{\gamma}{a^{2l+1}}$$
$$\delta =-\left[\frac{1}{(\mu_r'l+l+1)\left[\left(\frac{l+1}{l}\right)\mu_r'+1\right]-(l+1)\left(\frac{a}{b}\right)^{2l+1}(\mu_r'-1)^2}\right]$$ $$\left[(2l+1)\left(\left(\frac{l+1}{l}\right)\mu_r'+1\right)+(2l+1)(\mu_r'-1)\right]H$$
$$\delta =-\left[\frac{(2l+1)\left[\left(\frac{l+1}{l}\right)\mu_r'+1+\mu_r'-1\right]}{(\mu_r'l+l+1)\left[\left(\frac{l+1}{l}\right)\mu_r'+1\right]-(l+1)\left(\frac{a}{b}\right)^{2l+1}(\mu_r'-1)^2}\right]H$$
$$\delta =-\left[\frac{(2l+1)\left[\mu_r'\left(\frac{l+1}{l}+1\right)\right]}{(\mu_r'l+l+1)\left[\left(\frac{l+1}{l}\right)\mu_r'+1\right]-(l+1)\left(\frac{a}{b}\right)^{2l+1}(\mu_r'-1)^2}\right]H$$
So we get,
\begin{equation}
\delta =-\left[\frac{\frac{(2l+1)^2}{l}\mu_r'}{(\mu_r'l+l+1)\left[\left(\frac{l+1}{l}\right)\mu_r'+1\right]-(l+1)\left(\frac{a}{b}\right)^{2l+1}(\mu_r'-1)^2}\right]H    
\end{equation}


Therefore,

\begin{equation}\label{bt1}
\beta=-\left[\frac{(2l+1)\left(\left(\frac{l+1}{l}\right)\mu_r'+1\right)}{(\mu_r'l+l+1)\left[\left(\frac{l+1}{l}\right)\mu_r'+1\right]-(l+1)\left(\frac{a}{b}\right)^{2l+1}(\mu_r'-1)^2}\right]H
\end{equation}
\begin{equation}\label{g1}
\gamma=-\left[\frac{(2l+1)a^{2l+1}(\mu_r'-1)}{(\mu_r'l+l+1)\left[\left(\frac{l+1}{l}\right)\mu_r'+1\right]-(l+1)\left(\frac{a}{b}\right)^{2l+1}(\mu_r'-1)^2}\right] H    
\end{equation}
\begin{equation}\label{dt1}
\delta =-\left[\frac{\frac{(2l+1)^2}{l}\mu_r'}{(\mu_r'l+l+1)\left[\left(\frac{l+1}{l}\right)\mu_r'+1\right]-(l+1)\left(\frac{a}{b}\right)^{2l+1}(\mu_r'-1)^2}\right]H    
\end{equation}
Now using Eq.[\ref{H}] and boundary condition from Eq.[\ref{b3}] we get,
$$-\mu_r \frac{1}{r}\frac{\delta \Phi_2}{\delta \theta}(a)+\muo \frac{1}{r}\frac{\delta \Phi_1}{\delta \theta}(a)=\muo\bm{K_1}  $$
$$\frac{1}{r}\frac{\delta \Phi_1}{\delta \theta}(a)-\frac{1}{r}\mu'_r\frac{\delta \Phi_2}{\delta \theta}(a)=\bm{K_1}  $$
Using Eq.[\ref{a<r<b}, \ref{r<a}]
$$\frac{1}{a}\left[\delta a^l-\mu'_r\left(\beta a^l+\frac{\gamma}{a^{l+1}}\right)\right]\frac{\delta}{\delta \theta}(P_l(\cos\theta))= \bm{K_1}$$
$$\frac{1}{a}\left[\delta a^l-\mu'_r\left(\beta a^l+\frac{\gamma}{a^{l+1}}\right)\right][-P_l^1(u)]= K_1 P_l^1(u)$$
$$a K_1=\mu'_r\left(\beta a^l+\frac{\gamma}{a^{l+1}}\right)\right]-\delta a^l$$
$$\frac{(2l+1)a^{l-1+1}\mathcal{K}_1}{\muo}=\mu'_r\left(\beta a^l+\frac{\gamma}{a^{l+1}}\right)\right]-\delta a^l$$
$$\frac{(2l+1)a^l\mathcal{K}_1}{\muo}=a^l\left[\mu'_r\left(\beta +\frac{\gamma}{a^{2l+1}}\right)-\delta \right]$$
Using Eq. [\ref{bt1}, \ref{g1}, \ref{dt1}] we get,
$$\frac{(2l+1)\mathcal{K}_1}{\muo}=-\left[\frac{2l+1}{(\mu_r'l+l+1)\left[\left(\frac{l+1}{l}\right)\mu_r'+1\right]-(l+1)\left(\frac{a}{b}\right)^{2l+1}(\mu_r'-1)^2}\right]$$ $$\left[\mu'_r\left(\left(\frac{l+1}{l}\right)\mu_r'+1+\mu_r'-1\right)-\frac{(2l+1)}{l}\mu_r'}\right]H$$
$$\frac{\mathcal{K}_1}{\muo}=-\left[\frac{\mu'_r\left(\left(\frac{l+1}{l}\right)\mu_r'+\mu_r'-\frac{2l+1}{l}\right)}{(\mu_r'l+l+1)\left[\left(\frac{l+1}{l}\right)\mu_r'+1\right]-(l+1)\left(\frac{a}{b}\right)^{2l+1}(\mu_r'-1)^2}\right]H$$
$$\frac{\mathcal{K}_1}{\muo}=-\left[\frac{\mu'_r\left(\frac{\mu_r'l+\mu_r'+\mu_r'l-2l-1}{l}\right)}{(\mu_r'l+l+1)\left[\left(\frac{l+1}{l}\right)\mu_r'+1\right]-(l+1)\left(\frac{a}{b}\right)^{2l+1}(\mu_r'-1)^2}\right]H$$
$$\frac{\mathcal{K}_1}{\muo}=-\frac{1}{l}\left[\frac{\mu'_r\left(\mu_r'(2l+1)-1(2l+1)\right)}{(\mu_r'l+l+1)\left[\left(\frac{l+1}{l}\right)\mu_r'+1\right]-(l+1)\left(\frac{a}{b}\right)^{2l+1}(\mu_r'-1)^2}\right]H$$
So we get,
\begin{equation}\label{k1s}
\mathcal{K}_1=-\frac{\muo H}{l}\left[\frac{(2l+1)(\mu_r'-1)\mu'_r}{(\mu_r'l+l+1)\left[\left(\frac{l+1}{l}\right)\mu_r'+1\right]-(l+1)\left(\frac{a}{b}\right)^{2l+1}(\mu_r'-1)^2}\right]
\end{equation}
From bound surface currents method we got,
$$\mathcal{K}_1=-\frac{G_l(2l+1)(l+1)(\mu_r-\muo)\mu_r}{[\mu_r(l+1)+\muo l)][\mu_r l+\muo(l+1)]-l(l+1)\left(\dfrac{a}{b}\right)^{2l+1}(\mu_r-\muo)^2}$$
$$\mathcal{K}_1=-\frac{G_l(2l+1)(l+1)(\mu_r'-1)\muo\mu_r}{[\muo(\mu_r' l+\mu_r'+ l)\muo(\mu_r' l+l+1)]-l(l+1)\left(\dfrac{a}{b}\right)^{2l+1}\muo^2(\mu_r'-1)^2}$$
$$\mathcal{K}_1=-\frac{G_l(2l+1)(l+1)(\mu_r'-1)\mu_r'}{[(\mu_r' l+\mu_r'+ l)l(\mu_r'( 1+\frac{1}{l})+1)]-l(l+1)\left(\dfrac{a}{b}\right)^{2l+1}(\mu_r'-1)^2}$$
So we get,
\begin{equation}\label{k1b}
\mathcal{K}_1=-\frac{G_l (l+1)}{l}\left[\frac{(2l+1)(\mu_r'-1)\mu'_r}{(\mu_r'l+l+1)\left[\left(\frac{l+1}{l}\right)\mu_r'+1\right]-(l+1)\left(\frac{a}{b}\right)^{2l+1}(\mu_r'-1)^2}\right]
\end{equation}

So, comparing Eq.[\ref{k1s} and \ref{k1b}]
\begin{equation}\label{g}
G_l(l+1)=\muo H
\end{equation}

So, if Eq. [\ref{g}] is correct, then we are getting same result using scalar potential as we got from bound surface current method for \mathcal{K}_1.




So for \(a\textless r\textless b \) from Eq. [\ref{a<r<b}],

$$\Phi_m=\left(\beta r^l+\frac{\gamma}{r^{l+1}}\right)P_l(\cos\theta)$$
$$\Phi_m=\left[-r^l\left[\frac{(2l+1)\muo}{l\mu_r\left[1-\left(\frac{a}{b}\right)^{2l+1}}\right]\right]H-\frac{r^l}{r^{2l+1}}\left[\frac{(2l+1)a^{2l+1}\muo}{\mu_r(l+1)\left(1-\left(\frac{a}{b}\right)^{2l+1}\right)}\right] H}\right]P_l(\cos\theta)$$
$$\Phi_m=-\frac{(2l+1)Hr^lP_l(\cos\theta)}{\mu_r\left[1-\left(\frac{a}{b}\right)^{2l+1}}\right]}\left[\frac{\muo}{l}+\left(\frac{a}{r}\right)^{2l+1}\left(\frac{\muo}{l+1}\right)\right]$$
$$\Phi_m=-\frac{(2l+1)(\muo H)r^lP_l(\cos\theta)}{\mu_r(l+1)}\left[\frac{\frac{l+1}{l}+\left(\frac{a}{r}\right)^{2l+1}}{1-\left(\frac{a}{b}\right)^{2l+1}}\right]$$
$$\Phi_m=-\frac{(2l+1)Br^lP_l(\cos\theta)}{\mu_r(l+1)}\left[\frac{\frac{l+1}{l}+\left(\frac{a}{r}\right)^{2l+1}}{1-\left(\frac{a}{b}\right)^{2l+1}}\right]$$
\begin{equation}\label{p}
\Phi_m=-\frac{(2l+1)Br^lP_l(\cos\theta)}{\mu_r l}\left[\frac{1+\frac{l}{l+1}\left(\frac{a}{r}\right)^{2l+1}}{1-\left(\frac{a}{b}\right)^{2l+1}}\right]
\end{equation}
From Eq.[\ref{H}] we know,
$$\bm{H}=-\bm{\nabla}\Phi_m$$
$$H_r=-\frac{\delta \Phi_m}{\delta r}$$
Using eq. [\ref{p}] we get,
$$H_r=-\frac{\delta}{\delta r}\left[-\frac{(2l+1)Br^lP_l(\cos\theta)}{\mu_r l}\left[\frac{1+\frac{l}{l+1}\left(\frac{a}{r}\right)^{2l+1}}{1-\left(\frac{a}{b}\right)^{2l+1}}\right]\right]$$
$$H_r=\frac{(2l+1)BP_l(u)}{\mu_r l\left[1-\left(\frac{a}{b}\right)^{2l+1}\right]}\frac{\delta}{\delta r}\left[r^l+\frac{l}{l+1}a^{2l+1}r^{-2l-1+l}\right]$$
$$H_r=\frac{(2l+1)BP_l(u)}{\mu_r l\left[1-\left(\frac{a}{b}\right)^{2l+1}\right]}\left[l r^{l-1}+\frac{l}{l+1}a^{2l+1}\frac{(-l-1)}{r^{l+2}}\right]$$
$$H_r=\frac{(2l+1)BP_l(u)l r^{l-1}}{\mu_r l\left[1-\left(\frac{a}{b}\right)^{2l+1}\right]}\left[1-\left(\frac{a}{r}\right)^{2l+1}\right]$$
So we get,
\begin{equation}\label{Hr}
H_r=\frac{(2l+1)}{\mu_r}r^{l-1}BP_l(u)\left[\frac{1-\left(\frac{a}{r}\right)^{2l+1}}{1-\left(\frac{a}{b}\right)^{2l+1}}\right]
\end{equation}
Therefore,
\begin{equation}\label{Br}
B_r=(2l+1)r^{l-1}BP_l(u)\left[\frac{1-\left(\frac{a}{r}\right)^{2l+1}}{1-\left(\frac{a}{b}\right)^{2l+1}}\right]
\end{equation}

Similarly,
\begin{equation*}
\begin{split}
H_\theta &=-\frac{1}{r}\frac{\delta \Phi_m}{\delta \theta}\\
        & =-\frac{1}{r}\frac{\delta}{\delta \theta}\left[-\frac{(2l+1)Br^lP_l(\cos\theta)}{\mu_r l}\left[\frac{1+\frac{l}{l+1}\left(\frac{a}{r}\right)^{2l+1}}{1-\left(\frac{a}{b}\right)^{2l+1}}\right]\right]
\end{split}
\end{equation*}

So we get,
\begin{equation}\label{Ht}
H_\theta=-\frac{(2l+1)}{\mu_r l}Br^{l-1}P_l^1(u)}\left[\frac{1+\frac{l}{l+1}\left(\frac{a}{r}\right)^{2l+1}}{1-\left(\frac{a}{b}\right)^{2l+1}}\right]
\end{equation}

Therefore,
\begin{equation}\label{Bt}
B_\theta=-\frac{(2l+1)}{l}Br^{l-1}P_l^1(u)}\left[\frac{1+\frac{l}{l+1}\left(\frac{a}{r}\right)^{2l+1}}{1-\left(\frac{a}{b}\right)^{2l+1}}\right]
\end{equation}

So, finally,
\begin{equation}\label{B}
\bm{B}=(2l+1)r^{l-1}B_o P_l(u)\left[\frac{1-\left(\frac{a}{r}\right)^{2l+1}}{1-\left(\frac{a}{b}\right)^{2l+1}}\right] \hat{r}-\frac{(2l+1)}{l}B_o r^{l-1}P_l^1(u)}\left[\frac{1+\frac{l}{l+1}\left(\frac{a}{r}\right)^{2l+1}}{1-\left(\frac{a}{b}\right)^{2l+1}}\right] \hat{\theta}
\end{equation}





























\subsubsection{Using Equivalent Bound Surface Currents}

In general, any surface current bound to a sphere , and its resulting magnetic field, can be written in terms of spherical harmonics of order $m$ and degree $n$~\cite{CB1, smythe}.One can show , however, that the resulting equations arising from the boundary conditions on the tangential components of the magnetic field (i.e., $B$_\theta\  and \ $B$_\phi\)) are independent of the order $m$ of the spherical harmonic. Without loss of generality, then, we can restrict the analysis of spherical shields to zonal surface currents and field only (i.e. \phi-independent\ , \ $m$=0\) ).This also means that the following results can be applied to cases where tesseral components ($m$ \textgreater\) 0) do exist in the fields and currents, which is extremely valuable from the point of view of coil design, where the general spherical harmonics can be used as $building$ $blocks$ to produce a desire magnetic field.

From ~\cite{CB1, smythe}, the zonal surface current
\begin{equation}\label{i}
I_\phi\)=\bm{K}=KP_n^1(u)  \, \bm{\hat{\phi}}\)
\end{equation}
bound to a spherical surface $r=a$ gives rise to the vector potential
\begin{equation}\label{a}
\bm A =\mathcal{K} 
\left \{
  \begin{tabular}{ccc}
  r^n \bm{\hat{\phi}}\) &  & r\textless\) a  \\
  \(\frac{a^{2n+1}}{r^{n+1}}
\) &  & r\textgreater\) a  
  \end{tabular}
\right 
\end{equation}




where $P_n^1(u)$ is the associated Legendre function of order 1 and degree \($n$, $u$=cos\theta\), and the coefficient \(\mathcal{K}={\muo K}/({(2n+1)a^{n-1}})
\)has units T/m$^{n-1}$.

Consider a spherical $\mu_r$-metal shield of inner radius $r_1=R$ and outer radius $r_2=R+t$, and permeability \(\mu_r\) centered on the origin and exposed to the general zonal field (\textit{i.e.,} axisymmetric or $\phi$-independent) of order $n$~\cite{CB1, smythe}  in the presence of an externally applied magnetic field. 
\begin{equation}\label{bo}
\bm B_{\rm ext} = \bm{B_o} = G_n \, r^{n-1} \, (n+1)[n P_n(u) \, \bm{\hat{r}} -  P_n^1(u)  \, \bm{\hat{\theta}} ] \, ,
\label{BextS}
\end{equation}
where the magnitude $G_n$ is in units of T/m$^{n-1}$.
The magnetic fields arising from Eq. (\ref{a}) are
\begin{equation}\label{r=R}
\lim_{r\to R}\bm B =\mathcal{K}_1
\left \{
  \begin{tabular}{ccc}
  r^{n-1} \, (n+1)[n P_n(u) \, \bm{\hat{r}} -  P_n^1(u)  \, \bm{\hat{\theta}} ] \, &  & r\textless\) R  \\
  \(\frac{R^{2n+1}}{r^{n+2}}
\)n[(n+1) P_n(u) \, \bm{\hat{r}} +  P_n^1(u)  \, \bm{\hat{\theta}} ] \, &  & r\textgreater\) R  
  \end{tabular}
\right 
\end{equation}

\begin{equation}\label{r=R+t}
\lim_{r\to R+t}\bm B =\mathcal{K}_2
\left \{
  \begin{tabular}{ccc}
  r^{n-1} \, (n+1)[n P_n(u) \, \bm{\hat{r}} -  P_n^1(u)  \, \bm{\hat{\theta}} ] \, &  & r\textless\) R+t  \\
  \(\frac{{R+t}^{2n+1}}{r^{n+2}}
\)n[(n+1) P_n(u) \, \bm{\hat{r}} +  P_n^1(u)  \, \bm{\hat{\theta}} ] \, &  & r\textgreater\) R+t  
  \end{tabular}
\right \
\end{equation}
 The net field inside the shield (\textit{i.e.},  $r<R$) is 
\begin{equation}\label{r<R}
\lim_{r\to R}\bm B = (\mathcal{K}_1+\mathcal{K}_2+G_n) \, (n+1)\, r^{n-1} \, [n P_n(u) \, \bm{\hat{r}} -  P_n^1(u)  \, \bm{\hat{\theta}}
\end{equation}

The net field  within the bulk of the shield (\textit{i.e.}, $R<r<R+t$)  is 
\begin{equation}\label{R<r<R+t}
\lim_{r\to R<r<R+t}\bm B= \mathcal{K}_1 \, n\, \frac{R^{2n+1}}{r^{n+2}} \,[(n+1) P_n(u) \, \bm{\hat{r}} +  P_n^1(u)  \, \bm{\hat{\theta}} ] \,  +\, (\mathcal{K}_2+G_n) \, (n+1)\, r^{n-1} \, [n P_n(u) \, \bm{\hat{r}} -  P_n^1(u)  \, \bm{\hat{\theta}} ]
\end{equation}

The net field  outside the shield (\textit{i.e.}, $r>R+t$) is
\begin{multline}\label{r>R}
\lim_{r\to r>R+t}\bm B=\frac{\mathcal{K}_1R^{2n+1} + \mathcal{K}_2 (R+t)^{2n+1} }{r^{n+2}} \, n\, [(n+1) P_n(u) \, \bm{\hat{r}} +  P_n^1(u)  \, \bm{\hat{\theta}} ] \,  \\+\, G_n \, (n+1)\, r^{n-1} \, [n P_n(u) \, \bm{\hat{r}} - P_n^1(u)  \, \bm{\hat{\theta}} ]
\end{multline}

To find \(\mathcal{K}_1\) and \(\mathcal{K}_2\) , we have to apply the boundary condition for the tangential component of the magnetic field i.e.
\begin{equation}\label{bk}
\frac{1}{\muo} B_{1\theta}=  \frac{1}{\mu_r} B_{2\theta}  
\end{equation}

From Eq.  (\ref{r<R}), (\ref{R<r<R+t}) and (\ref{bk}) we get,
$$\(\mu_r B_{1\theta}=\muo B_{2\theta}\)$$
$$\(\mu_r[(\mathcal{K}_1+\mathcal{K}_2+G_n) \, (n+1)\, r^{n-1}(-  P_n^1(u))\ ]=\muo[\mathcal{K}_1 \, n\, \frac{R^{2n+1}}{r^{n+2}}P_n^1(u) \,  +\, (\mathcal{K}_2+G_n) \, (n+1)\, r^{n-1}( -  P_n^1(u))]\)$$

\begin{multline*}
-\mu_r \mathcal{K}_1 r^{n-1}(n+1) P_n^1(u)-\mu_r (\mathcal{K}_2+G_n)r^{n-1}(n+1) P_n^1(u)\\=\muo \mathcal{K}_1 n \frac{R^{2n+1}}{r^{n+2}}P_n^1(u)-\muo (\mathcal{K}_2+G_n)r^{n-1}(n+1) P_n^1(u)
\end{multline*}

$$\([\mu_r r^{n-1}(n+1)+\muo n \frac{R^{2n+1}}{r^{n+2}}]P_n^1(u)\mathcal{K}_1+(\mu_r-\muo)r^{n-1}(n+1)P_n^1(u)\mathcal{K}_2=-(\mu_r-\muo)r^{n-1}(n+1)P_n^1(u)G_n\)$$
Excluding $r^{n-1}(n+1)P_n^1(u)$ from both sides we get,
$$\([\mu_r+\muo\frac{n}{n+1}\left(\dfrac{R}{r}\right)^{2n+1}]\mathcal{K}_1 +(\mu_r-\muo)\mathcal{K}_2=-(\mu_r-\muo)G_n\)$$
So at r=R we get,
\begin{equation}\label{es1}
[\mu_r+\muo\frac{n}{n+1}]\mathcal{K}_1 +(\mu_r-\muo)\mathcal{K}_2=-(\mu_r-\muo)G_n
\end{equation}

Again, from Eq. (\ref{bk}) we get,
$$\(\frac{1}{\mu_r} B_{2\theta}=\frac{1}{\muo} B_{3\theta}\)$$
$$\(\muo B_{2\theta}=\mu_r B_{3\theta}\)$$

Using eq. , (\ref{R<r<R+t}), (\ref{r>R}) we get,
\begin{multline*}
\muo\left[\mathcal{K}_1 n\left(\frac{R^{2n+1}}{r^{n+2}}\right)P_n^1(u)+ (\mathcal{K}_2+G_n)(n+1)r^{n-1}\left(-P_n^1(u)\right)\right]\\= \mu_r\left[\frac{\mathcal{K}_1R^{2n+1} + \mathcal{K}_2 (R+t)^{2n+1} }{r^{n+2}}n P_n^1(u) +G_n (n+1) r^{n-1}\left(-  P_n^1(u)\right)\right ]
\end{multline*}

\begin{multline*}
(\mu_r-\muo)n\left(\frac{R^{2n+1}}{r^{n+2}}\right)P_n^1(u)\mathcal{K}_1+\left[\mu_r n\frac{(R+t)^{2n+1}}{r^{n+2}}+\muo(n+1)r^{n-1}\right]P_n^1(u)\mathcal{K}_2 \\= (\mu_r-\muo)r^{n-1}(n+1)P_n^1(u)G_n
\end{multline*}
Excluding $r^{n-1}(n+1)P_n^1(u)$ from both sides we get,
$$\((\mu_r-\muo)\frac{n}{n+1}\left(\dfrac{R}{r}\right)^{2n+1}\mathcal{K}_1+[\mu_r\frac{n}{n+1}\left(\dfrac{R+t}{r}\right)^{2n+1}+\muo]\mathcal{K}_2=(\mu_r-\muo)G_n\)$$
At $r=R+t$ we get,
\begin{equation}\label{es2}
(\mu_r-\muo)\frac{n}{n+1}\left(\dfrac{R}{R+t}\right)^{2n+1}\mathcal{K}_1+[\mu_r\frac{n}{n+1}+\muo]\mathcal{K}_2=(\mu_r-\muo)G_n
\end{equation}
Again from Eq. (\ref{es1}) we get,
$$\(\frac{[\mu_r+\muo\frac{n}{n+1}]}{(\mu_r-\muo)}\mathcal{K}_1 +\mathcal{K}_2=-G_n\)$$
\begin{equation}\label{k2}
\mathcal{K}_2=-G_n-\frac{\mu_r(n+1)+\muo n}{(\mu_r-\muo)(n+1)}\mathcal{K}_1
\end{equation}
Adding Eq. (\ref{es1}) and (\ref{es2}) we get,
\begin{multline*}
\left[\mu_r+\muo\frac{n}{n+1}\right]\mathcal{K}_1 +(\mu_r-\muo)\mathcal{K}_2+(\mu_r-\muo)G_n+(\mu_r-\muo)\frac{n}{n+1}\left(\dfrac{R}{R+t}\right)^{2n+1}\mathcal{K}_1\\+\left[\mu_r\frac{n}{n+1}+\muo\right]\mathcal{K}_2-(\mu_r-\muo)G_n=0
\end{multline*}
\begin{multline*}
[\mu_r+\muo\frac{n}{n+1}+(\mu_r-\muo)\frac{n}{n+1}\left(\dfrac{R}{R+t}\right)^{2n+1}]\mathcal{K}_1 +[\mu_r-\muo+\mu_r\frac{n}{n+1}+\muo]\mathcal{K}_2=0
\end{multline*}
Multiplying by $(n+1)$ we get,
\begin{multline*}
\left[\mu_r(n+1)+\muo{n}+(\mu_r-\muo){n}\left(\dfrac{R}{R+t}\right)^{2n+1}\right]\mathcal{K}_1 +\mu_r (2n+1)\mathcal{K}_2=0
\end{multline*}
Putting the value of \(\mathcal{K}_2\) from Eq.(\ref{k2}) we get,
\begin{multline*}
\left[\mu_r(n+1)+\muo{n}+(\mu_r-\muo){n}\left(\dfrac{R}{R+t}\right)^{2n+1}\right]\mathcal{K}_1 \\+\mu_r (2n+1) \left[-G_n-\frac{\mu_r(n+1)+\muo n}{(\mu_r-\muo)(n+1)}\mathcal{K}_1\right]=0
\end{multline*}

\begin{multline*}
\left[\frac{\mu_r^2 (n+1) (2n+1)+\muo\mu_r n(2n+1)}{(\mu_r-\muo)(n+1)}-\mu_r(n+1)-\muo{n}-(\mu_r-\muo){n}\left(\dfrac{R}{R+t}\right)^{2n+1}\right]\mathcal{K}_1\\=-\mu_r (2n+1)G_n
\end{multline*}

\begin{multline*}
[\mu_r^2 (n+1) (2n+1)+\muo\mu_r n(2n+1)-\mu_r^2(n+1)^2+\muo\mu_r(n+1)^2-\muo\mu_r n(n+1)
+\muo^2 n(n+1)\\-(\mu_r-\muo)^2 n(n+1)\left(\dfrac{R}{R+t}\right)^{2n+1}]\mathcal{K}_1=-\mu_r (\mu_r-\muo)(n+1)(2n+1)G_n
\end{multline*}


\begin{multline*}
[\mu_r^2(2n^2+2n+n+1-n^2-2n-1)+\muo\mu_r (2n^2+n-n^2-n)+\muo\mu_r(n+1)^2+\muo^2 n(n+1)\\-(\mu_r-\muo)^2 n(n+1)\left(\dfrac{R}{R+t}\right)^{2n+1}]\mathcal{K}_1=-\mu_r(\mu_r-\muo)(n+1) (2n+1)G_n
\end{multline*}

\begin{multline*}
[\mu_r^2 n(n+1)+\muo\mu_r n^2+\muo\mu_r(n+1)^2+\muo^2 n(n+1)-(\mu_r-\muo)^2 n(n+1)\left(\dfrac{R}{R+t}\right)^{2n+1}]\mathcal{K}_1\\=-\mu_r(\mu_r-\muo)(n+1) (2n+1)G_n
\end{multline*}

\begin{multline*}
[\mu_r n(\mu_r(n+1)+\muo n)+\muo(n+1)(\mu_r(n+1)+\muo n)-(\mu_r-\muo)^2 n(n+1)\left(\dfrac{R}{R+t}\right)^{2n+1}]\mathcal{K}_1\\=-\mu_r(\mu_r-\muo)(n+1) (2n+1)G_n
\end{multline*}

\begin{equation}\label{k1}
\mathcal{K}_1=-\frac{\mu_r(\mu_r-\muo)(n+1) (2n+1)G_n}{[\mu_r(n+1)+\muo n)][\mu_r n+\muo(n+1)]-(\mu_r-\muo)^2 n(n+1)\left(\dfrac{R}{R+t}\right)^{2n+1}}
\end{equation}
Now, from Eq. (\ref{R<r<R+t}), the net field  within the bulk of the shield (\textit{i.e.}, $R<r<R+t$)  is 
\begin{multline*}
\bm B_2= \left[\mathcal{K}_1 n(n+1)\frac{R^{2n+1}}{r^{n+2}}+(\mathcal{K}_2+G_n)n(n+1) r^{n-1}\right] P_n(u) \bm{\hat{r}}\\ + \left[\mathcal{K}_1 n\frac{R^{2n+1}}{r^{n+2}}-(\mathcal{K}_2+G_n)(n+1) r^{n-1}\right] P_n^1(u) \bm{\hat{\theta}}
\end{multline*}
\begin{multline}\label{B2}
% \begin{equation}\label{B2}
\bm B_2= n(n+1)r^{n-1}\left[\mathcal{K}_1 \left(\dfrac{R}{r}\right)^{2n+1}+(\mathcal{K}_2+G_n)\right] P_n(u) \bm{\hat{r}} \\+r^{n-1} \left[\mathcal{K}_1 n\left(\dfrac{R}{r}\right)^{2n+1}-(\mathcal{K}_2+G_n)(n+1) \right] P_n^1(u) \bm{\hat{\theta}}
% \end{equation}
\end{multline}
For \(\mu_r\gg\muo\), from Eq.(\ref{k1}) we get,
\begin{equation}\label{k12}
\begin{split}
\mathcal{K}_1^\infty & =-\frac{\mu_r^2(n+1) (2n+1)G_n}{\mu_r^2 n(n+1)-\mu_r^2 n(n+1)\left(\dfrac{R}{R+t}\right)^{2n+1}}\\
& =-\frac{(2n+1)G_n}{n-n\left(\dfrac{R}{R+t}\right)^{2n+1}}
\end{split}
\end{equation}


From Eq. (\ref{k2}) we get,
$$\(\mathcal{K}_2^\infty=-G_n-\frac{\mu_r(n+1)}{\mu_r(n+1}\mathcal{K}_1^\infty\)$$
$$\(\mathcal{K}_2^\infty=-G_n-\mathcal{K}_1^\infty\)$$
$$\(\mathcal{K}_2^\infty+G_n=-\mathcal{K}_1^\infty\)$$
Putting the value in Eq.(\ref{B2}) we get,
\begin{align*}
&\bm B_2= n(n+1)r^{n-1}[\mathcal{K}_1 \left(\dfrac{R}{r}\right)^{2n+1}-\mathcal{K}_1] P_n(u) \bm{\hat{r}} +r^{n-1} [\mathcal{K}_1 n\left(\dfrac{R}{r}\right)^{2n+1}+\mathcal{K}_1(n+1) ] P_n^1(u) \bm{\hat{\theta}}\\
&\bm B_2= n(n+1)r^{n-1}\mathcal{K}_1[\left(\dfrac{R}{r}\right)^{2n+1}-1] P_n(u) \bm{\hat{r}} +r^{n-1} \mathcal{K}_1[n\left(\dfrac{R}{r}\right)^{2n+1}+(n+1) ] P_n^1(u) \bm{\hat{\theta}}
\end{align*}

Now, using eq. (\ref{k12}) we get,
\begin{multline*}
\bm B_2= -n(n+1)r^{n-1}\left[-\frac{(2n+1)G_n}{n-n\left(\dfrac{R}{R+t}\right)^{2n+1}}][1-\left(\dfrac{R}{r}\right)^{2n+1}\right] P_n(u) \bm{\hat{r}} \\+r^{n-1} \left[-\frac{(2n+1)G_n}{n-n\left(\dfrac{R}{R+t}\right)^{2n+1}}\right]\left[(n+1)+n\left(\dfrac{R}{r}\right)^{2n+1}\right] P_n^1(u) \bm{\hat{\theta}}
\end{multline*}
\begin{multline*}
\bm B_2= (n+1)(2n+1)r^{n-1}G_n\left[\frac{1-\left(\dfrac{R}{r}\right)^{2n+1}}{1-\left(\dfrac{R}{R+t}\right)^{2n+1}}\right] P_n(u) \bm{\hat{r}} \\-(2n+1)r^{n-1}G_n\left [\frac{\frac{n+1}{n}+\left(\dfrac{R}{r}\right)^{2n+1}}{1-\left(\dfrac{R}{R+t}\right)^{2n+1}}\right] P_n^1(u) \bm{\hat{\theta}}
\end{multline*}
\begin{multline*}
\bm B_2= (n+1)(2n+1)r^{n-1}G_n\left[\frac{1-\left(\dfrac{R}{r}\right)^{2n+1}}{1-\left(\dfrac{R}{R+t}\right)^{2n+1}}\right] P_n(u) \bm{\hat{r}} \\-\frac{(2n+1)}{n}(n+1)r^{n-1}G_n \left[\frac{1+\frac{n}{n+1}\left(\dfrac{R}{r}\right)^{2n+1}}{1-\left(\dfrac{R}{R+t}\right)^{2n+1}}\right] P_n^1(u) \bm{\hat{\theta}}
\end{multline*}

Therefore,

$$\(\bm B_2= (n+1)(2n+1)r^{n-1}G_n\left[\left[\frac{1-\left(\dfrac{R}{r}\right)^{2n+1}}{1-\left(\dfrac{R}{R+t}\right)^{2n+1}}\right] P_n(u) \bm{\hat{r}} -\frac{1}{n}\left[\frac{1+\frac{n}{n+1}\left(\dfrac{R}{r}\right)^{2n+1}}{1-\left(\dfrac{R}{R+t}\right)^{2n+1}}\right] P_n^1(u) \bm{\hat{\theta}}\right]\)$$


\subsection{Uniform internal field generation}
From ~\cite{ smythe}, the field in the region r\textless a for a current loop at \(\theta=\alpha\) is,
\begin{equation}\label{bbr}
B_r=\frac{\muo I \sin \alpha}{2a}\sum_{n=1}^{\infty} \left(\frac{r}{a}\right)^{n-1}P_n^1(\cos\alpha) P_n(\cos\theta)
\end{equation}
\begin{equation}\label{bbt}
B_\theta=-\frac{\muo I \sin \alpha}{2a}\sum_{n=1}^{\infty} \frac{1}{n}\left(\frac{r}{a}\right)^{n-1}P_n^1(\cos\alpha) P_n^1(\cos\theta)
\end{equation}

For Helmholtz coils of radius \(r_c\) located at \(z=\pm z_o\) i.e. for two loops at \(\theta=\alpha, \pi-\alpha \), \(I_2=I_1=I\) and \(z_o=\frac{r_c}{2}\),
\begin{equation}\label{sinH}
\sin\alpha=\frac{r_c}{a}=\frac{r_c}{\sqrt{z_o^2+r_c^2}}=\frac{r_c}{\sqrt{\left(\frac{r_c}{2}\right)^2+r_c^2}}=\frac{r_c}{\sqrt{\frac{5r_c^2}{4}}}=\frac{2}{\sqrt{5}}
\end{equation}
\begin{equation}\label{sinHb}
\sin(\pi-\alpha)=\sin\pi\cos\alpha-\cos\pi\sin\alpha=\sin\alpha=\frac{2}{\sqrt{5}}
\end{equation}

\begin{equation}\label{cosH}
\cos\alpha=\frac{z_o}{a}=\frac{r_c}{2\sqrt{z_o^2+r_c^2}}=\frac{r_c}{2\sqrt{\left(\frac{r_c}{2}\right)^2+r_c^2}}=\frac{r_c}{2\sqrt{\frac{5r_c^2}{4}}}=\frac{1}{\sqrt{5}}
\end{equation}
\begin{equation}\label{cosHb}
\cos(\pi-\alpha)=\cos\pi\cos\alpha+\sin\pi\sin\alpha=-\cos\alpha=-\frac{1}{\sqrt{5}}
\end{equation}


So putting the values from eq. [\ref{sinH} and \ref{cosH}] in eq. [\ref{bbr}] we get,
$$B_r^{top}=\frac{\muo I 2}{2a\sqrt{5}}\sum_{n=1}^{\infty} \left(\frac{r}{a}\right)^{n-1}P_n^1\left(\frac{1}{\sqrt{5}}\right) P_n(\cos\theta)$$
So,
\begin{equation}\label{bbrH}
B_r^{top}=\frac{\muo I }{\sqrt{5}a}\sum_{n=1}^{\infty} \left(\frac{r}{a}\right)^{n-1}P_n^1\left(\frac{1}{\sqrt{5}}\right) P_n(\cos\theta)
\end{equation}
Again, putting the values from eq. [\ref{sinHb} and \ref{cosHb}] in eq. [\ref{bbr}] we get,
\begin{equation}\label{bbrHb}
B_r^{bottom}=\frac{\muo I }{\sqrt{5}a}\sum_{n=1}^{\infty} \left(\frac{r}{a}\right)^{n-1}P_n^1\left(-\frac{1}{\sqrt{5}}\right) P_n(\cos\theta)
\end{equation}
Therefore,
\begin{equation}\label{brtot}
B_r=B_r^{top}+B_r^{bottom}
\end{equation}
Using eq. [\ref{bbrH} and \ref{bbrHb}],
$$B_r=\frac{\muo I }{\sqrt{5}a}\sum_{n=1}^{\infty} \left(\frac{r}{a}\right)^{n-1}P_n^1\left(\frac{1}{\sqrt{5}}\right) P_n(\cos\theta)+\frac{\muo I }{\sqrt{5}a}\sum_{n=1}^{\infty} \left(\frac{r}{a}\right)^{n-1}P_n^1\left(-\frac{1}{\sqrt{5}}\right) P_n(\cos\theta)$$
Again,
\begin{equation}\label{leg}
P_n^1(-u)=(-1)^{n+1}P_n^1(u)
\end{equation}
From eq.[\ref{leg}] we get that only the odd n terms contribute to net field of eq.[\ref{brtot}]. Furthermore, since \(P_3^1\left(\pm\frac{1}{\sqrt{5}}\right)\) is uniquely zero – thereby eliminating the n = 3.


So we get,
\begin{equation}\label{brtotf}
B_r=\frac{2 \muo I }{\sqrt{5}a}\sum_{n=1,5,7,...}^{\infty} \left(\frac{r}{a}\right)^{n-1}P_n^1\left(\frac{1}{\sqrt{5}}\right) P_n(\cos\theta)
\end{equation}
Similarly, putting the values from eq. [\ref{sinH} and \ref{cosH}] in eq. [\ref{bbt}] we get,
$$B_\theta^{top}=-\frac{\muo I 2}{2a\sqrt{5}}\sum_{n=1}^{\infty} \frac{1}{n}\left(\frac{r}{a}\right)^{n-1}P_n^1\left(\frac{1}{\sqrt{5}}\right) P_n^1(\cos\theta)$$
So,
\begin{equation}\label{bbtH}
B_\theta^{top}=-\frac{\muo I }{\sqrt{5}a}\sum_{n=1}^{\infty}\frac{1}{n} \left(\frac{r}{a}\right)^{n-1}P_n^1\left(\frac{1}{\sqrt{5}}\right) P_n^1(\cos\theta)
\end{equation}

Again, putting the values from eq. [\ref{sinHb} and \ref{cosHb}] in eq. [\ref{bbt}] we get,
\begin{equation}\label{bbtHb}
B_\theta^{bottom}=-\frac{\muo I }{\sqrt{5}a}\sum_{n=1}^{\infty}\frac{1}{n} \left(\frac{r}{a}\right)^{n-1}P_n^1\left(\frac{1}{\sqrt{5}}\right) P_n^1(\cos\theta)
\end{equation}

Therefore,
\begin{equation}\label{bttot}
B_\theta=B_\theta^{top}+B_\theta^{bottom}
\end{equation}
Using eq. [\ref{bbtH} and \ref{bbtHb}],
$$B_\theta=-\frac{\muo I }{\sqrt{5}a}\sum_{n=1}^{\infty}\frac{1}{n} \left(\frac{r}{a}\right)^{n-1}P_n^1\left(\frac{1}{\sqrt{5}}\right) P_n^1(\cos\theta)-\frac{\muo I }{\sqrt{5}a}\sum_{n=1}^{\infty}\frac{1}{n} \left(\frac{r}{a}\right)^{n-1}P_n^1\left(\frac{1}{\sqrt{5}}\right) P_n^1(\cos\theta)$$

As discussed above we get,
\begin{equation}\label{brtotf}
B_\theta=-\frac{2 \muo I }{\sqrt{5}a}\sum_{n=1,5,7,...}^{\infty}
\frac{1}{n}\left(\frac{r}{a}\right)^{n-1}P_n^1\left(\frac{1}{\sqrt{5}}\right) P_n^1(\cos\theta)
\end{equation}



Now again for Anti-Helmholtz coils of radius \(r_c\) located at \(z=\pm z_o\) i.e. for two loops at \(\theta=\alpha, \pi-\alpha \), \(I_2=-I_1=I\) and \(z_o=\sqrt{3}\frac{r_c}{2}\),
\begin{equation}\label{sinAH}
\sin\alpha=\frac{r_c}{a}=\frac{r_c}{\sqrt{z_o^2+r_c^2}}=\frac{r_c}{\sqrt{\left(\frac{\sqrt{3}r_c}{2}\right)^2+r_c^2}}=\frac{r_c}{\sqrt{\frac{7r_c^2}{4}}}=\frac{2}{\sqrt{7}}
\end{equation}
\begin{equation}\label{sinAHb}
\sin(\pi-\alpha)=\sin\pi\cos\alpha-\cos\pi\sin\alpha=\sin\alpha=\frac{2}{\sqrt{7}}
\end{equation}
\begin{equation}\label{cosAH}
\cos\alpha=\frac{z_o}{a}=\frac{\sqrt{3}r_c}{2\sqrt{z_o^2+r_c^2}}=\frac{\sqrt{3}r_c}{2\sqrt{\left(\frac{\sqrt{3}r_c}{2}\right)^2+r_c^2}}=\frac{\sqrt{3}r_c}{2\sqrt{\frac{7r_c^2}{4}}}=\sqrt{\frac{3}{7}}
\end{equation}
\begin{equation}\label{cosAHb}
\cos(\pi-\alpha)=\cos\pi\cos\alpha+\sin\pi\sin\alpha=-\cos\alpha=-\sqrt{\frac{3}{7}}
\end{equation}

So puting the values from eq. [\ref{sinAH} and \ref{cosAH}] in eq. [\ref{bbr}] we get,
$$B_{r(anti)}^{top}=\frac{\muo (-I) 2}{2a\sqrt{7}}\sum_{n=1}^{\infty} \left(\frac{r}{a}\right)^{n-1}P_n^1\left(\sqrt{\frac{3}{7}}\right) P_n(\cos\theta)$$
So,
\begin{equation}\label{bbrAH}
B_{r(anti)}^{top}=-\frac{\muo I }{\sqrt{7}a}\sum_{n=1}^{\infty} \left(\frac{r}{a}\right)^{n-1}P_n^1\left(\sqrt{\frac{3}{7}}\right) P_n(\cos\theta)
\end{equation}
Again, putting the values from eq. [\ref{sinAHb} and \ref{cosAHb}] in eq. [\ref{bbr}] we get,
\begin{equation}\label{bbrAHb}
B_{r(anti)}^{bottom}=\frac{\muo I }{\sqrt{7}a}\sum_{n=1}^{\infty} \left(\frac{r}{a}\right)^{n-1}P_n^1\left(-\sqrt{\frac{3}{7}}\right) P_n(\cos\theta)
\end{equation}


Therefore,
$$B_{r(anti)}=B_{r(anti)}^{top}+B_{r(anti)}^{bottom}$$
Using eq. [\ref{bbrAH} and \ref{bbrAHb}],
$$B_{r(anti)}=-\frac{\muo I }{\sqrt{7}a}\sum_{n=1}^{\infty} \left(\frac{r}{a}\right)^{n-1}P_n^1\left(\sqrt{\frac{3}{7}}\right) P_n(\cos\theta)+\frac{\muo I }{\sqrt{7}a}\sum_{n=1}^{\infty} \left(\frac{r}{a}\right)^{n-1}P_n^1\left(-\sqrt{\frac{3}{7}}\right) P_n(\cos\theta)$$

From eq.[\ref{leg}] we get that only the even n terms contribute to net field of eq.[\ref{brtot}]. So, we get,
\begin{equation}\label{brtotAH}
B_{r(anti)}=-\frac{2 \muo I }{\sqrt{7}a}\sum_{n=2,4,6,...}^{\infty}
\left(\frac{r}{a}\right)^{n-1}P_n^1\left(\sqrt{\frac{3}{7}}\right) P_n(\cos\theta)
\end{equation}


Similarly,puting the values from eq. [\ref{sinAH} and \ref{cosAH}] in eq. [\ref{bbt}] we get,
$$B_{\theta(anti)}^{top}=-\frac{\muo (-I) 2}{2a\sqrt{7}}\sum_{n=1}^{\infty} \frac{1}{n}\left(\frac{r}{a}\right)^{n-1}P_n^1\left(\sqrt{\frac{3}{7}}\right) P_n^1(\cos\theta)$$
So,
\begin{equation}\label{bbtAH}
B_{\theta(anti)}^{top}=\frac{\muo I }{\sqrt{7}a}\sum_{n=1}^{\infty} \frac{1}{n}\left(\frac{r}{a}\right)^{n-1}P_n^1\left(\sqrt{\frac{3}{7}}\right) P_n^1(\cos\theta)
\end{equation}
Again, putting the values from eq. [\ref{sinAHb} and \ref{cosAHb}] in eq. [\ref{bbt}] we get,
\begin{equation}\label{bbtAHb}
B_{\theta(anti)}^{bottom}=-\frac{\muo I }{\sqrt{7}a}\sum_{n=1}^{\infty}\frac{1}{n} \left(\frac{r}{a}\right)^{n-1}P_n^1\left(-\sqrt{\frac{3}{7}}\right) P_n^1(\cos\theta)
\end{equation}


Therefore,
$$B_{\theta(anti)}=B_{\theta(anti)}^{top}+B_{\theta(anti)}^{bottom}$$
Using eq. [\ref{bbtAH} and \ref{bbtAHb}],
$$B_{\theta(anti)}=\frac{\muo I }{\sqrt{7}a}\sum_{n=1}^{\infty} \frac{1}{n}\left(\frac{r}{a}\right)^{n-1}P_n^1\left(\sqrt{\frac{3}{7}}\right) P_n^1(\cos\theta)-\frac{\muo I }{\sqrt{7}a}\sum_{n=1}^{\infty}\frac{1}{n} \left(\frac{r}{a}\right)^{n-1}P_n^1\left(-\sqrt{\frac{3}{7}}\right) P_n^1(\cos\theta)$$

As discussed above we get,
\begin{equation}\label{bttotAH}
B_{\theta(anti)}=\frac{2 \muo I }{\sqrt{7}a}\sum_{n=2,4,6,...}^{\infty}
\frac{1}{n}\left(\frac{r}{a}\right)^{n-1}P_n^1\left(\sqrt{\frac{3}{7}}\right) P_n^1(\cos\theta)
\end{equation}
%  \section{Uniform internal field generation}
From ~\cite{ smythe}, the field in the region $r< a$ for a current loop at \(\theta=\alpha\) is,
\begin{equation}\label{bbr}
B_r=\frac{\muo I \sin \alpha}{2a}\sum_{n=1}^{\infty} \left(\frac{r}{a}\right)^{l-1}P_l^1(\cos\alpha) P_l(\cos\theta)
\end{equation}
\begin{equation}\label{bbt}
B_\theta=-\frac{\muo I \sin \alpha}{2a}\sum_{n=1}^{\infty} \frac{1}{n}\left(\frac{r}{a}\right)^{l-1}P_l^1(\cos\alpha) P_l^1(\cos\theta)
\end{equation}

For Helmholtz coils of radius \(r_c\) located at \(z=\pm z_o\) i.e. for two loops at \(\theta=\alpha, \pi-\alpha \), \(I_2=I_1=I\) and \(z_o=\frac{r_c}{2}\),
\begin{equation}\label{sinH}
\sin\alpha=\frac{r_c}{a}=\frac{r_c}{\sqrt{z_o^2+r_c^2}}=\frac{r_c}{\sqrt{\left(\frac{r_c}{2}\right)^2+r_c^2}}=\frac{r_c}{\sqrt{\frac{5r_c^2}{4}}}=\frac{2}{\sqrt{5}}
\end{equation}
\begin{equation}\label{sinHb}
\sin(\pi-\alpha)=\sin\pi\cos\alpha-\cos\pi\sin\alpha=\sin\alpha=\frac{2}{\sqrt{5}}
\end{equation}

\begin{equation}\label{cosH}
\cos\alpha=\frac{z_o}{a}=\frac{r_c}{2\sqrt{z_o^2+r_c^2}}=\frac{r_c}{2\sqrt{\left(\frac{r_c}{2}\right)^2+r_c^2}}=\frac{r_c}{2\sqrt{\frac{5r_c^2}{4}}}=\frac{1}{\sqrt{5}}
\end{equation}
\begin{equation}\label{cosHb}
\cos(\pi-\alpha)=\cos\pi\cos\alpha+\sin\pi\sin\alpha=-\cos\alpha=-\frac{1}{\sqrt{5}}
\end{equation}


Using Eqs.~(\ref{sinH}), and (\ref{cosH}) in Eq.~(\ref{bbr}):
% $$B_r^{top}=\frac{\muo I 2}{2a\sqrt{5}}\sum_{n=1}^{\infty} \left(\frac{r}{a}\right)^{l-1}P_l^1\left(\frac{1}{\sqrt{5}}\right) P_l(\cos\theta)$$
% So,
\begin{equation}\label{bbrH}
B_r^{top}=\frac{\muo I }{\sqrt{5}a}\sum_{n=1}^{\infty} \left(\frac{r}{a}\right)^{l-1}P_l^1\left(\frac{1}{\sqrt{5}}\right) P_l(\cos\theta)
\end{equation}

Using Eqs.~(\ref{sinHb}), and (\ref{cosHb}) in Eq.~(\ref{bbr}):
\begin{equation}\label{bbrHb}
B_r^{bottom}=\frac{\muo I }{\sqrt{5}a}\sum_{n=1}^{\infty} \left(\frac{r}{a}\right)^{l-1}P_l^1\left(-\frac{1}{\sqrt{5}}\right) P_l(\cos\theta)
\end{equation}

\begin{equation}\label{brtot}
B_r=B_r^{top}+B_r^{bottom}
\end{equation}

\begin{multline}\label{brtot_1}
    B_r=\frac{\muo I }{\sqrt{5}a}\sum_{n=1}^{\infty} \left(\frac{r}{a}\right)^{l-1}P_l^1\left(\frac{1}{\sqrt{5}}\right) P_l(\cos\theta) \\+\frac{\muo I }{\sqrt{5}a}\sum_{n=1}^{\infty} \left(\frac{r}{a}\right)^{l-1}P_l^1\left(-\frac{1}{\sqrt{5}}\right) P_l(\cos\theta)
\end{multline}


\begin{equation}\label{leg}
P_l^1(-u)=(-1)^{l+1}P_l^1(u)
\end{equation}

From Eq.~(\ref{leg}) we get that only the odd n terms contribute to net field of Eq.~(\ref{brtot_1}). Furthermore, since \(P_3^1\left(\pm\frac{1}{\sqrt{5}}\right)\) is uniquely zero – thereby eliminating the n = 3.

\begin{equation}\label{brtotf}
B_r=\frac{2 \muo I }{\sqrt{5}a}\sum_{n=1,5,7,...}^{\infty} \left(\frac{r}{a}\right)^{l-1}P_l^1\left(\frac{1}{\sqrt{5}}\right) P_l(\cos\theta)
\end{equation}

Similarly, using Eqs.~(\ref{sinH}) and (\ref{cosH}) in Eq.~\ref{bbt}) :
% $$B_\theta^{top}=-\frac{\muo I 2}{2a\sqrt{5}}\sum_{n=1}^{\infty} \frac{1}{n}\left(\frac{r}{a}\right)^{l-1}P_l^1\left(\frac{1}{\sqrt{5}}\right) P_l^1(\cos\theta)$$
% So,
\begin{equation}\label{bbtH}
B_\theta^{top}=-\frac{\muo I }{\sqrt{5}a}\sum_{n=1}^{\infty}\frac{1}{n} \left(\frac{r}{a}\right)^{l-1}P_l^1\left(\frac{1}{\sqrt{5}}\right) P_l^1(\cos\theta)
\end{equation}

Again, using Eqs.~(\ref{sinHb}) and (\ref{cosHb}) in Eq.~\ref{bbt}) :
\begin{equation}\label{bbtHb}
B_\theta^{bottom}=-\frac{\muo I }{\sqrt{5}a}\sum_{n=1}^{\infty}\frac{1}{n} \left(\frac{r}{a}\right)^{l-1}P_l^1\left(\frac{1}{\sqrt{5}}\right) P_l^1(\cos\theta)
\end{equation}

\begin{equation}\label{bttot}
B_\theta=B_\theta^{top}+B_\theta^{bottom}
\end{equation}

\begin{multline}\label{bttot_1}
    B_\theta=-\frac{\muo I }{\sqrt{5}a}\sum_{n=1}^{\infty}\frac{1}{n} \left(\frac{r}{a}\right)^{l-1}P_l^1\left(\frac{1}{\sqrt{5}}\right) P_l^1(\cos\theta) \\-\frac{\muo I }{\sqrt{5}a}\sum_{n=1}^{\infty}\frac{1}{n} \left(\frac{r}{a}\right)^{l-1}P_l^1\left(\frac{1}{\sqrt{5}}\right) P_l^1(\cos\theta)
\end{multline}


As discussed above, Eq.~(\ref{bttot_1}) will be reduced to -
\begin{equation}\label{brtotf}
B_\theta=-\frac{2 \muo I }{\sqrt{5}a}\sum_{n=1,5,7,...}^{\infty}
\frac{1}{n}\left(\frac{r}{a}\right)^{l-1}P_l^1\left(\frac{1}{\sqrt{5}}\right) P_l^1(\cos\theta)
\end{equation}



Now again for Anti-Helmholtz coils of radius \(r_c\) located at \(z=\pm z_o\) i.e. for two loops at \(\theta=\alpha, \pi-\alpha \), \(I_2=-I_1=I\) and \(z_o=\sqrt{3}\frac{r_c}{2}\),
\begin{equation}\label{sinAH}
\sin\alpha=\frac{r_c}{a}=\frac{r_c}{\sqrt{z_o^2+r_c^2}}=\frac{r_c}{\sqrt{\left(\frac{\sqrt{3}r_c}{2}\right)^2+r_c^2}}=\frac{r_c}{\sqrt{\frac{7r_c^2}{4}}}=\frac{2}{\sqrt{7}}
\end{equation}
\begin{equation}\label{sinAHb}
\sin(\pi-\alpha)=\sin\pi\cos\alpha-\cos\pi\sin\alpha=\sin\alpha=\frac{2}{\sqrt{7}}
\end{equation}
\begin{equation}\label{cosAH}
\cos\alpha=\frac{z_o}{a}=\frac{\sqrt{3}r_c}{2\sqrt{z_o^2+r_c^2}}=\frac{\sqrt{3}r_c}{2\sqrt{\left(\frac{\sqrt{3}r_c}{2}\right)^2+r_c^2}}=\frac{\sqrt{3}r_c}{2\sqrt{\frac{7r_c^2}{4}}}=\sqrt{\frac{3}{7}}
\end{equation}
\begin{equation}\label{cosAHb}
\cos(\pi-\alpha)=\cos\pi\cos\alpha+\sin\pi\sin\alpha=-\cos\alpha=-\sqrt{\frac{3}{7}}
\end{equation}

Using Eqs.~(\ref{sinAH}) and (\ref{cosAH}) in Eq.~(\ref{bbr}) :
% $$B_{r(anti)}^{top}=\frac{\muo (-I) 2}{2a\sqrt{7}}\sum_{n=1}^{\infty} \left(\frac{r}{a}\right)^{l-1}P_l^1\left(\sqrt{\frac{3}{7}}\right) P_l(\cos\theta)$$
% So,
\begin{equation}\label{bbrAH}
B_{r(anti)}^{top}=-\frac{\muo I }{\sqrt{7}a}\sum_{n=1}^{\infty} \left(\frac{r}{a}\right)^{l-1}P_l^1\left(\sqrt{\frac{3}{7}}\right) P_l(\cos\theta)
\end{equation}

Using Eqs.~(\ref{sinAHb}) and (\ref{cosAHb}) in Eq.~(\ref{bbr}) :
\begin{equation}\label{bbrAHb}
B_{r(anti)}^{bottom}=\frac{\muo I }{\sqrt{7}a}\sum_{n=1}^{\infty} \left(\frac{r}{a}\right)^{l-1}P_l^1\left(-\sqrt{\frac{3}{7}}\right) P_l(\cos\theta)
\end{equation}

$$B_{r(anti)}=B_{r(anti)}^{top}+B_{r(anti)}^{bottom}$$

\begin{multline}\label{brtotAH_1}
    B_{r(anti)}=-\frac{\muo I }{\sqrt{7}a}\sum_{n=1}^{\infty} \left(\frac{r}{a}\right)^{l-1}P_l^1\left(\sqrt{\frac{3}{7}}\right) P_l(\cos\theta) \\+\frac{\muo I }{\sqrt{7}a}\sum_{n=1}^{\infty} \left(\frac{r}{a}\right)^{l-1}P_l^1\left(-\sqrt{\frac{3}{7}}\right) P_l(\cos\theta)
\end{multline}

From Eq.~(\ref{leg}) we get that only the even n terms contribute to net field of Eq.~(\ref{brtotAH_1}). So, we get,
\begin{equation}\label{brtotAH}
B_{r(anti)}=-\frac{2 \muo I }{\sqrt{7}a}\sum_{n=2,4,6,...}^{\infty}
\left(\frac{r}{a}\right)^{l-1}P_l^1\left(\sqrt{\frac{3}{7}}\right) P_l(\cos\theta)
\end{equation}


Similarly, using Eqs.~(\ref{sinAH}) and (\ref{cosAH}) in Eq.~(\ref{bbt}) :
% $$B_{\theta(anti)}^{top}=-\frac{\muo (-I) 2}{2a\sqrt{7}}\sum_{n=1}^{\infty} \frac{1}{n}\left(\frac{r}{a}\right)^{l-1}P_l^1\left(\sqrt{\frac{3}{7}}\right) P_l^1(\cos\theta)$$
% So,
\begin{equation}\label{bbtAH}
B_{\theta(anti)}^{top}=\frac{\muo I }{\sqrt{7}a}\sum_{n=1}^{\infty} \frac{1}{n}\left(\frac{r}{a}\right)^{l-1}P_l^1\left(\sqrt{\frac{3}{7}}\right) P_l^1(\cos\theta)
\end{equation}

Using Eqs.~(\ref{sinAHb}) and (\ref{cosAHb}) in Eq.~(\ref{bbt}) :
\begin{equation}\label{bbtAHb}
B_{\theta(anti)}^{bottom}=-\frac{\muo I }{\sqrt{7}a}\sum_{n=1}^{\infty}\frac{1}{n} \left(\frac{r}{a}\right)^{l-1}P_l^1\left(-\sqrt{\frac{3}{7}}\right) P_l^1(\cos\theta)
\end{equation}


$$B_{\theta(anti)}=B_{\theta(anti)}^{top}+B_{\theta(anti)}^{bottom}$$

\begin{multline}\label{bttotAH_1}
    B_{\theta(anti)}=\frac{\muo I }{\sqrt{7}a}\sum_{n=1}^{\infty} \frac{1}{n}\left(\frac{r}{a}\right)^{l-1}P_l^1\left(\sqrt{\frac{3}{7}}\right) P_l^1(\cos\theta) \\-\frac{\muo I }{\sqrt{7}a}\sum_{n=1}^{\infty}\frac{1}{n} \left(\frac{r}{a}\right)^{l-1}P_l^1\left(-\sqrt{\frac{3}{7}}\right) P_l^1(\cos\theta)
\end{multline}


As discussed above, Eq.~(\ref{bttotAH_1}) will be reduced to -
\begin{equation}\label{bttotAH}
B_{\theta(anti)}=\frac{2 \muo I }{\sqrt{7}a}\sum_{n=2,4,6,...}^{\infty}
\frac{1}{n}\left(\frac{r}{a}\right)^{l-1}P_l^1\left(\sqrt{\frac{3}{7}}\right) P_l^1(\cos\theta)
\end{equation}

\addtocontents{toc}{\vspace{2em}}  % Add a gap in the Contents, for aesthetics


%% ----------------------------------------------------------------

\label{Bibliography}
\lhead{\emph{Bibliography}} 

\bibliography{Bibliography}
 \bibliographystyle{unsrt}
% \bibliographystyle{plain}
% \bibliographystyle{ieeetr}


% \bibliography{Bibliography} \addcontentsline{toc}{chapter}{\bibname}
% \bibliography{url}
\end{document}
